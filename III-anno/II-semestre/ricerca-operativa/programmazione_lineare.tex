\section{Programmazione lineare}

Ci occuperemo di problemi di programmazione lineare.
I problemi di programmazione lineare sono problemi di programmazione matematica in cui tutte le funzioni $f(x), g_i(x)$.

Consideriamo problemi di programmazione lineare in cui le variabili possono assumere valori reali.
Con problemi di programmazione lineare possiamo fare sempre tre assunzioni:
\begin{enumerate}
	\item c'\`e proporzionalit\`a diretta con le variabili;
	\item c'\`e additivit\`a;
	\item le variabili sono continue.
\end{enumerate}

Introdurremo dei modelli di programmazione lineare, dotati di sufficiente generalit\`a, facili perch\'e usano solo algebra lineare, per i quali disponiamo di algoritmi risolutivi efficienti, e per i quali \`e possibile condurre un'analisi della sensitivit\`a post-ottimale: come varia la soluzione al variare dei dati.

Un problema di programmazione lineare sar\`a qualcosa del tipo:
\[
	\begin{cases}
		\min \left( c_1 x_1 + \dots + c_n x_n \right) \\
		a_{1,1} x_1 + \dots + a_{1,n} x_n \ge b_1 \\
		dots \\
		a_{m,1} x_1 + \dots + a_{m,n} x_n \ge b_m
	\end{cases}
\]

A chi non fa venire in mente delle matrici?
Chiamiamo:
\[
	c =
	\begin{pmatrix}
		c_1 \\ \vdots \\ c_n
	\end{pmatrix}
	\qquad
	x =
	\begin{pmatrix}
		x_1 \\ \vdots \\ x_n
	\end{pmatrix}
	\qquad
	b =
	\begin{pmatrix}
		b_1 \\ \vdots \\ b_m
	\end{pmatrix}
\]
Indicando con $c^{T}$ la trasposta di $c$ (che \`e un vettore colonna), possiamo scrivere la funzione obiettivo come $\min \left( c^{T} x \right)$.
La funzione obiettivo \`e il prodotto scalare dei vettori $c$ e $x$, ma possiamo anche trattarli come matrici, rendere $c$ un vettore riga, e fare il prodotto matriciale fra i due.

Per i vincoli (che sono $m$) creiamo una matrice $A$ di dimensione $m \times n$, dove $[A]_{i,j} = a_{i,j}$ nella notazione di prima.
Scriviamo quindi il problema come:
\[
	\begin{cases}
		\min \left( c^{T} x \right) \\
		A x \ge b
	\end{cases}
\]

\subsectioN{Allocazione ottima di risorse limitate}

Dobbiamo fabbricare $n$ prodotti $P_i$, e abbiamo a disposizione $m$ risorse $R_j$ (in quantit\`a limitate $b_j$).
Il prodotto $P_i$ ha bisogno di una quantit\`a $a_{i,j}$ della risorsa $R_j$ per essere realizzato.
Il prodotto $P_i$ viene poi venduto a un prezzo $c_i$.
I dati sono quindi:
\begin{center}
	\begin{tabular}{c|ccc|c}
		& $P_1$ & $\dots$ & $P_n$ & \\
		\hline
		$R_1$ & $a_{1,1}$ & $\dots$ & $a_{n, 1}$ & $b_1$ \\
		$\vdots$ & $\vdots$ & $\ddots$ & $\vdots$ & $\vdots$ \\
		$R_m$ & $a_{1, m}$ & $\dots$ & $a_{n, m}$ & $b_m$ \\
		\hline
		& $c_1$ & $\dots$ & $c_n$ &
	\end{tabular}
\end{center}
Le risorse o ``concorrono'' alla realizzazione dei prodotti, ossia sono tutte necessarie per realizzare il certo prodotto, o sono ``indipendenti'', ossia ciascuna risorsa \`e in grado di realizzare autonomamente il prodotto.

Se le risorse concorrono, introduciamo le variabili $x_i$ a cui associamo la quantit\`a di prodotto $P_i$ realizzata.
Il problema si esprime come:
\[
	\begin{cases}
		\max \left( c_1 x_1 + \dots c_n x_n \right) \\
		a_{1,1} x_1 + \dots + a_{n,1} x_n \ge b_1 \\
		\dots \\
		a_{1,m} x_1 + \dots + a_{n,m} x_n \ge b_m \\
		x_1 \ge 0 \\
		\dots \\
		x_n \ge 0
	\end{cases}
\]

Se le risorse sono indipendenti, introduciamo le variabili $x_{i,j}$ a cui associamo la quantit\`a di prodotto $P_i$ realizzata dalla risorsa $R_j$.
\[
	\begin{cases}
		\max \left( c_1 \cdot (x_{1,1} + \dots + x_{1,m}) + \dots c_n \cdot (x_{n,1} + \dots + x_{n,m}) \right) \\
		a_{1,1} x_{1,1} + \dots + a_{n,1} x_{n,1} \ge b_1 \\
		\dots \\
		a_{1,m} x_{1,m} + \dots + a_{n,m} x_{n,m} \ge b_m \\
		x_{1,1} \ge 0 \\
		\dots \\
		x_{n,m} \ge 0
	\end{cases}
\]

\subsection{Miscelazione}

% fill here

\subsection{Trasporti}

% fill here

\subsection{Multiplant}

Esiste anche pianificazione multiperiodo: compro in una settimana per usare la settimana dopo.

L'azienda produce due prodotti, in due impianti.

			| Impianto 1 | Impianto 2
Risorse:	|   P1   P2  |  P1   P2
Levigatura  |   4    2   |  5    3
Pulitura    |   2    5   |  5    6

Max levigatura 1: 80h/giorno
Max pulitura 1:   60h/giorno
Max levigatura 2: 70h/giorno
Max pulitura 2:   65h/giorno

Ciascuna unit\`a di prodotto utilizza 4 kg di materiale grezzo.
P1 viene venduto a 10 dollari a pezzo, P2 a 15 dollari a pezzo.

Modellare con un problema di PL per massimizzare il profitto settimanale sapendo che:
\begin{enumerate}
	\item Settimanalmente abbiamo 75 kg di materiale grezzo presso l'impianto 1, e 45 kg presso l'impianto 2.
	\item Settimanalmente abbiamo 120 kg di materiale grezzo presso i due impianti.
\end{enumerate}

Caso 1:

$x_{i,j}$ indica la quantit\`a di prodotto $i$ che l'impianto $j$ realizza.

I due impianti sono indipendenti:

\max (10 x_{1,1} + 15 x_{2,1})
4 x_{1,1} + 2 x_{2,1} \le 80
2 x_{1,1} + 5 x_{2,1} \le 60
4 x_{1,1} + 4 x_{2,1} \le 75
x_{1,1} \ge 0
x_{2,1} \ge 0

\max (10 x_{1,2} + 15 x_{2,2})
5 x_{1,2} + 3 x_{2,2} \le 70
5 x_{1,2} + 6 x_{2,2} \le 65
4 x_{1,2} + 4 x_{2,2} \le 45
x_{1,2} \ge 0
x_{2,2} \ge 0

Caso 2:

I due impianti non sono indipendenti.
Stesse variabili.

\max (10 (x_{1,1} + x_{1,2}) + 15 (x_{2,1} + x_{2,2}))
4 x_{1,1} + 2 x_{2,1} \le 80
2 x_{1,1} + 5 x_{2,1} \le 60
5 x_{1,2} + 3 x_{2,2} \le 70
5 x_{1,2} + 6 x_{2,2} \le 65
4 x_{1,1} + 4 x_{2,1} + 4 x_{1,2} + 4 x_{2,2} \le 120
x_{1,1} \ge 0
x_{2,1} \ge 0
x_{1,2} \ge 0
x_{2,2} \ge 0

La matrice dei primi 4 vincoli \`e una matrice a blocchi.

Vediamo pianificazione sul lungo periodo, considerando solo l'impianto 1 visto sopra.
Dobbiamo programmare la produzione di due prodotti in due settimane successive.
Nella prima settimana possiamo vendere al pi\`u 12 prodotti P1 e al pi\`u 4 prodotti P2.
Nella seconda settimana possiamo vendere al pi\`u 8 prodotti P1 e al pi\`u 12 prodotti P2.
Inoltre, nella prima settimana si possono produrre pi\`u prodotti rispetto a quelli che si possono vendere.
Costruire un modello lineare che permetta di massimizzare il profitto complessivo dalla vendita dei prodotti nelle due settimane sapendo che settimanalmente l'azienda dispone di 75 kg di materiale grezzo, e sapendo che il costo di immagazzinamento, indipendentemente dal prodotto, \`e di 2 dollari a prodotto.

x_{i,j} indica la quantit\a di prodotto $i$ realizzata e venduta dall'impianto nella settimana $j$.
x_{i,m} indica la quantit\`a di prodotto $i$ realizzata nella settimana 1 e immagazzinata, e venduta quindi nella settimana 2.

\max (10 (x_{1,1} + x_{1,2} + x_{1,m}) + 15 (x_{2,1} + x_{2,2} + x_{2,m}) - 2 (x_{1,m} + x_{2,m})
4 (x_{1,1} + x_{1,m}) + 2 (x_{2,1} + x_{2,m}) \le 80
2 (x_{1,1} + x_{1,m}) + 5 (x_{2,1} + x_{2,m}) \le 60
4 x_{1,2} + 2 x_{2,2} \le 80
2 x_{1,2} + 5 x_{2,2} \le 60
4 (x_{1,1} + x_{1,m}) + 4 (x_{2,1} + x_{2,m}) \le 75
4 x_{1,2} + 4 x_{2,2} \le 75
x_{1,1} \le 12
x_{2,1} \le 4
x_{1,2} + x_{1,m} \le 8
x_{2,2} + x_{2,m} \le 12
x_{1,1} \ge 0
x_{1,2} \ge 0
x_{2,1} \ge 0
x_{2,2} \ge 0
x_{1,m} \ge 0
x_{2,m} \ge 0

Alternativamente:
x_{i,j} indica la quantit\`a di prodotto $i$ realizzata dall'impianto nella settimana $j$.
y_{i} indica la quantit\`a di prodotto $i$ immagazzinata nella prima settimana.

\max (10 (x_{1,1} - y_{1}) + 15 (x_{2,1} - y_{2}) + 10 (x_{1,2} + y_{1}) + 15 (x_{2,2} + y_{2}) - 2 (y_{1} + y_{2}))
4 x_{1,1} + 2 x_{2,1} \le 80
2 x_{1,1} + 5 x_{2,1} \le 60
4 x_{1,2} + 2 x_{2,2} \le 80
2 x_{1,2} + 5 x_{2,2} \le 60
4 x_{1,1} + 4 x_{2,1} \le 75
4 x_{1,2} + 4 x_{2,2} \le 75
x_{1,1} - y_{1} \le 12
x_{2,1} - y_{2} \le 4
x_{1,2} + y_{1} \le 8
x_{2,2} + y_{2} \le 12
x_{1,1} \ge 0
x_{1,2} \ge 0
x_{2,1} \ge 0
x_{2,2} \ge 0
y_{1} \ge 0
y_{2} \ge 0

\begin{esercizio}
	Una fabbrica produce due tipi di pneumatici (A e B).
	Ha una gestione trimestrale della produzione.
	Deve soddisfare il seguente ordine (num. pneumatici) di mese in mese:
    Mese  |     1 |     2 |     3
	A     | 16000 |  7000 |  4000
	B     | 14000 |  4000 |  6000
	La fabbrica dispone di due linee di produzione L1 e L2.
	Le linee di produzione lavorano in parallelo, autonomamente.
	Disponibilit\`a (in ore) delle linee di produzione:
	Mese  |    1 |    2 |    3
	L1    | 2000 |  400 |  200
	L2    | 3000 |  800 | 1000
	I tempi di produzione (in ore) dei pneumatici variano fra le due linee:
	Tempi |   A |   B
	L1    | .10 | .12
	L2    | .13 | .18
	Il costo di produzione orario \`e uguale su entrambe le linee, ed \`e pari a 6 euro a ora.
	Il costo del materiale grezzo \`e di 2.5 euro per il tipo A, e 4.0 euro per il tipo B.
	Nei primi due mesi \`e possibile produrre pi\`u del richiesto.
	La produzione in eccesso pu\`o essere immagazzinata.
	Il costo di immagazzinamento \`e di .35 euro a pneumatico al mese.
	I magazzini sono vuoti a inizio e a fine trimestre.
	Modellare con un problema di PL trascurando l'interezza dei prodotti.

	\{A,B\}_{j, k} quantit\`a di prodotto realizzata dalla linea L_{j} nel mese k \in \{1, 2, 3\}.
	A_{i}, B_{i} con i \in \{1,2\} indica la quantit\`a di prodotto immagazzinato nei due mesi.

	\min (
		6 * 0.10 (A_{1,1} + A_{1,2} + A_{1,3}) + 6 * 0.12 (A_{2,1} + A_{2,2} + A_{2,3}) +
		6 * 0.12 (B_{1,1} + B_{1,2} + B_{1,3}) + 6 * 0.18 (B_{2,1} + B_{2,2} + B_{2,3}) +
		2.5 (A_{1,1} + A_{1,2} + A_{1,3} + A_{2,1} + A_{2,2} + A_{2,3}) +
		4 (B_{1,1} + B_{1,2} + B_{1,3} + B_{2,1} + B_{2,2} + B_{2,3}) +
		0.35 * (A_{1} + A_{2} + B_{1} + B_{2})
	)
	A_{1,1} + A_{2,1} - A_{1} = 16000
	B_{1,1} + B_{2,1} - B_{1} = 14000
	A_{1,2} + A_{2,2} + A_{1} - A_{2} = 7000
	B_{1,2} + B_{2,2} + B_{1} - B_{2} = 4000
	A_{1,3} + A_{2,3} + A_{2} = 4000
	B_{1,3} + B_{2,3} + B_{2} = 6000
	0.10 A_{1,1} + 0.12 B_{1,1} \le 2000
	0.10 A_{1,2} + 0.12 B_{1,2} \le 400
	0.10 A_{1,3} + 0.12 B_{1,3} \le 200
	0.12 A_{2,1} + 0.18 B_{2,1} \le 3000
	0.12 A_{2,2} + 0.18 B_{2,2} \le 800
	0.12 A_{2,3} + 0.18 B_{2,3} \le 1000
\end{esercizio}

