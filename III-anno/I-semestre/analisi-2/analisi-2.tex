\documentclass[11pt,a4paper,twoside]{report}
% draft mode prevents link generation

\usepackage[italian]{babel}
% per date e ToC in italiano
\usepackage{hyperref}
\usepackage{nameref}
\usepackage[table]{xcolor}
\usepackage{amsmath}        % matematica
\usepackage{amsthm}         % teoremi
\usepackage{amssymb}        % simboli
\usepackage{amsfonts}       % font matematicosi
\usepackage{mathrsfs}
\usepackage{mathtools}
\usepackage{thmbox}
\usepackage{centernot}      % per semplificare
\usepackage{pgfplots}       % per i grafici
\pgfplotsset{compat=1.5}    % consigliata compatibilita'
                            % con la versione 1.5

\usepackage{fullpage}       % troppo margine normalmente
\usepackage{parskip}        % preferisco spazio fra i paragrafi
                            % all'indentazione sulla prima riga
\usepackage{fancyhdr}
\usepackage[makeroom]{cancel}

% algorithms
\usepackage{algorithm}
\usepackage{algpseudocode}

\floatname{algorithm}{Algoritmo}
\renewcommand{\algorithmicrequire}{\textbf{Input:}}
\renewcommand{\algorithmicensure}{\textbf{Output:}}

\theoremstyle{plain}
\newtheorem{axiom}{Assioma}
\newtheorem{theorem}{Teorema}[section]
\newtheorem{lem}[theorem]{Lemma}
\newtheorem{prop}[theorem]{Proposizione}
\newtheorem{cor}[theorem]{Corollario}

\theoremstyle{definition}
\newtheorem{defn}{Definizione}[section]
\newtheorem{esercizio}{Esercizio}

\theoremstyle{remark}
\newtheorem{remark}{Remark}
\newtheorem{oss}{Osservazione}
\newtheorem{caso}{Caso}
\newtheorem{fact}{Fatto}
\newtheorem{claim}{Claim}

\newenvironment{exmp}[1][]
{ \par\medskip\textbf{Esempio\ #1\unskip\,:}\\*[1pt]\rule{\textwidth}{0.4pt}\\*[1pt] }
{ \\*[1pt]\rule{\textwidth}{0.4pt}\medskip }

\newenvironment{smallpmatrix}
{\left( \begin{smallmatrix}}
{\end{smallmatrix} \right)}

\newenvironment{completepmatrix}[1]
{\left(\begin{array}{*{#1}{c}|c}}
{\end{array}\right)}

\pagestyle{fancyplain}
\fancyhead{} % clear all header fields
\fancyfoot{} % clear all footer fields
\fancyfoot[C]{\today}
\fancyfoot[LE,RO]{\thepage}
\renewcommand{\headrulewidth}{0pt}
\renewcommand{\footrulewidth}{0pt}

\usetikzlibrary{shapes,arrows,calc,fit,backgrounds,positioning,automata}

\newcommand{\parity}[1]{\mathcal{Par}\left( { #1 } \right)}
% Languages & Automata
\newcommand{\langs}[1]{\mathcal{L}\left( {#1} \right)}
\newcommand{\lang}[1]{L\left( {#1} \right)}
\newcommand{\langor}{\, | \,}
\newcommand{\kleenestar}{\star}
\newcommand{\emptystring}{\varepsilon}

\newcommand{\bigo}[1]{O\left( {#1} \right)}
\newcommand{\probp}{\code{P}}
\newcommand{\probnp}{\code{NP}}
\newcommand{\probnpc}{\code{NP-C}}
\newcommand{\probexp}{\code{EXP}}
\newcommand{\covers}{<\!\cdot}      % simbolo di 'copre'
\newcommand{\scalar}{\centerdot}
\newcommand{\compl}{\mathsf{c}}     % C complementare
\newcommand{\dotcup}{\mathaccent\cdot\cup}  % unione disgiunta
\newcommand{\lateralsx}[1]{{}_{#1}\!\sim}   % equivalenza sinistra
\newcommand{\lateraldx}[1]{\sim_{#1}}       % equivalenza destra
\newcommand{\mcm}{\operatorname{mcm}}       % mcm
\newcommand{\mcd}{\operatorname{MCD}}       % MCD
\newcommand{\pow}[1]{<\!{#1}\!>}            % gruppo delle potenze (<a>)
\newcommand{\abs}[1]{\left\lvert{#1}\right\rvert}   % valore assoluto
\newcommand{\intsup}[1]{\left\lceil{#1}\right\rceil}   % intero superiore
\newcommand{\intinf}[1]{\left\lfloor{#1}\right\rfloor}   % intero inferiore
\newcommand{\norm}[1]{\left\lVert{#1}\right\rVert}  % norma di un vettore
\newcommand{\divides}{\bigm|}       % simbolo di 'divide'
\newcommand{\definition}{\overset{\underset{\mathrm{def}}{}}{=}}    % uguale con 'definizione' sopra
\newcommand{\supop}{\vee}           % simbolo dell'operatore sup
\newcommand{\infop}{\wedge}         % simbolo dell'operatore inf
\newcommand{\subgroupset}{\mathscr{S}}      % insieme dei sottogruppi (S corsiva)
\newcommand{\solutions}{\mathscr{S}}
\newcommand{\parts}[1]{\mathbb{P} \left( {#1} \right)}             % insieme delle parti
\newcommand{\naturals}{\mathbb{N}}          % insieme dei naturali
\newcommand{\integers}{\mathbb{Z}}          % insieme degli interi
\newcommand{\rationals}{\mathbb{Q}}         % insieme dei razionali
\newcommand{\reals}{\mathbb{R}}             % insieme dei reali
\newcommand{\complexes}{\mathbb{C}}         % insieme dei complessi
\newcommand{\field}{\mathbb{K}}             % simbolo di campo generico
\newcommand{\subfield}{\mathbb{F}}          % simbolo di sottocampo generico
\newcommand{\matrices}{\mathfrak{M}}        % insieme delle matrici
\newcommand{\nullelement}{\underline{0}}    % elemento neutro
\newcommand{\var}{\operatorname{Var}}       % varianza 
\newcommand{\seq}[3]{{#1}_{#2},\ldots,\,{#1}_{#3}}
\newcommand{\prob}[2][P]{#1 \left( {#2} \right)}
\newcommand{\probcond}[3][P]{#1 \left( #2 | #3 \right)}
\newcommand{\expect}[1]{E \left( #1 \right)}
\newcommand{\mass}[2][X]{p_{#1} \left( #2 \right)}
\newcommand{\distr}[2][X]{F_{#1} \left( #2 \right)}
\newcommand{\image}[1]{Im_{#1}}
\newcommand{\cov}{\operatorname{Cov}}
\newcommand{\detname}{\operatorname{det}}
\newcommand{\sgn}{\operatorname{sgn}}
\renewcommand{\det}[1]{\detname \left( {#1} \right)}
\newcommand{\algcompl}[2][A]{{\mathcal{#1}}_{#2}}
\newcommand{\agg}[1]{\operatorname{Agg} \left( {#1} \right)}
\newcommand{\id}{id}
\newcommand{\sfrac}[2]{{#1}/{#2}}
\newcommand{\requiv}[1][]{\ \varepsilon_{#1} \ }
\newcommand{\abslog}[1]{\log \left( \abs{#1} \right)}
\newcommand{\substitute}[2]{\left. {#1} \; \right|_{#2}}
\newcommand{\increment}[1]{{\left. {#1} \; \right|}}
\newcommand{\adj}{\sim}
\newcommand{\code}[1]{\texttt{#1}}
\newcommand{\isa}{\code{is-a}}
% intervalli aperti e chiusi
\newcommand{\ooint}[2]{\left( {#1}, {#2} \right)}
\newcommand{\coint}[2]{\left[ {#1}, {#2} \right)}
\newcommand{\ocint}[2]{\left( {#1}, {#2} \right]}
\newcommand{\ccint}[2]{\left[ {#1}, {#2} \right]}
\newcommand{\stereotipe}[1]{\ll \code{#1} \gg}

\renewcommand{\labelitemi}{$-$}
\let\oldforall\forall
\renewcommand{\forall}{\ \oldforall \ }
\let\oldexists\exists
\renewcommand{\exists}{\ \oldexists \ }
\renewcommand{\implies}{\Rightarrow}
\renewcommand{\iff}{\Leftrightarrow}

\makeatletter
\let\orgdescriptionlabel\descriptionlabel\renewcommand*{\descriptionlabel}[1]{%
\let\orglabel\label
\let\label\@gobble
\phantomsection
\edef\@currentlabel{#1}%
%\edef\@currentlabelname{#1}%
\let\label\orglabel
\orgdescriptionlabel{#1}%
}
\makeatother

\makeatletter
\providecommand*{\diff}%
    {\@ifnextchar^{\DIfF}{\DIfF^{}}}
\def\DIfF^#1{%
    \mathop{\mathrm{\mathstrut d}}%
        \nolimits^{#1}\gobblespace}
\def\gobblespace{%
        \futurelet\diffarg\opspace}
\def\opspace{%
    \let\DiffSpace\!%
    \ifx\diffarg(%
        \let\DiffSpace\relax
    \else
        \ifx\diffarg[%
            \let\DiffSpace\relax
        \else
            \ifx\diffarg\{%
                \let\DiffSpace\relax
            \fi\fi\fi\DiffSpace}

\providecommand*{\deriv}[1][]{%
    \frac{\diff}{\diff #1}}
%\providecommand*{\deriv}[2][]{%
%    \frac{\diff^{#1}}{\diff #2^{#1}}}
\providecommand*{\fderiv}[3][]{%
    \frac{\diff^{#1}#2}{\diff #3^{#1}}}
\providecommand*{\pderiv}[3][]{%
    \frac{\partial^{#1}#2}%
        {\partial #3^{#1}}}

\newcommand{\dx}[1][x]{\, \diff {#1}}

\author{Michele Laurenti \\ \href{mailto:asmeikal@me.com}{asmeikal@me.com}}
\date{\today}


\begin{document}

\title{Analisi Matematica 2}

\maketitle

\tableofcontents

\chapter{Equazioni differenziali ordinarie}

Hello world!

\chapter{Funzioni di pi\`u variabili}

Trattiamo inizialmente funzioni $f(x) : \reals^n \to \reals$, ossia che hanno
argomento vettoriale. Scriviamo comunque $x$ per indicare il vettore $(x_1, \ldots, x_n)$.

\section{Linee di livello}

Normalmente per disegnare una funzione ci \`e sufficiente sapere quanto vale in
un insieme di punti. Con le funzioni vettoriali a valori in $\reals$ non 
si pu\`o fare lo stesso: una volta tracciato un punto, dove andiamo? Abbiamo 
infinite altre direzioni in cui andare.

Pensiamo per semplicit\`a a funzioni $f(x) : \reals^2 \to \reals$, ma il 
ragionamento \`e generale. Indichiamo con $z = f(x)$  l'altezza nel punto 
(o vettore) $x$. Ma invece di chiederci quanto vale $z$ in un punto, chiediamoci 
dove $z$ assume un certo valore, ossia dove la funzione \`e alta quel certo valore.
Enter le \emph{linee di livello}.

\begin{defn}[Linea di livello]
        Chiamiamo \emph{insieme di livello $c$} di $f$ l'insieme di vettori definito
        cos\`i:
        \[
                I_{c}(f) = \{ x \in \reals^n f(x) = c \}
        \]
\end{defn}

Sono \emph{linee}, o curve, su cui la funzione \`e costante. Quindi il valore della funzione 
lo diamo su delle curve, non su dei punti.

Consideriamo la funzione $z = 2 \, x + 3 \, y - 2$. Le sue linee di livello saranno: 
\[
        I_{c} (f) : 2 \, x + 3 \, y - 2 = c \implies 2 \, x + 3 \, y - (2 + c) = 0
\]
Sono tutte rette!
Ora, ripassiamo cosa sono le curve.

\section{Curve}

Una curva possiamo pensarla come un insieme di infiniti vettori, parametrizzati in una 
variabile $t$:
\[
        \underline{\gamma} = \{ (x,y) \in \reals^2 : x = x(t), \, y = y(t), \, t \in \ooint{a}{b} \in I \}
\]

% disegnare esempio di curva \gamma = (t, t^2) con t \in \ooint{-1}{3}

Consideriamo la funzione $f(x) = x^2 + y^2$, e la sua curva di livello per $f(x) = 9$.
Possiamo scriverla come $I_9(f) : x^2 + y^2 = 9$. Questa si chiama \emph{forma cartesiana implicita}.
Ma possiamo anche scriverla come:
\[
\gamma (t) = \begin{pmatrix} 3 \, \sin(t) \\ 3 \, \cos(t) \end{pmatrix}
\]
Data una curva, ci interessano un paio di cose.

La prima \`e il vettore tangente. Data una curva, il vettore tangente nel punto $t$ \`e $\cap{t}$ e vale:
\[
\gamma'(t) = \begin{pmatrix} x'(t) \\ y'(t) \end{pmatrix}
\]
Riprendendo la funzione $\gamma_1$, vediamo che il suo vettore tangente \`e:
\[
\gamma_1'(t) = \begin{pmatrix} \deriv[t] t \\ \deriv[t] t^2 \end{pmatrix} = 
\begin{pmatrix} 1 \\ 2 \, t \end{pmatrix}
\]
Ricordiamo che sono sempre vettori.

L'altra cosa che ci interessa \`e il vettore normale (o perpendicolare) alla curva.

Consideriamo prima un vettore $n = (1,3)$. Il suo vettore perpendicolare $\cap{n}$ \`e tale che
$n \scalar \cap{n} = 0$. Quindi $n_1 \cdot \cap{n_1} + n_2 \cdot \cap{n_2} = 0$, e quindi basta:
\[
        \cap{n_1} = - n_2 \text{ e } \cap{n_2} = n_1
\]
Indichiamo il vettore perpendicolare alla curva nel punto $t$ con $\gamma_{\underline{n}}(t)$, e vale:
\[
\gamma_{\underline{n}} (t) = \begin{pmatrix} - \deriv[t] y(t) \\ \deriv[t] x(t) \end{pmatrix}
\]
Abbiamo scambiato le derivate delle componenti.

Con queste due informazioni, sappiamo come ``salire'' il pi\`u possibile di livello: seguendo la normale.

Consideriamo $f(x) = 1 - \sqrt{x^2 + y^2}$. Vediamo che $I_4 (f) : \sqrt{x^2 + y^2} = -3$. Significa che 
$I_4 (f) = \emptyset$: non esiste questa linea di livello.

Vediamo che in generale $I_c (f) : \sqrt{x^2 + y^2} = 1 - c$, da cui segue che le linee di livello 
esistono se $c \le 1$. Abbiamo una condizione di esistenza per le linee di livello, che ci dice anche
qualcosa anche sul dominio della funzione.

Vediamo che ogni linea di livello si pu\`o esprimere come $x^2 + y^2 = (1 - c)^2$, e quindi \`e 
una circonferenza di centro $(0,0)$ e raggio $1 - c$.

Oltre alle linee di livello, ci pu\`o interessare un'altra cosa: il valore che la funzione assume in 
una regione, come ad esempio sulla curva:
\[
\gamma(t) = \begin{pmatrix} 0 \\ t \end{pmatrix}
\]
Ossia, sulla retta $y$ ($x = 0$).

Per scoprirlo, sostituiamo nell'equazione $z = 2 \, x^2 + 2 \, y^2 - 3$, e otteniamo 
$z = 2 \, y^2 - 3$. Vuol dire che sulla curva che abbiamo scelto $z$ vale $2 \, y^2 - 3$. \`E come se 
avessimo sezionato la nostra superficie tridimensionale.

\begin{esercizio}
        Data la funzione f(x) = -2 \, x^2 - 5 \, y + y^2, quanto vale sulle seguenti curve:

\gamma_1 (t) = (x_0 t)

\`E una retta.

Quanto vale f(\unhderline{\gamma_1})? \`E semplicemente f(0, y) = -5 \, y + y^2. \`E una parabola rivolta verso 
l'alto.

\gamma_2 (t) = (t y=-t)

Anche qui abbiamo una retta. Per trovare quanto vale, di nuovo facciamo 
f(\underline{\gamma_2}) = f(x, -x) = -2 \, x^2 + 5 \, x + x^2 = -x^2 + 5 \, x
Ancora una volta una parabola, ma rivolta verso il basso.

Per disegnarla bisogna fare due passi: quando sono in x = x_0, devo trovare dove sono sulla curva, e 
da l\`i ``alzarmi'' di una certa quantit\`a.

\end{esercizio}

\section{Limiti}

I limiti di funzioni continue sono facili: sono i valori della funzione nel punto. I limiti 
nei punti di discontinuit\`a per funzioni definite a tratti sono pi\`u strani. E lo sappiamo questo.

Con funzioni a pi\`u variabili, un limite \lim_{x \to \lambda} f(x) = L esiste se non dipende 
da ``come mi avvicino'' a $x_0$. In \reals, ci sono 2 modi per avvicinarsi a un punto: da sinistra 
o da destra. Ma se andiamo in \reals^2?

\begin{defn}[Limite]

        \lim_{x \to x_0} f(x) = L \iff \forall \varepsilon > 0 \exists \delta > 0 : \abs{f(x) - L} \le \epsilon
        \forall \abs{x - x_0} \le \delta

\end{defn}

\abs{x - x_0} \le \delta \`e facile in \reals: \`e l'intervallo \ccint{x - x_0}{x + x_0}.

Passiamo a \reals^2:

Chiamiamo \underline{x} = (x,y) e \underline{x_0} = (x_0, y_0). La definizione di limite \`e:

\lim_{\underline{x} \to \underline{x_0}} f(\underline{x}) = L

La definizione \`e analoga, ma cambia l'idea di distanza: usiamo la distanza euclidea.

\begin{defn}[Limite a pi\`u variabili]


\forall \varepsilon > 0 \exists \delta > 0 : \abs{f(\underline{x}) - L} \le \varepsilon 
\forall \underline{x} t.c. \abs{\underline{x} - \underline{x_0}} \le \delta

\end{defn}

In due dimensioni, \abs{\underline{x} - \underline{x_0}} \le \delta individua il cerchio di 
centro \underline{x_0} e raggio \delta.

Adesso abbiamo un'infinit\`a di modi per avvicinarsi al punto \underline{x_0}.


f(x,y) = \frac{x \cdot y}{x^2 + y^2}

In (1,0) la funzione \`e continua, il limite \`e esattamente f(1,0).

Ma \lim_{(x,y) \to (0,0)} f(x,y) ci d\`a problemi: \`e \frac{0}{0}.

Mettiamoci su delle curve che passano per questo punto e vediamo come possiamo avvicinarci in questo punto.

Iniziamo con la curva \gamma_1 = (x 0). Vediamo che f(x,y) = 0.

Quindi il limite su questa direzione vale 0. Stesso discorso per la direzione (0 y).

Proviamo per \gamma_2 = (x x). f(x,x) = \frac{x^2}{2 \, x^2} = \frac{1}{2}.

E abbiamo trovato un limite differente... Quindi il limite per \underline{x} \to (0,0) non esiste.

Non basta provare per \gamma = (x \\ m \, x) e vedere che il limite \`e sempre uguale.

Prendiamo f(x,y) = \frac{x \, y^2}{x^2 + y^4}, e \gamma = (y^2 y).

Ma se vogliamo dimostrare che un limite esiste? Ci serve fare un cambio di variabili.

Prendiamo un punto (x,y), e scriviamolo come (\rho \cos(\theta), \rho \sin(\theta)). 
Possiamo individuarlo con i valori (\theta, \rho). 
Teniamo a mente che x^2 + y^2 = \rho^2. Inoltre \frac{y}[x} = \tan(\theta).
% TODO: controllare sta cosa

Prendiamo una funzione brutta:
f(x,y) = \frac{2 \, x^2 + 2 \, y^2 - 3 \, x \, y}{x^2 + y^2}

Possiamo riscriverla come:

f(\theta, \rho) = \frac{2 \, \rho ^2 - 3 \rho^2 \cos(\theta) \sin(\theta)}{\rho^2} = 
\frac{\rho^2 \left( 2 - 3 \cos(\theta) \sin(\theta) \right)}{\rho^2} = 
2 - 3 \cos(\theta) \sin(\theta)

Guarda un po': la funzione dipende solamente da \theta, non da \rho: solo dall'angolo 
in cui ci muoviamo, non dalla distanza.

Altra cosa che possiamo vedere: il modulo della funzione. 
Vale che \abs{f(\theta, \rho)} \le 2 + \abs{3 \, \cos(\theta) \sin(\theta)} = 5.
Quindi la funzione vale al massimo 5!

Torniamo ai limiti. Questo ci aiuta a dimostrare se un limite esiste o meno.
In due variabili si usa solo il teorema del confronto: ci sono metodi pi\`u sofisticati
ma noi non vogliamo vederli.

Se 0 \le f(x) \le g(x) e \lim_{x \to x_0} g(x) = 0 \implies \lim_{x \to x_0} f(x) = 0.

Vogliamo ora calcolare \lim_{\underline{x} \to (0,0)} f(\underline{x}). Se scopriamo che 
\abs{f(x)} \le g(x), e abbiamo che \lim_{\underline{x} \to (0,0)} g(\underline{x}} = 0, 
allora sappiamo che \lim_{x \to (0,0)} f(x) = 0.
Niente di nuovo. \`E il teorema del confronto. Bisogna saper fare le solite maggiorazioni.

Torniamo all'esempio di prima: \frac{x y²}{x² + y²} ... 
% TODO: finire di scrivere

Per i limiti, quindi, si prova su una \emph{restrizione} \gamma. Se l\`i vale $L$, proviamo 
a dimostrare che vale proprio $L$. Un modo \`e con il teorema del confronto: proviamo a dimostrare
che \abs{f(x) - L} \le g(x), con \lim_{x \to (x,y)} g(x) = 0.

Di solito facciamo limiti per \underline{x} \to 0.

Analisi 1 ci dice che se \lim_{\vector{x} \to \vector{0}} \abs{h(x)} = 0, allora 
\lim_{\vector{x} \to \vector{0}} h(x) = 0.

f - 3

x^2 y^3 / 2 \, x^4 + y^2
\le x^2 y^3 / y^2 = x^2 yx^2 y^3 / 2 \, x^4 + y^2
\le x^2 y^3 / y^2 = x^2 y















\end{document}

