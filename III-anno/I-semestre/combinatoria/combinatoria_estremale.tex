
\chapter{Combinatoria estremale}

Due delle branche pi\`u importanti della combinatoria e dell'informatica teorica sono la combinatoria estremale e la combinatoria probabilistica.

Partiamo con un po' di teoria estremale dei grafi: vogliamo capire il rapporto all'estremo di alcuni parametri, fissandone alcuni altri.
Un esempio di propriet\`a estremale \`e la minima superficie per volume che la sfera ha rispetto ad altri solidi.

Introduciamo la combinatoria estremale.
Fissato $n$, consideriamo tutti i grafi con $n$ vertici, e di questi grafi $G$ consideriamo due parametri: $\crom{G}$ (il numero cromatico) e $\abs{E(G)}$ (il numero degli archi).
Qual \`e il massimo numero degli archi in un grafo con $n$ vertici e con numero cromatico 2?
Un grafo con numero cromatico 2 \`e un grafo bipartito.
\[
	\Tau(n) = \max_{\abs{V(G)} = n \\ \crom{G} = 2} \abs{E(G)}
\]
Il grafo \`e bipartito: abbiamo due sottoinsieme dei vertici, $A$ e $[n] \setminus A$, ciascuno senza archi fra i suoi vertici.
Il massimo numero di archi \`e $\abs{A} \cdot (n - \abs{A})$: connettiamo ogni vertice in un insieme con ogni vertice nell'altro.
Dell'insieme $A$ ci interessa solo la cardinalit\`a.
Se la cardinalit\`a di $A$ \`e $x$, abbiamo $x \cdot (n - x)$ archi, e dobbiamo quindi massimizzare $x$ in funzione di $n$.
\[
	\Tau(n) = \max_{x} x \cdot (n - x)
\]
Quando $n$ \`e pari, questo valore \`e $\frac{n}{2} \cdot \frac{n}{2}$, quando $n$ \`e dispari \`e $\floor{\frac{n}{2} \cdot \frac{n}{2}V}$.
Ma perch\'e?

Dimentichiamo per un momento che $x$ debba essere un intero.
Quello che vale \`e questo:
\[
x \cdot (n - x) \le \left{( \frac{n}{2} \right)}^2 = \frac{n^2}{4}
\]
La verifica \`e immediata:
\begin{align*}
	x \cdot (n - x) & \le \frac{n^2}{4} \implies \\
	4 x^2 - 4 n x + n^2 & \ge 0
\end{align*}
L'ultimo \`e un quadrato, quindi sempre non negativo, quindi vale sempre quanto detto.

Riflettere su questo: la radice del prodotto \`e minore o uguale della media aritmetica, e la radice del prodotto \`e la media geometrica.

Notare che $\floor{\frac{n}{2} \cdot \frac{n}{2}V} = \floor{\frac{n}{2}} \cdot \ceil{\frac{n}{2}}$.
Notare inoltre che $\binom{n}{2} \simeq \frac{n^2}{2}$, quindi al massimo abbiamo la met\`a degli archi.

Vediamo che $\floor{\frac{n}{2} \cdot \frac{n}{2}V} = \floor{\frac{n}{2}} \cdot \ceil{\frac{n}{2}}$.
Se $n$ \`e pari, vale banalmente.
Se $n$ \`e dispari, possiamo scrivere $n$ come $2 l + 1$, e quindi:
\[
	\floor{\frac{n^2}{4}} = \floor{\frac{4 l^2 + 4 l + 1}{4}} =
	\floor{l^2 + l + \frac{1}{4}} = l^2 + l
\]
Dall'altro lato, possiamo scrivere questo:
\[
	\floor{\frac{n}{2}} = \floor{\frac{2 l + 1}{2}} = \floor{l + \frac{1}{2}} = l \\
	\ceil{\frac{n}{2}} = \ceil{\frac{2 l + 1}{2}} = \ceil{l + \frac{1}{2}} = l + 1
\]
E quindi abbiamo che $l^2 + l = l \cdot (l + 1)$.

Consideriamo ora tutti i grafi con $n$ vertici che non contengono triangoli.
Notare che tutti i grafi bipartiti non contengono triangoli, ma potrebbero esserci pi\`u grafi senza triangoli che grafi bipartiti.
\begin{theorem}[di Mandl-Tur\'an]
	\label{teorema_mandl_turan}
	Indicando con $\Delta$ il grafo triangolo, diciamo:
	\[
		T(n) = \max_{\abs{V(G)} = n \\ \Delta \nsubseteq G} \abs{E(G)}
	\]
	Questo numero \`e sempre $\floor{\frac{n^2}{4}}$.
\end{theorem}
\begin{proof}[del teorema \ref{teorema_mandl_turan}]
	L'assenza di triangoli caratterizza il grado dei vertici, ma lo vedremo poi.
	Intanto dimostriamo che $T(n) \ge \floor{\frac{n^2}{4}}$: \`e il lato pi\`u facile.
	Consideriamo il cosiddetto ``bipartito completo'', ossia il grafo $K_{\floor{\frac{n}{2}}\ceil{\frac{n}{2}}}$, che ha numero cromatico 2 e che quindi non contiene triangoli.
	Quindi il massimo numero di archi in un grafo che non contiene triangoli \`e certamente maggiore del numero di archi in questo grafo.

	Dimostriamo ora che $T(n) \le \floor{\frac{n^2}{4}}$.
	Ora dobbiamo sfruttare l'assenza di triangoli per limitare il numero degli archi.
	Consideriamo due vertici che formano un arco nel grafo, ossia $a,b \in V(G)$ tali che $\{a,b\} \in E(G)$.
	Avranno ciascuno il loro grado, $d_G(a)$ e $d_G(b)$.
	Non sappiamo niente su questi due gradi, ma sappiamo che la loro somma \`e limitata.

	Chiamando $\Gamma_a = \{ c : \{a,c\} \in E(G) \}$ l'insieme degli archi adiacenti a $a$.
	Chiaramente $a \notin \Gamma_a$, e $b \in \Gamma_a$.
	Inoltre $d_G(a) = \abs{\Gamma_a}$.
	Vale infine che $\Gamma_a \cap \Gamma_b = \emptyset$, altrimenti ci sarebbe un triangolo.
	Segue che, poich\'e $\abs{\Gamma_a \cup \Gamma_b} \le n$, $\abs{\Gamma_a} + \abs{\Gamma_b} \le n$.

	Ora introduciamo questa limitazione inferiore sul numero degli archi:
	\[
		\sum_{\{a,b\} \in E(G)} d_G(a) + d_G(b) \le n \cdot \abs{E(G)}
	\]

\end{proof}

\section{Ipegrafi}

Un ipergrafo \`e un insieme di vertici e una famiglia di sottoinsiemi dei vertici, che sono gli archi.
Sperner (shperner), 1928.

Una famiglia di Sperner \`e una famiglia di sottoinsiemi di $X$, con $\abs{X} = n$.
$\mathcal{F} \subseteq 2^X$, dove con $2^X = \{ A : A \subseteq X \}$.
\`E una notazione utile: la cardinalit\`a di $2^X$ \`e $2^n$.
Una famiglia \`e di Sperner se due suoi elementi non si contengono, ossia $\{a,b\} \in \binom{\mathca{F}}{2} \implies A \notsubseteq B$.

Gli archi di un grafo non si contengono: hanno la stessa cardinalit\`a, ma sono diversi, quindi non si contengono.
Quanto pu\`o essere grande una famiglia di Sperner su un insieme con $n$ elementi?
$M(n) = \max\limits_{\mathcal{F} \text{ di Sperner} \\ \mathcal{F} \subseteq 2^{[n]}} \abs{\mathcal{F}}$.
Il problema, a Sperner, lo fece venire in mente Schreier, un teorico dei gruppi.

\begin{theorem}[di Sperner (1928)]
	M(n) = \binom{n}{\floor{\frac{n}{2}}}
	La massima cardinalit\`a \`e il coefficiente binomiale ?? non ho capito cosa.

	Se $n$ \`e pari, la soluzione ottimale \`e unica.
	Se $n$ \`e dispari, ci sono due famiglie ottimali: sono una famiglia e il suo complementare.
\end{theorem}

Possiamo considerare un grafo i cui vertici sono sottoinsiemi di $X$, e mettiamo un arco fra due vertici se i due sottoinsiemi non si contengono.
Una famiglia di Sperner \`e un sottografo completo di questo grafo, e la massima cardinalit\`a di una famiglia di Sperner \`e la massima cardinalit\`a di una clique in questo grafo.

\begin{proof}[del teorema di Sperner]
	Scegliamo un $k$, e consideriamo tutti i sottoinsiemi di cardinalit\`a $k$.
	Questa \`e una famiglia di Sperner.
	Quindi $M(n) \ge \binom{n}{k} \forall k$.
	Possiamo affermare che $M(n) \ge \binom{n}{\floor{\frac{n}{2}}}$.
	Non stiamo dicendo che questo \`e il coefficiente binomiale pi\`u grande (anche se lo \`e).

	Non esistono solo famiglie uniformi, per\`o, in cui ogni insieme ha la stessa cardinalit\`a.
	Dobbiamo far vedere che le famiglie eterogenee hanno comunque al massimo questa cardinalit\`a, ossia che $M(n) \le \binom{n}{\floor{\frac{n}{2}}}$.

	Per vederlo utilizziamo questa disuguaglianza (disuguaglianza LYM):
	\[
		\mathcal{F} \subseteq 2^X \implies \sum_{F \in \mathcal{F}} \binom{n}{\abs{F}}^{-1} \le 1
	\]
	\`E stata ottenuta contemporaneamente da tre persone, indipendentemente: Lubell, Yamamoto e Meshalkin.
	\`E un risultato puramente tecnico, ma ci porter\`a alla dimostrazione.

	Se in una famiglia abbiamo un particolare insieme, tutti i suoi sottoinsiemi e tutti i suoi ``soprainsiemi'' non possono essere in quella famiglia.
	\`E questo che vogliamo cogliere con questa disuguaglianza.

	Consideriamo un insieme arbitrario $E \subseteq [n]$, costruiamo una catena che passa per questo insieme.
	Una catena \`e una sequenza di insiemi che si contengono strettamente.
	L'$i$-esimo di questi insiemi ha $i$ elementi.
	$D_i \setminus D_{i-1} = \{ x_i \}$, contiene un unico elemento.
	Quante catene ci sono?
	In generale, una catena \`e una successione di elementi, e possiamo pensarla come una permutazione.
	Ci sono quindi $n!$ catene.
	Per ogni catena, in una famiglia di Sperner abbiamo al pi\`u un insieme preso da questa catena.
	Questo ci dar\`a la limitazione superiore che cerchiamo.

	Fissato un insieme, una catena passa per questo insieme $E$ se $D_{\abs{E}} = E$.
	Quante catene passano per un insieme fissato $E$?
	Sono $\abs{E}! \cdot \abs{E^{\compl}}! = \abs{E}! \cdot (n - \abs{E})!$.
	Abbiamo prima gli elementi di $E$, poi gli elementi che non sono in $E$.

	Gli insiemi di una famiglia di Sperner non si contengono, quindi le catene che passano per uno di questi insiemi non possono contenere nessun altro insieme.
	Scriviamo in maniera formale:
	\[
		\sum_{E \in \mathcal{F}} \abs{\{ \text{catene che passano per un insieme } E \}} \le n!
	\]
	Ogni catena \`e conteggiata solo una volta.
	Questi insiemi di catene sono disgiunti.
	Quindi non stiamo contando pi\`u catene di quante esistano.
	Possiamo quindi scrivere questo:
	\[
		\sum_{E \in \mathcal{F}} \abs{E}! \cdot (n - \abs{E})! \le n!
	\]
	Dividendo i due lati per $n!$, abbiamo:
	\[
		\sum_{E \in mathcal{F}} \frac{\abs{E}! \cdot (n - \abs{E})!}{n!} \le 1
	\]
	E questa \`e proprio la disuguaglianza di LYM.

	Adesso possiamo dire questo:
	\[
		1 \ge \sum_{E \in \mathcal{F}} \frac{1}{\binom{n}{\abs{E}}} \ge \sum_{E \in \mathcal{F}} \frac{1}{\binom{n}{\floor{\frac{n}{2}}}} = \abs{F} \cdot \frac{1}{\binom{n}{\floor{\frac{n}{2}}}}
	\]
	Una limitazione inferiore alla sommatoria la otteniamo mettendo al posto di ogni coefficiente binomiale il coefficiente pi\`u grande.
	E abbiamo finito.
	Questo infatti implica che $\abs{\mathcal{F}} \le \binom{n}{\floor{\frac{n}{2}}}$.

	Facciamo vedere brevemente che $\binom{n}{k-1} \le \binom{n}{k}$ fin tanto che $k \le \floor{\frac{n}{2}}$.
	Sappiamo che $\binom{n}{k} = \binom{n}{n - k}$.
	Basta vedere che:
	\[
		\binom{n}{k-1} = \frac{n \cdot (n-1) \cdot \dots \cdot (n - k - 2)}{(k-1)!} \\
		\binom{n}{k} = \frac{n \cdot (n-1) \cdot \dots \cdot (n - k - 1)}{k!}
	\]
	Quindi $\binom{n}{k}$ lo otteniamo da $\binom{n}{k-1}$ in questo modo:
	\[
		\binom{n}{k} = \binom{n}{k-1} \cdot \frac{n - k - 1}{k}
	\]
	I coefficienti crescono fintanto che $\frac{n - k - 1}{k} \ge 1$, quindi $n - k - 1 \ge k \implies 2 k - 1 \le n$.
\end{proof}

Il vero teorema di Sperner riguarda anche l'unicit\`a di questa famiglia.
Lo vediamo solo per quando $n$ \`e pari.
Vogliamo sapere quando c'\`e l'uguaglianza, e vedremo che se $n$ \`e pari l'unicit\`a c'\`e per una sola famiglia.

Supponiamo che $\abs{\mathcal{F}} = \binom{n}{\frac{n}{2}}$.
La disuguaglianza di prima (TODO aggiungere riferimento) non deve essere stretta: deve essere un'uguaglianza.
Le due sommatorie al centro devono essere uguali, quindi $\forall E \binom{n}{\abs{E}} = \binom{n}{\frac{n}{2}}$.
Questa disuguaglianza la abbiamo solo se $\abs{E} = \frac{n}{2}$, e se li abbiamo tutti.
Quindi $\mathcal{F} = \left\{ A : \abs{A} = \frac{n}{2} \right\}$.

\section{Extremal set theory}

Cominciamo col grafo di Petersen, o grafo del comunista in prigione.
Il suo numero di clique \`e 2.
Il suo numero cromatico \`e 3.

Il grafo di Petersen appartiene alla famiglia dei grafi di Kneser.
Lo indichiamo con $K_{n,k}$, i cui vertici sono $V(K_{n,k} = \binom{[n]}{k}$, ossia tutti gli insiemi di $k$ elementi.
Mettiamo un arco $\{A,B\} \in E(K_{n,k})$ se $A \cap B = \emptyset$.

Kneser ha chiesto: qual \`e il numero cromatico di $K_{n,k}$?
Kneser ha notato che $\xhi(K_{n,k}) \le n - 2k + 2$.
Stiamo considerando solo $k \le \frac{n}{2}$.

Sappiamo che il suo numero cromatico \`e al pi\`u $n$: per ogni $n$ possiamo dare un colore a ogni vertice contenente $n$.
Nel momento in cui abbiamo usato $n - (2k - 1)$ colori, ci restano $2k -1$ elementi da colorare.
Da questi $2k -1$ elementi possiamo prendere un solo insieme di $k$ elementi, quindi usiamo al pi\`u $n - 2k +2$ colori.

E il numero di clique?
\`E al pi\`u $\frac{n}{k}$: dobbiamo dividere gli $n$ elementi in gruppi di $k$.

Laszlo Lovasz ha dimostrato nel '79 che $\xhi(K_{n,k}) = n - 2k +2$.
Il grafo di Petersen \`e un grafo di Kneser $K_{5,2}$.

Notare che $K_{n,1}$ \`e un grafo completo con $n$ vertici.

Vedremo prossimamente il teorema di Erd\H{o}s-Ko-Rado che riguarda il numero di stabilit\`a del grafo di Kneser.
Vedremo che $\alpha (K_{n,k}) = \binom{n-1}{k-1}$.

In questo grafo i vertici di una clique corrispondono a insiemi disgiunti di vertici.
Per $k = 1$ abbiamo un grafo completo.
Per $k = 2$ le cose sono pi\`u complicate.

Ogni grafo di Kneser si pu\`o sovrapporre a s\'e stesso in modo tale che i vertici si sovrappongano.
Ogni vertice all'interno del grafo ha lo stesso ruolo di ogni altro vertice.

Dimostriamo che il grafo di Petersen \`e un grafo di Kneser $K_{5,2}$.

Assegnamo a un vertice l'insieme $\{1,2\}$.
A un insieme adiacente assegniamo il primo insieme disgiunto, ossia $\{3,4\}$.
Fatto questo, dobbiamo trovare ancora un adiacente di $\{1,2\}$.
Ne abbiamo gi\`a trovati due: $\{3,4\}, \{4,5\}$.
Quindi manca $\{3,5\}$.
Possiamo anche pensare al complementare di ogni vertice: $\{3,4,5\}$.
Da qui prendiamo due elementi non consecutivi, ossia $\{3,5\}$.
Ripetiamo per gli altri vertici.

% TODO disegnare il grafo

Vediamo ora il numero di stabilit\`a di un grafo di Kneser $K_{n,k}$.
Consideriamo il problema astrattamente.

Una famiglia $\mathcal{F}$ di insiemi \`e intersecante se i suoi elementi non sono disgiunti, ossia se $\{A,B\} \in \binom{\mathcal{F}}{2} \implies A \cap B \neq \emptyset$.
Il problema di Erd\H{o}s-Ko-Rado \`e determinare la cardinalit\`a massima di una famiglia intersecante.
\[
	M(n,k) = \max\limits_{\mathcal{F}\subseteq\binom{[n]}{2}} \abs{\mathcal{F}}
\]
dove $\mathcal{F}$ \`e intersecante.

\`E facile costruire una famiglia intersecante di $k$-insiemi.
Fissiamo un elemento e consideriamo tutti i $k$ insiemi che contengono questo elemento.
Questi sono $\binom{n-1}{k-1}$: dobbiamo scegliere altri $k-1$ elementi fra gli $n-1$ elementi rimasti per formare un $k$ insieme.

Questo ci dice che $M(n,k) \ge \binom{n-1}{k-1}$.

C'\`e per\`o una considerazione da fare: se $k$ \`e maggiore di una certa soglia, non ci sono $k$-insiemi disgiunti.
Se $k \ge \frac{n}{2} \implies M(n,k) = \binom{n}{k}$: pi\`u grande non pu\`o essere.

\begin{theorem}[di Erd\H{o}s-Ko-Rado ('61)]
	\[
		M(n,k) =
		\begin{cases}
			\binom{n}{k} & \text{ se } k > \frac{n}{2} \\
			\binom{n-1}{k-1} & \text{ se } k \le \frac{n}{2}
		\end{cases}
	\]
\end{theorem}
Una struttura in cui fissiamo un elemento e cerchiamo insiemi che lo contengono \`e una struttura nucleare (??).
La struttura ottimale ha una struttura nucleare se si fissa un elemento piccolo e si considerano tutte le possibilit\`a.

\begin{proof}[del teorema di Erd\H{o}s-Ko-Rado]
	Tutto quello che dobbiamo mostrare \`e che $M(n,k) \le \binom{n-1}{k-1}$ per $k$ piccolo.
	Vuol dire che una famiglia intersecante non \`e ``troppo grande''.
	Invece di limitare superiormente la cardinalit\`a, limiteremo superiormente la \emph{densit\`a}.
	Possiamo quindi analizzare in maniera pi\`u intuitiva.
	\[
		\frac{M(n,k)}{\binom{n}{k}} \le \frac{\binom{n-1}{k-1}}{\binom{n}{k}} = \frac{k}{n}
	\]
	Questa \`e la frazione dell'universo che pu\`o formare una famiglia intersecante.
	L'ultima equivalenza \`e banale, ma possiamo vedere che $k \cdot \binom{n}{k} = n \cdot \binom{n-1}{k-1}$ che \`e pi\`u carino della banalit\`a.
	Scegliamo un elemento \emph{leader}.
	% TODO capire qui

	Facciamo un ``sondaggio'', o un censimento.
	Abbiamo gli studenti, divisi fra 100 corsi.
	Ciascuno frequenta cinque corsi.
	I professori fanno un sondaggio ai loro studenti, e ottengono tutti un certo risultato.
	Anche se gli insiemi di studenti di un corso non sono disgiunti, il risultato vale per tutti gli studenti.

	Se la famiglia di sondaggio \`e una $n$-copertura, e in ogni $n$-copertura la percentuale non supera una certa soglia, allora la soglia non \`e superata nell'insieme totale.

	\begin{lemma}[dei sondaggi]
		In realt\`a \`e un doppio conteggio, ma vabe.
		Abbiamo un insieme finito $U$, e denotiamo con $\Sigma$ una famiglia di sottoinsiemi di $U$, ossia $\Sigma \subseteq 2^{U}$.
		$\Sigma$ \`e una $c$-upla copertura di $U$, ossia il numero di insiemi contenenti un certo elemento $x$ \`e sempre $c$.
		$\abs{\{S : S \in \Sigma, x \in S \}} = c \forall x \in U$.
		Ogni elemento nell'universo \`e interpellato esattamente $c$ volte.
		Vogliamo calcolare la cardinalit\`a di un insieme $C$.
		Se vale che $\frac{\abs{C \cap S}}{\abs{S}} \le \lambda \forall S \in \Sigma$, allora $\frac{\abs{C}}{\abs{U}} \le \lambda$.
	\end{lemma}
	\begin{proof}[del lemma dei sondaggi]
		Se vale quanto detto, allora $\sum_{S \in \Sigma} \abs{S} = c \cdot \abs{U}$.
		Naturalmente vale anche che $\sum_{S \in \Sigma} \abs{S \cap C} = c \cdot \abs{C}$.
		Abbiamo detto che $\abs{C} = \sum_{S \in \Sigma} \frac{\abs{S \cap C}}{c}$.
		Scriviamo anche $c \cdot \abs{C} = \sum_{S \in \Sigma} \abs{S \cap C} = \sum_{S \in \Sigma} \abs{S} \cdot \frac{\abs{S \cap C}}{\abs{S}}$.
		L'ipotesi ci d\`a anche che la frazione $\frac{\abs{S \cap C}}{\abs{S}} \le \lambda$.
		Quindi $c \cdot \abs{C} \le \sum_{S \in \Sigma} \lambda \cdot \abs{S} = \lambda \sum_{S \in \Sigma} \abs{S} = \lambda \cdot c \cdot \abs{U}$.
		Quindi $\abs{C} \le \lambda \abs{U} \implies \frac{\abs{C}}{\abs{U}} \le \lambda$.
	\end{proof}

	Ora andiamo con la dimostrazione.
	Facciamo quella di Katona (che guarda caso \`e ungherese).
	Affermiamo che:
	\[
		\frac{\abs{\mathcal{F}}}{\binom{n}{k}} \le \frac{k}{n}
	\]
	Organizziamo i nostri $n$ elementi in una permutazione ciclica, che \`e una permutazione senza elemento iniziale o finale.
	Queste sono $\frac{n!}{n} = (n-1)!$.
	Fissiamo una permutazione ciclica e consideriamo i $k$ insiemi in questa permutazione ciclica che hanno elementi consecutivi.
	Sia $\pi$ una permutazione ciclica, $A$ \`e un intervallo ciclico se i suoi elementi sono consecutivi in $\pi$.
	Quanti intervalli ciclici ci sono?
	Il primo elemento di questo intervallo ciclico, se diamo un ordine alla permutazione ciclica, identifica un intervallo ciclico.
	Quindi abbiamo $n$ intervalli ciclici di una certa lunghezza.

	Ogni $k$-insieme sar\`a un intervallo ciclico in una qualche permutazione ciclica.
	E ogni $k$-insieme diventa un intervallo ciclico \emph{nello stesso numero di permutazioni cicliche}.

	Diamo un ordine agli elementi di $A$, per prima cosa.
	Possiamo dargli $k!$ ordini differenti.
	In quante permutazioni cicliche appare questo intervallo ciclico?
	In $(n - k)!$: il resto della permutazione ciclica \`e una permutazione \emph{vera} dei restanti $n - k$ elementi.
	Quindi un $k$-insieme a prescindere da chi sono i suoi elementi \`e una permutazione ciclica in $(n-k)! \cdot k!$.
	Ma possiamo anche capirlo per simmetria.

	Queste permutazioni cicliche formano una famiglia di sondaggio.
	Determiniamo quanti intervalli ciclici possono appartenere a una famiglia intersecante in una permutazione ciclica fissata.
	Ci sono al pi\`u $k$ intervalli ciclici che si intersecano, data una permutazione ciclica.
	% TODO rivedere un attimo i conteggi che stiamo facendo, specie che k <= n/2

	La massima cardinalit\`a di un insieme intersecante \`e almeno $\binom{n-1}{k-1}$, fissando un elemento e variando gli altri $k-1$ fra gli $n-1$ elementi ancora non scelti.

	Abbiamo visto che se $E \subseteq U$ e vogliamo limitare in qualche modo $\frac{\abs{E}}{\abs{U}} \le \lambda$, possiamo usare una famiglia $\Sigma \subseteq 2^{U}$ che ricopre $U$ $c$ volte, ossia ogni elemento $x \in U$ compare in $c$ insiemi in $\Sigma$.
	Possiamo limitare $U$ se scopriamo che $\forall S \in \Sigma$ $\frac{\abs{S \cap E}}{\abs{S}} \le \lambda$.

	Un metodo per trovare una $c$-copertura \`e considerare $c$ partizioni di un insieme.
	Non vale il contrario: il grafo del pentagono ha una 2-copertura data dagli archi che mancano.
	Questa bicopertura non \`e raggiunta da 2 partizioni, perch\'e partizioni in classi di 2 elementi non esistono su un insieme di 5 elementi.

	Consideriamo una permutazione $\pi$ ciclica di $[n]$.
	\`E una permutazione con un solo ciclo, in cui non c'\`e un primo elemento.
	In una configurazione ciclica abbiamo degli ``intervalli ciclici''.
	Un intervallo ciclico \`e una $k$-upla di elementi consecutivi nella configurazione ciclica, ossia una sequenza $a, a+1, \dots, a+k$.
	Ci sono $n$ intervalli ciclici lunghi $k$ in ogni permutazione ciclica.
	Gli intervalli ciclici individuano $k$-insiemi.

	Per utilizzare il lemma dei sondaggi dobbiamo innanzitutto capire che sono soddisfatte le condizioni che il lemma richiede.
	A una permutazione ciclica $\pi$ possiamo associare i $k$-insiemi che formano un intervallo ciclico in $\pi$:
	\[
		S(\pi) = \{ A; \abs{A} = k, A \text{ \`e un intervallo ciclico in } \pi \}
	\]
	Ogni $k$-insieme \`e un intervallo ciclico nello stesso numero di permutazioni cicliche.

	Sia ora $C = \binom{[n]}{k}$ una famiglia intersecante, vogliamo quanti suoi elementi possono essere contenuti nel nostro insieme di sondaggio $S(\pi)$, ossia $\frac{\abs{C \cap S(\pi)}}{\abs{S(\pi)}}$.
	Sappiamo che il denominatore di questa frazione \`e $n$.
	Dobbiamo solo capire quanto \`e grande l'intersezione $C \cap S(\pi)$.
	Se due insiemi si intersecano, anche i corrispondenti intervalli ciclici si intersecano.
	Quello che ora affermiamo \`e:
	\[
		\frac{\abs{C \cap S(\pi)}}{\abs{S(\pi)}} \le \frac{k}{n}
	\]
	Fissiamo un intervallo ciclico $[a_1, a_2, \dots, a_k]$ in $\pi$ che appartiene a $C$.
	Se non ce n'\`e siamo felici.
	Se ce n'\`e, ne prendiamo uno arbitrariamente.
	Ricordiamo che $k \le \frac{n}{2}$, altrimenti due intervalli ciclici si intersecherebbero sempre.

	Gli intervalli che intersecano l'intervallo scelto non possono essere troppo \emph{lontani} da questo.
	Se l'ultimo elemento dell'intervallo intersecante \`e $a_1$, questo interseca l'intervallo scelto solo in $a_1$.
	Anche l'intervallo che inizia in $a_2$ interseca l'intervallo.
	Possiamo considerare $k$ coppie di intervalli: che terminano in $a_i$ e che iniziano in $a_{i+1}$.
	Gli elementi di una coppia di intervalli non sono intersecanti.
	In ogni coppia al pi\`u un elemento appartiene a $C$, quindi al pi\`u $k$ intervalli ciclici appartengono alla famiglia intersecante $C$.
\end{proof}

La struttura nucleare delle soluzioni non funziona sempre.
Due permutazioni $\pi$ e $\rho$ si intersecano se $\exists i \in [n]$ che ha la stessa immagine, ossia $\pi(i) = \rho(i)$.
Qual \`e la massima cardinalit\`a di una famiglia intersecante di permutazioni?
Possiamo considerare tutte le permutazioni che fissano un elemento: sono $(n-1)!$.
Non ce ne sono di pi\`u.
Non pi\`u dell'$n$-esima parte di tutte le permutazioni sono intersecanti.
Troviamo quindi delle permutazioni \emph{non} intersecanti, troviamo la famiglia pi\`u grande possibile non intersecante.
Ogni famiglia di permutazioni intersecanti conterr\`a al pi\`u una di queste permutazioni non intersecanti.

Prendiamo una permutazione qualsiasi.
Spostando gli elementi (modulo $n$) di un passo verso destra, otteniamo una permutazione non intersecante.
Possiamo ripetere l'operazione $n-1$ volte, e ottenere $n$ permutazioni non intersecanti.
% TODO disegnare le permutazioni
Infatti all'$i$-esimo posto hanno $i,i+1,\dots,i+n-1$.
Di questo insieme una famiglia intersecante pu\`o contenere al pi\`u una permutazione, quindi al pi\`u l'$n$-esima parte della famiglia di tutte le permutazioni pu\`o contenere una famiglia intersecante.
Stiamo dicendo che se $P$ \`e intersecante, $\frac{\abs{P}}{n!} \le \frac{1}{n}$, e quindi $\abs{P} \le \frac{n!}{n} = (n-1)!$.

Ora, un problema che Korner ha inventato e che non sappiamo risolvere.
Le permutazioni $\pi$ e $\rho$ sono in collisione se $\exists i$ tale che $\abs{\pi(i) - \rho(i)} = 1$.
Le permutazioni si \emph{quasi} intersecano: non sono uguali, ma sono a distanza minima.
Quante permutazioni possono formare, al massimo, una famiglia in collisione?
Qua non si possono usare strutture nucleari.
La congettura \`e che $\abs{Q}$ sia al pi\`u il coefficiente binomiale centrale, $\binom{n}{\floor{\frac{n}{2}}}$.

A ogni permutazione possiamo associare un profilo: $(\pi(1), \dots, \pi(n)) \to (\varepsilon_1, \dots, \varepsilon_n)$, tale che $\varepsilon_i = \pi(i) \mod 2$.
Il profilo ci dice se l'immagine dell'$i$-esimo elemento \`e pari o dispari.
Due permutazioni in collisione devono avere profili differenti, e il numero dei profili differenti \`e quello che abbiamo detto prima.

% TODO fill here
\section{Principio di inclusione/esclusione}

Abbiamo un insieme $X$ di cardinalit\`a finita.
Questo insieme ha due sottoinsiemi $A$ e $B$.
Conosciamo le cardinalit\`a di $A$ e di $B$.
Vogliamo sapere quanti elementi di $X$ non appartengono a $A$ o a $B$.
Potremmo dire che sono $\abs{X} - \abs{A} - \abs{B}$, ma non va bene se $A$ e $B$ hanno elementi in comune.
Noi vorremmo esattamente $\abs{X} - \abs{A \cup B}$.

La relazione fra i due \`e questa:
\[
	\abs{X} - \abs{A \cup B} \ge \abs{X} - ( \abs{A} + \abs{B} )
\]
Poich\'e $\abs{A \cup B} \le \abs{A} + \abs{B}$, con l'uguaglianza solo se $A$ e $B$ sono disgiunti.

L'osservazione che possiamo fare \`e che gli elementi in $A$ e in $B$ sono stati ``tolti due volte''.
Se li aggiungiamo otteniamo un'espressione corretta: $\abs{X} - \abs{A \cup B} = \abs{X} - (\abs{A} + \abs{B}) + \abs{A \cap B}$.

Ci interessa una generalizzazione di questa espressione.
Abbiamo $r$ sottoinsiemi di un insieme $X$, indicati con $P_i \subseteq X$ con $i = 1 \dots r$.
Possiamo pensare che $P_i$ indica l'insieme degli elementi che godono della propriet\`a $i$-esima.
Noi vogliamo ora la cardinalit\`a dell'insieme di elementi che non godono di nessuna di queste propriet\`a, ossia $\abs{X \setminus \bigcup_{i=1}^{r} P_i}$.

Prendiamo un sottoinsieme $A \subseteq [r]$, da cui definiamo $N(A) = \abs{\{ x : x \text{ ha tutte le propriet\`a } A\}}$.
Questo insieme non esclude niente: l'altro insieme che abbiamo definito esclude tutte le propriet\`a.

Quello che diremo \`e:
\[
	\abs{X \setminus \bigcup_{i=1}^{r}} =
	\sum_{A \subseteq [r]} (-1)^{\abs{A}} N(A)
\]
Per quale $A \subseteq [r]$, $N(A) = \abs{X}$?
Per $A = \emptyset$.

Notare poi che $(-1)^{\abs{A}}$ significa che insiemi di cardinalit\`a dispari sono nella formula con segno negativo, mentre quelli di cardinalit\`a pari sono con segno positivo.

\begin{proof}
	Prendiamo un arbitrario $x \in X$: questo pu\`o non avere nessuna propriet\`a o averne alcune.
	Sia $B \subseteq [r]$ l'insieme delle propriet\`a di $x$ (quindi $x$ non ha propriet\`a che non sono in $B$).

	$x$ contribuir\`a un certo numero di volte a $N(A)$.
	Dobbiamo notare che per\`o non contribuisce necessariamente a tutti i ``conteggi''.
	Vorremmo capire il contributo totale di questo generico $x$.

	Se $x$ ha alcune propriet\`a, $B$ non \`e vuoto.
	Vogliamo dimostrare che se $B$ non \`e vuoto il contributo di $x$ al lato destro della formula \`e nullo.
	Il contributo complessivo di $x$ \`e:
	\[
		\sum_{A : A \subseteq [r]} (-1)^{\abs{A}}
	\]
	Ma non per tutti gli $A$ $x$ d\`a un contributo a questa sommatoria.
	Poich\'e $x$ ha tutte e sole le propriet\`a di $B$, $x$ contribuisce alla sommatoria solo per quegli insiemi $A$ che sono all'interno di $B$.
	Quindi possiamo riscrivere:
	\[
		\sum_{A : A \subseteq [r] \\ A \subseteq B} (-1)^{\abs{A}}
	\]
	Dobbiamo quindi mostrare che questa sommatoria \`e 0 se $B \neq \emptyset$.

	Se definiamo la funzione caratteristica $\xi(x,A)$ come segue:
	\[
		\xi(x,A) =
		\begin{cases}
			1 \text{ se } x \text{ ha tutte le propriet\`a di } A \\
			0 \text{ altrimenti}
		\end{cases}
	\]
	Potremmo scrivere che il contributo di $x$ \`e:
	\[
		\sum_{A \subseteq [r]} (-1)^{\abs{A}} \xi(x,A)
	\]
	E in generale che $N(A) = \sum_{x \in X} \xi(x,A)$.

	Un po' di algebra.
	Possiamo scrivere:
	\[
		\sum_{A: A \subseteq B} (-1)^{\abs{A}} =
		\sum_{A : A \subseteq B} (-1)^{\abs{A}} \cdot 1^{\abs{B} - \abs{A}}
	\]
	Abbiamo due tipi di fattori, e in ogni termine sono sempre $\abs{B}$.
	Ogni sommando ha lo stesso numero di fattori.
	\`E un caso speciale del binomio di Newton: $(1-1)^{\abs{B}} = 0^{\abs{B}}$.
	Se $B$ non \`e vuoto, questa sommatoria vale 0.
	Se $B$ \`e l'insieme vuoto, questa sommatoria contribuisce 1.
\end{proof}

\begin{oss}
	Possiamo riscrivere la formula in questo modo:
	\[
		\abs{X \setminus \bigcup_{i=1}^{r}} =
		\sum_{t = 0}^{r} \sum_{A : \abs{A} = t} (-1)^{\abs{A}} N(A)
	\]
\end{oss}

Vediamo un'applicazione del principio di inclusione/esclusione.
L'applicazione riguarda le permutazioni: una permutazione \`e una biezione $\pi : [n] \to [n]$.
\begin{defn}
	Una permutazione $\pi$ \`e un \emph{d\'erangement} (uno ``sconvolgimento'') se non ha punti fissi, ossia se $\pi(i) \neq i \forall i \in [n]$.
\end{defn}
Prendiamo $X = \{ \text{permutazioni di } [n]\}$.
$\abs{X} = n!$.
Consideriamo come $i$-esima propriet\`a $P_i$ quella di lasciare $i$ al suo posto, ossia $P_i = \{ \pi : \pi(i) = i \}$.
Abbiamo $n$ propriet\`a, e una permutazione \`e sconvolgente se non ha nessuna di queste propriet\`a.

Sia $A \subseteq [n]$, $N(A) = \abs{\{ \pi : \pi(i) = i \forall i \in A \}}$ \`e il numero delle permutazioni che lasciano al suo posto ognuno degli elementi di $A$.
Queste permutazioni che lasciano gli elementi di $A$ al loro posto sono $(n - \abs{A})!$.

\`E facile ottenere tutti gli $N(A)$.
Il numero delle permutazioni sconvolgenti quindi \`e:
\[
	\sum_{A \subseteq [n]} (-1)^{\abs{A}} N(A) =
	\sum_{t = 0}^{n} (-1)^{t} \sum_{A : A \in \binom{[n]}{t}} (n - t)!
\]

La probabilit\`a di prendere una permutazione sconvolgente ``a caso'' \`e:
\[
	\sum_{t = 0}^{n} (-1)^{t} \sum_{A : A \in \binom{[n]}{t}} \frac{(n - t)!}{n!} =
	\sum_{t = 0}^{n} (-1)^{t} \binom{n}{t} \frac{(n-t)!}{n!} =
	\sum_{t = 0}^{n} (-1)^{t} \frac{\cancel{n!}}{t! \cdot \cancel{(n-t)!}} \frac{\cancel{(n-t)!}}{\cancel{n!}}
	\sum_{t = 0}^{n} \frac{(-1)^{t}}{t!} = e^{-1} > \frac{1}{3}
\]
L'ultima \`e la serie di Taylor della funzione esponenziale!
Funny.








