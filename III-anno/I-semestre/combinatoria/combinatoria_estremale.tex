
\chapter{Combinatoria estremale}

Due delle branche pi\`u importanti della combinatoria e dell'informatica teorica sono la combinatoria estremale e la combinatoria probabilistica.

Partiamo con un po' di teoria estremale dei grafi: vogliamo capire il rapporto all'estremo di alcuni parametri, fissandone alcuni altri.
Un esempio di propriet\`a estremale \`e la minima superficie per volume che la sfera ha rispetto ad altri solidi.

Introduciamo la combinatoria estremale.
Fissato $n$, consideriamo tutti i grafi con $n$ vertici, e di questi grafi $G$ consideriamo due parametri: $\crom{G}$ (il numero cromatico) e $\abs{E(G)}$ (il numero degli archi).
Qual \`e il massimo numero degli archi in un grafo con $n$ vertici e con numero cromatico 2?
Un grafo con numero cromatico 2 \`e un grafo bipartito.
\[
	\Tau(n) = \max_{\abs{V(G)} = n \\ \crom{G} = 2} \abs{E(G)}
\]
Il grafo \`e bipartito: abbiamo due sottoinsieme dei vertici, $A$ e $[n] \setminus A$, ciascuno senza archi fra i suoi vertici.
Il massimo numero di archi \`e $\abs{A} \cdot (n - \abs{A})$: connettiamo ogni vertice in un insieme con ogni vertice nell'altro.
Dell'insieme $A$ ci interessa solo la cardinalit\`a.
Se la cardinalit\`a di $A$ \`e $x$, abbiamo $x \cdot (n - x)$ archi, e dobbiamo quindi massimizzare $x$ in funzione di $n$.
\[
	\Tau(n) = \max_{x} x \cdot (n - x)
\]
Quando $n$ \`e pari, questo valore \`e $\frac{n}{2} \cdot \frac{n}{2}$, quando $n$ \`e dispari \`e $\floor{\frac{n}{2} \cdot \frac{n}{2}V}$.
Ma perch\'e?

Dimentichiamo per un momento che $x$ debba essere un intero.
Quello che vale \`e questo:
\[
	x \cdot (n - x) \le \left{( \frac{n}{2} \right)}^2 = \frac{n^2}{4}
\]
La verifica \`e immediata:
\begin{align*}
	x \cdot (n - x) & \le \frac{n^2}{4} \implies \\
	4 x^2 - 4 n x + n^2 & \ge 0
\end{align*}
L'ultimo \`e un quadrato, quindi sempre non negativo, quindi vale sempre quanto detto.

Riflettere su questo: la radice del prodotto \`e minore o uguale della media aritmetica, e la radice del prodotto \`e la media geometrica.

Notare che $\floor{\frac{n}{2} \cdot \frac{n}{2}V} = \floor{\frac{n}{2}} \cdot \ceil{\frac{n}{2}}$.
Notare inoltre che $\binom{n}{2} \simeq \frac{n^2}{2}$, quindi al massimo abbiamo la met\`a degli archi.

Vediamo che $\floor{\frac{n}{2} \cdot \frac{n}{2}V} = \floor{\frac{n}{2}} \cdot \ceil{\frac{n}{2}}$.
Se $n$ \`e pari, vale banalmente.
Se $n$ \`e dispari, possiamo scrivere $n$ come $2 l + 1$, e quindi:
\[
	\floor{\frac{n^2}{4}} = \floor{\frac{4 l^2 + 4 l + 1}{4}} = 
	\floor{l^2 + l + \frac{1}{4}} = l^2 + l
\]
Dall'altro lato, possiamo scrivere questo:
\[
	\floor{\frac{n}{2}} = \floor{\frac{2 l + 1}{2}} = \floor{l + \frac{1}{2}} = l \\
	\ceil{\frac{n}{2}} = \ceil{\frac{2 l + 1}{2}} = \ceil{l + \frac{1}{2}} = l + 1
\]
E quindi abbiamo che $l^2 + l = l \cdot (l + 1)$.

Consideriamo ora tutti i grafi con $n$ vertici che non contengono triangoli.
Notare che tutti i grafi bipartiti non contengono triangoli, ma potrebbero esserci pi\`u grafi senza triangoli che grafi bipartiti.
\begin{theorem}[di Mandl-Tur\'an]
	\label{teorema_mandl_turan}
	Indicando con $\Delta$ il grafo triangolo, diciamo:
	\[
		T(n) = \max_{\abs{V(G)} = n \\ \Delta \nsubseteq G} \abs{E(G)}
	\]
	Questo numero \`e sempre $\floor{\frac{n^2}{4}}$.
\end{theorem}
\begin{proof}[del teorema \ref{teorema_mandl_turan}]
	L'assenza di triangoli caratterizza il grado dei vertici, ma lo vedremo poi.
	Intanto dimostriamo che $T(n) \ge \floor{\frac{n^2}{4}}$: \`e il lato pi\`u facile.
	Consideriamo il cosiddetto ``bipartito completo'', ossia il grafo $K_{\floor{\frac{n}{2}}\ceil{\frac{n}{2}}}$, che ha numero cromatico 2 e che quindi non contiene triangoli.
	Quindi il massimo numero di archi in un grafo che non contiene triangoli \`e certamente maggiore del numero di archi in questo grafo.

	Dimostriamo ora che $T(n) \le \floor{\frac{n^2}{4}}$.
	Ora dobbiamo sfruttare l'assenza di triangoli per limitare il numero degli archi.
	Consideriamo due vertici che formano un arco nel grafo, ossia $a,b \in V(G)$ tali che $\{a,b\} \in E(G)$.
	Avranno ciascuno il loro grado, $d_G(a)$ e $d_G(b)$.
	Non sappiamo niente su questi due gradi, ma sappiamo che la loro somma \`e limitata.

	Chiamando $\Gamma_a = \{ c : \{a,c\} \in E(G) \}$ l'insieme degli archi adiacenti a $a$.
	Chiaramente $a \notin \Gamma_a$, e $b \in \Gamma_a$.
	Inoltre $d_G(a) = \abs{\Gamma_a}$.
	Vale infine che $\Gamma_a \cap \Gamma_b = \emptyset$, altrimenti ci sarebbe un triangolo.
	Segue che, poich\'e $\abs{\Gamma_a \cup \Gamma_b} \le n$, $\abs{\Gamma_a} + \abs{\Gamma_b} \le n$.

	Ora introduciamo questa limitazione inferiore sul numero degli archi:
	\[
		\sum_{\{a,b\} \in E(G)} d_G(a) + d_G(b) \le n \cdot \abs{E(G)}
	\]

\end{proof}


