
\chapter{Grafi}

La combinatoria \`e la matematica del finito.
Tratteremo molto di grafi.
I grafi vanno considerati oggetti astratti.

\begin{defn}[Grafo]
	Un grafo (semplice) \`e una coppia di insiemi $(V,E)$ dove $\abs{V} < \infty$ e $E \subseteq \binom{V}{2}$.
\end{defn}
Con $\binom{V}{2}$ indichiamo che stiamo ``prendendo due elementi'' da $V$.
$\binom{V}{2}$ \`e l'insieme delle coppie non ordinate di elementi distinti di $V$.

L'insieme $E$ \`e l'insieme degli archi (\emph{edges}), mentre $V$ \`e l'insieme dei vertici collegati da questi archi.

Possiamo scrivere che $E \subseteq \{ X : X \in \parts{V} \land \abs{X} = 2 \}$.

Quanti archi possiamo avere al massimo?
Vale questo:
\[
	\abs{\binom{V}{2}} = \binom{\abs{V}}{2} = \frac{\abs{V} \cdot \left( \abs{V} - 1 \right)}{2}
\]

Il solito problema dei ponti di K\"onigsberg si rappresenta facilmente con un grafo.
Bisogna aggiungere vertici ``fittizzi'', visto che i nostri grafi non possono avere archi multipli o cappi.
Un cappio \`e un arco le cui estremit\`a coincidono.

Nel problema dei ponti di K\"onigsberg vogliamo trovare (guarda caso) un circuito euleriano, un circuito che attraversa tutti gli archi una sola volta, e inizia e finisce sullo stesso vertice.
Ma prima definiamo cosa intendiamo per spostamenti all'interno di un grafo.

\begin{defn}[Passeggiata (walk)]
	Una passeggiata su un grafo $G$ \`e una sequenza alternata di vertici e archi, in cui ogni arco \`e preceduto e seguito dai suoi estremi.
	Ossia, \`e qualcosa del tipo:
	\[
		v_1 e_1 v_2 e_2 v_3 \dots v_s e_s v_{s+1} \dots v_t
	\]
\end{defn}
In una passeggiata vertici e archi possono ripetersi.
Ora inventiamo due passeggiate vincolate.

\begin{defn}[Cammino (path)]
	Un cammino \`e una passeggiata in cui $i \neq j \implies e_i \neq e_j$, ossia in cui non possiamo percorrere lo stesso arco due volte.
\end{defn}

\begin{defn}[Cammino semplice (simple path)]
	Un cammino semplice \`e un cammino in cui $i \neq j \implies v_i \neq v_j$, ossia un cammino in cui non si passa pi\`u di una volta per lo stesso vertice.
\end{defn}

Chiamiamo ``chiuso'' un cammino semplice in cui $v_1 = v_t$, ossia che inizia e finisce con lo stesso nodo.
Chiamiamo ``circuito'' un cammino in cui vale la stessa propriet\`a.

Sappiamo ora come descrivere un percorso all'interno di un grafo.
Problema: partendo da $x \in V(G)$, dove possiamo arrivare tramite una passeggiata, un cammino, o un cammino semplice?

Concettualmente, una passeggiata \`e la cosa pi\`u comoda.
Se esiste una passeggiata $x \leadsto y$, ossia con $v_1 = x$ e con $v_t = y$, allora $y$ \`e raggiungibile da $x$.
Questa \`e una relazione, e non una relazione qualsiasi.
\`E una relazione transitiva, riflessiva e, nei grafi semplici, simmetrica.
\begin{enumerate}
	\item Che $x \leadsto x$, ossia che sia riflessiva, lo postuliamo, considerando $v_1$ una passeggiata.
	\item Che $x \leadsto y \implies y \leadsto x$ si vede facilmente: scriviamo vertici e archi della passeggiata in ordine inverso.
	\item $x \leadsto y \land y \leadsto z \implies x \leadsto z$, infatti abbiamo i due cammini e, togliendo il vertice finale al primo possiamo concatenarli e ottenre un cammino da $x$ a $z$.
		Formalmente, scriviamo il cammino da $x$ a $y$ come $x e_1 v_2 \dots e_s y$ e il cammino da $y$ a $z$ come $y \cap{e}_1 \cap{v}_2 \dots \cap{e}_t z$, e scriviamo poi il cammino da $x$ a $z$ come $x e_1 v_2 \dots e_s y \cap{e}_1 \cap{v}_2 \dots \cap{e}_t z$.
\end{enumerate}

$V(G)$ \`e quindi partizionato da questa relazione, ed \`e quindi unione disgiunta di classi di vertici.
I vertici in una classe sono raggiungibili gli uni dagli altri.

\`E chiaro che se un nodo \`e raggiungibile da $x$ tramite un cammino semplice, lo \`e anche con una passeggiata, poich\'e un cammino semplice \`e un caso particolare di passeggiata.
Ma vale anche l'opposto.

Una nota: ``essere una passeggiata'' \`e una propriet\`a locale, che riguarda i singoli archi.
``Essere un cammino semplice'', invece, \`e una propriet\`a globale.

Per trasformare una passeggiata in un cammino semplice, prendiamo un vertice $z \in V(G)$ che appare nella passeggiata.
Se $z$ appare pi\`u di una volta nella passeggiata, possiamo ``tagliare'' parte della passeggiata.

Siano $i$ e $j$ rispettivamente il pi\`u piccolo e il pi\`u grande indice per cui vale $z = v_i = v_j$.
Cancelliamo la passeggiata fra $v_i$ e $v_j$, lasciando $v_j$.
Ora $z$ non si ripete pi\`u nella passeggiata.
Iterando questo per ogni $z \in V(G)$ otteniamo un cammino semplice.

\begin{defn}[Grafo connesso]
	Se in $G$ la relazione di raggiungibilit\`a ha una sola classe di equivalenza il grafo $G$ \`e connesso.
\end{defn}

\begin{defn}[Circuito euleriano]
	Un circuito euleriano \`e un circuito che comprende tutti gli archi e tutti i vertici.
\end{defn}
Un grafo non connesso non ha circuiti euleriani.
Se dalla definizione di circuito euleriano togliessimo il vincolo che questo deve comprendere ``tutti i vertici'', potremmo trovare un circuito euleriano in un grafo con pi\`u di una componente connessa ma di cui una sola ha archi.

Possiamo pensare anche a un cammino euleriano, ossia un cammino che comprende tutti gli archi e tutti i vertici.
\begin{oss}
	Se $G$ ha un circuito euleriano, $G$ \`e connesso.
\end{oss}
Alcune altre osservazioni:
\begin{itemize}
	\item Un circuito \`e chiuso, quindi possiamo iniziare da un nodo qualsiasi.
	\item Non potendo passare pi\`u di una volta sull ostesso arco, se non riusciamo a trovare un cammino significa che in qualche vertice del grafo ci ``blocchiamo''.
\end{itemize}
Se rimaniamo ``bloccati'' in un vertice possiamo dire qualcosa sul numero di archi di questo vertice: sono dispari.
Per ``entrare'' e ``uscire'' da un vertice questo deve avere due archi incidenti.

\begin{defn}[Grado di un vertice]
	Il grado del vertice $v$ nel grafo $G$ \`e il numero degli archi che incidono a $v$.
	\[
		\deg{v} = \abs{\{ e \in E(G) : v \in e \}}
	\]
\end{defn}

\begin{theorem}[Teorema di Euler]
	Il grafo semplice $G$ ammette un circuito euleriano $\iff$ $G$ \`e connesso e ogni suo vertice ha grado pari.
\end{theorem}

\begin{proof}[del teorema di Euler]
	Vediamo il teorema verso destra: se in un grafo $G$ c'\`e un circuito euleriano $C$, allora ogni vertice di $G$ ha grado pari.
	Tutti gli archi $e$ presenti in $C$ sono diversi, e $\forall e \in E(G)$, vale che $e \in C$.

	Sia $v \in V(G)$ un vertice tale che $v \neq v_1$ (in $C$).
	$v$ appare in $C$ un certo numero di volte, ossia abbiamo che $v = v_{i_1} = \dots = v_{i_r}$.
	Sicuramente nel cammino non avviene $\dots v e v \dots$.
	Ogni volta che attraversiamo $v$, passiamo per due archi mai attraversati prima.
	Poich\'e in $C$ non compaiono archi ripetuti, e tutti gli archi compaiono una volta, ogni $v \in V(G)$ (escluso $v_1 = v_t$) ha grado pari.
	Ma per partire da $v_1$ e arrivare in $v_t$ usiamo due archi distinti, e nel resto del cammino il vertice $v_1 = v_t$ \`e come uno degli altri vertici ``di passaggio''.
	Quindi anche $v_1$ ha grado pari.

	Vediamo il teorema verso sinistra: un grafo connesso con ogni vertice di grado pari ha un circuito euleriano.
	0 non conta come pari, ma in ogni caso un vertice di grado 0 \`e isolato e, se il grafo ha pi\`u di un nodo e ha un vertice di grado 0, il grafo non \`e connesso.

	Prendiamo un vertice $x \in V(G)$ e un arco $e$ incidente a $x$ e, partendo da $x$ e passando per $e$, possiamo trovare un circuito (che torna quindi in $x$).
	Se cos\`i non fosse rimarremmo bloccati in qualche $y \neq x$: ``entriamo'' in $y$ ma non ne ``usciamo''.
	``Entrare'' e ``uscire'' ha bisogno di un numero pari di archi, se quindi non riusciamo a ``uscire'', $y$ \`e di grado dispari.
	Contraddizione.

	Abbiamo trovato un circuito: se questo comprende tutti gli archi, \`e Euleriano.
	Se non comprende tutti gli archi vogliamo ampliare il circuito.
	Consideriamo uno de icircuiti che comprendono il circuito trovato, e in particolare il circuito massimale che lo contiene.
	Dimostriamo che questo circuito massimale \`e Euleriano.

	Se un circuito massimale non ha archi ``pendenti'', ossia archi che incidono a un vertice del circuito senza farne parte, allora il circuito contiene tutti gli archi.
	E se non contiene tutti gli archi, allora ha archi pendenti!
	Se c'\`e un arco pendente dal circuito, rimuoviamo gli archi del circuito dal grafo.
	Vogliamo poter dire che anche in questo grafo possiamo trovare un circuito.
	Ma attenzione: il grafo pu\`o non essere connesso dopo aver tolto il circuito.
	Nella dimostrazione precedente, per\`o, il fatto che il grafo fosse connesso non ci \`e servito.
	Ogni vertice, ora, o \`e di grado pari o \`e isolato.
	Considerando la componente di cui fa parte l'arco pendente, questa sar\`a connessa.
	Avendo tolto un numero pari di archi da ogni vertice, ora i vertici avranno o grado pari o grado 0.

	Ripetendo quindi il ragionamento precedente, l'arco appeso \`e contenuto in un nuovo circuito.
	Questo circuito ha un vertice in comune col vecchio circuito e nessun arco in comune.
	I due circuiti formano, se uniti, un nuovo circuito, ma questo contraddice l'ipotesi che il vecchio circuito fosse massimale.
	Quindi un circuito massimale non ha archi pendenti, ed \`e quindi Euleriano.
\end{proof}

\begin{defn}[Sottografo]
	$F$ \`e un sottografo di $G$ se $V(F) \subseteq V(G)$ e $E(F) \subseteq E(G)$.
	Ovviamente $F$ deve essere un grafo, e quindi $E(F) \subseteq \binom{V(F)}{2}$.
\end{defn}

Ci interessano due tipi di sottografi importanti.
\begin{defn}[Sottografo indotto (induced subgraph]
	$F$ \`e un sottografo indotto di $G$ se $V(F) \subset V(G)$ e $E(F) = \binom{V(F)}{2} \cap E(G)$.
	\`E un sottografo definito unicamente dai vertici.
\end{defn}

\begin{defn}[Sottografo di copertura (spanning subgraph)]
	$F$ \`e un sottografo di copertura di $G$ se ha gli stessi vertici, ossia $V(F) = V(G)$ ma $E(F) \subseteq E(G)$.
\end{defn}
Un sottografo indotto e di copertura non \`e un sottografo proprio.
(Ma questo significa che un sottografo indotto $F$ pu\`o avere gli stessi vertici di $G$?)

\begin{defn}[Grafo completo]
	Un grafo $G$ \`e completo se $E(G) = \binom{V(G)}{2}$.
\end{defn}
Un grafo completo si indica spesso con $K$.
Un sottografo completo \`e necessariamente un sottografo indotto.
I sottografi completi si chiamano anche \emph{clique}.

$\clique{G}$ \`e il massimo numero di vertici in un sottografo completo di $G$.

\begin{defn}[Grafo complementare]
	$G$ e $\bar{G}$ sono complementari se:
	\begin{itemize}
		\item $V(G) = V(\bar{G}) = V$
		\item $E(G) \cap E(\bar{G}) = \emptyset$
		\item $E(G) \cup E(\bar{G}) = \binom{V}{2}$
	\end{itemize}
\end{defn}

\begin{prop}
	\label{grado_n_archi}
	Sia $G$ un grafo, vale:
	\[
		\sum_{x \in V(G)} \degree{x} = 2 \abs{E(G)}
	\]
\end{prop}

\begin{proof}
	Immaginiamo una matrice a valori in $\{0,1\}$ in cui ogni riga corrisponde a un arco e ogni colonna a un vertice del grafo, definita cos\`i:
	\[
		M(e,x) =
		\begin{cases}
			1 & \text{ se } x \in e \\
			0 & \text{ altrimenti}
		\end{cases}
	\]
	\`E una matrice di incidenza.
	Quanti 1 ci sono in $M$?
	Usiamo il metodo del doppio conteggio.

	Ogni riga contiene esattamente due ``1'', quindi sono $2 \abs{E(G)}$.
	Ogni colonna contiene un numero di ``1'' pari al numero di archi incidenti al vertice relativo alla colonna su cui siamo, quindi $\sum_{x \in V(G)} \degree{x}$.
\end{proof}

La sommatoria dei gradi di un grafo \`e pari!
Quindi il numero di vertici di grado dispari \`e pari o \`e 0.
Infatti:
\[
	\underbrace{\sum_{x \in V(G) \\ 2 \not\divides \degree{x}} \degree{x}}{\text{deve essere un numero pari}} +
	\underbrace{\sum_{x \in V(G) \\ 2 \divides \degree{x}} \degree{x}{\text{\`e un numero pari}}} =
	\underbrace{\sum_{x \in V(G)} \degree{x}}{\text{\`e un numero pari}}
\]
Il numero di sommandi (tutti dispari) deve essere pari.

Se in un grafo ci sono $>2$ vertici di grado dispari, non esiste un circuito euleriano.
Se il numero di vertici di grado dispari \`e esattamente 2, esiste per\`o un cammino euleriano.
\begin{defn}[Cammino Euleriano]
	Un cammino Euleriano in $G$ \`e un cammino che comprende tutti gli archi e tutt i vertici di $G$.
\end{defn}

\begin{proof}
	Ogni vertice \`e un vertice di passaggio, tranne gli estremi (che non devono coincidere), e ogni vertice tranne gli estremi \`e di grado pari.

	Possiamo ridurre la dimostrazione alla dimostrazione precedente per i circuiti Euleriani.
	Introduciamo un nuovo vertice con archi incidenti ai due vertici di grado dispari.
	Nel grafo ora c'\`e un circuito Euleriano, che inizia da dove vogliamo noi, quindi anche da un nodo di grado dispari, e passa per l'arco che abbiamo aggiunto.
	Il circuito sar\`a, ad esempio, del tipo $aecfb \dots a$, dove $c$ \`e il nuovo vertice e $e,f$ sono gli archi $\{a,c\},\{c,b\}$.
	Togliendo $aecf$ otteniamo un cammino Euleriano da $b$ ad $a$.
\end{proof}

\begin{defn}[Albero]
	Un albero \`e un grafo connesso senza cicli.
	Con ciclo indichiamo un cammino semplice chiuso.
\end{defn}
Un albero \`e ``radicato'' se si identifica un vertice come ``radice''.

\begin{prop}
	In un albero non banale (ossia con $\abs{V(G)} \ge 2$) ci sono almeno due vertici di grado 1.
\end{prop}

\begin{proof}
	Dimostriamo una proposizione pi\`u forte:
	\[
		\# \text{ di vertici di grado } 1 \ge \max_{x \in V(G)} \degree{x}
	\]
	Ossia, il numero di vertici di grado 1 \`e maggiore del grado del grafo.
	Il grado del grafo \`e il massimo grado di un vertice nel grafo.

	Questo implica la proposizione. C'\`e il caso in cui $\max \degree{g} = 1$, ma se il grafo \`e connesso (e un albero deve esserlo) l'unico caso in cui questo \`e possibile \`e se $V(G) = \{a,b\}$ e $E(G) = \{\{a,b\}\}$, che \`e subito verificato.

	Negli altri casi, quindi, il grafo deve avere grado $>1$.
	Se $\max \degree{x} = 0$, il grafo non \`e connesso, ma composto da vertici isolati (ossia $E(G) = \emptyset$).

	Percorriamo un cammino semplice da un vertice $x$ (cammino semplice con l'estremit\`a fissa $x$) per trovare un vertice di grado $1$.
	Vogliamo qualcosa di ``estremo'', qualcosa oltre cui non si pu\`o andare.
	Cerchiamo quindi un cammino semplice massimale con estremit\`a fissa $x$.
	Non \`e prolungabile, non \`e contenuto in altri cammini semplici con estremit\`a fissa $x$.
	Il cammino deve terminare in un vertice di grado dispari, in particolare di grado 1.
	Non possiamo passare due volte per lo stesso vertice, quindi il grado deve essere 1 (se ci ``blocchiamo'').
	Il numero di questi cammino \`e almeno $\degree{x}$.
	Prendendo $x$ con $\degree{x} \ge 2$, e sappiamo che c'\`e, abbiamo fatto.
\end{proof}
% TODO rivedere questa dimostrazione

Un grafo $G$ connesso ha un sottografo di copertura connesso e minimale, ed \`e interessante cercarlo.
Un grafo connesso di questo tipo \`e senza cicli.
Se consideriamo un grafo con $n$ archi e $n$ vertici, che formano un ciclo, togliendo un arco qualunque abbiamo un sottografo connesso e minimale con $n-1$ archi.
Vale per tutti i sottografi di copertura connessi e minimali, e per tutti gli archi.

\begin{prop}
	Un grafo connesso \`e un albero $\iff \abs{V(G)} = 1 + \abs{E(G)}$.
\end{prop}
Gli alberi sono molto diversi gli uni dagli altri, ma questo vale sempre.

\begin{proof}
	Primo lato: in un albero $\abs{V(G)} = 1 + \abs{E(G)}$.
	Sia $x_0 \in V(G)$ un vertice arbitrario fissato.
	Per ogni $x \neq x_0$, poich\'e il grafo \`e connesso, c'\`e un unico cammino semplice da $x$ a $x_0$.
	\`E semplice poich\'e il grafo non ha cicli.
	\`E unico perch\'e altrimenti i due cammini biforcherebbero e si ricongiungirebbero da qualche parte individuando un ciclo.
	Creiamo una bisezione $f: V(G) \setminus \{x_0\} \to E(G)$ dove $f(x)$ \`e il primo arco dell'unico cammino da $x$ a $x_0$.
	Poich\'e il cammino \`e unico, la funzione \`e ben definita.
	\`E suriettiva: se un arco non fosse nell'immagine della funzione, avremmo un ciclo fra $x_0$ e i vertici su cui questo arco incide.
	Possiamo quindi gi\`a vedere che $\abs{E(G)} \le \abs{V(G)} - 1$.
	La funzione \`e anche iniettiva.
	Consideriamo due vertici $a \neq b$, con $a \neq x_0$ e $b \neq x_0$.
	Se il primo arco nel cammino da $a \to x_0$ e $b \to x_0$ \`e lo stesso, allora $a$ e $b$ devono essere adiacenti.
	Quindi c'\`e un cammino $a e b \dots x_0$ e un cammino $b e a \dots x_0$.
	Possiamo individuare due nuovi cammini $a \to x_0$ e $b \to x_0$ che non passano per $e$.
	Sono entrambi due cammini semplici, e hanno almeno un vertice in comune ($x_0$).
	Questo primo vertice in comune chiude un ciclo, in contraddizione con l'ipotesi che il grafo fosse un albero.
	Quindi la funzione \`e una biezione, e $\abs{E(G)} + 1 = \abs{V(G)}$.

	Secondo lato: se un grafo connesso $G$ ha $\abs{V(G)} = \abs{E(G)} + 1$, $G$ \`e un albero (non ha cicli).
	Chiamiamo $\abs{V(G)} = n$.
	$G$ ha almeno un vertice di grado 1.
	Infatti per la proposizione \ref{grado_n_archi}, sappiamo che:
	\[
		2 (n - 1) = 2 \abs{E(G)} = \sum_{x \in V(G)} \degree{x}
	\]
	Essendo il grafo connesso, $\nexists x : \degree{x} = 0$.
	Quindi:
	\begin{align*}
		2 (n-1) =
		\sum_{x : \degree{x} = 1} \degree{x} + \sum_{x : \degree{x} > 1} \degree{x} \ge
		\sum_{x : \degree{x} = 1} 1 + \sum_{x : \degree{x} > 1} 2 =
		c + 2 (n - c) = 2n - c
	\end{align*}
	Dove $c$ \`e il numero di vertici di grado 1.
	Quindi $2n - 2 \ge 2n - c \impies c \ge 2$, abbiamo almeno 2 vertici di grado 1.

	Facciamo induzione sul numero dei vertici.
	Supponiamo sia vero per $\abs{\verts{\bar{G}}} < n$ che se $\abs{\verts{\bar{G}}} = \abs{\edges{\bar{G}}} + 1 \implies \bar{G}$ \`e un albero.
	Abbiamo visto che $\exists y \in \verts{G}$ tale che $\degree{y} = 1$.
	$G$ induce $\bar{G}$ su $\verts{G} \setminus \{y\}$.
	$\bar{G}$ \`e connesso, come $G$, infatti da $y$ possiamo andare ovunque.
	Inoltre $\abs{\verts{\bar{G}}} = \abs{\verts{G}} - 1$ e $\abs{\edges{\bar{G}}} = \abs{\edges{G}} - 1$.
	Il passo base dell'induzione \`e con $\abs{\verts{G}} = 1$ e $\edges{G} = \emptyset$, e si verifica banalmente che questo grafo \`e un albero.
	Ora, per ipotesi di induzione $\bar{G}$ \`e un albero, ossia non ha cicli.
	Se $G$ avesse unciclo, questo dovrebbe coinvolgere l'arco incidente in $y$, poich\'e $\bar{G}$ non ha cicli.
	Ma $\degree{y} = 1$, quindi $G$ neanche ha cicli.
\end{proof}

\begin{defn}
	Un grafo senza cicli \`e una foresta.
\end{defn}

\begin{oss}
	Sia $F$ una foresta, $\abs{\verts{F}} = \abs{\edges{F}} + \lambda$, dove $\lambda$ \`e il numero di componenti connesse di $F$.
\end{oss}

\begin{proof}
	Siano $C_1, C_2, \dots C_{\lambda}$ le componenti connesse di $F$.
	Ciascuna componente connessa \`e un albero, quindi $\forall i \abs{\verts{C_i}} = \abs{\edges{C_i}} + 1$.
	Quindi:
	\begin{align*}
		\abs{\verts{F}} =
		\sum_{i = 1}^{\lambda} \abs{\verts{C_i}} =
		\sum_{i = 1}^{\lambda} \left( \abs{\edges{C_i}} + 1 \right) =
		\sum_{i = 1}^{\lambda} \abs{\edges{C_i}} + \lambda =
		\abs{\edges{F}} + \lambda
	\end{align*}
\end{proof}

\newpage

% TODO catch up here

Abbiamo una endofunzone $f : [n] \to [n]$.
La rappresentiamo con un grafo diretto $G_f$ in cui $V(G) = [n]$ e con archi $(a, f(a)) \forall a \in [n]$.
Tutti i cicli sono ``ben orientati''.
Tutti gli archi con un vertice in comune sono ``consecutivi''.

Consideriamo un cammino in cui gli archi sono orientati consecutivamente, e che non pu\`o essere esteso ulteriormente.
O questo cammino contiene tutti gli archi del ciclo, e quindi il ciclo \`e ben orientato, o non li comprende.
Se il cammino fosse prolungabile, avremmo un vertice con due archi che partono dallo stesso nodo, ma ogni nodo del grafo ha grado uscente 1.
Nel grafo non ci sono cicli non orientati bene.
Ogni ciclo \`e nei ricorrenti.

Abbiamo due insiemi.
L'insieme $\mathcal{V}(n)$ dei vertebrati su $K_n$, e l'insieme $\mathcal{F}(n)$ delle endofunzioni su $[n]$.
Vogliamo articolare i due insiemi allo stesso modo, per mettere in biezione vertebrati e endofunzioni e dimostrare che i due insiemi hanno
la stessa cardinalit\`a.

Un vertebrato generico \`e una terna (C, \overset{L}{<}, \{ T_x \}_{x \in C} ) dove C \subseteq [n], \overset{L}{<} \`e un ordinamento lineare di C, e T_x \`e un albero radicato in x \in C e i vertici V(T_x) degli alberi partizionano [n], ossia lo ricoprono senza avere vertici in comune.

Una endofunzione \`e un insieme (C, partizione di C in cicli ben orientati, \{ T_x \}_{x \in C}) dove C \subseteq [n] sono i vertici ricorrenti della funzione, poi abbiamo questa partizione, e poi gli alberi T_x radicati in x \in C che partizionano [n].

Se C e \{ T_x \} coincidono, cosa possibilissima, le due terne possono essere in corrispondenza.
Dobbiamo vedere cosa \`e l'elemento centrale.

Abbiamo \abs{C}! ordinamenti lineari, nei vertebrati.
Per la biezione, ci basterebbe vedere che il numero di partizioni \`e lo stesso.

Nelle endofunzioni, i cicli in C rappresentano l'azione della endofunzione sui ricorrenti.
Anche questa \`e una endofunzione.
Ma non una qualsiasi.
La endofunzione ristretta ai ricorrenti \`e \emph{suriettiva}: ogni vertice ha grado entrante 1, quindi \`e immagine di qualche altro vertice (anche di s\'e stesso).
Una endofunzione suriettiva \`e una biezione (quindi invertibile).
Il numero di endofunzioni biettive su un insieme di cardinalit\`a \abs{C}, \`e \abs{C}!.

So, we are done.
\qed

Ora vogliamo parlare di endofunzioni suriettive.
Sono endofunzioni che non hanno elementi \emph{transienti}, come li abbiamo chiamati in precedenza.
Una generica endofunzione suriettiva \pi : [n] \to [n] la chiamiamo \emph{permutazione}.
(Questa \`e una definizione)

Possiamo rappresentarla naturalmente con una tabella con nella prima riga gli argomenti e nella seconda riga le immagini.

\begin{tabular}{*{5}{c}}
	1 & 2 & 3 & \dots & n \\
	\pi (1) & \pi (2) & \pi (3) & \dots & \pi (n)
\end{tabular}

Non abbiamo bisogno, in realt\`a, della prima riga.
Ci basta la seconda, e quindi ci basta un ordinamento lineare.
Sono due cose ben differenti,
endofunzioni suriettive e ordinamenti lineari, ma guarda un po' sono uguali.
Ora per\`o pensiamo al fatto che una endofunzione suriettiva \`e anche una partizione del suo insieme di definizione.

Sappiamo che le endofunzioni possono essere iterate.
Per comporre due funzioni $f$ e $g$, deve essere che \mathcal{R}(f) \subseteq \mathcal{D}(g), ossia l'immagine
di f deve essere un sottoinsieme del dominio di g.

Questo vale con le endofunzioni: dominio e immagine coincidono.
Le permutazioni di [n] possono essere applicate a [n] in maniera consecutiva.

Date le permutazioni \pi : [n] \to [n] e \rho : [n] \to [n], la permutazione \rho (\pi (a)) \`e ben definita \forall a \in [n].
Questa nuova permutazione la indichiamo con \pi \rho, e intendiamo dire che \pi \rho (a) = \rho (\pi (a)).

Sull'insieme delle n! permutazioni di un insieme di [n] elementi, possiamo considerare l'applicazione consecutiva come un \emph{prodotto}.
Diciamo prodotto e non addizione, per vari motivi.
L'addizione \`e considerata commutativa, il prodotto pu\`o non esserlo.

Questa operazione non \`e commutativa.
Vediamolo.

n = 3
\pi (1,2) (2,1) (3,3)
\rho (1,3) (3,1) (2,2)

\pi \rho \`e l'applicazione successiva di prima \pi e dopo \rho.
\pi \rho (1) = 2, \pi \rho (2) = 3, \pi \rho (3) = 1.
\rho \pi \`e l'applicazione successiva di prima \rho e dopo \pi.
\rho \pi (1) = 3, \rho \pi (2) = 1, \rho \pi (3) = 2.

Fatto.

Ora, di solito, con i prodotti abbiamo un elemento neutro.
La permutazione identit\`a \`e questo elemento neutro.

L'applicazione consecutiva \`e un'operazione associativa.


Guarda caso \`e anche un'operazione invertibile.
Rappresentando la permutazione come cicli, per trovare la permutazione inversa basta invertire l'orientamento degli archi.
Questo insieme \`e un gruppo: ha un'operazione associativa, invertibile e con elemento neutro.


Una permutazione \`e un automorfismo di un grafo: trasforma un grafo in s\'e stesso.
(???)
Una permutazione che porta un vertice di un grado in un vertice di un altro grado non \`e un automorfismo.
Gli automorfismi di una struttura formano un gruppo.

Gli automorfismi di una struttura mi dicono il grado di simmetria di una struttura.
La simmetria si definisce (in generale e in astratto) con il gruppo di automorfismi della struttura.

Gli insiemi dei vertici dei cicli sono disgiunti (e ben orientati).
Possiamo parlare di decomposizione ciclica di una permutazione.

Introduciamo la notazione per i cicli.
Indichiamo con (c_1 c_2 c_3 \dots c_t) il ciclo con archi (c_1, c_2), (c_2, c_3), \dots, (c_t, c_1).

La decomposizione ciclica di una permutazione la scriviamo mettendo i cicli uno di fianco all'altro.
I cicli non sono ordinati.
I cappi non li rappresentiamo.

Sappiamo che (1,3) (1,2) \neq (1,2) (1,3).
Ma nella decomposizione ciclica l'ordine non \`e importante.
Il prodotto di cicli disgiunti \`e commutativo!

Una permutazione si pu\`o anche vedere come prodotto di trasposizioni.
Una trasposizione \tau \`e una permutazione (ab), ossia \tau(a) = b e \tau(b) = a.

\begin{oss}
	La permutazione ciclica $a_1 a_2 \dots a_t$ \`e una serie di trasposizioni: $(a_1 a_2) (a_1 a_3) \dots (a_1 a_t)$.
\end{oss}

Attenzione: inserendo un elemento in questa serie di trasposizioni, si passa da sinistra verso destra!
Se scriviamo la permutazione come trasposizioni del tipo $(a_1, a_2) (a_2, a_3) \dots (a_{t-1} a_{t})$, l'``argomento'' della trasposizione passa da destra verso sinistra.

C'\`e correlazione fra permutazioni e algoritmi di sorting.
Un insieme disordinato pu\`o essere visto come una permutazione, e trovare come ordinare l'insieme vuol dire trovare la permutazione inversa.

Cambiando l'ordine degli elementi nella permutazione, non sembre otteniamo delle trasposizioni in ordine differente.
Invece, cambiando l'ordine delle trasposizioni, cambia la permutazione.

Perch\'e per ordinamenti diversi otteniamo una stessa rappresentazione ciclica?
Basta scegliere un elemento differente da cui partire.

Per cui, possiamo scrivere la permutazione come $(a_1 a_2 \dots a_t)$ o come $(a_2 a_3 \dots a_t a_1)$ senza che l'ordine delle trasposizioni cambi.
Ma possiamo anche scrivere la stessa permutazione come un prodotto di \emph{trasposizioni differenti}, ad esempio come $(a_2 a_3) (a_2 a_4) \dots (a_2 a_t) (a_2 a_1)$.

\begin{prop}
	Ogni permutazione \`e un prodotto di trasposizioni.
\end{prop}

Questo perch\'e ogni permutazione \`e un prodotto di permutazioni cicliche, e ogni permutazione ciclica \`e un prodotto di trasposizioni.
I componenti di questo prodotto \emph{non} sono univocamente determinati.

Now, attenzione: quando scriviamo una permutazione come prodotto di trasposizioni, il numero di trasposizioni e le trasposizioni possono essere differenti.
Quel che non cambia \`e la parit\`a del numero di permutazioni: o sono sempre un numero pari o sono sempre un numero dispari.

Vediamo ora che il gruppo delle permutazioni ha un sottogruppo non banale molto grande.

\begin{defn}[Elementi in inversione]
	Per la permutazione $\rho : [n] \to [n]$ gli elementi $a \in [n]$ e $b \in [n]$ sono ``in inversione'' se $(a - b) \cdot (\rho(a) - \rho(b)) < 0$.
	Stiamo quindi dicendo che $a$ e $b$ sono diversi, e che $a < b \implies \rho(a) > \rho(b)$ o che $a > b \implies \rho(a) < \rho(b)$.
\end{defn}

Una permutazione \emph{mette in inversione} coppie di elementi.

\begin{defn}[Parit\`a di una permutazione]
	La parit\`a di $\rho$ \`e la parit\`a del numero di coppie che $\rho$ mette in inversione.
\end{defn}

La permutazione identit\`a mette in inversione 0 elementi.
In questo caso consideriamo 0 un numero pari, o comunque consideriamo la permutazione identit\`a \`e pari.

Una trasposizione $(c,d)$, \`e una permutazione di che parit\`a?
Guardiamo la tabella della permutazione, supponendo $c < d$:

\begin{tabular}{*{10}{c}}
	1 & 2 & \dots & c & \dots & x & \dots & d & \dots & n \\
	1 & 2 & \dots & d & \dots & x & \import Data.Array.IArraydots & c & \dots & n
\end{tabular}

$c$ e $d$ sono in inversione, certamente, ma anche $x$ e $c$, o $d$ e $x$, e via dicendo.

Siano $t$ gli elementi fra $c$ e $d$, ciascuno di questi $t$ elementi \`e sia in inversione con $c$ che con $d$.
Quindi abbiamo $2 \, t$ inversioni per gli elementi fra $c$ e $d$, pi\`u un'inversione per $c$ e $d$ stessi.
Quindi una trasposizione \`e una permutazione di ordine dispari, avendo $2 \, t + 1$ inversioni.

\begin{prop}
	Per un'arbitraria permutazione $\rho$ e la trasposizione $(a,b)$, la parit\`a di $\rho$, che indichiamo con $\parity{\rho}$, \`e diversa dalla parit\`a di $\rho \cdot (a,b)$, ossia:
	\[
		\parity{\rho} \neq \parity{\rho \cdot (a,b)}
	\]
\end{prop}
Abbiamo appena visto che la parit\`a dell'identit\`a \`e pari, e la parit\`a di una trasposizione \`e dispari.
Questo implicher\`a che la parit\`a della permutazione ciclica $(a_1 a_2 \dots a_t)$ \`e opposta alla parit\`a di $t$.
Infatti possiamo scrivere la permutazione come $(a_1 a_2) (a_1 a_3) \dots (a_1 a_t)$, che sono $t-1$ trasposizioni.
Sappiamo che $(a_1 a_2)$ ha parit\`a dispari, e che $(a_1 a_2) (a_1 a_3)$ ha parit\`a differente, ossia pari.
E via dicendo, cambiamo parit\`a $t-1$ volte.

Possiamo vedere anche che il prodotto di permutazioni pari \`e ancora una permutazione pari.
Quindi l'insieme delle permutazioni pari \`e chiuso rispetto al prodotto.

Tra le permutazioni di $[n]$, il numero delle permutazioni pari \`e $\frac{n!}{2}$.
Anche questa \`e una conseguenza della proposizione sopra.
Serve assumere $n > 1$, perch\'e in questo caso abbiamo una sola permutazione, pari, e $\frac{1}{2}$ non \`e un naturale.

Come mai vale questo?
Prendiamo una permutazione pari $\rho$, e moltiplichiamola per la trasposizione $(1,2)$.
Possiamo mettere in biezione le permutazioni $\rho$ e $\rho \cdot (1,2)$.
Riapplicando il prodotto per $(1,2)$ riotteniamo $\rho$.
Abbiamo quindi una biezione, quindi entrambi gli insiemi hanno cardinalit\`a $\frac{n!}{2}$.

Piccola nota: il sottogruppo delle permutazioni pari \`e il sottogruppo pi\`u grande del gruppo delle permutazioni.

Passiamo alla dimostrazione.

Senza vincolare la generalit\`a, supponiamo $\rho^{-1}(a) < \rho^{-1}(b)$, e scriviamo la permutazione come segue.

\begin{tabular}{*{n}{c}}
	1 & 2 & \dots & $\rho^{-1}(a)$ & \dots & $\rho^{-1}(b)$ & \dots & n \\
	\rho(1) & \rho(2) & \dots & a & \dots & b & \dots & \rho(n)
\end{tabular}

Quali coppie in inversione in $\rho$ non sono in inversione in $\rho \cdot (a,b)$?
Scrivendo la permutazione $\rho \cdot (a,b)$, abbiamo questo:

\begin{tabular}{*{n}{c}}
	1 & 2 & \dots & $\rho^{-1}(a)$ & \dots & $\rho^{-1}(b)$ & \dots & n \\
	\rho(1) & \rho(2) & \dots & b & \dots & a & \dots & \rho(n)
\end{tabular}

Possiamo dire in generale che ci sono coppie di colonne che ``cambiano situazione'', e coppie che non la cambiano.
Non sappiamo chi sono.
Identifichiamo intanto le colonne con la loro posizione.
Quali sono le colonne in \rho che sono in inversione e che nel prodotto non lo sono?
Quali colonne da non invertite diventano invertite?

Una coppia di colonne cambia situazione se almeno una \`e tra le colonne $\begin{smallpmatrix}\rho^{-1}(a) \\ a \end{smallpmatrix}$ e $\begin{smallpmatrix}\rho^{-1}(b) \\ b \end{smallpmatrix}$.
Le colonne intermedie saranno colonne del tipo $\begin{smallpmatrix} x \\ \rho(x) \end{smallpmatrix}$ con $\rho^{-1} (a) < x < \rho^{-1} (b)$.
	Questa colonna pu\`o essere in inversione con entrambe, con nessuna, o con una ma non con l'altra.
	Abbiamo quindi 4 casi distinti.

	Sia $i$ il numero delle coppie in inversione in $\rho$ tali da non essere una coppia dove almeno una coppia \`e ``di bordo'' e l'altra \`e intermedia o di bordo.
	Sia $R(a,b)$ il numero delle colonne intermedie in inversione con le due colonne di bordo.
	Sia $N(a,b)$ il numero delle colonne intermedie non in inversione con nessuna colonna di bordo.
	Sia $R(a)$ il numero delle colonne intermedie in inversione con solo la colonna di $a$.
	Sia $R(b)$ il numero delle colonne intermedie in inversione con solo la colonna di $b$.

	Il totale delle colonne in inversione \`e dato da:
	\[
		i + 2 \, R(a,b) + R(a) + R(b) + \chi (a,b)
	\]
	Dove $\chi (a,b)$ \`e 1 se le colonne di $a$ e $b$ sono in inversione in $\rho$, 0 altrimenti.

	Come cambia la situazione?
	Le $i$ colonne erano in inversione e cos\`i rimangono. Quelle in inversione con le due colonne di bordo non ci saranno pi\`u.
	Quelle in inversione con la colonna di $a$ saranno in inversione con la colonna di $b$, e viceversa con la colonna di $b$.
	Quelle non in inversione con le due colonne di bordo ora saranno in inversione con entrambe.

	Otteniamo quindi:
	\[
		i + R(a) + R(b) + 2 \, N(a,b) + \chi'(a,b)
	\]
	Ponendo $\chi'(a,b) = 1$ se $\chi (a,b) = 0$ o $\chi'(a,b) = 0$ se $\chi(a,b) = 1$.
	La differenza fra prima e dopo \`e:
	\[
		2 \, R(a,b) + \chi (a,b) - 2 \, N(a,b) + \chi'(a,b)
	\]
	Che \`e un numero dispari.
	Quindi se il numero di inversioni era pari in $\rho$, ora \`e dispari, e viceversa se era dispari ora \`e pari.

	Abbiamo visto che ogni oggetto matematico pu\`o avere diverse rappresentazioni.
	Una permutazione pu\`o essere scritta come prodotto di trasposizioni, e queste possono essere completamente diverse.
	C'\`e qualcosa che non cambia scrivendo una permutazione come prodotto di trasposizioni?
	S\`i: la parit\`a del numero di trasposizioni.

	Abbiamo detto che data una permutazione $\rho$, alcune coppie di elementi sono in ``inversione''.
	La parit\`a del numero di coppie in inversione \`e la parit\`a della permutazione, e questa \`e una definizione.

	Gli automorfismi descrivono quello che non cambia in un oggetto matematico. (???)
	La parit\`a \`e un invariante relativamente semplice: ha solo due valori, anche se la definizione \`e un po' contorta.
	Avendo solo due valori, divide le permutazioni in due classi.

	Abbiamo parlato di una funzione ``Parit\`a'':
	\[
		\par{\rho} =
		\begin{cases}
			-1 \\
			1
		\end{cases}
	\]
	Vale che $\par{\rho \sigma} = \par{\rho} \par{\sigma}$.
	\`E una conseguenza di quello che abbiamo dimostrato, ossia che data una permutazione $\rho : [n] \to [n]$ qualsiasi e una trasposizione $(a,b)$ con $a,b \in [n]$, vale $\par{\rho (a,b)} \neq \par{\rho}$.

	Per dimostrarlo abbiamo contato il numero di inversioni in $\rho$ prima e dopo averlo moltiplicato per $(a,b)$.
	Rappresentando la permutazione con una tabella, chiamiamo ``colonne di bordo'' le colonne di $a$ e di $b$, e le colonne fra queste due le chiamiamo ``colonne intermedie''.

	Sia $i$ il numero delle coppie di colonne in inversione in $\rho$ in cui al pi\`u una \`e una colonna di bordo e l'altra \`e intermedia.
	Queste colonne saranno presenti anche in $\rho (a,b)$.
	Poi ci sono alcune colonne che cambiano situazione: sono le coppie di colonne in cui una \`e una colonna di bordo e una \`e una colonna intermedia.
	Chiamiamo poi $R(a,b)$ il numero delle colonne intermedie in inversione con tutte e due le colonne di bordo, e $N(a,b)$ il numero di colonne che non sono in inversione con nessuna delle colonne di bordo.
	Poi, chiamiamo $V(a,b) = R(a) + R(b)$ il numero delle colonne in inversione con esattamente una fra le colonne di bordo.
	Infine, $\chi(a,b)$ \`e definito come $1$ se le due colonne di bordo sono in inversione in $\rho$, $0$ altrimenti.

	Non conosciamo nessuno di questi numeri, ci interessa solo sapere come cambiano.
	Il numero di colonne in inversione in $\rho$ \`e:
	\[
		i + 2 \, R(a,b) + V(a,b) + \chi(a,b)
	\]
	$R(a,b)$ sono le coppie in inversione sia con $a$ che con $b$, quindi dobbiamo ``contarle'' due volte.
	Le colonne che non sono in inversione con $a$ o con $b$ non dobbiamo contarle.

	Il numero delle coppie in inversione in $\rho (a,b)$ \`e, invece:
	\[
		i + 2 \, N(a,b) + V(a,b) + \chi'(a,b)
	\]
	Le uniche colonne che possono cambiare situazione, e che anzi necessariamente la cambiano, saranno solo le coppie di colonne in cui una colonna \`e di bordo e l'altra \`e o una colonna di bordo o \`e una colonna intermedia.
	Le colonne in inversione e con $a$ e con $b$ ora non saranno in inversione con nessuna delle due.
	Al loro posto avremo in inversione le colonne che in $\rho$ non erano in inversione n\'e con $a$ n\'e con $b$.
	Quelle che erano in inversione con esattamente con una colonna di bordo, saranno ora in inversione ancora con esattamente una colonna di bordo, ma con l'altra.
	Infine, possiamo dire che $\chi'(a,b) + \chi(a,b) = 1$.
	Se $a,b$ non erano in inversione, ora lo saranno, e viceversa se erano in inversione ora non lo saranno.

	Ora affermiamo che la parit\`a del primo numero \`e diversa dalla parit\`a del secondo numero, ossia che la differenza fra i due numeri \`e un numero dispari.
	\[
		i + 2 \, N(a,b) + V(a,b) + 1 - \chi(a,b) - \left[ i + 2 \, R(a,b) + V(a,b) + \chi (a,b) \right]
	\]
	Questo numero \`e dispari.
	Infatti vale:
	\[
		2 \, N(a,b) - 2 \, R(a,b) + 1 - 2 \, \chi(a,b)
	\]
	che \`e evidentemente dispari.

	\subsection{Numero cromatico}

	Parliamo di numero cromatico in relazione alla colorazione dei vertici di un grafo.
	Si pu\`o parlare di numero cromatico anche in relazione al colore degli archi di un grafo.
	Il problema \`e nato prima della teoria dei grafi, nel contesto di colorazione delle carte geografiche.

	Appel-Haken hanno dimostrato nel '79, alla University of Illinois at Urbana Champwtf, che bastano 4 colori per colorare ogni carta geografica.

	I due hanno ridotto il numero di grafi da analizzare da infiniti che erano, a meno di duemila.
	Li hanno fatti analizzare tutti a un computer e hanno ``dimostrato'' che l'ipotesi \`e vera.
	Ma negli anni '90 \`e stato dimostrato che questa dimostrazione era incompleta.
	Hanno ridotto nuovamente il numero di grafi da analizzare a circa 600, e controllato i casi col computer.

	\`E un problema centrale nell'ottimizzazione combinatoria, nella quale dobbiamo colorare un grafo col minor numero possibile di colori.

	Consideriamo una carta geografia.
	A ogni regione sulla carta associamo un vertice di un grafo.
	Vogliamo associare a ogni regione un colore, e lo facciamo con una funzione $f: V(G) \to C$ che a ogni vertice del grafo assegna un elemento di un insieme astratto $C$, che chiamiamo colore.
	Nel problema della carta geografica dobbiamo avere che regioni adiacenti hanno colori differenti.
	Nei grafi, $f$ \`e una colorazione se vertici adiacenti hanno colori diversi.

	\begin{defn}[Colorazione di un grafo]
		Una colorazione di un grafo \`e una funzione $f : V(G) \to C$ con $C$ insieme astratto se per ogni coppia di vertici $a,b$ abbiamo che $\{a,b\} \in E(G) \implies f(a) \neq f(b)$.
	\end{defn}
	Ogni grafo ha una colorazione: qualsiasi funzione iniettiva (con valori tutti diversi) soddisfa il vincolo.

	Il problema \`e capire quanto pu\`o essere piccolo $C$, ossia il minimo $\abs{C}$ per cui esiste una colorazione $f : V(G) \to C$.
	La cardinalit\`a $\abs{C}$ \`e il numero cromatico del grafo.

	L'ottimizzazione combinatoria \`e anche chiamata ricerca operativa, o \emph{operations research}.
	Operations = operazioni militari, la ricerca operativa \`e nata per ottimizzare le risorse.

	\begin{defn}[Numero cromatico]
		Il numero cromatico di un grafo semplice \`e la pi\`u piccola cardinalit\`a di $C$ per cui esiste una funzione $f : V(G) \to C$ di colorazione che ha vertici adiacenti con colori diversi.
	\end{defn}

	Chiamiamo $\xhi(G) = \min \abs{f(V(G))}$ questo numero cromatico.
	Indichiamo, data $f : V \to C$ e $A \subseteq V$, con $f(A) = \{ f(v) : v \in A \}$.

	I grafi delle carte geografiche sono particolari.

	Indichiamo con $K_{3,3}$ un grafo \emph{bipartito}.
	Stiamo dicendo che abbiamo una partizione dei vertici in due sottoinsiemi, e i vertici in un sottoinsieme sono adiacenti a tutti i vertici nell'altro insieme.
	Su $K_{3,3}$ il problema tipico \`e disegnare il grafo in modo che gli archi non si incrocino.
	Non \`e possibile.

	I grafi che possono essere disegnati sul piano in modo tale che archi che non hanno vertici in comune non si incrocino, sono chiamati \emph{grafi planari}.
	Un grafo planare ha numero cromatico al pi\`u 4.

	Un grafo senza archi ha numero cromatico 1, ossia $E(G) = \emptyset \implies \xhi(G) = 1$.
	Un grafo con almeno un arco ha numero cromatico maggiore di 1.

	\begin{defn}
		Un grafo \`e  bipartito se il suo numero cromatico \`e al pi\`u due.
	\end{defn}

	V(G) = A \cup B
	A \cap B = \emptyset
	\forall \{a, b\} \in E(G) \abs{\{a,b\} \cap A } = \abs{\{a,b\} \cap B} = 1

	Si possono usare grafi bipartiti per rappresentare grafi arbitrari.
	Si duplica l'insieme dei vertici V, si toglie ogni arco, e si inserisce un arco fra il nodo $a$ e il nodo $b$, uno preso dal primo insieme e uno preso dall'insieme duplicato, se $a$ e $b$ sono adiacenti nel grafo originale.

	Un cammino (semplice) ha numero cromatico 2.
	Un ciclo con numero pari di vertici ha anche questo numero cromatico 2.
	Bisogna essere in grado di alternare strettamente i colori.

	Il pi\`u piccolo ciclo dispari \`e il grafo completo con 3 vertici, e ha numero cromatico 3.

	Ogni sottografo ha numero cromatico non superiore al numero cromatico del grafo.

	\begin{prop}
		\label{prop_grafo_bipartito_cicli_dispari}
		$G$ \`e bipartito $\iff$ non contiene cicli dispari.
	\end{prop}

	\begin{proof}[della proposizione \ref{prop_grafo_bipartito_cicli_dispari}]
		Verso destra: \`e sufficiente vedere che un ciclo dispari non \`e bicolorabile, poich\'e ha numero cromatico 3.
		Quindi ogni grafo contenente un ciclo dispari ha numero cromatico almeno 3.
		Se il ciclo dispari fosse bicolorabile, l'unica bicolorazione assegnerebbe a un nodo un colore, e a ogni nodo a distanza pari lo stesso colore.
		Ma il ciclo ha lunghezza dispari, quindi il suo adiacente ha distanza pari.

		Verso sinistra: il problema si riduce, banalmente, al caso di un grafo connesso.
		Se G non \`e connesso, ha pi\`u componenti connesse, che non hanno archi tra loro.
		La colorazione deve dare valori diversi a vertici adiacenti, quindi colorando una componente connessa non vincoliamo la coloraione delle altre componenti.
		Siamo $L_i$ le sue componenti connesse, con $i = 1 \dots t$.
		$\xhi(G) = \max_{i} \xhi(L_i)$, poich\'e la colorazione pu\`o essere fatta in maniera indipendente.
		Ora \`e sufficiente mostrare che se $G$ non ha cicli dispari ed \`e connesso, allora $\xhi(G) \le 2$.
		Un grafo connesso ha un albero di copertura, e un albero ha una sola colorazione.
		Possiamo pensare al fatto che l'albero \`e composto da una ramificazione di cammini, ed \`e senza cicli.
		Questa colorazione imposta dall'albero di copertura, \`e una colorazione di tutto il grafo.

		Quindi, $G$ \`e connesso.
		Prendiamo un suo arbitrario albero di copertura $T$, e sia $x_0 \in V(T)$ un vertice in $T$ (e in $G$, essendo un albero di copertura).
		Prendendo un altro vertice $x \neq x_0$, abbiamo una distanza $d_T(x,x_0)$ fra i due vertici data dal numero degli archi nell'unico cammino fra $x$ e $x_0$.
		La colorazione $f : V(T) \to \{ 0, 1 \}$ la definiamo come $f(x) = d_T(x, x_0) \pmod{2}$.
		In quest'unica colorazione possibile, prendiamo due vertici $a$ e $b$ che hanno la stessa colorazione, ossia $f(a) = f(b)$.
		Se questi due vertici non sono adiacenti in $G$, stiamo a cavallo, ma se lo sono la colorazione dell'albero non va bene per il grafo.
		Il colore di $a$ \`e $1$ se il cammino da $x_0$ a $a$ \`e di lunghezza dispari, $0$ se \`e pari.
		Se $a$ e $b$ sono dello stesso colore, significa che la parit\`a dei cammini da $x_0$ a uno dei nodi \`e la stessa.
		Sia $c$ (con $c$ possibilmente, ma non necessariamente, uguale a $x_0$) il punto in cui questi due cammini (in $T$) si congiungono.
		Poich\'e l'inizio del cammino da $x_0$ ai nodi coincide fino a $c$, e la lunghezza dei cammini \`e congrua modulo 2, la lunghezza del cammino da $a$ a $c$ a $b$ \`e pari.
		I cammini da $a$ a $c$ e da $c$ a $b$ sono infatti entrambi di lunghezza pari o entrambi di lunghezza dispari, essendo congrui modulo 2.
		Ma quindi $\{a,b\}$ \`e un arco che chiude un ciclo dispari.
		Contraddizione.

	\end{proof}

	Abbiamo detto che una colorazione \`e una funzione con certe caratteristiche, e che il numero cromatico \`e il numero minimo di colori per una colorazione.
	Se il numero cromatico \`e 2, il grafo non ha cicli di lunghezza dispari.

	Il numero cromatico $\xhi(G)$ \`e almeno pari al numero di clique, $\omega(G)$, ossia il numero massimo di vertici in un sottografo completo.
	Non \`e un'uguaglianza: $\xhi(G) \ge \omega(G)$.
	Un grafo per cui non vale \`e il grafo rappresentato da un pentagono.
	Il grafo pentagono \`e curioso, perch\'e, ad esempio, il grafo complementare \`e isomorfo.

	% TODO disegnare grafo complementare e grafo pentagono

	Claude Berge ha introdotto il concetto di grafo perfetto, ossia un grafo per cui $\xhi(G) = \omega(G)$ per il grafo e per ogni suo sottografo indotto.
	Un grafo perfetto ha molte conseguenze interessanti, per Berge, che ne ha congetturate alcune.

	Possiamo limitare superiormente il numero cromatico.
	\begin{prop}
		\label{prop_num_cromatico_minore_grado}
		Abbiamo gi\`a visto che $\xhi(G) \le d(G) + 1$, dove $d(G)$ indica il massimo numero di archi incidenti a un vertice in un grafo.
		Abbiamo definito $d_G(x) = \abs{\{ \{x,a\} : a \in V(G), \{x,a\} \in E(G) \}}$, e definiamo quindi $d(G) = \max\limits_{x \in V(G)} d_G(x)$
	\end{prop}
	Ma \`e una limitazione debole, questa.
	Per un ciclo dispari, $d(G)$ \`e 2, e quindi $\xhi(G) \le 3$, ma lo stesso vale per un ciclo pari.
	Con un grafo a stella, il massimo grado \`e $\abs{V(G)} - 1$, ma basterebbero due colori.

	\begin{proof}[della proposizione \ref{prop_num_cromatico_minore_grado}]
		Faremo una colorazione on-line del grafo.
		Sia $x_1, x_2, \dots, x_{\abs{V(G)}}$ un ordinamento dei vertici di $G$.
		\emph{Colorazione on-line} significa che non appena ci viene ``presentato'' un vertice, dobbiamo assegnargli un colore.
		Non \`e necessariamente il modo ottimale di colorare il grafo.

		Vediamo che in un qualsiasi ordine di presentazione i vertici possono essere colorati con al pi\`u $d(G) + 1$ colori.
		Usiamo i naturali come insieme dei colori, e stabiliamo che $f(x_1) = 1$.
		Dopo aver colorato i primi $t-1$ vertici, vogliamo colorare il vertice $x_t$.
		Dobbiamo stare attenti che $f(x_t) \neq f(x_i)$ con $i < t$, se $x_t$ e $x_i$ sono adiacenti.
		\`E sufficiente dare a $x_t$ un colore che sia diverso dal colore degli adiacenti.
		Il caso peggiore \`e quello in cui ho gi\`a colorato tutti gli adiacenti, e questi hanno tutti colori differenti.
		Gli adiacenti sono al massimo $d_G(x_t) \le d(G)$.
		Nel caso peggiore abbiamo usato $d(G)$ colori per gli adiacenti.
	\end{proof}

	Diamo anche un'altra dimostrazione.
	\begin{proof}[della proposizione \ref{prop_num_cromatico_minore_grado}]
		Sia $S \subseteq G$ un sottoinsieme che forma uno \emph{stabile} nel grafo.
		Uno stabile \`e un sottografo indotto senza archi fra i suoi vertici.
		Chiaramente questi vertici possono avere tutti lo stesso colore, non avendo archi fra di loro.
		Ci conviene prendere uno stabile massimale, ovviamente.

		Questo algoritmo (greedy) si pu\`o ripetere togliendo di volta in volta i vertici di $S$ dal grafo.
		Per renderlo effettivamente greedy bisognerebbe dare un qualche ordinamento alla scelta che si fa, scegliendo ad esempio ogni volta lo stabile pi\`u grande.

		In ogni caso, questo algoritmo ci d\`a una colorazione con un numero di colori limitato.

		Sia $S_1$ il primo stabile, coloriamo i suoi vertici col colore 1.
		Sia poi $G_2$ il grafo con vertici $V(G_2) = V(G_1) \setminus V(S_1)$ indotto da $G$.
		Vale che $d(G_2) \le d(G)$.

		Infatti ogni vertice in $S_1$ ha almeno un arco verso un vertice in $G_2$, poich\'e $S_1$ \`e massimale.
		Quindi ogni vertice che non appartiene a $S_1$ perde almeno un arco, e quindi $\forall x \in V(G_2)$ vale che $d_{G_2}(x) < d_G (x)$.

		Dimostriamo ora la proposizione per induzione sul massimo grado del grafo.
		Il caso base \`e per $d(G) = 0$, ossia il grafo \`e composto da vertici non connessi, e per cui vale $\xhi(G) = 1$.

		Consideriamo il passo generale, in cui $d(G) = n$.
		Per ipotesi, considerando $S_1$ stabile massimale in $G$, sappiamo che $\xhi(G \setminus S_1) \le d(G - S_1) + 1 \le d(G)$.
		Segue quindi, dal fatto che $\xhi(G \setminus S_1) \le d(G)$, che $\xhi(G) \le \xhi(G \setminus S_1) + 1 \le d(G) + 1$.
	\end{proof}

	% TODO disegnare grafo triangolo con un triangolo su ogni lato

	Con il grafo in figura \ref{}, il numero cromatico \`e 3.
	Ma usando l'algoritmo di prima, prendendo uno stabile che ha massima cardinalit\`a, usiamo 4 colori.
	Siamo poi costretti a prendere altri tre stabili con un solo vertice ciascuno.

	Passiamo ora a parlare della disuguaglianza $\omega(G) \le \xhi(G)$.
	I due parametri possono essere diversi, anche di 1.
	$\omega(G)$ \`e la massima cardinalit\`a di una clique di $G$.

	\`E evidente che una clique di $n$ vertici ha numero cromatico $n$.

	\subsection{Teoria Ramsey}

	Prendiamo un $k \in \naturals$.
	$K_k$, la clique con $k$ vertici, \`e un sottografo completo.
	$S_k$ \`e il sottografo stabile di $k$ vertici.
	Ramsey ha dimostrato qualcosa che c'entra con la ``regolarit\`a inevitabile''.
	Fissato $k$, considerando grafi abbastanza grandi rispetto a $k$, questi grafi contengono sempre un sottografo completo di $k$ vertici o uno stabile di $k$ vertici.
	Non abbiamo specificato quanto deve essere grande il grafo rispetto a $k$.
	Strutture grandi, quindi, hanno aree omogenee grandi.

	Parlando dei naturali, c'\`e un teorema, di tal van der Waerden, che dice che se si partiziona l'insieme degli interi in un numero finito di classi, ciascuna di queste conterr\`a una successione aritmetica di $k$ numeri, o una cosa simile.
	Erd\"os ha congetturato che, sia $A \subseteq \naturals$ un insieme, se $\lim_{n \to \infty} \frac{\abs{A \cap [n]}}{n} > 0$, allora $A$ contiene una progressione aritmetica di $k$ numeri $\forall k \in \naturals$.

	Tornando indietro, non sappiamo la soglia oltre cui vale quanto detto da Ramsey, non sappiamo neanche come cresce questa soglia.
	Ci serve perch\'e un grafo con un piccolo numero di clique e un piccolo numero di stabili, ha un numero cromatico enorme.

	La soglia di Ramsey per 3 \`e pari a 6.
	Indichiamo la soglia di Ramsey con R(3,3).
	Si vede facilmente che R(3,3) > 5, trovando un grafo con 5 vertici che non ha triangoli e che non ha stabili con 3 vertici: basta prendere il grafo pentagono.

	Invece, R(3,3) \le 6.
	In un grafo di grado 5, ogni vertice pu\`o essere adiacente al pi\`u a 5 altri vertici.
	Supponiamo ci sia un vertice di grado 3: se non c'\`e, consideriamo il grafo complementare.
	Questo vertice ha 3 adiacenti: o questi 3 formano uno stabile, o non formano uno stabile, e quindi almeno due di questi sono adiacenti, e abbiamo quindi un triangolo con il vertice di grado 3.

	Ma R(k,k) cresce molto in fretta.
	Erd\"os ha dimostrato che $R(k,k) \ge 2^{\frac{k}{2}}$.

	Il risultato fondatore della teoria Ramsey \`e che $\forall k$ $\exists R(k,k) \in \naturals$ tale che ogni grafo con almeno $R(k,k)$ vertici o ha una clique $K_k$ di $k$ vertici o lo ha il suo grafo complementare.
	Quindi se prendiamo un grafo completo con $R(k,k)$ vertici, comunque coloriamo gli archi con due colori ci sar\`a un sottografo completo di $k$ vertici con archi dello stesso colore.

	Sappiamo che $R(k,k)$ non \`e cos\`i grande: vale che $R(k,k) \le 2^{2 \, k} = 4^k$.
	Il teorema di Erd\"os ci dice che $2^{\frac{k}{2}} \le R(k,k)$.

	\begin{theorem}[di Erd\"os]
		$\forall k \ge 4$ $\exists G_k$ tale che $\abs{V(G_k)} \ge 2^{\frac{k}{2}}$ e che $\omega(G_k) < k$ e $\alpha(G_k) < k$, ossia il suo pi\`u grande completo ha meno di $k$ vertici e il suo pi\`u grande stabile ha meno di $k$ vertici.
	\end{theorem}
	Quello che ci fa vedere questo teorema \`e che esiste un grafo con un numero cromatico molto alto, ma con delle clique non grandi.
	Il numero cromatico sar\`a esponenziale in $k$.

	La dimostrazione sar\`a \emph{probabilistica}.
	Facciamo vedere che se prendiamo un grafo con quella cardinalit\`a, sar\`a del tipo che cerchiamo.
	Ma non lo esibiremo, questo grafo.
	La teoria dei grafi aleatori \`e dovuta a Erd\"os e a R\'enyi.

	\begin{proof}[del teorema di Erd\"os]
		Un grafo aleatorio \`e un grafo scelto a caso.
		Fissiamo $n \in \naturals$, e consideriamo tutti i grafi con $n$ vertici, ossia che hanno come insieme di vertici $[n]$.
		Su $n$ vertici possiamo creare $2^{\binom{n}{2}}$ grafi differenti.

		Possiamo pensare di tirate una moneta per ogni coppia di vertici, e mettere un arco fra questi due vertici solo se esce testa.
		La probabilit\`a di scegliere uno specifico grafo \`e ${\left( \frac{1}{2} \right)}^{\binom{n}{2}}$.

		Se la ``moneta'' con cui scegliamo di inserire un arco non \`e onesta, ma d\`a testa con probabilit\`a $p$, i grafi saranno differenti.
		Al variare di $p$ si vede che esistono valori soglia, oltre i quali una certa propriet\`a \`e verificata con probabilit\`a 1, e prima dei quali una certa probabilit\`a \`e verificata con probabilit\`a 0.

		Sia $\graphs_n$ l'insieme di tutti questi grafi.
		La sua cardinalit\`a $\abs{\graphs_n}$ \`e, abbiamo detto, $2^{\binom{n}{2}}$.
		Denotiamo con $\Alpha_n(k)$ i grafi con numero di stabilit\`a maggiore o uguale a $k$, ossia:
		\[
			\Alpha_n(k) = \{ G : G \in \graphs_n \land \alpha(G) \ge k \}
		\]
		Definiamo poi l'insieme $\Omega_n (k)$ come l'insieme dei grafi con un numero di clique maggiore di $k$.

		Ai fini della dimostrazione vogliamo far vedere che $\Alpha_n (k) \cup \Omega_n (k) \subsetneq \graphs_n$.

		Ci aspettiamo che questa stretta inclusione non valga per $n$ molto grandi.
		Vale sicuramente se $n < k$. % TODO check here

		Questi due insiemi non sono disgiunti fra loro.
		Un grafo pu\`o avere una clique grande e un sottografo stabile grande.

		% TODO definire graphs come \mathcal{G}
		Dovremmo far vedere che $\abs{\Alpha_n (k) \cup \Omega_n (k)} < \abs{\graph_n}$.
		La prima cosa che vogliamo dire \`e che le due famiglie di grafi hanno la stessa cardinalit\`a.
		Per farlo vedere, introduciamo la biezione $G \leftrightarrow \bar{G}$ su $\graphs_n$, che manda $\Alpha_n (k)$ in $\Omega_n (k)$, mandando ogni grafo nel suo complementare.

		Sappiamo che $\abs{\Alpha_n (k) \cup \Omega_n (k)} \le \abs{\Alpha_n (k)} + \abs{\Omega_n (k)} = 2 \, \abs{\Omega_n (k)}$.

		Un grafo \`e dentro $\Omega_n (k)$ se ha una clique su $k$ vertici.
		Introduciamo la notazione $\Omega_n (K) = \{ G : G \in \graphs_n \land G \text{ induce un completo su } K \subseteq [n] \}$

		Quindi scriviamo:
		\[
			\Omega_n (k) = \bigcup_{K \in \binom{[n]}{k}} \Omega_n (K)
		\]
		Dove con $\binom{[n]}{k} = \{ A : A \subseteq [n] \land \abs{A} = k \}$ indichiamo i sottoinsiemi di $[n]$ con cardinalit\`a $k$.
		Questa notazione ci aiuter\`a con i calcoli.

		Dato $K$, quanto vale $\abs{\Omega_n (K)}$?
		Abbiamo gi\`a scelto gli archi fra i vertici di $K$, quindi il numero di scelte che dobbiamo fare \`e $\binom{n}{2} - \binom{k}{2}$.
		Quindi $\abs{\Omega_n (K)} = 2^{\binom{n}{2} - \binom{k}{2}}$.

		Possiamo quindi dire questo, sulla cardinalit\`a di $\Omega_n (k)$:
		\[
			\abs{\Omega_n (k)} = \abs{\bigcup_{K \in \binom{[n]}{k}} \Omega_n (K)} \le
			\binom{n}{k} 2^{\binom{n}{2} - \binom{k}{2}}
		\]

		Quindi fino ad ora possiamo dire quanto segue:
		\[
			\abs{\Alpha_n (k) \cup \Omega_n (k)} \le 2 \, \binom{n}{k} 2^{\binom{n}{2} - \binom{k}{2}}
		\]
		Ora: quanto pu\`o essere grande $n$ in funzione di $k$ affinch\'e valga questa disuguaglianza?
		\[
			2 \, \binom{n}{k} 2^{\binom{n}{2} - \binom{k}{2}} < 2^{\binom{n}{2}}
		\]
		Se la disuguaglianza vale, allora in $\graphs_n$ esiste il grafo che vogliamo.
		Possiamo limitare $\binom{n}{k}$ con $\frac{n^k}{2^k}$, tanto $k \ge 4$.
		Facciamo il conto:
		\begin{align*}
			2 \, \binom{n}{k} 2^{\binom{n}{2} - \binom{k}{2}} < 2^{\binom{n}{2}} \implies \\
			\binom{n}{k} \cdot 2^{- \binom{k}{2} + 1} < 1 \implies \\
			\frac{n^k}{2^k} \cdot 2^{- \binom{k}{2} + 1} < 1
		\end{align*}
		Prendiamo $k$ pari, per semplicit\`a, e vogliamo far vedere che questa disuguaglianza vale per $n = 2^{\frac{k}{2}}$.
		Segue quindi:
		\[
			2^{k \cdot \frac{k}{2} - k - \binom{k}{2} + 1} < 1
		\]
		Che vale se l'esponente \`e minore di $0$.
		Infatti:
		\[
			\frac{k^2}{2} - k - \frac{k \cdot (k - 1)}{2} + 1 < 0 \implies
			- k + \frac{k}{2} + 1 < 0 \implies
			\frac{k}{2} + 1 < k
		\]
		Che \`e sempre vero per $k \ge 4$.
	\end{proof}

	Noi vogliamo sfruttare il teorema di Erd\"os.
	\begin{oss}
		$\xhi (G) \ge \frac{V(G)}{\alpha(G)}$, ossia il numero cromatico \`e maggiore o uguale al numero di vertici diviso per la cardinalit\`a massima di uno stabile in $G$.
	\end{oss}

	V(G) = \bigcup_{i = 1}^{\xhi(G)} \Alpha_i con \Alpha_i stabile.
	La sua cardinalit\`a \`e:

	\abs{V(G)} = \sum_{i = 1}^{\xhi(G)} \abs{\Alpha_i} \le \xhi(G) \cdot \alpha(G)

	Erd\"os ci dice che esiste un grafo $G_k$ con V(G_k) \ge 2^{\frac{k}{2}} e con \alpha(G_k) < k.
	Da questo sappiamo che il numero cromatico \`e minorato da qualcosa di enorme:
	\[
		\xhi(G_k) \ge \frac{2^{\frac{k}{2}}}{k}
	\]
	E allo stesso tempo il numero di clique \`e minore di $k$.

