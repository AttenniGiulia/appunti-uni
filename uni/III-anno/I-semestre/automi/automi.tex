
\chapter{Introduzione}

Gli argomenti del corso sono:
\begin{description}
	\item[Automi] modelli di calcolo, ossia astrazioni di calcolatore.
	\item[Calcolabilit\`a] Cosa possiamo e non possiamo calcolare.
	\item[Complessit\`a] Cosa non possiamo calcolare perch\'e intrattabile (ossia, esponenziale).
\end{description}

I modelli di calcolo si (potrebbero) possono classificare come ``modelli matematici di computazione della nuova gerarchia di Chomsky''.
\begin{center}
\begin{tabular}{c|c}
	Modelli di macchine & Classi di linguaggi \\
	\hline
	Macchina di Turing & ? \\
	\emph{Linear bound automata} & ? \\
	Automa a pila & Linguaggi liberi dal contesto \\
	Automa a stati finiti & Linguaggi regolari
\end{tabular}
\end{center}

Pi\`u si sale nella tabella pi\`u aumenta la complessit\`a del modello.

Nella colonna di destra abbiamo dei nomi di \emph{grammatiche}
Una grammatica \`e qualcosa del tipo:
\begin{verbatim}
<roba> ::= <altra roba> <ancora roba> [<roba opzionale>]
<roba> ::= s | i | n | g | o | l | i | c | a | r | a | t | t | e | r | i
\end{verbatim}
Una grammatica \`e sia un insieme di regole che definiscono la sintassi di un linguaggio, che un insieme di regole per generare frasi nel linguaggio.
La grammatica specifica la \emph{sintassi}, non la semantica.
Ogni linguaggio ha una macchina associata che lo riconosce.

Calcolabilit\`a: esistono problemi che non hanno soluzione algoritmica.
Per formalizzare un algoritmo di solito si d\`a la specifica della macchina che lo realizza.

Intuitivamente, il numero di algoritmi \`e numerabile, ma il numero di problemi non \`e numerabile, quindi esistono (molti) pi\`u problemi che algoritmi.

Complessit\`a: consideriamo solo i problemi che possiamo risolvere, e mettiamo in una certa classe tutti i problemi che possono essere risolti nel caso peggiore in un certo tempo.
La classe $\probp$ corrisponde a problemi risolvibili in $\bigo{n^k}$ per $k \ge 1$, mentre la classe $\probexp$ corrisponde a problemi risolvibili in $\bigo{2^{n^k}}$ sempre per $k \ge 1$.
Ovviamente per individuare la classe di un problema guardiamo i tempi dell'algoritmo migliore.

La classe $\probnp$ corrisponde alla classe dei problemi la cui soluzione \`e verificabile in tempo polinomiale.
La classe $\probnpc$ \`e la classe dei problemi ($\probnp$-completi) che, se risolvibili in tempo polinomiale, implicherebbe la risolvibilit\`a in tempo polinomiale di tutti i problemi in $\probnp$.

\chapter{Automi (modelli di calcolo)}

Cosa \`e una computazione? Cosa \emph{calcola} un modello di calcolo?

Partiamo da classificare cosa \`e un problema.
Ci interesseremo solo di problemi decisionali (la cui risposta \`e un s\`i o un no).
\`E possibile passare da un algoritmo che risolve un problema decisionale a un algoritmo che trova la soluzione del problema.

Un'istanza di un problema \`e l'input del problema.

Abbiamo accennato che un problema decisionale identifica un linguaggio.
Un linguaggio \`e un insieme di parole, dove con parola indichiamo una combinazione di elementi di un alfabeto (finito) di simboli.
Il linguaggio associato al problema \`e quello delle stringhe che codificano istanze ``s\`i'' del problema (decisionale).

\begin{description}
	\item[Computazione] riconoscimento di una parola.
	\item[Algoritmo] macchina che riconosce un linguaggio.
	\item[Alfabeto] insieme finito di simboli.
	\item[Parola] sequenza di lettere di un alfabeto.
	\item[$\lambda$ (lambda) o $\emptystring$ (epsilon)] stringa vuota.
\end{description}

Se $\Sigma$ \`e un alfabeto, $\Sigma^{\star}$ \`e l'insieme di tutte le parole costruibili in quell'alfabeto
Un linguaggio $L$ su $\Sigma$ \`e un sottoinsieme di $\Sigma^{\star}$.

Per come abbiamo definito un'istanza di un problema, ogni parola \`e un'istanza.
Quindi un linguaggio \`e un insieme di parole che corrispondono a istanze s\`i di un dato problema.

Tutti i nostri modelli di calcolo sono riconoscitori di linguaggi.

\section{Automi a stati finiti}

Consideriamo automi senza output, che si limitano a riconoscere l'input.

In un FSA la testina si sposta sempre a destra di un passo.
L'automa \`e rappresentabile con un grafo diretto in cui ogni nodo \`e uno stato, e un arco etichettato da una lettera dell'alfabeto di input  indica che si pu\`o passare da uno stato all'altro con l'input indicato.
Lo stato iniziale \`e individuato da un arco entrante che viene dal nulla.
Gli stati finali sono evidenziati.

La lunghezza di un cammino \`e data dal numero di vertici (nodi) visitati.
Ogni nodo deve avere un arco uscente per ogni lettera dell'alfabeto.
Uno stato da cui non si esce, ossia da cui non si pu\`o raggiungere uno stato finale, \`e uno ``stato pozzo'', e pu\`o non essere disegnato.
Un automa accetta almeno una parola se c'\`e almeno un cammino dallo stato iniziale a uno degli stati finali

La formalizzazione degli automi a stati finiti \`e la solita.
Ci concentriamo in particolare, per adesso, sui DFA, o \emph{Deterministic Finite state Automata}.
Un DFA \`e una quintupla: 
\[
	(Q, \Sigma, \delta, q_0. F)
\]
dove $Q$ \`e l'insieme degli stati, $\Sigma$ \`e l'alfabeto di input, $\delta : Q \times \Sigma \to Q$ \`e la funzione di transizione, $q_0 \in Q$ \`e lo stato iniziale e $F \subseteq Q$ \`e l'insieme di stati finali.

Parleremo spesso di configurazione: non tiene conto solo dello stato effettivo dell'automa, ma anche dello stato della sua memoria.
In un FSA non c'\`e memoria, ma ci interessa tenere conto della stringa ancora da leggere.

La configurazione di partenza \`e $(q_0, x)$, con $x \in \Sigma^{\star}$ a indicare la stringa di input.

Definiamo la relazione ``porta a'' come segue:
\[
	(p, ax) \underset{M}{\to} (q,x) \iff \delta(p,a) = q
\]
dove $p,q \in Q$ sono due stati, $a \in \Sigma$ \`e una lettera dell'alfabeto e $x \in \Sigma^{\star}$ \`e una stringa.

Notare: questa relazione \`e definita sull'insieme delle configurazioni.

La chiusura transitiva e riflessiva di questa relazione ci dice se da una certa configurazione arriviamo a un'altra.
Per chiusura transitiva, ricordiamo, si intende che le due coppie $((p, abx), (q,bx))$ e $((q,bx),(t,x))$ nella relazione portano alla coppia $((p,abx),(t,x))$ nella chiusura transitiva della relazione stessa.
La riflessivit\`a, invece, serve a dirci che senza input la macchina non cambia stato.

Una parola $x$ \`e accettata da $M$ se $(q_0, x) \underset{M}{\to} (q,\emptystring)$ con $q \in F$.
L'insieme delle parole accettate formano il linguaggio associato a $M$.

Possiamo estendere la funzione di transizione $\delta : Q \times \Sigma \to Q$ a una funzione $\delta^{\star}: Q \times \Sigma^{\star} \to Q$ definita come:
\begin{itemize}
	\item se $\abs{x} = 0$, allora $\delta^{\star}(q,x) = q$.
	\item se $\abs{x} > 0$, allora $x = ya$ dove $y \in \Sigma^{\star}$ e $a \in \Sigma$, e quindi $\delta^{\star}(q,x) = \delta \left( \delta^{\star} (q, y), a \right)$.
\end{itemize}

Segue immediatamente che:
\[
	\delta^{\star}(q,a) = \delta \left( \delta^{\star} (q, \emptystring) , a \right) = \delta (q,a)
\]
Inoltre, una definizione alternativa di ``accettare una parola'' \`e dire che:
\[
	\delta^{\star} (q_0, x) = r \in F
\]
La funzione $\delta^{\star}$, dato un input, ci porta nello stato a cui il DFA arriva consumando tutto l'input.

Possiamo formalizzare il linguaggio accettato da un DFA come segue:
\[
	\lang{M} = \{ x \in \Sigma^{\star} \text{ t.c. } \delta^{\star} (q_0, x) \in F \}
\]
E ora, pensiamo alla \emph{classe} dei linguaggi accettati dai DFA.

Un DFA accetta almeno una parola se esiste un cammino da $q_0$ a uno stato in $F$.
Se vogliamo trovare una parola accettata a forza bruta, prendiamo tutte le parole di lunghezza strettamente minore di $n$, dove $n$ \`e il numero degli stati dell'automa, e le proviamo tutte.

Supponiamo nessuna di queste parole sia accettata, e che esista una parola $w$ con $\abs{w} \ge n$ che viene accettata.
Per almeno uno stato passiamo due volte.
La parte di cammino che passa fra i due vertici ripetuti pu\`o essere tolta, e da $w$ possiamo togliere i caratteri che hanno portato a compiere questo ciclo.
Otteniamo quindi una parola $w'$ accettata dall'automa con $\abs{w'} < \abs{w}$.
Iterando il procedimento finch\'e la parola non \`e pi\`u corta di $n$, arriviamo all'assurdo.
Quindi se l'automa accetta almeno una parola, esiste una parola pi\`u corta di $n$ che viene accettata.

Con un altro giro di parole, se abbiamo un cammino di accettazione che ha un ciclo, togliendo le ``etichette'' relative al ciclo otteniamo una parola che \`e ancora accettata ma il cui cammino di accettazione non ha cicli e che quindi \`e lungo meno di $n$.

\subsection{Propriet\`a dei linguaggi dei DFA}

Ogni modello di calcolo ha delle propriet\`a.
I linguaggi dei DFA godono della propriet\`a di chiusura rispetto ad alcune operazioni.
I linguaggi sono insiemi, quindi abbiamo le tipiche operazioni di unione, intersezione e complemento.
In generale, i linguaggi sono chiusi rispetto a queste operazioni.

Un'altra operazione possibile \`e la concatenazione fra parole, che ci porta a pensare all'operazione di concatenazione \emph{fra linguaggi}: concateniamo tutte le parole del primo con tutte le parole del secondo.
Si indica tipicamente con $L L'$.
Due propriet\`a banali:
\[
	\{ \emptystring \} L = L = L \{ \emptystring \} \qquad \emptyset L = \emptyset = L \emptyset
\]
La concatenazione ci porta poi a definire il concetto di potenza di un linguaggio (ripetizione della concatenazione).
Il ``caso base'' \`e $L^0 = \{ \emptystring \}$.

Il simbolo ``stella di Kleene'' indica l'unione di tutte le potenze di un linguaggio:
\[
	L^{\kleenestar} = \{ \emptystring \} \cup L \cup L^2 \cup L^3 \cup \dots = \bigcup_{n \ge 0} L^n
\]

Un insieme $C$ \`e chiuso rispetto all'operazione (binaria) $\cdot$ se $\forall x, y \in C$, vale che $x \cdot y \in C$.

Vogliamo dimostrare (o comunque vedere) che $\langs{\text{DFA}}$, o classe dei linguaggi regolari, ossia l'insieme di tutti i linguaggi accettati da una qualche DFA, \`e chiuso rispetto alle operazioni di unione, intersezione, complemento, concatenazione e stella di Kleene.

\begin{itemize}
	\item $L, L' \in \langs{\text{DFA}} \implies L \cup L' \in \langs{\text{DFA}}$.
		La nuova macchina che accetta l'unione dei due linguaggi possiamo costruirla a partire dalle due macchine $M_1, M_2$.
		Date le due DFA:
		\[
			M_1 = (Q_1, \Sigma, \delta_1, q_{01}, F_1) \qquad M_2 = (Q_2, \Sigma, \delta_2, q_{02}, F_2)
		\]
		Costruiamo la macchina che accetta l'unione in questo modo:
		\[
			M = \left( (Q_1 \times Q_2), \Sigma, \delta, (q_{01}, q_{02}), F \right)
		\]
		L'insieme degli stati \`e il prodotto cartesiano degli stati delle due macchine.
		Dove la nuova funzione \`e definita come $\delta \left( (q_1, q_2), a \right) = \left( \delta_1(q_1, a), \delta_2(q_2, a) \right)$, e l'insieme degli stati finali $F$ \`e dato dall'unione $F_1 \times Q_2 \cup Q_1 \times F_2$.
	\item $L, L' \in \langs{\text{DFA}} \implies L \cap L' \in \langs{\text{DFA}}$.
		La costruzione \`e identica a quella per l'unione, con la differenza che l'insieme degli stati finali \`e ora dato da $F_1 \times F_2$.
	\item $L \in \langs{\text{DFA}} \implies \complement L \in \langs{\text{DFA}}$.
		Il complemento di $M = (Q, \Sigma, \delta, q_0, F)$ \`e dato da $\complement M = (Q, \Sigma, \delta, q_0, Q \setminus F)$.
\end{itemize}

Si dimostra quindi per induzione, per l'unione e l'intersezione, che:
\[
	\delta^{\star} \left( ( q_{01}, q_{02}), x \right) = \left( \delta_1^{\star} (q_{01}, x), \delta_2^{\star} (q_{02}, x) \right)
\]
Non ci interessa degli stati non raggiungibili dallo stato iniziale.

Per il complemento, invece, la prova \`e immediata:
\[
	\left( x \in \lang{M} \iff \delta^{\star} (q_0, x) \in F \right) \equiv \left( x \in \lang{\complement M} \iff \delta^{\star} (q_0, x) \in Q \setminus F \right)
\]

\subsection{Automi non deterministici (NFA)}

Con i NFA da uno stato, con un input, possiamo raggiungere pi\`u stati.
Se fra tutti i cammini individuati da una parola ce ne \`e uno che termina con uno stato finale, l'automa accetta la parola.
Con i NFA introduciamo anche le $\emptystring$-transizioni o $\emptystring$-mosse, che non hanno bisogno di una parola in input.

Non determinismo significa che possiamo avere cammini diversi con lo stesso input, ossia l'input non determina il cmamino.

Si possono introdurre le $\emptystring$-transizioni anche nelle DFA, ma gli stati con $\emptystring$-transizioni non devono avere altre transizioni possibili.

La formalizzazione degli NFA \`e praticamente identica:
\[
	M = (Q, \Sigma, \delta, q_0, F)
\]
Ma adesso la funzione di transizione \`e una funzione $\delta : Q \times \left( \Sigma \cup \{ \emptystring \} \right) \to \parts{Q}$.
Non porta a un singolo stato, ma a un insieme di stati.

Il concetto di configurazione \`e identico, ma un automa pu\`o avere pi\`u configurazioni.
Vale ancora la relazione definita per i DFA.
\[
	(p, ax) \underset{M}{\to} (q,x) \iff q \in \delta(p,a)
\]
Una configurazione \`e di accettazione se \`e $(q, \emptystring)$ con $q \in F$.

Una parola \`e rifiutata da un NFA quando non esistono cammini di accettazione, ossia tutti i cammini si bloccano prima della fine dell'input o terminano in uno stato non in $F$.

\subsection{Equivalenza fra NFA e DFA}

Abbiamo un algoritmo per creare un DFA a partire da un NFA, che ne dimostra quindi l'equivalenza.
Con $\emptystring$-chiusura di uno stato $q$ indichiamo l'insieme degli stati raggiungibili da $q$ usando solo $\emptystring$-mosse.

\begin{algorithm}
	\label{alg:algoritmo_dfa_eq_nfa}
	\caption{Algoritmo per creare un DFA a partire da un NFA.}
	\begin{algorithmic}
		\Require un NFA con $\emptystring$-mosse $M = (Q, \Sigma, \delta, q_0, F)$
		\Ensure un DFA $M' = (Q', \Sigma, \delta', q_0', F')$ equivalente
		\State $Q' \gets \{ \emptystring$-chiusura($q_0$)$\}$
		\While{$\exists T \in Q'$ non marcato}
			\State marca $T$
			\ForAll{$a \in \Sigma$}
				\State $V \gets \bigcup_{q \in T} \delta(q, a)$
				\State $V' \gets \bigcup_{q \in V}$ $\emptystring$-chiusura($q$)
				\If{$V' \notin Q'$}
				\State $Q' \gets Q' \cup \{ V' \}$
				\EndIf
				\State $\delta'(T, a) \gets V'$
			\EndFor
		\EndWhile
	\end{algorithmic}
\end{algorithm}

Lo stato iniziale, dopo aver usato l'algoritmo \ref{alg:algoritmo_dfa_eq_nfa}, sar\`a l'$\emptystring$-chiusura di $q_0$.
Gli stati finali in $F'$ saranno tutti gli stati in $Q'$ che contengono uno stato $q \in F$.

La complessit\`a dell'algoritmo \`e un problema: potremmo essere costretti a creare $2^n$ nuovi stati.
Prendiamo ad esempio l'automa in figura \ref{fig:nfa_con_problemi}.
Questo automa accetta tutte le stringhe con un 1 nella posizione $n$-ultima (o una cosa simile).
Il DFA di questo automa ha $2^n$ stati.

\begin{figure}
	\centering
	\begin{tikzpicture}[
			->,
			>=stealth',
			shorten >=1pt,
			auto,
			node distance=2.8cm,
			semithick
		]

		\node[initial, state] (q0) {$q_0$};
		\node[state] (q1) [right of=q0] {$q_1$};
		\node[state] (q2) [right of=q1] {$q_2$};
		\node[] (dots) [right of=q2] {$\dots$};
		\node[state, accepting] (qn) [right of=dots] {$q_n$};
	
		\path (q0)	edge [loop, above]	node {0,1}	(q0)
			(q0)	edge	node {1}	(q1)
			(q1)	edge	node {0,1}	(q2)
			(q2)	edge	node {0,1}	(dots)
			(dots)	edge	node {0,1}	(qn);
	\end{tikzpicture}
	\label{fig:nfa_con_problemi}
	\caption{Automa con problemi}
\end{figure}

Bisognerebbe dimostrare l'equivalenza fra l'automa che costruiamo con l'algoritmo e l'automa iniziale, ossia che:
\[
	x \in \lang{M} \iff x \in \lang{M'}
\]
Ma chi siamo noi per dimostrarlo?

\subsection{\emph{Pumping Lemma}}

Abbiamo parlato di cicli nei cammini di parole accettate.
Siamo sicuri che c'\`e un ciclo nel cammino di una parola accettata, quando la parola \`e pi\`u lunga di $n = \abs{Q}$.
Ogni parola pi\`u lunga di $n$ si pu\`o scomporre in tre sottoparole: la parola dall'inizio a prima del ciclo, il ciclo, e la fine della parola.
Sicuramente l'etichetta del ciclo \`e non vuota.

Possiamo quindi scomporre una parola $w$ come la concatenazione delle tre parole $xyz$, con $\abs{y} > 0$.
Da questa scomposizione possiamo ricavare infinite parole $x y^i z$, ``percorrendo'' pi\`u volte il ciclo.

Con il \emph{pumping lemma} dimostriamo che esiste un linguaggio che non \`e regolare (ossia per cui non esiste un automa a stati finiti che lo riconosca).

Partiamo dal fatto che ogni parola accettata pi\`u lunga del numero degli stati contiene un ciclo. 
Possiamo ricavare infinite parole ripetendo le etichette relative al ciclo.
Questo vale \emph{per ogni} parola accettata pi\`u lunga del numero degli stati, e per qualsiasi numero di volte ripetiamo le etichette del ciclo.

Per dimostrare, grazie al pumping lemma, che un certo linguaggio non \`e regolare, dobbiamo far vedere che non esiste una macchina che lo accetti.
Quindi per ogni possibile numero $k$ di stati troviamo una parola dentro il linguaggio che \`e pi\`u lunga di $k$ ma che non rispetta il pumping lemma.
Quindi per ogni sottoparola non vuota che termina entro il $k$-esimo carattere troviamo un indice $i$ tale che, ripetendo la sottoparola $i$ volte, otteniamo una parola non nel linguaggio.

\`E come se fissassimo di volta in volta il numero di nodi nell'ipotetico automa, e trovassimo una parola che, essendo pi\`u lunga del numero di nodi, dovrebbe contenere un ciclo.
Ma il ciclo non c'\`e, perch\'e ogni sottoparola candidata a esserlo, se iterata un certo numero di volte, ci porta fuori dal linguaggio.
Questo contraddice il pumping lemma: percorrendo il ciclo un numero arbitrario di volte otteniamo sempre parole nel linguaggio.

\begin{theorem}
	Un linguaggio $L$ non \`e regolare se $\forall k > 0$ $\exists z \in L : z \ge k$, e per ogni fattorizzazione di $z = u v w$ con $\abs{v} \ge 1$ e $\abs{uv} \le k$ $\exists i \ge 0$ tale che $u v^i w \notin L$.
\end{theorem}

\section{Automa a pila (\emph{Push-Down Automata})}

I PDA sono automi a stati finiti con una pila.
\`E possibile accedere a una memoria, che funziona come uno stack, e a cui possiamo accedere solo al \emph{top}
Come in tutte queste astrazioni, la pila \`e infinita.


% TODO catch up here

Consideriamo il linguaggio $\lang{G} = \{ a^m b^n \text{ t.c. } m \ge n \ge 0 \}$.
Qual \`e la grammatica per generare questo linguaggio?
\begin{align*}
	S &\to aSb \langor aA \langor \emptystring \\
	A &\to aA \langor \emptystring
\end{align*}

Passiamo al linguaggio $\lang(G) = \{ a^i b^j c^k \text{ t.c. } i = j \lor j = k \text{ con } i,j,k \ge 0 \}$.
La sua grammatica \`e:
\begin{align*}
	S &\to AT \langor RC \langor \emptystring \\
	A &\to aA \langor \emptystring \\
	C &\to Cc \langor \emptystring \\
	T &\to bTc \langor \emptystring \\
	R &\to aRb \langor \emptystring 
\end{align*}
Come sappiamo che questo \`e corretto?
La grammatica $A$ descrive il linguaggio $\lang{A} = \{ a^n : n \ge 0 \}$, cos\`i come la grammatica $C$ descrive il linguaggio $\lang{C} = \{c^n : n \ge 0\}$.
Questi due sono linguaggi regolari.
Le altre due grammatiche descrivono rispettivamente i linguaggi $\lang{T} = \{ b^i c^i : i \ge 0 \}$ e $\lang{R} = \{ a^i b^i : i \ge 0 \}$.
Quindi il nostro linguaggio \`e dato da $\lang{G} = \lang{A}\lang{T} \cup \lang{R}\lang{C}$.

Questo \`e un linguaggio \emph{inerentemente ambiguo}.
Non si pu\`o trovare una grammatica non ambigua equivalente.

Passiamo al linguaggio $\lang{G} = \{ a^i b^j c^k \text{ t.c. } i + j = k \text{ con } i,j,k \ge 0 \}$.
\begin{align*}
	S &\to aSc \langor B \langor \emptystring \\\
	B &\to bBc \langor \emptystring
\end{align*}

Passiamo al linguaggio in cui ogni prefisso di ogni parola ha un numero di $a$ maggiore o uguale al numero delle $b$.
Come troviamo la grammatica di questo linguaggio?
\begin{align*}
	S &\to aSbS \langor aA \langor \emptystring \\
	A &\to aA \langor \emptystring
\end{align*}

Basta esempi.
Ora vediamo che i linguaggi liberi dal contesto corrispondono ai linguaggi riconosciuti dagli automi a pila.

Partendo da una grammatica, costruiamo un PDA che lavora non deterministicamente in tempo lineare per riconoscere la parola di input.

Sia $G = (T, V, S, P)$ una grammatica libera dal contesto.

1. Impila il simbolo di pila vuota e poi la prima variabile della CFG
2. Se nella pila c'\`e una variabile A, scegli non deterministicamente una regola con a sinistra A e in cima alla pila sostituisci la parte destra della regola.
3. ?
4. ?

Si pu\`o rimpiazzare una grammatica ambigua con una grammatica non ambigua equivalente.
Una grammatica \`e ambigua se ha pi\`u alberi di derivazione per uno stesso input.

Il metodo che usiamo \`e quello di imporre una precedenza e l'associativit\`a a sinistra.
\begin{verbatim}
E -> E + T | T 
T -> T * F | F
F -> (E) | id
\end{verbatim}
Abbiamo tolto l'ambiguit\`a a questa grammatica:
\begin{verbatim}
E -> E + E
E -> E * E
E -> (E)
E -> id
\end{verbatim}
Date le grammatiche $G_1$ e $G_2$, vediamo che i linguaggi generati sono chiusi rispetto alle solite operazioni.
\begin{align*}
	G_1 \cup G_2 & : S \to S_1 \langor S_2 \\
	G_1 G_2 & : S \to S_1 S_2 \\
	G_1^{\star} & : S \to S S_1 \langor \emptystring
\end{align*}
I linguaggi liberi dal contesto non sono chiusi rispetto all'intersezione.
Consideriamo i linguaggi $a^n b^n c^m$ e $a^m b^n c^n$.
La loro intersezione \`e il linguaggio $a^n b^n c^n$ e non \`e un linguaggio libero dal contesto.
Ci\`o implica che i linguaggi context-free non sono chiusi neanche rispetto al complemento, poich\'e l'intersezione si pu\`o esprimere in termini di unione e complemento.

Il linguaggio $\{ ww : w \in \{0,1\}^{\star}\}$ non \`e context-free, ma il suo complemento s\`i.

Il suo complemento \`e l'unione del linguaggio delle parole di lunghezza dispari (banale) e del linguaggio delle parole di lunghezza pari le cui due met\`a non sono uguali.
\[
	L_1 = \{ xy : x,y \in \{0,1\}^{\star} \land \abs{x} = \abs{y} \land x \neq y \}
\]
Possiamo considerare solo parole del tipo $xax'za'z'$, con $x,z,x',z' \in \{0,1\}^{\star}$, e $a \neq a'$ elemento di $\{0,1\}$, e con $\abs{x} = \abs{x'}$ e $\abs{z} = \abs{z'}$.
Notare che le stringhe $xax'$ e $za'z'$ hanno lunghezza dispari.
Quindi $L_1$ \`e l'unione dei due linguaggi dati uno dalla concatenazione di $\{ x0z : \abs{x} = \abs{z} \}$ e $\{ x1z : \abs{x} = \abs{z} \}$ e l'altro dalla concatenazione di $\{ x1z : \abs{x} = \abs{z} \}$ e di $\{ x0z : \abs{x} = \abs{z} \}$.

Scrivendolo come grammatica, abbiamo che $G_1$ \`e:
\begin{align*}
G_1 :& S \to S_0 S_1 \langor S_1 S_0 \\
& S_0 \to 0 \langor 0 S_0 0 \langor 0 S_0 1 \langor 1 S_0 0 \langor 1 S_0 1 \\
& S\_1 \to 1 \langor 0 S\_1 0 \langor 0 S\_1 1 \langor 1 S\_1 0 \langor 1 S\_1 1 
\end{align*}

\section{\emph{Linear bound automata}}

Sono una versione limitata delle macchine di Turing.

