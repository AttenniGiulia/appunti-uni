
\chapter{Calcolabilit\`a}

\section{Linguaggi indecidibili}

Vediamo ora che esistono problemi indecidibili.
Lo facciamo con un confronto fra le cardinalit\`a dell'insieme degli algoritmi e l'insieme dei problemi.
Ogni problema \`e associato a un linguaggio, quindi la cardinalit\`a dell'insieme dei problemi \`e uguale alla cardinalit\`a dell'insieme dei linguaggi. 
Un algoritmo, invece, \`e una stringa finita su un alfabeto.
Intuitivamente l'insieme degli algoritmi \`e un'infinit\`a numerabile, mentre l'insieme dei problemi \`e un'infinit\`a non numerabile.

Un insieme \`e numerabile se pu\`o essere messo in corrispondenza biunivoca con l'insieme dei naturali.

Vediamo la non-numerabilit\`a dell'insieme dei linguaggi.
Ordiniamo quasi-lessigograficamente le parole su un certo alfabeto, ordinandole prima per lunghezza della parola e poi lessicograficamente se le parole hanno lunghezza uguale.
Dopo aver ordinato tutte le parole su un alfabeto $\Sigma$, rappresentiamo un linguaggio con una sequenza binaria infinita.
Alla $i$-esima posizione inseriamo un 1 se la $i$-esima parola (secondo il nostro ordinamento) \`e presente nel linguaggio, 0 altrimenti.
Supponendo che l'insieme dei linguaggi sia numerabile, e usando il metodo della diagonale, vediamo che l'insieme dei linguaggi non \`e numerabile.

Un insieme, nella teoria ingenua degli insiemi, \`e una collezione di oggetti che soddisfano una certa propriet\`a, e di cui (della collezione) ignoriamo ordine o molteplicit\`a.
L'antinomia di Russel si basa sull'autoreferenzialit\`a: insiemi che sono elementi di s\'e stessi.

\begin{esercizio}
	Dimostrare che l'insieme dei linguaggi regolari \`e numerabile e che quello dei linguaggi non regolari non \`e numerabile.
\end{esercizio}

I linguaggi regolari sono anche \emph{enumerabili}, ossia esiste una funzione $f: \{0,1\}^{\star} \to \naturals$ calcolabile, che associa a ogni parola binaria un intero.
Questo vale anche per ogni altro alfabeto, non solo su quello binario.

\begin{esercizio}
	Sia $B$ l'insieme delle stringhe binarie infinite, e sia $C$ il sottoinsieme di $B$ contenente solo le stringhe con un numero di 1 minore o uguale a 25.
	Dimostrare che $C$ \`e numerabile (e enumerabile).
\end{esercizio}

