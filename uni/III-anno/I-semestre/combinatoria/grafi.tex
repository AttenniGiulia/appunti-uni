
Abbiamo una endofunzone $f : [n] \to [n]$.
La rappresentiamo con un grafo diretto G_f in cui V(G) = [n] e con archi (a, f(a)) \forall a \in [n].
Tutti i cicli sono ``ben orientati''.
Tutti gli archi con un vertice in comune sono ``consecutivi''.

Consideriamo un cammino in cui gli archi sono orientati consecutivamente, e che non pu\`o essere esteso ulteriormente.
O questo cammino contiene tutti gli archi del ciclo, e quindi il ciclo \`e ben orientato, o non li comprende.
Se il cammino fosse prolungabile, avremmo un vertice con due archi che partono dallo stesso nodo, ma ogni nodo del grafo ha grado uscente 1.
Nel grafo non ci sono cicli non orientati bene.
Ogni ciclo \`e nei ricorrenti.

Abbiamo due insiemi.
L'insieme \mathcal{V}(n) dei vertebrati su K_n, e l'insieme \mathcal{F}(n) delle endofunzioni su [n].
Vogliamo articolare i due insiemi allo stesso modo, per mettere in biezione vertebrati e endofunzioni e dimostrare che i due insiemi hanno 
la stessa cardinalit\`a.

Un vertebrato generico \`e una terna (C, \overset{L}{<}, \{ T_x \}_{x \in C} ) dove C \subseteq [n], \overset{L}{<} \`e un ordinamento lineare di C, e T_x \`e un albero radicato in x \in C e i vertici V(T_x) degli alberi partizionano [n], ossia lo ricoprono senza avere vertici in comune.

Una endofunzione \`e un insieme (C, partizione di C in cicli ben orientati, \{ T_x \}_{x \in C}) dove C \subseteq [n] sono i vertici ricorrenti della funzione, poi abbiamo questa partizione, e poi gli alberi T_x radicati in x \in C che partizionano [n].

Se C e \{ T_x \} coincidono, cosa possibilissima, le due terne possono essere in corrispondenza.
Dobbiamo vedere cosa \`e l'elemento centrale.

Abbiamo \abs{C}! ordinamenti lineari, nei vertebrati.
Per la biezione, ci basterebbe vedere che il numero di partizioni \`e lo stesso.

Nelle endofunzioni, i cicli in C rappresentano l'azione della endofunzione sui ricorrenti.
Anche questa \`e una endofunzione.
Ma non una qualsiasi.
La endofunzione ristretta ai ricorrenti \`e \emph{suriettiva}: ogni vertice ha grado entrante 1, quindi \`e immagine di qualche altro vertice (anche di s\'e stesso).
Una endofunzione suriettiva \`e una biezione (quindi invertibile).
Il numero di endofunzioni biettive su un insieme di cardinalit\`a \abs{C}, \`e \abs{C}!.

So, we are done.
\qed

Ora vogliamo parlare di endofunzioni suriettive.
Sono endofunzioni che non hanno elementi \emph{transienti}, come li abbiamo chiamati in precedenza.
Una generica endofunzione suriettiva \pi : [n] \to [n] la chiamiamo \emph{permutazione}.
(Questa \`e una definizione)

Possiamo rappresentarla naturalmente con una tabella con nella prima riga gli argomenti e nella seconda riga le immagini.

\begin{tabular}{*{5}{c}}
1 & 2 & 3 & \dots & n \\
\pi (1) & \pi (2) & \pi (3) & \dots & \pi (n)
\end{tabular}

Non abbiamo bisogno, in realt\`a, della prima riga.
Ci basta la seconda, e quindi ci basta un ordinamento lineare.
Sono due cose ben differenti,
endofunzioni suriettive e ordinamenti lineari, ma guarda un po' sono uguali.
Ora per\`o pensiamo al fatto che una endofunzione suriettiva \`e anche una partizione del suo insieme di definizione.

Sappiamo che le endofunzioni possono essere iterate.
Per comporre due funzioni $f$ e $g$, deve essere che \mathcal{R}(f) \subseteq \mathcal{D}(g), ossia l'immagine
di f deve essere un sottoinsieme del dominio di g.

Questo vale con le endofunzioni: dominio e immagine coincidono.
Le permutazioni di [n] possono essere applicate a [n] in maniera consecutiva.

Date le permutazioni \pi : [n] \to [n] e \rho : [n] \to [n], la permutazione \rho (\pi (a)) \`e ben definita \forall a \in [n].
Questa nuova permutazione la indichiamo con \pi \rho, e intendiamo dire che \pi \rho (a) = \rho (\pi (a)).

Sull'insieme delle n! permutazioni di un insieme di [n] elementi, possiamo considerare l'applicazione consecutiva come un \emph{prodotto}.
Diciamo prodotto e non addizione, per vari motivi.
L'addizione \`e considerata commutativa, il prodotto pu\`o non esserlo.

Questa operazione non \`e commutativa.
Vediamolo.

n = 3
\pi (1,2) (2,1) (3,3)
\rho (1,3) (3,1) (2,2)

\pi \rho \`e l'applicazione successiva di prima \pi e dopo \rho.
\pi \rho (1) = 2, \pi \rho (2) = 3, \pi \rho (3) = 1.
\rho \pi \`e l'applicazione successiva di prima \rho e dopo \pi.
\rho \pi (1) = 3, \rho \pi (2) = 1, \rho \pi (3) = 2.

Fatto.

Ora, di solito, con i prodotti abbiamo un elemento neutro.
La permutazione identit\`a \`e questo elemento neutro.

L'applicazione consecutiva \`e un'operazione associativa.


Guarda caso \`e anche un'operazione invertibile.
Rappresentando la permutazione come cicli, per trovare la permutazione inversa basta invertire l'orientamento degli archi.
Questo insieme \`e un gruppo: ha un'operazione associativa, invertibile e con elemento neutro.


Una permutazione \`e un automorfismo di un grafo: trasforma un grafo in s\'e stesso.
(???)
Una permutazione che porta un vertice di un grado in un vertice di un altro grado non \`e un automorfismo.
Gli automorfismi di una struttura formano un gruppo.

Gli automorfismi di una struttura mi dicono il grado di simmetria di una struttura.
La simmetria si definisce (in generale e in astratto) con il gruppo di automorfismi della struttura.

Gli insiemi dei vertici dei cicli sono disgiunti (e ben orientati).
Possiamo parlare di decomposizione ciclica di una permutazione.

Introduciamo la notazione per i cicli.
Indichiamo con (c_1 c_2 c_3 \dots c_t) il ciclo con archi (c_1, c_2), (c_2, c_3), \dots, (c_t, c_1).

La decomposizione ciclica di una permutazione la scriviamo mettendo i cicli uno di fianco all'altro.
I cicli non sono ordinati.
I cappi non li rappresentiamo.

Sappiamo che (1,3) (1,2) \neq (1,2) (1,3).
Ma nella decomposizione ciclica l'ordine non \`e importante.
Il prodotto di cicli disgiunti \`e commutativo!

Una permutazione si pu\`o anche vedere come prodotto di trasposizioni.
Una trasposizione \tau \`e una permutazione (ab), ossia \tau(a) = b e \tau(b) = a.

\begin{oss}
	La permutazione ciclica $a_1 a_2 \dots a_t$ \`e una serie di trasposizioni: $(a_1 a_2) (a_1 a_3) \dots (a_1 a_t)$.
\end{oss}

Attenzione: inserendo un elemento in questa serie di trasposizioni, si passa da sinistra verso destra!
Se scriviamo la permutazione come trasposizioni del tipo $(a_1, a_2) (a_2, a_3) \dots (a_{t-1} a_{t})$, l'``argomento'' della trasposizione passa da destra verso sinistra.

C'\`e correlazione fra permutazioni e algoritmi di sorting.
Un insieme disordinato pu\`o essere visto come una permutazione, e trovare come ordinare l'insieme vuol dire trovare la permutazione inversa.

Cambiando l'ordine degli elementi nella permutazione, non sembre otteniamo delle trasposizioni in ordine differente.
Invece, cambiando l'ordine delle trasposizioni, cambia la permutazione.

Perch\'e per ordinamenti diversi otteniamo una stessa rappresentazione ciclica?
Basta scegliere un elemento differente da cui partire.

Per cui, possiamo scrivere la permutazione come $(a_1 a_2 \dots a_t)$ o come $(a_2 a_3 \dots a_t a_1)$ senza che l'ordine delle trasposizioni cambi.
Ma possiamo anche scrivere la stessa permutazione come un prodotto di \emph{trasposizioni differenti}, ad esempio come $(a_2 a_3) (a_2 a_4) \dots (a_2 a_t) (a_2 a_1)$.

\begin{prop}
	Ogni permutazione \`e un prodotto di trasposizioni.
\end{prop}

Questo perch\'e ogni permutazione \`e un prodotto di permutazioni cicliche, e ogni permutazione ciclica \`e un prodotto di trasposizioni.
I componenti di questo prodotto \emph{non} sono univocamente determinati.

Now, attenzione: quando scriviamo una permutazione come prodotto di trasposizioni, il numero di trasposizioni e le trasposizioni possono essere differenti.
Quel che non cambia \`e la parit\`a del numero di permutazioni: o sono sempre un numero pari o sono sempre un numero dispari.

Vediamo ora che il gruppo delle permutazioni ha un sottogruppo non banale molto grande.

\begin{defn}[Elementi in inversione]
	Per la permutazione $\rho : [n] \to [n]$ gli elementi $a \in [n]$ e $b \in [n]$ sono ``in inversione'' se $(a - b) \cdot (\rho(a) - \rho(b)) < 0$.
	Stiamo quindi dicendo che $a$ e $b$ sono diversi, e che $a < b \implies \rho(a) > \rho(b)$ o che $a > b \implies \rho(a) < \rho(b)$.
\end{defn}

Una permutazione \emph{mette in inversione} coppie di elementi.

\begin{defn}[Parit\`a di una permutazione]
	La parit\`a di $\rho$ \`e la parit\`a del numero di coppie che $\rho$ mette in inversione.
\end{defn}

La permutazione identit\`a mette in inversione 0 elementi.
In questo caso consideriamo 0 un numero pari, o comunque consideriamo la permutazione identit\`a \`e pari.

Una trasposizione $(c,d)$, \`e una permutazione di che parit\`a?
Guardiamo la tabella della permutazione, supponendo $c < d$:

\begin{tabular}{*{10}{c}}
	1 & 2 & \dots & c & \dots & x & \dots & d & \dots & n \\
	1 & 2 & \dots & d & \dots & x & \dots & c & \dots & n 
\end{tabular}

$c$ e $d$ sono in inversione, certamente, ma anche $x$ e $c$, o $d$ e $x$, e via dicendo.

Siano $t$ gli elementi fra $c$ e $d$, ciascuno di questi $t$ elementi \`e sia in inversione con $c$ che con $d$.
Quindi abbiamo $2 \, t$ inversioni per gli elementi fra $c$ e $d$, pi\`u un'inversione per $c$ e $d$ stessi.
Quindi una trasposizione \`e una permutazione di ordine dispari, avendo $2 \, t + 1$ inversioni.

\begin{prop}
	Per un'arbitraria permutazione $\rho$ e la trasposizione $(a,b)$, la parit\`a di $\rho$, che indichiamo con $\parity{\rho}$, \`e diversa dalla parit\`a di $\rho \cdot (a,b)$, ossia:
	\[
		\parity{\rho} \neq \parity{\rho \cdot (a,b)}
	\]
\end{prop}
Abbiamo appena visto che la parit\`a dell'identit\`a \`e pari, e la parit\`a di una trasposizione \`e dispari.
Questo implicher\`a che la parit\`a della permutazione ciclica $(a_1 a_2 \dots a_t)$ \`e opposta alla parit\`a di $t$.
Infatti possiamo scrivere la permutazione come $(a_1 a_2) (a_1 a_3) \dots (a_1 a_t)$, che sono $t-1$ trasposizioni.
Sappiamo che $(a_1 a_2)$ ha parit\`a dispari, e che $(a_1 a_2) (a_1 a_3)$ ha parit\`a differente, ossia pari.
E via dicendo, cambiamo parit\`a $t-1$ volte.

Possiamo vedere anche che il prodotto di permutazioni pari \`e ancora una permutazione pari.
Quindi l'insieme delle permutazioni pari \`e chiuso rispetto al prodotto.

Tra le permutazioni di $[n]$, il numero delle permutazioni pari \`e $\frac{n!}{2}$.
Anche questa \`e una conseguenza della proposizione sopra.
Serve assumere $n > 1$, perch\'e in questo caso abbiamo una sola permutazione, pari, e $\frac{1}{2}$ non \`e un naturale.

Come mai vale questo? 
Prendiamo una permutazione pari $\rho$, e moltiplichiamola per la trasposizione $(1,2)$.
Possiamo mettere in biezione le permutazioni $\rho$ e $\rho \cdot (1,2)$.
Riapplicando il prodotto per $(1,2)$ riotteniamo $\rho$.
Abbiamo quindi una biezione, quindi entrambi gli insiemi hanno cardinalit\`a $\frac{n!}{2}$.

Piccola nota: il sottogruppo delle permutazioni pari \`e il sottogruppo pi\`u grande del gruppo delle permutazioni.

Passiamo alla dimostrazione.

Senza vincolare la generalit\`a, supponiamo $\rho^{-1}(a) < \rho^{-1}(b)$, e scriviamo la permutazione come segue.

\begin{tabular}{*{n}{c}}
	1 & 2 & \dots & $\rho^{-1}(a)$ & \dots & $\rho^{-1}(b)$ & \dots & n \\
	\rho(1) & \rho(2) & \dots & a & \dots & b & \dots & \rho(n)
\end{tabular}

Quali coppie in inversione in $\rho$ non sono in inversione in $\rho \cdot (a,b)$?
Scrivendo la permutazione $\rho \cdot (a,b)$, abbiamo questo:

\begin{tabular}{*{n}{c}}
	1 & 2 & \dots & $\rho^{-1}(a)$ & \dots & $\rho^{-1}(b)$ & \dots & n \\
	\rho(1) & \rho(2) & \dots & b & \dots & a & \dots & \rho(n)
\end{tabular}

Possiamo dire in generale che ci sono coppie di colonne che ``cambiano situazione'', e coppie che non la cambiano.
Non sappiamo chi sono.
Identifichiamo intanto le colonne con la loro posizione.
Quali sono le colonne in \rho che sono in inversione e che nel prodotto non lo sono?
Quali colonne da non invertite diventano invertite?

Una coppia di colonne cambia situazione se almeno una \`e tra le colonne $\begin{smallpmatrix}\rho^{-1}(a) \\ a \end{smallpmatrix}$ e $\begin{smallpmatrix}\rho^{-1}(b) \\ b \end{smallpmatrix}$.
Le colonne intermedie saranno colonne del tipo $\begin{smallpmatrix} x \\ \rho(x) \end{smallpmatrix}$ con $\rho^{-1} (a) < x < \rho^{-1} (b)$.
Questa colonna pu\`o essere in inversione con entrambe, con nessuna, o con una ma non con l'altra.
Abbiamo quindi 4 casi distinti.

Sia $i$ il numero delle coppie in inversione in $\rho$ tali da non essere una coppia dove almeno una coppia \`e ``di bordo'' e l'altra \`e intermedia o di bordo.
Sia $R(a,b)$ il numero delle colonne intermedie in inversione con le due colonne di bordo.
Sia $N(a,b)$ il numero delle colonne intermedie non in inversione con nessuna colonna di bordo.
Sia $R(a)$ il numero delle colonne intermedie in inversione con solo la colonna di $a$.
Sia $R(b)$ il numero delle colonne intermedie in inversione con solo la colonna di $b$.

Il totale delle colonne in inversione \`e dato da:
\[
	i + 2 \, R(a,b) + R(a) + R(b) + \chi (a,b)
\]
Dove $\chi (a,b)$ \`e 1 se le colonne di $a$ e $b$ sono in inversione in $\rho$, 0 altrimenti.

Come cambia la situazione?
Le $i$ colonne erano in inversione e cos\`i rimangono. Quelle in inversione con le due colonne di bordo non ci saranno pi\`u.
Quelle in inversione con la colonna di $a$ saranno in inversione con la colonna di $b$, e viceversa con la colonna di $b$.
Quelle non in inversione con le due colonne di bordo ora saranno in inversione con entrambe.

Otteniamo quindi:
\[
	i + R(a) + R(b) + 2 \, N(a,b) + \chi'(a,b)
\]
Ponendo $\chi'(a,b) = 1$ se $\chi (a,b) = 0$ o $\chi'(a,b) = 0$ se $\chi(a,b) = 1$.
La differenza fra prima e dopo \`e:
\[
	2 \, R(a,b) + \chi (a,b) - 2 \, N(a,b) + \chi'(a,b)
\]
Che \`e un numero dispari.
Quindi se il numero di inversioni era pari in $\rho$, ora \`e dispari, e viceversa se era dispari ora \`e pari.

Abbiamo visto che ogni oggetto matematico pu\`o avere diverse rappresentazioni.
Una permutazione pu\`o essere scritta come prodotto di trasposizioni, e queste possono essere completamente diverse.
C'\`e qualcosa che non cambia scrivendo una permutazione come prodotto di trasposizioni?
S\`i: la parit\`a del numero di trasposizioni.

Abbiamo detto che data una permutazione $\rho$, alcune coppie di elementi sono in ``inversione''.
La parit\`a del numero di coppie in inversione \`e la parit\`a della permutazione, e questa \`e una definizione.

Gli automorfismi descrivono quello che non cambia in un oggetto matematico. (???)
La parit\`a \`e un invariante relativamente semplice: ha solo due valori, anche se la definizione \`e un po' contorta.
Avendo solo due valori, divide le permutazioni in due classi.

Abbiamo parlato di una funzione ``Parit\`a'':
\[
	\par{\rho} = 
	\begin{cases}
	-1 \\
	1
	\end{cases}
\]
Vale che $\par{\rho \sigma} = \par{\rho} \par{\sigma}$.
\`E una conseguenza di quello che abbiamo dimostrato, ossia che data una permutazione $\rho : [n] \to [n]$ qualsiasi e una trasposizione $(a,b)$ con $a,b \in [n]$, vale $\par{\rho (a,b)} \neq \par{\rho}$.

Per dimostrarlo abbiamo contato il numero di inversioni in $\rho$ prima e dopo averlo moltiplicato per $(a,b)$.
Rappresentando la permutazione con una tabella, chiamiamo ``colonne di bordo'' le colonne di $a$ e di $b$, e le colonne fra queste due le chiamiamo ``colonne intermedie''.

Sia $i$ il numero delle coppie di colonne in inversione in $\rho$ in cui al pi\`u una \`e una colonna di bordo e l'altra \`e intermedia.
Queste colonne saranno presenti anche in $\rho (a,b)$.
Poi ci sono alcune colonne che cambiano situazione: sono le coppie di colonne in cui una \`e una colonna di bordo e una \`e una colonna intermedia.
Chiamiamo poi $R(a,b)$ il numero delle colonne intermedie in inversione con tutte e due le colonne di bordo, e $N(a,b)$ il numero di colonne che non sono in inversione con nessuna delle colonne di bordo.
Poi, chiamiamo $V(a,b) = R(a) + R(b)$ il numero delle colonne in inversione con esattamente una fra le colonne di bordo.
Infine, $\chi(a,b)$ \`e definito come $1$ se le due colonne di bordo sono in inversione in $\rho$, $0$ altrimenti.

Non conosciamo nessuno di questi numeri, ci interessa solo sapere come cambiano.
Il numero di colonne in inversione in $\rho$ \`e:
\[
i + 2 \, R(a,b) + V(a,b) + \chi(a,b)
\]
$R(a,b)$ sono le coppie in inversione sia con $a$ che con $b$, quindi dobbiamo ``contarle'' due volte.
Le colonne che non sono in inversione con $a$ o con $b$ non dobbiamo contarle.

Il numero delle coppie in inversione in $\rho (a,b)$ \`e, invece:
\[
i + 2 \, N(a,b) + V(a,b) + \chi'(a,b)
\]
Le uniche colonne che possono cambiare situazione, e che anzi necessariamente la cambiano, saranno solo le coppie di colonne in cui una colonna \`e di bordo e l'altra \`e o una colonna di bordo o \`e una colonna intermedia.
Le colonne in inversione e con $a$ e con $b$ ora non saranno in inversione con nessuna delle due.
Al loro posto avremo in inversione le colonne che in $\rho$ non erano in inversione n\'e con $a$ n\'e con $b$.
Quelle che erano in inversione con esattamente con una colonna di bordo, saranno ora in inversione ancora con esattamente una colonna di bordo, ma con l'altra.
Infine, possiamo dire che $\chi'(a,b) + \chi(a,b) = 1$.
Se $a,b$ non erano in inversione, ora lo saranno, e viceversa se erano in inversione ora non lo saranno. 

Ora affermiamo che la parit\`a del primo numero \`e diversa dalla parit\`a del secondo numero, ossia che la differenza fra i due numeri \`e un numero dispari.
\[
i + 2 \, N(a,b) + V(a,b) + 1 - \chi(a,b) - \left[ i + 2 \, R(a,b) + V(a,b) + \chi (a,b) \right]
\]
Questo numero \`e dispari.
Infatti vale:
\[
2 \, N(a,b) - 2 \, R(a,b) + 1 - 2 \, \chi(a,b)
\]
che \`e evidentemente dispari.

\subsection{Numero cromatico}

Parliamo di numero cromatico in relazione alla colorazione dei vertici di un grafo.
Si pu\`o parlare di numero cromatico anche in relazione al colore degli archi di un grafo.
Il problema \`e nato prima della teoria dei grafi, nel contesto di colorazione delle carte geografiche.

Appel-Haken hanno dimostrato nel '79, alla University of Illinois at Urbana Champwtf, che bastano 4 colori per colorare ogni carta geografica.

I due hanno ridotto il numero di grafi da analizzare da infiniti che erano, a meno di duemila.
Li hanno fatti analizzare tutti a un computer e hanno ``dimostrato'' che l'ipotesi \`e vera.
Ma negli anni '90 \`e stato dimostrato che questa dimostrazione era incompleta.
Hanno ridotto nuovamente il numero di grafi da analizzare a circa 600, e controllato i casi col computer.

\`E un problema centrale nell'ottimizzazione combinatoria, nella quale dobbiamo colorare un grafo col minor numero possibile di colori.

Consideriamo una carta geografia.
A ogni regione sulla carta associamo un vertice di un grafo.
Vogliamo associare a ogni regione un colore, e lo facciamo con una funzione $f: V(G) \to C$ che a ogni vertice del grafo assegna un elemento di un insieme astratto $C$, che chiamiamo colore.
Nel problema della carta geografica dobbiamo avere che regioni adiacenti hanno colori differenti.
Nei grafi, $f$ \`e una colorazione se vertici adiacenti hanno colori diversi.

\begin{defn}[Colorazione di un grafo]
Una colorazione di un grafo \`e una funzione $f : V(G) \to C$ con $C$ insieme astratto se per ogni coppia di vertici $a,b$ abbiamo che $\{a,b\} \in E(G) \implies f(a) \neq f(b)$.
\end{defn}
Ogni grafo ha una colorazione: qualsiasi funzione iniettiva (con valori tutti diversi) soddisfa il vincolo.

Il problema \`e capire quanto pu\`o essere piccolo $C$, ossia il minimo $\abs{C}$ per cui esiste una colorazione $f : V(G) \to C$.
La cardinalit\`a $\abs{C}$ \`e il numero cromatico del grafo.

L'ottimizzazione combinatoria \`e anche chiamata ricerca operativa, o \emph{operations research}.
Operations = operazioni militari, la ricerca operativa \`e nata per ottimizzare le risorse.
