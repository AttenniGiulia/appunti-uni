
Abbiamo una endofunzone $f : [n] \to [n]$. La rappresentiamo con un grafo diretto G_f in cui V(G) = [n] e con archi (a, f(a)) \forall a \in [n].
Tutti i cicli sono ``ben orientati''. Tutti gli archi con un vertice in comune sono ``consecutivi''.

Consideriamo un cammino in cui gli archi sono orientati consecutivamente, e che non pu\`o essere esteso ulteriormente.
O questo cammino contiene tutti gli archi del ciclo, e quindi il ciclo \`e ben orientato, o non li comprende.
Se il cammino fosse prolungabile, avremmo un vertice con due archi che partono dallo stesso nodo, ma ogni nodo del grafo ha grado uscente 1.
Nel grafo non ci sono cicli non orientati bene. Ogni ciclo \`e nei ricorrenti.

Abbiamo due insiemi. L'insieme \mathcal{V}(n) dei vertebrati su K_n, e l'insieme \mathcal{F}(n) delle endofunzioni su [n].
Vogliamo articolare i due insiemi allo stesso modo, per mettere in biezione vertebrati e endofunzioni e dimostrare che i due insiemi hanno 
la stessa cardinalit\`a.

Un vertebrato generico \`e una terna (C, \overset{L}{<}, \{ T_x \}_{x \in C} ) dove C \subseteq [n], \overset{L}{<} \`e un ordinamento lineare di C, e T_x \`e un albero radicato in x \in C e i vertici V(T_x) degli alberi partizionano [n], ossia lo ricoprono senza avere vertici in comune.

Una endofunzione \`e un insieme (C, partizione di C in cicli ben orientati, \{ T_x \}_{x \in C}) dove C \subseteq [n] sono i vertici ricorrenti della funzione, poi abbiamo questa partizione, e poi gli alberi T_x radicati in x \in C che partizionano [n].

Se C e \{ T_x \} coincidono, cosa possibilissima, le due terne possono essere in corrispondenza. Dobbiamo vedere cosa \`e l'elemento centrale.

Abbiamo \abs{C}! ordinamenti lineari, nei vertebrati. Per la biezione, ci basterebbe vedere che il numero di partizioni \`e lo stesso.

Nelle endofunzioni, i cicli in C rappresentano l'azione della endofunzione sui ricorrenti. Anche questa \`e una endofunzione. Ma non una qualsiasi.
La endofunzione ristretta ai ricorrenti \`e \emph{suriettiva}: ogni vertice ha grado entrante 1, quindi \`e immagine di qualche altro vertice (anche di s\'e stesso).
Una endofunzione suriettiva \`e una biezione (quindi invertibile). Il numero di endofunzioni biettive su un insieme di cardinalit\`a \abs{C}, \`e \abs{C}!.

So, we are done. \qed

Ora vogliamo parlare di endofunzioni suriettive. Sono endofunzioni che non hanno elementi \emph{transienti}, come li abbiamo chiamati in precedenza.
Una generica endofunzione suriettiva \pi : [n] \to [n] la chiamiamo \emph{permutazione}. (Questa \`e una definizione)

Possiamo rappresentarla naturalmente con una tabella con nella prima riga gli argomenti e nella seconda riga le immagini.

\begin{tabular}{*{5}{c}}
1 & 2 & 3 & \dots & n \\
\pi (1) & \pi (2) & \pi (3) & \dots & \pi (n)
\end{tabular}

Non abbiamo bisogno, in realt\`a, della prima riga. Ci basta la seconda, e quindi ci basta un ordinamento lineare. Sono due cose ben differenti,
endofunzioni suriettive e ordinamenti lineari, ma guarda un po' sono uguali.
Ora per\`o pensiamo al fatto che una endofunzione suriettiva \`e anche una partizione del suo insieme di definizione.

Sappiamo che le endofunzioni possono essere iterate. Per comporre due funzioni $f$ e $g$, deve essere che \mathcal{R}(f) \subseteq \mathcal{D}(g), ossia l'immagine
di f deve essere un sottoinsieme del dominio di g.

Questo vale con le endofunzioni: dominio e immagine coincidono. Le permutazioni di [n] possono essere applicate a [n] in maniera consecutiva. 
Date le permutazioni \pi : [n] \to [n] e \rho : [n] \to [n], la permutazione \rho (\pi (a)) \`e ben definita \forall a \in [n].
Questa nuova permutazione la indichiamo con \pi \rho, e intendiamo dire che \pi \rho (a) = \rho (\pi (a)).

Sull'insieme delle n! permutazioni di un insieme di [n] elementi, possiamo considerare l'applicazione consecutiva come un \emph{prodotto}.
Diciamo prodotto e non addizione, per vari motivi. L'addizione \`e considerata commutativa, il prodotto pu\`o non esserlo.

Questa operazione non \`e commutativa. Vediamolo.

n = 3
\pi (1,2) (2,1) (3,3)
\rho (1,3) (3,1) (2,2)

\pi \rho \`e l'applicazione successiva di prima \pi e dopo \rho. \pi \rho (1) = 2, \pi \rho (2) = 3, \pi \rho (3) = 1.
\rho \pi \`e l'applicazione successiva di prima \rho e dopo \pi. \rho \pi (1) = 3, \rho \pi (2) = 1, \rho \pi (3) = 2.

Fatto.

Ora, di solito, con i prodotti abbiamo un elemento neutro. La permutazione identit\`a \`e questo elemento neutro. 
L'applicazione consecutiva \`e un'operazione associativa. 

Guarda caso \`e anche un'operazione invertibile. Rappresentando la permutazione come cicli, per trovare la permutazione inversa basta invertire l'orientamento degli archi.
Questo insieme \`e un gruppo: ha un'operazione associativa, invertibile e con elemento neutro. 

Una permutazione \`e un automorfismo di un grafo: trasforma un grafo in s\'e stesso. (???)
Una permutazione che porta un vertice di un grado in un vertice di un altro grado non \`e un automorfismo.
Gli automorfismi di una struttura formano un gruppo.

Gli automorfismi di una struttura mi dicono il grado di simmetria di una struttura.
La simmetria si definisce (in generale e in astratto) con il gruppo di automorfismi della struttura.

Gli insiemi dei vertici dei cicli sono disgiunti (e ben orientati). Possiamo parlare di decomposizione ciclica di una permutazione.

Introduciamo la notazione per i cicli. Indichiamo con (c_1 c_2 c_3 \dots c_t) il ciclo con archi (c_1, c_2), (c_2, c_3), \dots, (c_t, c_1). 
La decomposizione ciclica di una permutazione la scriviamo mettendo i cicli uno di fianco all'altro.
I cicli non sono ordinati. I cappi non li rappresentiamo.

Sappiamo che (1,3) (1,2) \neq (1,2) (1,3). Ma nella decomposizione ciclica l'ordine non \`e importante.
Il prodotto di cicli disgiunti \`e commutativo!



























