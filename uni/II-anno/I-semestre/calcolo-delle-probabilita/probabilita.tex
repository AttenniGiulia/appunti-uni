\documentclass[12pt,a4paper,draft]{article}
% caratteri grandi che non ci vedo!

\usepackage[italian]{babel}
% per date e ToC in italiano

\usepackage{amsmath}        % matematica
\usepackage{amsthm}         % teoremi
\usepackage{amssymb}
\usepackage{amsfonts}       % font matematicosi
\usepackage{centernot}      % per semplificare
\usepackage{pgfplots}       % per i grafici
\pgfplotsset{compat=1.5}    % consigliata compatibilita'
                            % con la versione 1.5

\usepackage{fullpage}       % troppo margine normalmente
\usepackage{parskip}        % preferisco spazio fra i paragrafi
                            % all'indentazione sulla prima riga

% per le liste non numeriche preferisco un bel dash
\renewcommand{\labelitemi}{$-$}

\theoremstyle{plain}% default
\newtheorem{theorem}{Teorema}[section]
\newtheorem{lem}[theorem]{Lemma}
\newtheorem{prop}[theorem]{Proposizione}
\newtheorem*{cor}{Corollario}
\newtheorem*{KL}{Lemma di Klein}

\theoremstyle{definition}
\newtheorem{defn}{Definizione}[section]
\newtheorem{conj}{Congettura}[section]
\newtheorem{exmp}{Esempio}[section]
\newtheorem{esercizio}{Esercizio}[section]

\theoremstyle{remark}
\newtheorem*{oss}{Osservazione}
\newtheorem*{comm}{Commento}
\newtheorem*{note}{Nota}
\newtheorem*{fact}{Fatto}
\newtheorem{caso}{Caso}

\newcommand{\compl}{\mathsf{c}}
\newcommand{\abs}[1]{\left\lvert{#1}\right\rvert}
\newcommand{\norm}[1]{\left\lVert{#1}\right\rVert}
\newcommand{\reals}{\mathbb{R}}
\newcommand{\naturals}{\mathbb{N}}
\newcommand{\var}{\operatorname{Var}} % varianza 

\begin{document}

\title{Calcolo delle probabilit\`a}
\author{Michele Laurenti}
\date{\today}

\maketitle

\newpage

\tableofcontents
\newpage

\section{Calcolo combinatorio}

\subsection{Principio fondamentale del calcolo combinatorio}

\section{Principio fondamentale del calcolo combinatorio}

\begin{theorem}
Consideriamo $n$ esperimenti (un meta-esperimento di $n$ esperimenti, come il lancio di due dadi = 2 esperimenti). Supponiamo che il primo esperimento abbia $m_1$ esiti possibili. Qualunque sia l'esito del primo esperimento il secondo esperimento ha sempre $m_2$ esiti possibili. Qualunque sia il risultato (coppia di esiti) dei primi due esperimenti, il terzo esperimento ha $m_3$ esiti possibili, e cos\`i via. Gli esiti possibili di ogni esperimento sono sempre gli stessi, qualunque sia il risultato degli esperimenti precedenti.

Allora il meta-esperimento ha $ m_1 \times m_2 \times m_3 \times \dots \times m_n $ stringhe possibili nella forma $( a_1, a_2, a_3, \dots a_n)$, dove $a_1$ \`e l'esito dell'esperimento $i$-esimo $\forall \ i \in [1, n]$.
\end{theorem}

Dati due insiemi finiti $A$ e $B$, voglio determinare la cardinalit\`a delle funzioni da $A$ a $B$.
\begin{gather*}
\sharp \left\{ f: A \to B \right\} \\
\abs{\left\{ f: A \to B \right\}}
\end{gather*}
$\abs{A} = n$, $\abs{B} = m$. Considero un elemento di $A$, $a_1$, ho $m$ modi di decidere la sua immagine in $B$. Per ogni altro elemento di $A$, ho sempre $m$ modi di decidere la sua immagine in $B$. Ho quindi $n$ fattori tutti pari ad $m$, per cui il risultato \`e $m^n$.

Altro esempio: il numero delle sigle di cinque caratteri ottenute usando le 26 lettere dell'alfabeto inglese. $26^5$

Se la sigla fosse composta da cinque caratteri senza ripetizioni? $26 \times 25 \times 24 \times 23 \times 22$.

\section{Permutazioni ed anagrammi}

Le permutazioni sono i possibili ordinamenti di un insieme di $n$ oggetti distinti. I due insiemi $ \left\{ A, C, Z, X \right\}$ e $ \left\{ A, Z, X, C \right\} $ sono lo stesso insieme. ACZX e AZXC \textit{non} sono la stessa permutazione.

Con l'insieme $\{A,C,Z,X\}$ posso scegliere il primo elemento di una permutazione in 4 modi possibili, il secondo in 3, il terzo in 2, il quarto in 1. Quindi ho $4!$ permutazioni.

Le permutazioni di un insieme di $n$ oggetti distinti sono $n!$

\textit{Gli aggettivi in matematica sono colpi d'accetta.}

Ma ora voglio gli anagrammi di MATTEO. Non sono lettere distinte. Considero quindi un alfabeto esteso in cui ho $T_1$ e $T_2$. In questo alfabeto esteso, il numero di anagrammi possibili sono 6! Da questa lista devo cancellare i numeri sotto le T. Quante ripetizioni ho per ogni anagramma? Tante quanto il numero di modi in cui posso combinare le lettere ripetute, in questo caso $2!$

Ciascun anagramma \`e ripetuto $2!$ volte. Il numero di anagrammi \`e:
\[
\frac{6!}{2!}
\]
VITTORIO, tre lettere si ripetono due volte. La parola ha 8! anagrammi, se numero le lettere ripetute. Considero un anagramma, ORIOVITT. Posso numerare ciascuna lettera ripetuta in 2! modi, quindi ho $2! \times 2! \times 2!$ modi per mettere gli indici alle lettere.
\[
\frac{8!}{2! \times 2! \times 2!}
\]
ZOZZO, $5!$ anagrammi numerando le lettere. Z ripetuta 3 volte, O ripetuta 2 volte.
\[
\frac{5!}{3! \times 2!}
\]
Consideriamo una famiglia di $n$ oggetti con ripetizioni: vi sono $r$ tipi di oggetti, il primo tipo si presenta $n_1$ volte, il secondo tipo $n_2$ volte e cos\`i via. Notiamo: $ n_1 + n_2 + \dots + n_r = n$. I possibili ordinamenti di questa famiglia sono:
\[
\frac{ n! }{ n_1! \times n_2! \times \cdots \times n_r!}
\]

\section{Sottoinsiemi di una certa dimensione}

Ho un insieme di $n$ oggetti distinti. In quanti modi posso scegliere $r$ oggetti da questo insieme, indipendentemente dall'ordine in cui li scelgo? Coincide con la domanda: quanti sono i possibili sottoinsiemi di cardinalit\`a $r$ di un insieme di cardinalit\`a $n$? Necessariamente $r \leq n$, altrimenti la risposta sarebbe 0.

Se l'ordine contasse, avrei $n \times (n-1) \times (n-2) \times \cdots \times (n-r+1)$ modi per scegliere gli $r$ oggetti.

Ogni gruppo di $r$ oggetti pu\`o essere disposto in $r!$ modi differenti. Quindi se dimentico l'ordine avr\`o $r!$ ``copie'' ci ciascun gruppo di $r$ oggetti. Un insieme di $r$ elementi ha $r!$ permutazioni (ordinate) $\implies$ per ogni $r$-upla non ordinata ho $r!$ $r$-uple ordinate.

Rispondendo alla domanda:
\[
\frac{(n) \ (n-1) \ \ldots \ (n - r + 1)}{r!} =
\frac{ n! }{ r! \ (n-r)! } = 
\binom{n}{r}
\]
Ricordiamo i coefficienti binomiali. Con $0 \le r \le n$:
\[
\binom{n}{k} = \frac{n!}{k! \ (n-k)!} = 
\frac{n \ (n - 1) \ (n - 2) \ldots (n - k + 1)}{k!}
\]
Se $r \geq n \implies \binom{n}{r} = 0$. Propriet\`a dei coefficienti binomiali:
\[
\binom{n}{r} = \binom{n}{n-r}
\]
Da cui $\binom{n}{n} = \binom{n}{0} = 1$. \`E una ``bigezione''. Se ho un insieme $X$ di cardinalit\`a $\abs{X} = n$, ed un suo sottoinsieme $A$ con il suo complementare $B = X \setminus A$, il numero di modi in cui posso scegliere $A$ sono uguali al numero di modi in cui posso scegliere $B$.
\begin{align*}
\left\{A \subset X : \abs{A} = r\right\} \\
\left\{B \subset X : \abs{B} = n - r\right\}
\end{align*}
Con $1 \leq r \leq n$:
\[
\binom{n}{r} = \binom{n-1}{r-1} + \binom{n-1}{r}
\]
Ciao Salvo. Con $X = V \cup W$ e $X \setminus V = W$, $\abs{X} = \abs{V} + \abs{W}$.

I coefficienti binomiali si leggono anche ``$n$ scelgo $k$''. I coefficienti binomiali servono con i binomi:
\begin{multline*}
(a+b)^n = 
a^n 
+ \binom{n}{1} \, b \, a^{n-1} 
+ \binom{n}{2} \, b^2 \, a^{n-2} 
+ \ldots 
+ \binom{n}{n-2} \, b^{n-2} \, a^2
+ \binom{n}{n-1} \, b^{n-1} \, a
+ b^n
\end{multline*}
\[
(x+y)^n = \sum_{k=0}^{n} \binom{n}{k} \, x^k \, y^{n-k}
\]
Un esempio lievemente pi\`u complicato: devo scegliere un rappresentante e due segretari da un gruppo di ottanta persone. Ho ottanta modi per scegliere il rappresentante. Mi restano 79 persone da cui devo scegliere una coppia non ordinata. Per il principio fondamentale del calcolo combinatorio:
\[
80 \times \binom{79}{2} = \frac{80 \times 79 \times 78}{2!}
\]
Ho $r$ tipi di oggetti, ciascun tipo ha $n_i$ oggetti distinti, con $i \in [1,r]$. In quanti modi posso \textit{raggruppare} gli oggetti per tipo? Ho $r!$ modi per disporre i tipi di oggetti, gli oggetti del tipo $i$ possono essere disposti in $n_i!$ modi. Per il principio fondamentale del calcolo combinatorio:
\[
r! \times n_1! \times n_2! \times \cdots \times n_r! = 
r! \prod_{k = 1}^{r} n_r!
\]
Remember: $0! = 1$.
\[
\sum_{k=0}^{n} \binom{n}{k} = \sum_{k=0}^{n} \binom{n}{k} \, 1^k \, 1^{n-k} = (1 + 1)^n = 2^n
\]
O anche mi basta sapere che $\binom{n}{k}$ \`e il numero di sottoinsiemi di $k$ elementi di un insieme di $n$ elementi. Tutti i sottoinsiemi sono l'insieme delle parti, che ha cardinalit\`a $2^n$.

$X$ insieme con cardinalit\`a $\abs{X} = n$. $\parts(X) = \left \{ Y : Y \subset X \right\}$. Un sottoinsieme $Y \subset X$ pu\`o o non pu\`o contenere ciascun elemento di $X$. Quanti sottoinsiemi $Y$ posso costruire?
\[
\overbrace{2 \times 2 \times 2 \times \dots \times 2}^{n} = 2^n
\]
Un modo pi\`u complesso per dimostrarlo:
\begin{align*}
\parts(X) = \left \{Y :  Y \subset X \right\} = \bigcup_{k=0}^{n} \left \{ Y \subset X : \abs{Y} = k \right\} \\
\abs{\parts(X)} = \sum_{k=0}^{n} \abs{\left\{ Y \subset X : \abs{Y} = k\right\}}  = \sum_{k = 0}^{n} \binom{n}{k}
\end{align*}

\section{Coefficienti multinomiali}

Ho 10 persone. Devo assegnare 5 persone al compito 1, 2 al compito 2, 3 al compito 3. In quanti modi posso farlo?
\[
\binom{10}{5} \binom{5}{2} \binom{3}{3} = 
\frac{10!}{5! \ 5!} \ \frac{5!}{2! \ 3!} \ \frac{3!}{3! \ 0!} =
\frac{10!}{5! \ 3! \ 2!}
\]
Ho un insieme di $n$ oggetti distinti e $k$ categorie in cui inserirli. Devo inserire $n_1$ elementi nella categoria 1, $n_2$ elementi nella categoria 2, \dots $n_k$ elementi da inserire nella categoria $k$, senza che nessuno resti fuori, ossia $\sum_{i = 1}^{k} n_i = n$, con $n_i \in \naturals \forall i \in [1, k]$. In quanti modi lo posso fare?

\begin{align*}
\binom{n}{n_1} \binom{n - n_1}{n_2} \binom{n - (n_1 + n_2)}{n_3}\dots \binom{n - \left( \sum_{i = 1}^{k-1} n_i \right)}{n_k} = \\
\binom{n}{n_1} \binom{n - n_1}{n_2} \binom{n - (n_1 + n_2)}{n_3}\dots \binom{n_k}{n_k} = \\
\frac{n!}{n_1! \left( n - n_1\right)!} \ 
\frac{(n - n_1)!}{n_1! \left( n - n_1 - n_2\right)!} \ 
\frac{(n - n_1 - n_2)!}{n_1! \left( n - n_1 - n_2 - n_3\right)!} \ 
\dots \\
\frac{(n - n_1 - n_2 \dots - n_{k-2})!}{n_1! \left( n - n_1 - n_2 \dots - n_{k-2} - n_{k-1} \right)!} \ 
\frac{(n - n_1 - n_2 \dots - n_{k-1})!}{n_1! \ 0!} = \\
\frac{n!}{\prod_{i=1}^{k}n_k!} =
\binom{n}{n_1, n_2, \dots, n_k}
\end{align*}
Detto anche coefficiente multinomiale. Ricordiamo che $\sum_{i = 1}^{k} n_i = n$. I coefficienti binomiali sono un caso particolare dei coefficienti multinomiali. Sapendo che $n = n_1 + n_2$:
\[
\binom{n}{n_1, n_2} = \binom{n}{n_1, n - n_1} = \frac{n!}{n_1! (n - n_1)!} = \binom{n}{n_1}
\]
Devo dividere $n$ persone ($n$ pari) in 2 gruppi. Dopo generalizziamo a $m$ gruppi (spero) con $n$ multiplo di $m$. Se i gruppi fossero distinguibili avrei:
\[
\binom{n}{\frac{n}{2}, \frac{n}{2}}
\] 
modi per formare i gruppi. Ma non essendo distinguibili i gruppi avr\`o:
\[
\frac{\binom{n}{\frac{n}{2}, \frac{n}{2}}}{2} = \frac{n!}{2 \ \frac{n}{2}! \ \frac{n}{2}!} = \frac{n \ (n - 1) \dots (\frac{n}{2}+1)}{2 \frac{n}{2}!} = \frac{(n-1)(n-2) \dots (\frac{n}{2} + 1)}{(\frac{n}{2}-1)!} = \binom{n-1}{\frac{n}{2}-1}
\] 
\section{Teorema multinomiale}
Con $n_i \ge 0$ interi:
\[
\left( x_1 + \dots + x_r \right)^n = 
\sum_{n_1 + \dots + n_r = n} \binom{n}{n_1, \dots, n_r} \cdot x_1^{n_1} \cdots x_r^{n_r}
\]
Wikipedia lo scrive come:
\[
\left( \sum_{i=1}^r x_i \right)^n=\sum_{k_1+\ldots+k_r=n}{n!\cdot \prod_{i=1}^r \frac{x_i^{k_i}}{k_i!}}
\]





















\newpage

\section{Calcolo della Probabilit\`a}


\section{Spazio di probabilit\`a}

Consideriamo un esperimento con pi\`u esiti possibili. Vogliamo modellizzarlo con uno ``spazio di probabilit\`a''.

\begin{defn}[Spazio campionario]
Chiamiamo l'insieme dei possibili esiti dell'esperimento $S$, ``spazio campionario''.
\end{defn}

\begin{defn}[Evento]
Matematicamente, un \emph{evento} \`e un sottoinsieme dello spazio campionario. Ad esempio, con lo spazio campionario dei risultati di un dado $\{ 1, 2, 3, 4, 5, 6\}$ un evento potrebbe essere ``esce un numero pari'', che corrisponde all'insieme di esiti $\{2, 4, 6\}$.

Un evento elementare \`e un sottoinsieme dello spazio campionario consistente di un singolo elemento (o esito): $\{ s \}, s \in S$.
\end{defn}

\begin{defn}[Spazio e funzione di probabilit\`a]
Lo spazio di probabilit\`a che modellizza l'esperimento \`e una coppia $(S,P)$ dove $S$ \`e lo spazio campionario e $P$ \`e una funzione reale, detta ``funzione di probabilit\`a'' definita su $\parts(S)$ t.c. valgono i seguenti assiomi:
\begin{description}
    \item[P1\label{itm:P1}] $0 \le \prob{E} \le 1 \forall  E \subset S$, la funzione di probabilit\`a associa un numero fra 0 ed 1 ad ogni \emph{evento}, non ad ogni esito. La probabilit\`a di un evento \`e un numero fra 0 ed 1.
    \item[P2\label{itm:P2}] $\prob{S} = 1$. $S$ \`e l'``evento certo'', la cui probabilit\`a \`e 1.
    \item[P3\label{itm:P3}] Assioma di numerabile additivit\`a: se $E_1, E_2, \dots$ \`e una successione di eventi ($E_1 \subset S \forall  i$), a 2 a 2 disgiunti (o incompatibili) ($E_i \cap E_j = \emptyset$ se $i \neq j$), allora:
    \[
    \prob{\bigcup_{i=1}^{\infty} E_i} = \sum_{i=1}^{\infty} \prob{E_i}
    \]
    Vuol dire che, considerando eventi incompatibili, la probabilit\`a dell'unione \`e la somma delle probabilit\`a. 
\end{description}
\end{defn}

L'insieme delle parti di $S$ \`e la ``famiglia degli eventi'', la funzione $P: \parts(S) \to [0,1]$. La funzione $P$ \`e definita sull'insieme delle parti di $S$.

L'intero insieme $S$ rappresenta l'evento certo. L'insieme vuoto $\emptyset$ \`e l'evento impossibile.

Se $E, F$ sono eventi con $E \cap F = \emptyset \implies E$ ed $F$ sono detti ``incompatibili'': il realizzarsi di un evento implica che l'altro non si \`e realizzato. 

$\prob{\emptyset} = 0$ non \`e un'assioma. Prendo infiniti insiemi a due a due disgiunti $E_1, E_2, \dots, E_n$ dove $E_i = \emptyset$, e quindi $\bigcup_{n=1}^{\infty} E_n = \emptyset$. Per l'assioma di numerabile additivit\`a:
\[
\prob{\emptyset} = \sum_{n=1}^{\infty} \prob{\emptyset} \implies \prob{\emptyset} = 0
\]
L'unico elemento che sommato infinite volte a s\'e stesso d\`a sempre s\'e stesso \`e lo 0.

\begin{prop}[Additivit\`a finita \label{additivita_finita}]
$P$ soddisfa l'additivit\`a finita. Considerata una famiglia finita $E_1, E_2, \dots, E_k$ di eventi a due a due disgiunti, vale che $\prob{\bigcup_{n=1}^k E_n} = \sum_{n=1}^k \prob{E_n}$.
\end{prop}
\begin{proof}
Consideriamo la successione di eventi $F_1, F_2, F_3, \dots$ cos\`i definita:
\[
F_n =
\begin{cases}
E_n \text{ se } 1 \le n \le k \\
\emptyset \text{ se } n > k
\end{cases}
\]
Ho quindi una successione infinita di eventi a due a due disgiunti. Posso applicare l'assioma di numerabilit\`a additiva:
\[
\prob{\bigcup_{n = 1}^{\infty} F_n} = \sum_{n = 1}^{\infty} \prob{F_n}
\]
Ma $\bigcup_{n = 1}^{\infty} F_n = \bigcup_{n = 1}^{k} E_n$, quindi:
\[
\prob{\bigcup_{n = 1}^{\infty} F_n} =
\prob{\bigcup_{n = 1}^{k} E_n} =
\sum_{n = 1}^{\infty} \prob{F_n}
\]
Inoltre $\prob{F_n}$ \`e definita come:
\[
\prob{F_n} =
\begin{cases}
\prob{E_n} \text{ se } 1 \le n \le k \\
\prob{\emptyset} \text{ se } n >k
\end{cases}
\]
Quindi:
\[
\sum_{n = 1}^{\infty} \prob{F_n} =
\sum_{n = 1}^{k} \prob{E_n}
\]
Per transitivit\`a ho dimostrato la tesi.
\[
\prob{\bigcup_{n=1}^k E_n} = \sum_{n=1}^k \prob{E_n}
\]
\end{proof}

L'additivit\`a richiede la disgiunzione.

\begin{prop}[Probabilit\`a dell'evento complementare]
Dato un evento $E$, l'evento complementare $E^{\compl} = S \setminus E$ ha probabilit\`a:
\[
\prob{E^{\compl}} = 1 - \prob{E}
\]
\end{prop}
\begin{proof}
Applico la proposizione \ref{additivita_finita} con $k = 2$, dato che i due insiemi sono disgiunti.
\[
\prob{E \cup E^{\compl}} = \prob{E} + \prob{E^{\compl}}
\]
Ma poich\'e $E + E^{\compl} = S$:
\[
\prob{E \cup E^{\compl}} = \prob{E} + \prob{E^{\compl}} = P(S) = 1 \implies
\prob{E^{\compl}} = 1 - \prob{E}
\]
\end{proof}

\begin{prop}[Monotonia della compatibilit\`a]
Se $E$ e $F$ sono eventi compatibili ($E \subset F$), allora $\prob{E} \le \prob{F}$. Sapendo che $F = E \cup (F \setminus E)$:
\[
\prob{F} = \prob{E \cup (F \setminus E)} = \prob{E} + \prob{F \setminus E} \ge \prob{E}
\]
\end{prop}

Dati due eventi $A, B \subset S$ tali che $A \cap B = \emptyset \implies \prob{A \cup B} = \prob{A} + \prob{B}$. Ma nel caso generale?

\begin{prop}
Dati due eventi $A$, $B \subset S$, vale che $\prob{A \cup B} = \prob{A} + \prob{B} - \prob{A \cap B}$. La cardinalit\`a, come la misura della probabilit\`a, \`e additiva: $\abs{A \cup B} = \abs{A} + \abs{B} - \abs{A \cap B}$.
\end{prop}
\begin{proof}
Dati due eventi congiunti $A, B$, chiamiamo $I = A \setminus B$, $II = A \cap B$, $III = B \setminus A$. $A \cup B = I \cup II \cup III$. Essendo i tre insiemi disgiunti, posso applicare l'additivit\`a finita.
\[
\prob{A \cup B} = \prob{I} + \prob{II} + \prob{III}
\]
Sempre per l'additivit\`a finita: $\prob{A} = \prob{I} + \prob{II}$ e $\prob{B} = \prob{II} + \prob{III}$, e $\prob{A \cap B} = \prob{II}$. Riscrivo quindi il membro destro della proposizione come:
\begin{align*}
\prob{A} + \prob{B} - \prob{A \cap B} = \\
\prob{I} + \prob{II} + \prob{II} + \prob{III} - \prob{II} = \\
\prob{I} + \prob{II} + \prob{III} = \\
\prob{A \cup B}
\end{align*}
E ho dimostrato la tesi.
\end{proof}

\section{Spazi campionar\^i  a esiti equiprobabili}

Quale \`e lo spazio di probabilit\`a $(S,P)$ che descrive il lancio di un dado?
\[
S = \{ 1, 2, 3, 4, 5, 6 \}
\]
Ma qual \`e la funzione di probabilit\`a $P : \parts(S) \to [0,1]$? Non ho motivi per ritenere che una faccia esca pi\`u spesso rispetto alle altre. Per simmetria, le probabilit\`a degli eventi elementari $\prob{\{1\}} = \prob{\{2\}} = \dots = \prob{\{6\}}$ sono tutte uguali.
\begin{align*}
\prob{S} &= 1 = \prob{\{1,2,\dots, 6\}} = \tag{per la propriet\`a \ref{itm:P2}} \\
&= \underbrace{\prob{\{1\}}}_{x} + \prob{\{2\}} + \dots + \prob{\{6\}} = \tag{per additivit\`a finita} \\
&= 6 \cdot x \implies \\
& x = \frac{\prob{S}}{6} = \frac{1}{6} = \prob{\{k\}}
\end{align*}
% Riprendendo la propriet\`a \ref{itm:P2}, $\prob{S} = 1 = \prob{\{1,2,\dots, 6\}}$. Ma per l'additivit\`a finita, $\prob{S} = P(\{1\}) + P(\{2\}) + \dots + P(\{6\})$. Sono tutti numeri uguali, per cui $P(S) = 6 \ x$ con $x = P(\{1\})$. Quindi:
% \[
% x = \frac{P(S)}{6} = \frac{1}{6} = P(\{k\})
% \]

Sappiamo quindi la probabilit\`a di un evento elementare. Ma qual \`e $\prob{E}$ con $E \subseteq S$?
\[
\prob{E} = \prob{\bigcup_{k \in E} \{k\}}
\]
\`E una famiglia finita di insiemi a due a due disgiunti, quindi per l'additivit\`a finita:
\[
\prob{E} = \sum_{k \in E} \prob{\{k\}} = \frac{\abs{E}}{6} = \frac{\abs{E}}{\abs{S}}
\]
La probabilit\`a di un evento \`e data dal numero degli esiti favorevoli diviso il numero degli esiti equiprobabili, ma solo se gli esiti \emph{sono} equiprobabili.

\begin{defn}[Spazio di probabilit\`a con esiti equiprobabili]
Uno spazio di probabilit\`a $(P,S)$ \`e detto con esiti equiprobabili se $\prob{\{s\}} = \prob{\{s'\}} \forall s, s' \in S$.
\end{defn}
\begin{prop}
Sia $S$ finito e $(S,P)$ con esiti equiprobabili, allora:
\[
\prob{E} = \frac{\abs{E}}{\abs{S}} \qquad \forall  E \subseteq S
\]
\end{prop}
\begin{proof}
Dato un evento $E$, posso scrivere:
\[
E = \bigcup_{s \in E} \{ s \}
\]
ossia, $E$ \`e un'unione finita di eventi a due a due disgiunti. Posso quindi applicare l'additivit\`a finita:
\[
\prob{E} = \sum_{s \in E} \prob{\{s\}}
\]
Consideriamo il caso in cui $E = S$:
\[
\prob{S} = \sum_{s \in S} \underbrace{\prob{\{ s \}}}_{x} = x \cdot \abs{S}
\]
poich\'e $\prob{\{s\}} = x$ non dipende da $s \in S$. Quindi $x = \frac{1}{\abs{S}}$, con $x = \prob{\{s\}} \forall  s \in S$. Sostituisco nell'equazione precedente:
\[
P(E) = \sum_{s \in E} \frac{1}{|S|} = \frac{|E|}{|S|}
\]
\end{proof}

\begin{prop}
Sia $\abs{S} = + \infty$ e $S$ numerabile, non esiste uno spazio di probabilit\`a con esiti equiprobabili. 
\end{prop}
\begin{proof}
Supponiamo per assurdo che esista $(S,P)$ con esiti equiprobabili. So che $P(S) = 1$, e che:
\[
S = \bigcup_{s \in S} \{s\}
\]
Ho un'unione infinita numerabile di eventi a due a due disgiunti. Applico l'additivit\`a numerabile:
\[
P(S) = \sum_{s \in S} \underbrace{P \left( \{ s \} \right)}_{x} = 1
\]
Non esiste un numero $x$ che sommato infinite volte da 1, o qualunque altro numero finito.
\end{proof}

\begin{esercizio}
Lancio due volte un dado.
\begin{itemize}
    \item Modellizare l'esperimento con uno spazio di probabilit\`a;
    \item Calcolare la probabilit\`a che escano gli stessi numeri.
\end{itemize}
Lo spazio campionario \`e $S = \{ (a,b) : a,b \in \{ 1 \dots 6 \} \}$, con $a$ faccia del primo lancio e $b$ faccia del secondo lancio. In un singolo lancio ogni faccia \`e equivalente ad ogni altra faccia. La simmetria vale ancora se faccio due lanci. Quindi:
\[
\forall  E \subseteq S \qquad P(E) = \frac{|E|}{|S|} = \frac{|E|}{36}
\]

Prendiamo l'evento $E = $ ``le due facce sono uguali''. $E = \{ (1,1), (2,2) \dots (6,6) \}$.
\[
P(E) = \frac{6}{36} = \frac{1}{6}
\]
\end{esercizio}

\begin{esercizio}
Ho un'urna contenente 8 palline blu, 5 gialle e 3 nere. Estraggo tre palline senza rimpiazzo. Qual \`e la probabilit\'a di estrarre tre palline blu?

Non posso definire lo spazio campionario basandomi sul colore: estrarre una pallina blu e una gialla, ad esempio, non \`e un esito equiprobabile. Posso distinguere le palline dello stesso colore numerandole. In questo caso estrarre una pallina rispetto ad un'altra \`e equiprobabile.
\end{esercizio}

\section{Principio di inclusione-esclusione}

So che, dati gli eventi $A$ e $B$ non necessariamente disgiunti, la probabilit\`a della loro unione \`e $P(A \cup B) = P(A) + P(B) - P(A \cap B)$. Se volessi generalizzare a $n$ eventi?

\begin{prop}
Dati $n$ eventi $E_1, E_2 \dots E_n$, vale che:
\begin{align*}
P(E_1 \cup E_2 \cup \dots \cup E_n) = \\
P(E_1) + P(E_2) + \dots + P(E_n) \\
- \sum_{1 \le i_1 < i_2 \le n} P(E_{i_1} \cap E_{i_2}) \\
+ \sum_{1 \le i_1 < i_2 < i_3 \le n} P(E_{i_1} \cap E_{i_2} \cap E_{i_3}) \\
\dots \\
+ (-1)^{n+1} P(E_1 \cap E_2 \cap \dots \cap E_n)
\end{align*}
In maniera pi\`u compatta:
\begin{equation}
P(E_1 \cup E_2 \cup \dots \cup E_n) =
\sum_{r = 1}^{n} (-1)^{r+1} 
\left(
\sum_{1 \le i_1 < i_2 < \dots < i_r \le n} P(E_{i_1} \cap E_{i_2} \cap \dots \cap E_{i_r})
\right)
\end{equation}
\end{prop}
\begin{exmp}
Prendiamo tre eventi $E, F, G$ (in seguito scriveremo $EF$ per intendere $E \cap F$) e vediamo la probabilit\`a di $P(E \cup F \cup G)$.
\begin{align*}
P(E \cup F \cup G) = \\
(-1)^{1+1} \ (P(E) + P(F) + P(G)) \\
+ (-1)^{2+1} \ (P(E \cap F) + P(E \cap G) + P(F \cap G)) \\
+ (-1)^{3+1} \ (P(E \cap F \cap G)) = \\
P(E) + P(F) + P(G) - P(E \cap F) - P(E \cap G) - P(F \cap G) + P(E \cap F \cap G)
\end{align*}
\end{exmp}
\begin{proof}
Limitiamo la dimostrazione a $S$ numerabile. Se lo spazio campionario \`e numerabile, vale la proposizione \ref{numerabile}.

Dato $s \in S$ e dato $E \subseteq S$ indichiamo con il simbolo $\mathbf{1}(s \in E)$ (funzione caratteristica associata alla propriet\`a ``$s \in E$'') la funzione:
\[
\mathbf{1}(s \in E) =
\begin{cases}
1 \text{ se } s \in E \\
0 \text{ se } s \notin E 
\end{cases}
\]
Quindi la proposizione \ref{numerabile} equivale a scrivere:
\[
P(E) = \sum_{s \in E} P(\{s\}) = \sum_{s \in S} \mathbf{1}(s \in E) P(\{s\})
\]
Posso riscrivere il membro sinistro della proposizione come:
\[
P(E_1 \cup E_2 \cup \dots \cup E_n) = 
\sum_{a \in S} \mathbf{1}(a \in E_1 \cup \dots \cup E_n) P(\{a\})
\]
Posso riscrivere il membro destro della proposizione come:
\[
\sum_{r = 1}^{n} (-1)^{r+1} 
\left(
\sum_{1 \le i_1 < i_2 < \dots < i_r \le n} 
\overbrace{
P(E_{i_1} \cap E_{i_2} \cap \dots \cap E_{i_r})
}^{ \sum_{a \in S} \mathbf{1}(a \in E_1 \cap \dots \cap E_n) P(\{a\}) }
\right) 
\]
Il risultato di una serie a termini non costanti dipende dall'ordine di somma. Ma queste serie sono assolutamente convergenti, quindi posso scambiare l'ordine di somma. Quindi:
\[
\sum_{a \in S} P(\{a\}) \left[ \sum_{r = 1}^n (-1)^{r+1} \sum_{1 \le i_1 < \dots < i_r \le n} \mathbf{1}(a \in E_{i_1} \cap E_{i_2} \cap \dots \cap E_{i_r}) \right]
\]
Per dimostrare il principio di inclusione-esclusione basta dimostrare che, $\forall  a \in S$:
\[
\sum_{r = 1}^n (-1)^{r+1} \sum_{1 \le i_1 < \dots < i_r \le n} \mathbf{1}(a \in E_{i_1} \cap E_{i_2} \cap \dots \cap E_{i_r}) = \mathbf{1}(a \in E_1 \cup \dots \cup E_n)
\]
Il membro destro sar\`a sempre o 0 o 1, il membro sinistro sar\`a una somma di 1, 0 e $-1$. Distinguiamo due casi:
\begin{enumerate}
    \item $a \notin E_1 \cup \dots \cup E_n$, allora il membro destro vale 0. Ma quindi $a$ non pu\`o appartenere all'intersezione di eventi, se non appartiene alla loro unione. Quindi anche il membro sinistro \`e 0 (o una somma di 0).
    \item $a \in E_1 \cup \dots \cup E_n$, allora il membro destro vale 1. Sia $m$ il numero degli eventi di tipo $E_i$ a cui $a$ appartiene (ossia per cui $a \in E_i$). Per semplicit\`a, ma senza perdita di generalit\`a, supponiamo che $a$ appartenga ai primi $m$ elementi. Abbiamo quindi che $\mathbf{1}(a \in E_{i_1} \cap \dots \cap E_{i_r}) = 1 \Leftrightarrow i_1, i_2, \dots i_r \le m$.

    Quindi la somma dei valori della funzione caratteristica per valori fino a $m$ \`e uguale al numero di possibili tuple di numeri distinti ordinati in senso crescente compresi fra 1 e $m$.
    \[
    \sum_{1 \le i_1 < \dots < i_r \le n} \mathbf{1}(a \in E_{i_1} \cap E_{i_2} \cap \dots \cap E_{i_r}) = \left| \{(i_1,i_2, \dots i_r) : 1 \le i_1 < i_2 < \dots < i_r \le m\} \right|
    \]
    Altro non \`e che $\binom{m}{r}$. C'\`e una bigezione fra il numero di insiemi non ordinati e il numero di $n$-uple ordinate in senso crescente.

    La parte sinistra dell'equazione vale quindi:
    \[
    \sum_{r = 1}^{n} (-1)^{r+1} \binom{m}{r}
    \]
    Dobbiamo dimostrare che \`e uguale ad 1. Se $r > m$ il coefficiente binomiale vale 0. La somma si pu\`o semplificare ancora e diventare quindi:
    \[
    \sum_{r = 1}^{m} (-1)^{r+1} \binom{m}{r} =
    \sum_{r = 0}^{m} (-1)^{r+1} \binom{m}{r} + 1 =
    - \sum_{r = 0}^{m} (-1)^{r} \binom{m}{r} + 1
    \]
    Posso riscrivere tutto con il teorema del binomio applicato a $(1 - 1)^m$:
    \[
    \sum_{r = 0}^{m} \binom{m}{r} 1^{m-r} (-1)^r + 1 = (1-1)^m + 1 = 0 + 1 = 1
    \]
\end{enumerate}
\end{proof}
\begin{prop}\label{numerabile}
Sia $S$ numerabile (finito o infinito), Allora $\forall  E \subset S$ vale:
\[
P(E) = \sum_{s \in E} P(\{s\})
\]
\end{prop}
\begin{proof}
Se $S$ \`e numerabile $\implies E$ \`e numerabile (perch\'e la cardinalit\`a di un insieme \`e maggiore della cardinalit\`a di un suo sottoinsieme proprio). Posso scrivere $E$ come:
\[
E = \bigcup_{s \in E} \{ s \}
\]
ossia, un'unione numerabile di eventi a due a due disgiunti. Vale quindi che:
\[
P(E) = \sum_{s \in E} P(\{s\})
\]
\end{proof}

\begin{prop}
Supponiamo di avere una famiglia di eventi $E_1, E_2, \dots E_n$ vale che:
\[
P(E_1 \cup E_2 \cup \dots \cup E_n) \le P(E_1) + P(E_2) + \dots + P(E_n)
\]
Inoltre, data una successione di eventi $E_1, E_2, \dots E_n \dots$ vale che:
\[
P \left( \bigcup_{n = 1}^{\infty} E_n \right) \le \sum_{n = 1}^{\infty} P(E_n)
\]
\end{prop}
\begin{proof}
Dimostriamo solo la prima parte. Per $n = 2$ vale che $P(E_1 \cup E_2) = P(E_1) + P(E_2) - P(E_1 \cap E_2) \le P(E_1) + P(E_2)$.

Concludiamo per induzione. Se vale per $n$, vale per $n + 1$. Pensiamo $ P(E_1 \cup E_2 \cup \dots \cup E_n \cup E_{n+1}) $ come $P( A \cup E_{n+1})$ con $A = E_1 \cup E_2 \cup \dots \cup E_n$.
\[
P(E_1 \cup E_2 \cup \dots \cup E_n \cup E_{n+1}) \le P(E_1 \cup \dots \cup E_n) + P(E_{n+1}) \le P(E_1) + \dots + P(E_n) + P(E_{n+1})
\]
\end{proof}

\begin{exmp}
Abbiamo cinque coppie, ciascuna formata da un uomo e una donna. Vogliamo sapere qual \`e la probabilit\`a di prendere $G_i = F_i$ balla con $M_i$.
\[
E^{\compl} = G_1 \cup G_2 \cup G_3 \cup G_4 \cup G_5
\]
\[
P(E^c) = P(G_1 \cup G_2 \cup G_3 \cup G_4 \cup G_5) = \sum_{r = 1}^{5} (-1)^{r +1} \sum_{1 \le i_1 < i_2 \dots i_r \le 5} P( G_{i_1} \cap G_{i_2} \dots G_{i_r})
\]
$P(G_i)$ non dipende da $i$, per simmetria.
\[
P(G_1) = \frac{4!}{5!} = \frac{1}{5}
\]
Sappiamo che \`e vero $\forall i = 1 \dots 5$. In generale:
\[
P( G_{i_1} \cap \dots \cap G_{i_r}) = \frac{(5 - r)!}{5!}
\]
\[
P(E) = 1 - \sum_{r = 1}^{5} (-1)^{r+1} \frac{1}{r!} = 
\sum_{k = 0}^{5} \frac{(-1)^k}{k!}
\]
Quindi:
\[
P(E) = 1 - \frac{1}{1!} + \frac{1}{2!} - \frac{1}{3!} + \frac{1}{4!} - \frac{1}{5!} = 
\frac{3 \cdot 4 \cdot 5 - 4 \cdot 5 + 5 - 1}{5!}
\]
In generale:
\[
\sum_{k = 0}^{N} \frac{(-1)^k}{k!}
\]
Per $N$ all'infinito, questo \`e lo sviluppo in serie di $e^{-1}$
\end{exmp}
\section{Interpretazione frequentistica della probabilit\`a}

Ripeto un esperimento $N$ volte. Chiamo $K(E, N)$ il numero di volte in cui $E$ si \`e verificato tra gli $N$ esperimenti. Quindi il rapporto $\frac{K(E, N)}{N}$ \`e la frequenza relativa con cui $E$ si \`e verificato negli $N$ esperimenti.

Per $N$ grande la frequenza relativa \`e circa la probabilit\`a di $E$.
\[
\frac{K(E,N)}{N} \approx P(E)
\]
Di norma aggiungo informazioni per rendere gli esiti simmetrici.

La modellizzazione si basa sul buon senso. La funzione di probabilit\`a in caso di esiti non equiprobabili si costruisce statisticamente.

Su 1000 lanci, il mio dado truccato mi d\`a:

\begin{tabular}{cc}
1 & 405 \\
2 & 200 \\
3 & 001 \\
4 & 103 \\
5 & 050 \\
6 & 241
\end{tabular}

Modellizzo quindi il mio dado con una funzione di probabilit\`a $P : \parts(S) \to [0,1]$ viene definita su $S = \{ 1 \dots 6 \}$ come:
\begin{align*}
P(\{1\}) &= \frac{405}{1000} \\
P(\{2\}) &= \frac{200}{1000} \\
\dots & \\
P(\{6\}) &= \frac{241}{1000}
\end{align*}

Abbiamo la probabilit\`a degli eventi elementari. Se sappiamo la probabilit\`a degli eventi elementari, sappiamo tutto. Infatti per $S$ numerabile (finito o infinito):
\[
P(E) = \sum_{s \in E} P(\{s\})
\]
\begin{prop}
Se $S$ \`e numerabile, allora per ogni funzione $p : S \to [0,1]$ tale che $\sum_{s \in S} p(s) = 1 \exists ! P$ funzione di probabilit\`a con $P(\{s\}) = p(s) \forall  s \in S$.
\end{prop}

\section{Probabilit\`a condizionata}

Probabilit\`a condizionata all'informazione. Abbiamo un'informazione parziale sul risultato dell'esperimento.

\begin{defn}[Probabilit\`a condizionata]
Dati due eventi $E$ e $F$ con $P(F) > 0$ si definisce la probabilit\`a di $E$ condizionata dall'evento $F$ come:
\[
P(E|F) = \frac{P(E \cap F)}{P(F)} 
\]
\end{defn}

\begin{oss}
Se $S$ ha esiti equiprobabili e se $\abs{S} < + \infty$, allora:
\[
P(E|F) = \frac{\abs{E \cap F}}{\abs{F}}
\]
\end{oss}

\begin{proof}
Siccome lo spazio ha esiti equiprobabili, scriviamo la formula come cardinalit\`a di insiemi:
\[
P(E|F) = \frac{P(E \cap F)}{P(F)} = \frac{\frac{\abs{E \cap F}}{\abs{S}}}{\frac{\abs{F}}{\abs{S}}} = \frac{\abs{E \cap F}}{\abs{F}}
\]
Se moltiplico per $N$ numero di esperimenti, ho che $P(F)\ N$ \`e il numero di volte in cui si \`e verificato $P(F)$ negli $N$ esperimenti, e $P(E \cap F) \ N$ \`e il numero di volte in cui si \`e verificato sia $E$ che $F$.
\end{proof}

Lancio due dadi. $F =$ ``la somma dei due dadi \`e 7''. Qual \`e la probabilit\`a $E =$ ``il secondo lancio ha dato 3''?

% gioco delle tre carte, RR RN NN

\subsection{Legge del prodotto (o legge delle probabilit\`a composte)}
\begin{theorem}[Legge del prodotto]
Siano $E_1, E_2, \ldots, E_n$ eventi con $P(E_1 \cap \ldots \cap E_{n-1}) > 0$, allora:
\[
P(E_1 \cap \ldots \cap E_{n}) = P(E_1) \ P(E_2 | E_1) \ P(E_3 | E_1 \cap E_2) \ \ldots \ P(E_n | E_1 \cap E_2 \cap \ldots \cap E_{n-1})
\]
$P(E|F)$ si legge ``probabilit\`a di E condizionato F''.
\end{theorem}

\begin{exmp}
Ho un'urna contenente 5 palline blu, 3 rosse e 6 gialle. Estraggo 3 palline senza rimpiazzo e voglio sapere la probabilit\`a che siano tutte gialle. Definisco, immaginando di estrarre le palline una dopo l'altra, l'evento $E_i =$ ``la pallina $i$-esima \`e gialla'', per $i = 1, 2, 3$. $P(E_1 \cap E_2 \cap E_3)$ \`e la probabilit\`a che tutte le palline siano gialle. Per la legge delle probabilit\`a composte quindi:
\[
P(E_1 \cap E_2 \cap E_3) = 
P(E_1) \ P(E_2 | E_1) \ P(E_3 | E_1 \cap E_2)
\]
Al momento di estrarre la prima pallina so che la probabilit\`a di estrarre una pallina gialla \`e $P(E_1) = \frac{6}{14}$
\end{exmp}

La probabilit\`a condizionata ha senso quando l'evento condizionante ha cardinalit\`a positiva.

\begin{oss}
$P(E_1 \cap \ldots \cap E_{n_1}) > 0 \implies \forall r = 1 \dots n - 1 , \, P(E_1 \cap E_2 \cap \ldots \cap E_r) > 0$.
\end{oss}

Se $F \subset E$ e $P(F) > 0$ allora $P(E) > 0$

\begin{proof}
Il membro destro \`e:
\[
P(E_1) \cdot \frac{P(E_1 \cap E_2)}{P(E_1)} \cdot \frac{P(E_1 \cap E_2 \cap E_3)}{P(E_1 \cap E_2)} \dots \frac{P(E_1 \cap E_2 \cap \dots \cap E_{n-1} \cap E_n)}{P(E_1 \cap E_2 \cap \dots \cap E_{n-1})}
\]
Semplificando ottengo il membro sinistro.
\end{proof}

\subsection{Legge della probabilit\`a totale}
\begin{theorem}[Legge della probabilit\`a totale]
Siano $F_1, F_2, \dots F_n$ eventi a due a due disgiunti la cui unione \`e $S$, ossia $F_1 \dots F_n$ \`e una partizione di $S$. Supponiamo che $P(F_i) > 0 \forall i = 1 \dots n$. Allora $\forall E \in S$ vale:
\[
P(E) = \sum_{1 = 1}^{n} P(E | F_i) \cdot P(F_i)
\]
\end{theorem}
\begin{proof}
\[
E = \bigcup_{i = 1}^{n} (E \cap F_i)
\]
Perch\'e:
\[
E = E \cap S = E \cap (\bigcup_{i = 1}^{n} f_i)
\]
Questi eventi sono a due a due disgiunti al variare di $i$, perch\'e lo sono gli $F_i$ e quindi a maggior ragione qualcosa di pi\`u piccolo.

Per additivit\`a finita:
\[
P(E) = \sum_{i = 1}^{n} P(E \cap F_i)
\]
Ma $P(E \cap F_i) = P(E | F_i) P(F_i)$ per la definizione di probabilit\`a condizionata (o per la legge del prodotto). Finito.
\end{proof}

\begin{exmp}
Lancio una moneta, se esce Testa estraggo 3 carte da un mazzo di 40, se esce croce estraggo una sola carta da un mazzo di 40. Vinco se estraggo almeno un Re.

Con una partizione in due eventi $F$ e $F^c$ ho che $P(E) = P(E|F) P(F) + P(E|F^c) P(F^c)$.
\[
P(E|F) = 1 - \frac{\binom{36}{3}}{\frac{40}{3}}
\]
\end{exmp}

\subsection{Teorema di Bayes}

\begin{theorem}[Teorema di Bayes]
Si legge ``beis''. Siano $E$ e $F$ eventi, allora:
\[
P(F|E) = \frac{P(E|F)P(F)}{P(E)} = \frac{P(E|F)P(F)}{P(E|F)P(F) + P(E|F^{c})P(F^{c})}
\]
La prima formula vale se $P(E)$ e $P(F) > 0$, la seconda se $P(F^{c}) > 0$.
\end{theorem}

\begin{proof}
\begin{align*}
P(F|E) &= \frac{P(FE)}{P(E)} = \tag{per definizione} \\
&= \frac{P(E|F) \ P(F)}{P(E)} = \tag{per la legge del prodotto} \\
&= \frac{P(E|F) \ P(F)}{P(E|F) \ P(F) + P(E|F^{c}) \ P(F^{c})} \tag{per la legge delle probabilit\`a totali}
\end{align*}
\end{proof}
\`E utile quando bisogna scambiare l'evento condizionante con l'evento condizionato.

\begin{theorem}[Versione generale del teorema di Bayes]
Sia $F_1 \dots F_n$ una partizione dello spazio campionario $S$ (a due a due disgiunti e la cui unione da tutto $S$), allora per ogni evento $E \subset S$ e $\forall i = 1 \dots n$ vale:
\[
P(F_i|E) = \frac{P(E|F_i)P(F_i)}{P(E)} = \frac{P(E|F_i) P(F_i)}{\sum_{j = 1}^{n} P(E|F_j)P(F_j)}
\]
Per $n = 2$ e $F_1 = F$ e $F_2 = F^{c}$, il teorema di Bayes generale si riduce al teorema precedente.
\end{theorem}

\begin{proof}
\begin{align*}
P(F_i|E ) &= \frac{P(F_i E)}{P(E)} = \tag*{per definizione} \\ 
&= \frac{P(E | F_i) P(F_i)}{P(E)} = \tag*{per la regola del prodotto} \\ 
&= \frac{P(E | F_i) P(F_i)}{\sum_{j = 1}^{n} P(E | F_j) P(F_j)} \tag*{per la legge della probabilit\`a totale}
\end{align*}
\end{proof}

\begin{exmp}
Due persone estraggono una carta ciascuno da un mazzo di carte. Qual \`e la probabilit\`a che solo uno dei due estragga un asso?
\begin{gather*}
S = \{ (x_1, x_2) : x_1 \neq x_2 \text{ con } x_1, x_2 \in M \} \\
P(x_1 \text{ sia asso}), \, P(x_2 \text{ sia asso})
\end{gather*}
$S \ni (x_1, x_2) \to (x_2, x_1) \in S$, ossia le coppie in cui $x_1$ \`e asso sono in bigezione con quelle in cui $x_2$ \`e asso. La probabilit\`a che la prima persona estragga un asso \`e uguale alla probabilit\`a che la seconda persona estragga un asso.
\end{exmp}

\begin{defn}[Rapporto a favore di un evento]
Il rapporto a favore di un evento $A$ \`e definito come:
\[
\frac{P(A)}{P(A^{\compl})}
\]
\end{defn}
\begin{exmp}
``La Juve \`e data 10 a 2'' vuol dire che $\frac{P(A)}{P(A^{\compl})} = \frac{10}{2}$.
\[
\frac{P(A)}{P(A^{\compl})} = c \implies P(A) = c (1 - P(A)) \implies P(A) [1 + c] = c \implies P(A) = \frac{c}{1 + c}
\]
\end{exmp}

\begin{exmp}[3a dal libro]
Una compagnia di assicurazioni suddivide le persone in due classi: quelle che sono propense a incidenti e quelle che non lo sono.

La probabilit\`a che una persona propensa a incidenti faccia un incidente \`e $0{,}4$ percento, la probabilit\`a che una persona non propensa a incidenti faccia un incidente \`e $0{,}2$ percento. Il 30 percento della popolazione \`e propensa agli incidenti.

Qual \`e la probabilit\`a che un nuovo cliente abbia un incidente entro un anno. Non sappiamo se il cliente \`e propenso o meno agli incidenti.

$A_1 = $ ``il cliente ha un incidente entro l'anno''.

$A = $ ``il cliente \`e propenso agli incidenti''.
\[
P(A) = 30 \%
\]
\[
P(A_1 | A) = 0{,}4 \%
\]
\[
P(A_1 | A^{\compl}) = 0{,}2 \%
\]
\[
P(A_1) = P(A_1 | A) P(A) + P(A_1 | A^{\compl}) P(A^{\compl})
\]
Dopo un anno il cliente ha avuto un incidente. Qual \`e ora la probabilit\`a che \`e propenso agli incidenti?
\[
P(A | A_1)
\]
L'ideale \`e applicare Bayes, perch\'e aiuta a invertire evento condizionato ed evento condizionante.
\[
P(A | A_1) = \frac{P(A \cap A_1)}{P(A_1)} = \frac{P(A) P(A_1 | A)}{P(A_1)}
\]
\end{exmp}

Le ricerche a campionamento permettono di capire, sapendo che, ad esempio, non si \`e trovato l'aereo disperso in una certa zona, quanto vale la probabilit\`a che l'aereo disperso \emph{non sia} nella zona.

\begin{exmp}[3m dal libro, seconda edizione]
Tre tipi di lampadine. Tipo 1 ha 0,7 \% di probabilit\`a di durare $> 1000 h$, Tipo 2 ha 0,4 \% di probabilit\`a di durare $>$ di $1000 h$, tipo 3 ha 0,3 \% di probabilit\`a di durare $> 1000 h$.

Ho una scatola in cui il 20 \% delle lampadine sono di tipo 1, 30 \% di tipo 2 e le restanti di tipo 3.

Qual \`e la probabilit\`a che una lampadina scelta a caso duri pi\`u di mille ore.

$A =$ ``la lampadina scelta dura pi\`u di 1000 h''

$F_i =$ ``la lampadina scelta \`e del tipo $i$''
\begin{align*}
P(A | F_1) &= 0{,}7 \\
P(A | F_2) &= 0{,}4 \\
P(A | F_3) &= 0{,}3
\end{align*}
\begin{align*}
P(F_1) &= 0{,}2 \\
P(F_2) &= 0{,}3 \\
P(F_3) &= 0{,}5
\end{align*}
\[
P(A) = \sum_{i = 1}^{3} P(A | F_i) P(F_i)
\]
Sapendo che la lampadina funziona pi\`u 1000 h, qual \`e la probabilit\`a che sia di uno dei tre tipi?
\[
P(F_j | A) = \frac{P(A | F_j) P(F_j)}{P(A)}
\]
\end{exmp}

\section{Indipendenza probabilistica di due eventi}

Abbiamo due eventi $E$ e $F$. $P(F) > 0$, consideriamo il caso particolare in cui $P(E | F) = P(E)$. Sapere che si \`e verificato l'evento $F$ non cambia la probabilit\`a che si verifichi l'evento $E$. Vuol dire che $E$ \`e indipendente da $F$.

Poich\`e $\frac{P(E \cap F)}{P(F)} = P(E)$ vuol dire richiedere che $P(E \cap F) = P(E) P(F)$.

Se $P(E) > 0$ possiamo dire $P(F | E) = P(F)$.

\begin{defn}[Indipendenza probabilistica]
Due eventi E e F si dicono ``indipendenti'' se:
\[
P(E \cap F) = P(E) \ P(F)
\]
\end{defn}
La premessa \`e questa osservazione:
\begin{oss}
Se $P(F) > 0$, $E$ e $F$ sono indipendenti se e solo se
\[
P(E| F) = P(E)
\]
\end{oss}

\begin{exmp}
Lanciamo due monete. $S = \{ (T, T), (C, C), (T, C), (C, T) \}$. $E =$ ``la prima moneta d\`a Testa'', $F =$ ``la seconda moneta d\`a croce''.
\begin{align*}
E &= \{(T, T), (T, C)\}
F &= \{(C, C), (T, C)\}
\end{align*}
Abbiamo che $P(E) = \frac{2}{4}$, $P(F) = \frac{2}{4}$ e $P(E \cap F) = \frac{1}{4}$. \`E verificato che:
\[
P(E) P(F) = P(E F)
\]
$E$ ed $F$ sono quindi indipendenti.
\end{exmp}

\begin{defn}[Eventi indipendenti]
Gli eventi $E, F, G$ sono indipendenti se:
\begin{itemize}
    \item sono indipendenti a coppie:
    \begin{itemize}
        \item $P(E F) = P(E) P(F)$
        \item $P(E G) = P(E) P(G)$
        \item $P(F G) = P(F) P(G)$
    \end{itemize}
    \item $P(EFG) = P(E) P(F) P(G)$
\end{itemize}
\end{defn}

\begin{exmp}
$E =$ ``il primo lancio d\`a Testa''.
$F =$ ``il secondo lancio d\`a Testa''.
$G =$ ``i due lanci danno lo stesso risultato''.
\[
P(E F) = \frac{1}{4} = P(E) P(F) = \frac{1}{2} \frac{1}{2}
\]
\[
P(E G) = \frac{1}{4} = P(E) P(G) = \frac{1}{2} \frac{1}{2}
\]
\[
P(F G) = P(F) P(G)
\]
\[
P(E F G) = \frac{1}{4} \neq P(E) P(F) P(G) = \frac{1}{2} \frac{1}{2} \frac{1}{2}
\]
Non sono indipendenti.
\end{exmp}

\begin{prop}
Se $E$ e $F$ sono indipendenti, allora $E$ e $F^{\compl}$ sono indipendenti.
\end{prop}
\begin{proof}
Ipotesi: $P(E F) = P(E) P(F)$
Tesi: $P(E F^{\compl}) = P(E) P(F^{\compl})$

Abbiamo che $E \cap F^{\compl} = E \setminus (E \cap F)$

Quindi $P(E F^{\compl}) = P(E) - P(EF) = P(E) - P(E) P(F) = P(E) (1 - P(F)) = P(E) P(F^{\compl})$
\end{proof}
Con il complementare, l'unione e l'intersezione, dato F posso generare solo $\emptyset$, $S$ e $F^{\compl}$.

L'evento vuoto (impossibile) \`e indipendente da tutti. L'evento certo anche \`e indipendente da tutti gli altri.

Quindi date $E$, $F$ con $E$ ed $F$ indipendenti, $E$ \`e indipendente da ogni insieme generabile da $F$ con unione, intersezione e complementare.

Vale che se $E$, $F$ e $G$ sono indipendenti, allora che $E$ \`e indipendente da ogni evento costruito a partire da $F$ e $G$ tramite unione, intersezione e complementare.

\begin{defn}[Indipendenza generalizzata]
Dati $n$ eventi $E_1, E_2 \dots E_n$, questi sono detti indipendenti se comunque ne prendo una sottofamiglia, la probabilit\`a dell'intersezione \`e il prodotto delle singole probabilit\`a.
\begin{gather*}
\forall r \ge 2, \forall \seq{i}{1}{r} : 1 \le i_1 < i_2 < \ldots < i_r \le n \\ P \left( E_{i_1} \cap E_{i_2} \cap \ldots \cap E_{i_r} \right) = P(E_{i_1}) \cdot P(E_{i_2}) \cdot \ldots \cdot P(E_{i_r})
\end{gather*}
\end{defn}

Due lanci di un dado non hanno memoria, quindi non interferiscono fra loro.

Consideriamo un esperimento fatto da $n$ o $\infty$ sottoesperimenti che tra loro non interferiscono. Quando i sottoesperimenti sono identici, vengono detti \emph{prove}.

Se $P$ descrive l'esperimento, vale la seguente propriet\`a (ossia se $P$ \`e una buona descrizione dell'esperimento): eventi riferiti a sottoesperimenti diversi sono matematicamente indipendenti.

Posso avere un sottoesperimento con due esiti, 0 e 1. L'esito 0 viene chiamato insuccesso, 1 viene chiamato successo.

Prove = sottoesperimenti operativamente indipendenti e identici.

Consideriamo una successione (idealmente infinita) di prove dove si possono avere due risultati: ``succeso'' (1) e ``insuccesso'' (0). Il successo avviene con probabilit\`a $p$, l'insuccesso con probabilit\`a $q = 1 - p$.

\begin{exmp}
Lancio infinite volte una moneta, ``esce testa'' lo identifico con il successo, ``esce croce'' con l'insuccesso.

Oppure, gioco infinite volte al lotto. Viene estratta una pallina da un'urna di novanta palline numerate da 1 a 90. Scommetto ogni volta lo stesso numero, il successo \`e ``esce quel numero'', se non esce quel numero si realizza un insuccesso.

Calcoliamo la probabilit\`a che nelle prime $n$ prove ci sia sempre un successo.

Lo spazio di probabilit\`a non \`e numerabile, ha $2^n$ esiti possibili, e la cardinalit\`a di questo insieme \`e la cardinalit\`a del continuo. ``Spazio degli stati non numerabili''.

$P(\text{``nelle prime $n$ prove ho successo''}) = P( \bigcap_{i=1}^{n} F_i )$ con $F_i =$ ``ho successo alla prova $i$-esima.'' Eventi relativi a prove diverse devono essere matematicamente indipendenti. Ciascun evento $F_i$ \`e matematicamente indipendente dagli altri perch\'e riferiti a esperimenti operativamente indipendenti. Se sono indipendenti, la probabilit\`a dell'intersezione \`e il prodotto delle singole probabilit\`a.
\[
P \left( \bigcap_{i=1}^{n} \right) = \prod_{i=1}^{n} \overbrace{P(F_i)}^{p} = p^n
\]
Vediamo la probabilit\`a che ci sia almeno un successo nelle prime $n$ prove. $P($``almeno un successo nelle prime $n$-prove''$) = 1 -  P($``sempre insuccesso nelle prime $n$-prove''$)$.
\[
1 - P \left( \bigcap_{i=1}^{n} F_{i}^{\compl} \right) = 1 - \prod_{i=1}^{n} P(F_{i}^{\compl}) = 1 - q^n = 1 - (1 - p)^{n}
\]
Anche in questo caso ci riferiamo a prove operativamente indipendenti. Non \`e necessario pensare a una successione infinita di prove, questi esempi valgono anche nel caso di successioni finite.
\end{exmp}

C'\`e un teorema (che dobbiamo ancora vedere) che dice che:
\[
\lim_{n \to \infty} P(\text{``successo alle prime n-prove''}) = P(\text{``ho sempre successo''})
\]
Se $p$ \`e minore di 1, questo limite \`e 0. Quindi la probabilit\`a di avere sempre successo \`e nulla.

Consideriamo $n$ prove e vediamo la probabilit\`a di avere esattamente $k$ successi nelle $n$ prove (o nelle prime $n$ prove se ne faccio infinite).
\[
P(\text{``esattamente $k$ successi nelle prime $n$ prove''})
\]
Consideriamo prima la probabilit\`a che nelle prime $n$ prove ho che le prime $k$ prove sono successi, le restanti $n - k$ sono insuccessi.
\[
P(F_1 \cap F_2 \cap \dots \cap F_k \cap F_{k+1}^{\compl} \cap \dots \cap F_{n}^{\compl})
\]
Sono matematicamente indipendenti, quindi per indipendenza questo \`e uguale a:
\[
P(F_1) \cdot P(F_2) \cdot \ldots \cdot P(F_k) \cdot P(F_{k+1}^{\compl}) \cdot \ldots \cdot P(F_{n}^{\compl}) = p^{k} \cdot q^{n-k}
\]
Anche cambiando l'ordine dei successi, la probabilit\`a \`e sempre $p^k \cdot q^{n-k}$.

Fissato $A \subset \{ 1 \dots n \}$ con $|A| = k$, $P($``la prova $i$-esima \`e successo $\forall  i \in A$ e la prova $j$-esima \`e insuccesso $\forall  j \in \{1 \dots n \} \setminus A$''$)$.

\begin{proof}
\[
P \left( \left( \bigcap_{i \in A} F_i \right) \cap \left( \bigcap_{j \in B \setminus A} F_j^{\compl} \right) \right) = 
\prod_{i \in A} P(F_i) \cdot \prod_{j \in B \setminus A} P(F_{j}^{\compl}) = p^{|A|} \cdot q^{|B \setminus A|} = p^k q^{n-k}
\]
\end{proof}

Torniamo all'esempio di prima. $P($``nelle prime $n$ prove ho $k$ successi e $n-k$ insuccessi''$)$. \`E uguale a:
\[
P \left( \bigcup_{A \subset B, \abs{A} = k} E_A \right) =
\sum_{A \subset B, \abs{A} = k} P(E_A) = \sum_{A \subset B, \abs{A} = k} p^k \cdot q^{n - k} =
\binom{n}{k} \cdot p^k \cdot q^{n - k} =
\binom{n}{k} \cdot p^k \cdot {(1 - p)}^{n - k}
\]
Con $E_A$ l'evento ``le prove indicizzate da $A$ sono successi, le prove indicizzate da $B \setminus A$ sono insuccessi''.

\begin{prop}
Dato $F \subset S$ con $\prob{F} > 0$, allora $(S, \probcond{\cdot}{F})$ \`e uno spazio di probabilit\`a.

Cio\`e la mappa $E \to \probcond{E}{F}$ con $E \subset S$ \`e una funzione di probabilit\`a, ossia la funzione che ad un evento associa la probabilit\`a dell'evento condizionato $F$ \`e una funzione di probabilit\`a.
\end{prop}
\begin{proof}
Verifichiamo il primo assioma: $0 \le P(E|F) \le 1 \forall E \subset S$

$\probcond{E}{F} = \frac{\prob{E \cap F}}{\prob{F}}$ per definizione, essendo numeratore e denominatore maggiori di 0, anche $\probcond{E}{F}$ \`e maggiore di 0. Inoltre $E \cap F \subset F \implies \prob{E \cap F} \le \prob{F}$ per monotonia. Quindi $\frac{ \prob{E \cap F} }{\prob{F}} \le 1$.

Secondo assioma: $\probcond{S}{F} = 1$. Infatti $ \probcond{S}{F} = \frac{\prob{S \cap F}}{\prob{F}} = \frac{\prob{F}}{\prob{F}} = 1$

Terzo assioma: additivit\`a numerabile. Sia $(E_n)$ con $N \ge 1$ una successione di eventi a 2 a 2 disgiunti. Dobbiamo verificare che:
\[
\probcond{\bigcup_{n = 1}^{\infty} E_n}{F} = \sum_{n = 1}^{\infty} \probcond{E_n}{F}
\]
Al membro sinistro abbiamo:
\[
\probcond{\bigcup_{n = 1}^{\infty} E_n}{F} = \frac{\prob{\left( \bigcup_{n = 1}^{\infty} E_n \right) \cap F}}{\prob{F}} = \frac{ \prob{ \bigcup_{n = 1}^{\infty} (E_n \cap F)}}{\prob{F}}
\]
Gli eventi $E_n \cap F$ sono a due a due disgiunti. Posso applicare l'additivit\`a finita:
\[
\sum_{n = 1}^{\infty} \frac{\prob{E_n \cap F}}{\prob{F}} = \sum_{n=1}^{\infty} \probcond{E_n}{F}
\]
\end{proof}

Per questo, le propriet\`a che valgono per le probabilit\`a, valgono con le probabilit\`a condizionate.
\[
\probcond{E^{\compl}}{F} = 1 - \probcond{E}{F}
\]
\[
\probcond{A \cup B}{F} = \probcond{A}{F} + \probcond{B}{F} - \probcond{A \cap B}{F}
\]
\begin{fact}
Ho una funzione $P$. So che si \`e verificato l'evento $F$, quindi posso passare alla funzione $Q = \probcond{\cdot}{F}$. So poi che si \`e verificato l'evento $G$, quindi posso passare alla funzione $\probcond[Q]{\cdot}{G}$, che equivale a $\probcond{\cdot}{FG}$.
\end{fact}
\begin{proof}
\[
\probcond[Q]{E}{G} = \frac{\prob[Q]{E \cap G}}{\prob[Q]{G}} = \frac{\frac{\prob{E \cap G \cap F}}{\prob{F}}}{\frac{\prob{G \cap F}}{\prob{F}}} = \frac{\prob{E \cap G \cap F}}{\prob{G \cap F}} = \probcond{E}{FG}
\]
\end{proof}

Consideriamo un esperimento fatto da $n$ sottoesperimenti operativamente indipendenti. Supponiamo che il $k$-esimo sottoesperimento sia modellizzato (ossia, descritto) da uno spazio di probabilit\`a $(S_k, P_k) \forall k = 1 \ldots n$.

Nell'esempio del lancio del dado e della moneta truccata, il lancio del dado \`e descritto da $(S_1, P_1)$ con $S_1 = \{ 1, \ldots, 6 \}$ e $\prob[P_1]{E} = \frac{\abs{E}}{6}$ con $E \subset S_1$, e il lancio della moneta \`e descritto da $(S_2, P_2)$ con $S_2 = \{ T, C \}$ e $\prob[P_2]{E} = \frac{\abs{E}}{2}$ con $E \subset S_2$.

Supponiamo di avere $S_1, S_2, \ldots, S_k$ spazi numerabili. L'esperimento si modellizza con lo spazio di probabilit\`a $(S, P)$ dove $S = S_1 \times S_2 \times \ldots \times S_n = \{ (x_1, x_2, \ldots, x_n) : x_i \in S_i$ con $i = 1, \ldots, n \}$. $P$ \`e l'unica funzione di probabilit\`a definita su $\parts(S)$ tale che:
\[
\prob{\{(x_1, x_2, \ldots, x_n)\}} = \prob[P_1]{\{x_1\}} \cdot \prob[P_2]{\{x_2\}} \cdot \ldots \cdot \prob[P_n]{\{x_n\}}
\]
La funzione $P$ \`e unica se l'insieme su cui \`e definita \`e numerabile.

Eventi riferiti a sottoesperimenti diversi sono indipendenti in $(S, P)$.

\begin{esercizio}
Dimostrare che se $\abs{S_k} < \infty \forall  k = 1, \ldots, n$ e $(S_k, P_k)$ ha esiti equiprobabili $\forall  k = 1 \dots n$, allora $(S,P)$ ha esiti equiprobabili.
\end{esercizio}

% SONO ARRIVATO QUI A SISTEMARE

Supponiamo di avere $n$ sottoesperimenti.
\[
(S_1, P_1), \ldots, (S_n, P_n)
\]
Ogni $S_i$ \`e numerabile.
\[
S = S_1 \times \ldots \times S_n = \{ (\seq{s}{1}{n}) : s_i \in S_i \text{ con } i = 1, \ldots, n \}
\]
$P$ \`e l'unica misura di probabilit\`a tale che $\prob{\{(\seq{s}{1}{n})\}} = \prob[P_1]{\{s_1\}} \cdot \ldots \cdot \prob[P_n]{\{s_n\}}$.

\begin{prop}
Se $\abs{S_1}, \ldots, \abs{S_n} < + \infty$ e $(S_i, P_i)$  ha esiti equiprobabili $\forall  i = 1, \ldots, n \implies (S, P)$ ha esiti equiprobabili.
\end{prop}
\begin{proof}
\begin{align*}
\prob{\{s_1 \ldots s_n\}} &= \prob[P_1]{\{s_1\}} \cdot \ldots \cdot \prob[P_n]{\{s_n\}} \tag{per definizione} \\
&= \frac{1}{\abs{S_1}} \cdot \frac{1}{\abs{S_2}} \cdot \ldots \cdot \frac{1}{\abs{S_n}} \tag{avendo ciascun esperimento esiti equiprobabili} \\
&= \frac{1}{\abs{S}}
\end{align*}
\end{proof}

Considero $n$ prove, ciascuna con esito di tipo successo/insuccesso (1/0). Supponiamo che $p$ sia la probabilit\`a che si verifichi il successo in una singola prova. La prova $i$-esima \`e descritta dallo spazio di probabilit\`a $(S_i, P_i)$. $S_i = \{0,1\}$, e $\prob[P_i]{\{1\}} = p$. $\prob[P_i]{\{0\}} = 1 - p = q$.
\[
S = S_1 \times \dots \times S_n = \{0,1\} = \{ (\seq{x}{1}{n}) : x \in \{ 0, 1 \}\}
\]
$P$ \`e l'unica funzione di probabilit\`a tale che $\prob{\{(\seq{x}{1}{n})\}} = \prod_{i = 1}^{n} \prob[P_i]{\{x_i\}} = p^m \cdot q^{m-n}$ con $m $ a indicare il numero degli 1 presenti in $(\seq{x}{1}{n})$.

\begin{esercizio}
Lancio 3 volte un dado truccato. Ciascuna faccia esce con probabilit\`a:
\begin{align*}
1 &\to \frac{1}{2} \\
2 &\to \frac{1}{10} \\
& \vdots \\
6 &\to \frac{1}{10}
\end{align*}
Calcolare la probabilit\`a che esca sempre lo stesso valore.
\begin{gather*}
S = \left\{ 1, \ldots, 6 \}^3 = \{ (x_1, x_2, x_3) : x_i \in \{1, \ldots, 6\} \right\} \\
\prob{\{(x_1, x_2, x_3)\}} = \prob[\hat{P}]{\{x_1\}} \cdot \prob[\hat{P}]{\{x_2\}} \cdot \prob[\hat{P}]{\{x_3\}}
\end{gather*}
L'evento ``esce sempre lo stesso valore'' \`e:
\[
E = \left\{ (x, x, x) : x \in \{1, \ldots, 6\} \right\} \subset S
\]
La probabilit\`a di $E$ \`e:
\begin{align*}
\prob{E} &= \prob{\{(1,1,1)\}} + \prob{\{(2,2,2)\}} + \prob{\{(3,3,3)\}} + \\
&+ \prob{\{(4,4,4)\}} + \prob{\{(5,5,5)\}} + \prob{\{(6,6,6)\}} = \\
&= \left( \frac{1}{2} \right)^3 + 5 \cdot \left( \frac{1}{10} \right)^3 = \frac{1}{8} + \frac{1}{200}
\end{align*}
\end{esercizio}
















\newpage

\section{Esercizi}


\chapter{esercizi}

\begin{esercizio}
Ogni sottoinsieme di un linguaggio regolare \`e regolare?
\end{esercizio}

No, il linguaggio dato da tutto $\Sigma^{\star}$ \`e regolare, e ogni linguaggio \`e sottoinsieme di questo.
Ma sappiamo che esistono linguaggi non regolari, quindi almeno uno fra questi linguaggi non lo \`e.
Deve essere $\abs{\Sigma} > 1$.

\begin{esercizio}
Se $L_1$ e $L_2$ sono linguaggi non regolari, la loro intersezione pu\`o essere regolare.
Dare un esempio in cui questo succede, e uno in cui non succede.
\end{esercizio}

Se l'intersezione fra due linguaggi non regolari (come $a^nb^n$ e $b^na^n$) \`e l'insieme vuoto o $\{\emptystring\}$, abbiamo trovato due linguaggi per cui l'intersezione \`e regolare.
Prendendo invece $L_1 = L_2$, l'intersezione \`e uguale a entrambi, e non \`e regolare.

\begin{esercizio}
Se $L_1$ \`e regolare e $L_2$ non lo \`e, l'unione non necessariamente \`e un linguaggio regolare.
Dare un esempio in cui questo succede, e uno in cui non succede.
\end{esercizio}

Se $L_2 \subset L_1$, l'unione \`e necessariamente un linguaggio regolare.
Un esempio \`e quando $L_1 = \{0,1\}^{\star}$, e $L_2$ \`e un qualsiasi sottoinsieme proprio di $L_1$ che identifichi un linguaggio non regolare (come il solito $a^n b^n : n \ge 0$).
Se invece $L_1 \subset L_2$, l'unione \`e un linguaggio non regolare.
Un esempio \`e $L_1 = \{ a^n b^n : 0 \le n \le k \}$, e $L_2 = \{ a^n b^n : n \ge 0 \}$.

\begin{esercizio}
$F$ \`e un linguaggio finito, $L \setminus F$ \`e un linguaggio regolare. $L$ \`e regolare?
\end{esercizio}

S\`i: $L = (L \setminus F) \cup (F \cap L)$.
Sappiamo che l'insieme dei linguaggi regolari \`e chiuso rispetto alle operazioni insiemistiche, e che ogni linguaggio finito \`e regolare.
$F \cap L$ \`e un linguaggio regolare, perch\'e sicuramente $\abs{F \cap L} \le \abs{F}$,   quindi anche l'intersezione \`e un linguaggio finito (e quindi regolare).
Quindi $L$ \`e l'unione di due linguaggi regolari, che \`e a sua volta regolare.

\begin{esercizio}
Dimostrare che il linguaggio $L = \{ w : w \in \{a,b\}^{\star} \land n_a(w) \le n_b(w) + 2 \}$ non \`e regolare.
\end{esercizio}

Consideriamo un $k$ generico, e la parola $a^k b^{k+2}$, che appartiene al linguaggio.
Per ogni scomposizione $a^r a^s a^t b^{k+2}$, con $s \ge 1$ e $r + s + t = k$, esiste un indice $i \ge 0$ per cui $r + i \cdot s + t > k + 2$, che ci porta alla parola $a^r a^{i \cdot s} a^t b^{k+2}$ che \emph{non} appartiene al linguaggio.
\`E sufficiente prendere $i$ tale che $i > \frac{s + 2}{s}$.

\begin{esercizio}
Dimostrare, usando il pumping lemma, che il linguaggio $L = \{ a^i b^j c^k : i,j,k \ge 0 e k = i * j \}$ non \`e regolare.
\end{esercizio}

Consideriamo un $n$ positivo generico, e la parola $a^n b^n c^{n^2}$ nel linguaggio.
Per ogni scomposizione $a^r a^s a^t b^n c^{n^2}$, con $s \ge 1$ e $r + s + t = n$, esiste un indice $i \ge 0$ per cui $(r + i \cdot s + t) \cdot n > n^2$, portando alla parola $a^{r + i \cdot s + t} b^n c^{n^2}$ che non appartiene al linguaggio.
\`E sufficiente prendere $i > 1$, che verifica la disuguaglianza:
\[
i > \frac{n^2 - t \cdot n - r \cdot n}{s \cdot n} = \frac{n - t - r}{s} = \frac{s}{s} = 1 
\]

\begin{esercizio}
Scrivere la grammatica del linguaggio $L = \{ a x a : x \in \{0,1\}^{\star} e a \in \{0,1\}^{\star} \}$.
\end{esercizio}

La grammatica di $L$ \`e:
\begin{align*}
G :& S \to 0 R 0 \langor 1 R 1 \\
& R \to 0R \langor 1R \langor \emptystring
\end{align*}

\begin{esercizio}
Scrivere la grammatica del linguaggio $L = \{ w # x : w, x \in \{0,1\}^{\star} \text{ e } w^R \text{ \`e una sottostringa di } x \} = \{ w # u w^R v : w, u, v \in \{0,1\}^{\star} \}$.
\end{esercizio}




\end{document}