\section{13 ottobre 2014}

\begin{enumerate}
    \item (2.10) Il 60 per cento degli studenti di una scuola non indossano n\'e un anello n\'e una collana. Il 20 per cento porta un anello e il 30 per cento una collana. Se scegliamo uno studente a caso, qual \`e la probabilit\`a che indossi:
    \begin{enumerate}
        \item un anello o una collana? $P(A \cup C) = 1 - P((A \cup C)^{\mathsf{c}}) = 1 - 0{,}6 = 0{,}4$
        \item un anello e una collana? $P(A \cup C) = P(A) + P(C) - P(A \cap C) \Rightarrow P(A \cap C) = P(A) + P(C) - P(A \cup C) = 0{,}2 + 0{,}3 - 0{,}4 = 0{,}1$
    \end{enumerate}
    \item (2.12) Una scuola elementare offre corsi di tre lingue: uno dispagnolo, uno di francese e uno di tedesco. Ognuna di queste classi \`e aperta a ognuno dei 100 studenti della scuola. Nella classe di spagnolo ci sono 28 studenti, 26 in quella di francese e 16 in quella di tedesco. 12 studenti frequentano sia il corso di spagnolo che di francese, 4 sia quello di spagnolo che quello di tedesco e 6 sia quello di francese che di tedesco. Inoltre ci sono due studenti che frequentano tutti e tre i corsi di lingua.
    \begin{enumerate}
        \item Se scegliamo uno studente a caso, qual \`e la probabilit\`a che non segua alcun corso di lingua? $P(S^{\mathsf{c}} \cap F^{\mathsf{c}} \cap T^{\mathsf{c}}) = P((S \cup F \cup T)^{\mathsf{c}}) = 1 - P(S \cup F \cup T)$. Usiamo quindi il principio di inclusione ed esclusione per trovare $P(S \cup F \cup T) = P(S) + P(F) + P(T) - P(SF) - P(ST) -P(FT) + P(SFT) = \frac{28}{100} + \frac{26}{100} + \frac{16}{100} - \frac{12}{100} - \frac{4}{100} - \frac{6}{100} + \frac{2}{100} = \frac{1}{2}$.
        \item Se scegliamo uno studente a caso, qual \`e la probabilit\`a che frequenti un solo corso di lingua?
        \item Se scegliamo a caso 2 studenti, qual \`e la probabilit\`a che almeno 1 segua un corso di lingua?
    \end{enumerate}
    \item (2.15) Se supponiamo che tutte le $\binom{52}{5}$ mani di poker siano equiprobabili, qual \`e la probabilit\`a che venga servito:
    \begin{enumerate}
        \item un colore? (Una mano si definisce un colore se tutte e 5 le carte hanno lo stesso seme) Le possibili combinazioni di colore sono $4 \times \binom{13}{5}$.
        \item una coppia? (Questo capiter\`a quando le carte saranno $a, a, b, c, d$ con $a, b, c,$ e $d$ con valori distinti tra loro) Le possibili coppie sono $\frac{52 \times 48 \times 44 \times 40 \times 3}{5!}$, ossia il numero di modi in cui posso scegliere 4 carte con valori distinti per il numero di modi in cui posso scegliere la quinta carta in modo che abbia lo stesso valore della prima.
        \item una doppia coppia? (Questo capiter\`a quando le carte saranno $a, a, b, b, c$ con $a, b,$ e $c$ con valori distinti tra loro) Le possibili doppie coppie sono $\frac{52 \times 48 \times 44 \times 3^2}{5!} $, ossia il numero di modi in cui posso scegliere 3 carte con valori distinti per il numero di modi in cui posso scegliere la quarta e la quinta carta in modo che siano uguali rispettivamente alla prima ed alla seconda.
        \item un tris? (Questo capiter\`a quando le carte saranno $a, a, a, b, c$ con $a, b$ e $c$ con valori distinti tra loro)  I possibili tris sono $\frac{52 \times 48 \times 44 \times 6}{5!}$, ossia il numero di modi in cui posso scegliere 3 carte con valori distinti per il numero di modi in cui posso scegliere la quarta e la quinta carta in modo che siano uguali alla prima.
        \item un poker? (Questo capiter\`a quando le carte saranno $a, a, a, a, b$ con $a$ e $b$ con valori distinti tra loro) I possibili poker sono $\frac{52 \times 48 \times 6}{5!}$, ossia il numero di modi in cui posso scegliere 2 carte con valori distinti per il numero di modi in cui posso scegliere la terza, la quarta e la quinta carta in modo che siano uguali  alla prima.
    \end{enumerate}
\end{enumerate}