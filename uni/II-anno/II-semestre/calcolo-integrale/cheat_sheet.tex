\documentclass[11pt,a4paper,twoside]{report}
% draft mode prevents link generation

\usepackage[italian]{babel}
% per date e ToC in italiano
\usepackage{hyperref}
\usepackage{nameref}
\usepackage[table]{xcolor}
\usepackage{amsmath}        % matematica
\usepackage{amsthm}         % teoremi
\usepackage{amssymb}        % simboli
\usepackage{amsfonts}       % font matematicosi
\usepackage{mathrsfs}
\usepackage{mathtools}
\usepackage{thmbox}
\usepackage{centernot}      % per semplificare
\usepackage{pgfplots}       % per i grafici
\pgfplotsset{compat=1.5}    % consigliata compatibilita'
                            % con la versione 1.5

\usepackage{fullpage}       % troppo margine normalmente
\usepackage{parskip}        % preferisco spazio fra i paragrafi
                            % all'indentazione sulla prima riga
\usepackage{fancyhdr}
\usepackage[makeroom]{cancel}

% algorithms
\usepackage{algorithm}
\usepackage{algpseudocode}

\floatname{algorithm}{Algoritmo}
\renewcommand{\algorithmicrequire}{\textbf{Input:}}
\renewcommand{\algorithmicensure}{\textbf{Output:}}

\theoremstyle{plain}
\newtheorem{axiom}{Assioma}
\newtheorem{theorem}{Teorema}[section]
\newtheorem{lem}[theorem]{Lemma}
\newtheorem{prop}[theorem]{Proposizione}
\newtheorem{cor}[theorem]{Corollario}

\theoremstyle{definition}
\newtheorem{defn}{Definizione}[section]
\newtheorem{esercizio}{Esercizio}

\theoremstyle{remark}
\newtheorem{remark}{Remark}
\newtheorem{oss}{Osservazione}
\newtheorem{caso}{Caso}
\newtheorem{fact}{Fatto}
\newtheorem{claim}{Claim}

\newenvironment{exmp}[1][]
{ \par\medskip\textbf{Esempio\ #1\unskip\,:}\\*[1pt]\rule{\textwidth}{0.4pt}\\*[1pt] }
{ \\*[1pt]\rule{\textwidth}{0.4pt}\medskip }

\newenvironment{smallpmatrix}
{\left( \begin{smallmatrix}}
{\end{smallmatrix} \right)}

\newenvironment{completepmatrix}[1]
{\left(\begin{array}{*{#1}{c}|c}}
{\end{array}\right)}

\pagestyle{fancyplain}
\fancyhead{} % clear all header fields
\fancyfoot{} % clear all footer fields
\fancyfoot[C]{\today}
\fancyfoot[LE,RO]{\thepage}
\renewcommand{\headrulewidth}{0pt}
\renewcommand{\footrulewidth}{0pt}

\usetikzlibrary{shapes,arrows,calc,fit,backgrounds,positioning,automata}

\newcommand{\parity}[1]{\mathcal{Par}\left( { #1 } \right)}
% Languages & Automata
\newcommand{\langs}[1]{\mathcal{L}\left( {#1} \right)}
\newcommand{\lang}[1]{L\left( {#1} \right)}
\newcommand{\langor}{\, | \,}
\newcommand{\kleenestar}{\star}
\newcommand{\emptystring}{\varepsilon}

\newcommand{\bigo}[1]{O\left( {#1} \right)}
\newcommand{\probp}{\code{P}}
\newcommand{\probnp}{\code{NP}}
\newcommand{\probnpc}{\code{NP-C}}
\newcommand{\probexp}{\code{EXP}}
\newcommand{\covers}{<\!\cdot}      % simbolo di 'copre'
\newcommand{\scalar}{\centerdot}
\newcommand{\compl}{\mathsf{c}}     % C complementare
\newcommand{\dotcup}{\mathaccent\cdot\cup}  % unione disgiunta
\newcommand{\lateralsx}[1]{{}_{#1}\!\sim}   % equivalenza sinistra
\newcommand{\lateraldx}[1]{\sim_{#1}}       % equivalenza destra
\newcommand{\mcm}{\operatorname{mcm}}       % mcm
\newcommand{\mcd}{\operatorname{MCD}}       % MCD
\newcommand{\pow}[1]{<\!{#1}\!>}            % gruppo delle potenze (<a>)
\newcommand{\abs}[1]{\left\lvert{#1}\right\rvert}   % valore assoluto
\newcommand{\intsup}[1]{\left\lceil{#1}\right\rceil}   % intero superiore
\newcommand{\intinf}[1]{\left\lfloor{#1}\right\rfloor}   % intero inferiore
\newcommand{\norm}[1]{\left\lVert{#1}\right\rVert}  % norma di un vettore
\newcommand{\divides}{\bigm|}       % simbolo di 'divide'
\newcommand{\definition}{\overset{\underset{\mathrm{def}}{}}{=}}    % uguale con 'definizione' sopra
\newcommand{\supop}{\vee}           % simbolo dell'operatore sup
\newcommand{\infop}{\wedge}         % simbolo dell'operatore inf
\newcommand{\subgroupset}{\mathscr{S}}      % insieme dei sottogruppi (S corsiva)
\newcommand{\solutions}{\mathscr{S}}
\newcommand{\parts}[1]{\mathbb{P} \left( {#1} \right)}             % insieme delle parti
\newcommand{\naturals}{\mathbb{N}}          % insieme dei naturali
\newcommand{\integers}{\mathbb{Z}}          % insieme degli interi
\newcommand{\rationals}{\mathbb{Q}}         % insieme dei razionali
\newcommand{\reals}{\mathbb{R}}             % insieme dei reali
\newcommand{\complexes}{\mathbb{C}}         % insieme dei complessi
\newcommand{\field}{\mathbb{K}}             % simbolo di campo generico
\newcommand{\subfield}{\mathbb{F}}          % simbolo di sottocampo generico
\newcommand{\matrices}{\mathfrak{M}}        % insieme delle matrici
\newcommand{\nullelement}{\underline{0}}    % elemento neutro
\newcommand{\var}{\operatorname{Var}}       % varianza 
\newcommand{\seq}[3]{{#1}_{#2},\ldots,\,{#1}_{#3}}
\newcommand{\prob}[2][P]{#1 \left( {#2} \right)}
\newcommand{\probcond}[3][P]{#1 \left( #2 | #3 \right)}
\newcommand{\expect}[1]{E \left( #1 \right)}
\newcommand{\mass}[2][X]{p_{#1} \left( #2 \right)}
\newcommand{\distr}[2][X]{F_{#1} \left( #2 \right)}
\newcommand{\image}[1]{Im_{#1}}
\newcommand{\cov}{\operatorname{Cov}}
\newcommand{\detname}{\operatorname{det}}
\newcommand{\sgn}{\operatorname{sgn}}
\renewcommand{\det}[1]{\detname \left( {#1} \right)}
\newcommand{\algcompl}[2][A]{{\mathcal{#1}}_{#2}}
\newcommand{\agg}[1]{\operatorname{Agg} \left( {#1} \right)}
\newcommand{\id}{id}
\newcommand{\sfrac}[2]{{#1}/{#2}}
\newcommand{\requiv}[1][]{\ \varepsilon_{#1} \ }
\newcommand{\abslog}[1]{\log \left( \abs{#1} \right)}
\newcommand{\substitute}[2]{\left. {#1} \; \right|_{#2}}
\newcommand{\increment}[1]{{\left. {#1} \; \right|}}
\newcommand{\adj}{\sim}
\newcommand{\code}[1]{\texttt{#1}}
\newcommand{\isa}{\code{is-a}}
% intervalli aperti e chiusi
\newcommand{\ooint}[2]{\left( {#1}, {#2} \right)}
\newcommand{\coint}[2]{\left[ {#1}, {#2} \right)}
\newcommand{\ocint}[2]{\left( {#1}, {#2} \right]}
\newcommand{\ccint}[2]{\left[ {#1}, {#2} \right]}
\newcommand{\stereotipe}[1]{\ll \code{#1} \gg}

\renewcommand{\labelitemi}{$-$}
\let\oldforall\forall
\renewcommand{\forall}{\ \oldforall \ }
\let\oldexists\exists
\renewcommand{\exists}{\ \oldexists \ }
\renewcommand{\implies}{\Rightarrow}
\renewcommand{\iff}{\Leftrightarrow}

\makeatletter
\let\orgdescriptionlabel\descriptionlabel\renewcommand*{\descriptionlabel}[1]{%
\let\orglabel\label
\let\label\@gobble
\phantomsection
\edef\@currentlabel{#1}%
%\edef\@currentlabelname{#1}%
\let\label\orglabel
\orgdescriptionlabel{#1}%
}
\makeatother

\makeatletter
\providecommand*{\diff}%
    {\@ifnextchar^{\DIfF}{\DIfF^{}}}
\def\DIfF^#1{%
    \mathop{\mathrm{\mathstrut d}}%
        \nolimits^{#1}\gobblespace}
\def\gobblespace{%
        \futurelet\diffarg\opspace}
\def\opspace{%
    \let\DiffSpace\!%
    \ifx\diffarg(%
        \let\DiffSpace\relax
    \else
        \ifx\diffarg[%
            \let\DiffSpace\relax
        \else
            \ifx\diffarg\{%
                \let\DiffSpace\relax
            \fi\fi\fi\DiffSpace}

\providecommand*{\deriv}[1][]{%
    \frac{\diff}{\diff #1}}
%\providecommand*{\deriv}[2][]{%
%    \frac{\diff^{#1}}{\diff #2^{#1}}}
\providecommand*{\fderiv}[3][]{%
    \frac{\diff^{#1}#2}{\diff #3^{#1}}}
\providecommand*{\pderiv}[3][]{%
    \frac{\partial^{#1}#2}%
        {\partial #3^{#1}}}

\newcommand{\dx}[1][x]{\, \diff {#1}}

\author{Michele Laurenti \\ \href{mailto:asmeikal@me.com}{asmeikal@me.com}}
\date{\today}

\renewcommand*\rmdefault{ppl}

\begin{document}

\chapter{Integrali}

Con il simbolo di \emph{integrale} si indicano tutte le primitive di una funzione $f(x)$:
\[
\int f(x) \dx = F(x) + c
\]
Una primitiva di $f(x)$ verifica la propriet\`a $[F(x)]' = f(x)$, ossia
la sua derivata \`e $f(x)$.
Il simbolo $\diff x$ significa ``rispetto a $x$'', e indica la variabile 
che si sta integrando.

\emph{In un intervallo}, le primitive di una funzione differiscono tutte 
di una costante. Quindi se $F(x)$ \`e una primitiva di $f(x)$, scrivendo
$F(x) + c$ stiamo indicando \emph{tutte} le primitive di $f(x)$ 
(nell'intervallo di integrazione).

\emph{L'integrale \`e l'operazione inversa della derivata.}

\section{Regole di derivazione}

Dalle regole di derivazione si ricavano diverse regole di integrazione 
facilmente applicabili.

La prima regola da ricordare \`e che la derivata \`e \emph{lineare}:
\[
\left[ a \cdot f(x) + b \cdot g(x) \right]' =
a \cdot f'(x) + b \cdot g'(x)
\]

La derivata di una costante $c$ \`e sempre 0.

La derivata di una potenza di $x$ \`e:
\[
\left[ x^{\alpha} \right]' = \alpha \cdot x^{\alpha - 1}
\]
Chiaramente se $\alpha = 1$, la derivata di $x^{\alpha}$ \`e $x^0 = 1$.

La derivata di un prodotto di funzioni \`e:
\[
\left[ F(x) \cdot G(x) \right]' = 
f(x) \cdot G(x) + F(x) \cdot g(x)
\]
Indichiamo con $f(x)$ la derivata di $F(x)$, e con $g(x)$ la derivata 
di $G(x)$.

La derivata del rapporto di due funzioni \`e:
\[
\left[ \frac{F(x)}{G(x)} \right]' =
\frac{f(x) \cdot G(x) - F(x) \cdot g(x)}{\left(G(x)\right)^2}
\]
Da questa regola si ricava anche una regola per la derivata del
reciproco di una funzione:
\[
\left[ \frac{1}{G(x)} \right]' =
- \frac{g(x)}{\left(G(x)\right)^2}
\]
\`E come se avessimo applicato la regola della derivata del rapporto, 
ponendo $F(x) = 1$, e quindi $f(x) = 0$.

La derivata di una funzione composta \`e:
\[
\left[ F \left(G(x) \right) \right]' =
f \left( G(x) \right) \cdot g(x)
\]
Si legge come ``la derivata della funzione $F(x)$ \emph{calcolata} 
in $G(x)$, per la derivata della funzione $G(x)$''.

\section{Metodo di sostituzione}
La regola di derivazione delle funzioni composte ci dice questo:
\[
\left[ F \left(G(x) \right) \right]' =
f \left( G(x) \right) \cdot g(x)
\]
Quindi, integrando entrambi i membri (e tenendo a mente che l'integrale
\`e l'operazione inversa della derivata):
\[
\int f \left( G(x) \right) \cdot g(x) \dx =
F \left(G(x) \right) + C
\]
Tipicamente si scrive cos\`i:
\begin{align*}
\int f \left( G(x) \right) \cdot g(x) \dx &=
\int \substitute{f (y) \dx[y]}{y = G(x) ,\, \dx[y] = g(x) \dx} = \\
&= \substitute{F(y) + C}{y = G(x)} = \\
&= F \left( G(x) \right) + C
\end{align*}
Bisogna riconoscere una funzione $G(x)$, composta per una funzione $f(x)$,
e la sua derivata $g(x)$. $G(x)$ diventa la nuova variabile di integrazione
$y$, e $g(x) \dx$ diventa la nuova $\diff y$.

\section{Integrali per parti}

La regola di derivazione del prodotto ci dice:
\[
\left[ F(x) \cdot G(x) \right]' = 
f(x) \cdot G(x) + F(x) \cdot g(x)
\]
Possiamo riscriverla cos\`i:
\[
f(x) \cdot G(x) = 
\left[ F(x) \cdot G(x) \right]' - F(x) \cdot g(x)
\]
Integrando entrambi i membri (e tenendo a mente che l'integrale \`e
l'operazione inversa della derivata):
\[
\int \left( f(x) \cdot G(x) \right) \dx = F(x) \cdot G(x) - \int \left( F(x) \cdot g(x) \right) \dx
\]
Per risolvere un integrale per parti, bisogna riconoscere due funzioni:
una funzione $f(x)$ che si sa integrare, e una funzione $G(x)$ che si sa
derivare.

\section{Integrali razionali}

Gli integrali razionali sono integrali del tipo:
\[
\int \frac{p(x)}{q(x)}
\]
dove $p(x)$ e $q(x)$ sono funzioni razionali (ossia, polinomi).

Se il grado di $p(x)$ \`e maggiore o uguale al grado di $q(x)$,
va fatta una divisione fra polinomi. Il grado di una funzione 
razionale \`e l'esponente maggiore delle $x$ nella funzione.

Un esempio di divisione fra polinomi:
\[
\int \frac{x^2}{x - 4} \dx
\]
Il grado al numeratore \`e maggiore del grado al numeratore, quindi
possiamo dividere:
\begin{center}
\begin{tabular}{rrr|r}
$x^2$ & & & $x - 4$ \\ \cline{4-4}
$-x^2$ & $+4 \, x$ & & $x + 4$ \\ \cline{1-2}
$0$ & $4 \, x$ & & \\
& $- 4 \, x$ & $+ 16$ & \\ \cline{2-3}
& $0$ & $16$ & 
\end{tabular}
\end{center}
Il risultato della divisione \`e quindi:
\[
\int \left[ x + 4 + \frac{16}{x-4} \right] \dx 
\]
Nel caso generale, abbiamo una funzione $\frac{p(x)}{q(x)}$, con il grado 
di $p(x)$ maggiore o uguale al grado di $q(x)$. La divisione sar\`a:
\begin{center}
\begin{tabular}{c|r}
$p(x)$ & $q(x)$ \\ \cline{2-2}
$\vdots$ & $h(x)$ \\ \cline{1-1}
$r(x)$ & 
\end{tabular}
\end{center}
E il risultato, quindi:
\[
h(x) + \frac{r(x)}{q(x)}
\]
Ora il grado di $r(x)$ \`e minore del grado di $q(x)$ (grazie all'algebra).

Gli integrali di funzioni razionali sono \emph{sempre} combinazioni 
lineari di polinomi, logaritmi e arcotangenti.

\subsection{Integrali razionali di secondo grado con $\Delta < 0$}

Consideriamo integrali del tipo:
\[
\int \frac{\diff x}{x^2 \pm b \, x + c}
\]
In cui l'equazione di secondo grado al denominatore non ha soluzioni reali.
Se il coefficiente di $x^2$ non \`e 1, possiamo sempre riscriverlo in questo
modo:
\[
\int \frac{\diff x}{a \, x^2 + b \, x + c} = 
\frac{1}{a} \int \frac{\diff x}{x^2 + \frac{b}{a} \, x + \frac{c}{a}}
\]
Per risolvere l'integrale, riscriviamolo come:
\[
\int \frac{\diff x}{\left( x \pm \sqrt{\frac{b}{2}} \right)^2 + 
\left( c - \frac{b}{2} \right)}
\]
\emph{Attenzione:} il segno di $\sqrt{\frac{b}{2}}$ concorda con il segno 
di $b \, x$.
Poniamo $d = c - \frac{b}{2}$, per semplicit\`a, e facciamo la 
sostituzione $y = x \pm \sqrt{\frac{b}{2}}$ (e $\diff y = \diff x$).
\[
\int \frac{\diff y}{y^2 + d}
\]
Vogliamo che questo integrale ci dia un'arcotangente. Per ottenerlo, dobbiamo
avere un 1 al posto della $d$. Mettiamo quindi in evidenza in questo modo:
\[
\frac{1}{d} \int \frac{\diff y}{\frac{y^2}{d} + 1}
\]
Ora, sostituiamo $z^2 = \frac{y^2}{d}$, ossia $z = \frac{y}{\sqrt{d}}$, 
e quindi $\diff y = \sqrt{d} \dx[z]$, ottenendo:
\[
\frac{\sqrt{d}}{d} \int \frac{\diff z}{z^2 + 1} =
\frac{1}{\sqrt{d}} \int \frac{\diff z}{z^2 + 1}
\]
La soluzione \`e, quindi:
\begin{align*}
\frac{1}{\sqrt{d}} \, \arctan \left( z \right) + C &=
\frac{1}{\sqrt{d}} \, \arctan \left( \frac{y}{\sqrt{d}} \right) + C = \\
&= \frac{1}{\sqrt{d}} \, \arctan \left( \frac{x \pm \sqrt{\frac{b}{2}}}{\sqrt{d}} \right) + C = \\
&= \frac{1}{\sqrt{c - \frac{b}{2}}} \, \arctan \left( \frac{x \pm \sqrt{\frac{b}{2}}}{\sqrt{c - \frac{b}{2}}} \right) + C
\end{align*}


\chapter{Integrali definiti}

\section{Integrali impropri}

\chapter{Equazioni differenziali}

\chapter{Serie}

\section{Criteri di convergenza}

Si parla di convergenza assoluta quando vale:
\[
\lim_{N \to \infty} \sum_{n = n_0}^{N} \abs{a_n} = L
\]
Si parla di convergenza semplice quando vale:
\[
\lim_{N \to \infty} \sum_{n = n_0}^{N} a_n = L
\]
Le serie a termini di segno alterno spesso convergono semplicemente,
senza convergere assolutamente. Le serie a termini di segno concorde
se convergono semplicemente convergono anche assolutamente, e viceversa.

Si parla di divergenza quando il limite o non esiste, o tende all'infinito.

\subsection{Criterio del confronto}

Consideriamo due serie $\sum a_n$ e $\sum b_n$, e supponiamo valga questo:
\[
a_n \le b_n \forall n \ge n_0
\]
Vale, di conseguenza, questo:
\[
\sum_{n = n_0}^{\infty} a_n \le \sum_{n = n_0}^{\infty} b_n
\]
Il criterio del confronto ci dice, in questo caso, che:
\begin{itemize}
    \item se $a_n$ diverge, anche $b_n$ diverge;
    \item se $b_n$ converge, anche $a_n$ converge.
\end{itemize}
Se $b_n$ diverge, non sappiamo nulla su $a_n$, e se $a_n$ converge, non 
sappiamo nulla su $b_n$.

\subsection{Criterio integrale}

\subsection{Criterio asintotico}

Consideriamo due serie $\sum a_n$ e $\sum b_n$. Supponiamo valga quanto
segue:
\[
\lim_{n \to \infty} \frac{a_n}{b_n} = L \in \ooint{0}{\infty}
\]
Se questa ipotesi \`e verificata, alla le due serie $\sum a_n$ e $\sum b_n$ 
o convergono entrambe o divergono entrambe.

\subsection{Criterio del rapporto}

Consideriamo una serie $\sum a_n$ tale per cui $a_n \ge 0$. Studiamo il 
limite per $n$ che tende all'infinito del rapporto fra il termine $n$-esimo
e il termine che lo precede. Vale questo:
\[
\lim_{n \to \infty} \frac{a_{n+1}}{a_n} =
\begin{cases}
L < 1 &\implies \text{ la serie converge} \\
L > 1 &\implies \text{ la serie diverge} \\
L = 1 &\implies \text{ il criterio non dice niente sulla serie}
\end{cases}
\]

\subsection{Criterio della radice}

Consideriamo una serie $\sum a_n$ tale per cui $a_n \ge 0$. Studiamo il 
limite per $n$ che tende all'infinito della radice $n$-esima del termine
$n$-esimo. Vale questo:
\[
\lim_{n \to \infty} \sqrt[n]{a_n} =
\begin{cases}
L < 1 &\implies \text{ la serie converge} \\
L > 1 &\implies \text{ la serie diverge} \\
L = 1 &\implies \text{ il criterio non dice niente sulla serie}
\end{cases}
\]

\subsection{Criterio di Leibniz}

Il criterio di Leibniz si applica a serie di segno alterno.
Sia $\sum a_n$ una serie, e supponiamo valga questo:
\begin{enumerate}
    \item la serie \`e a termini di segno alterno, ossia ogni termine
    ha segno opposto rispetto al successivo;
    \item $\abs{a_n} \ge \abs{a_{n+1}}$, ossia i termini della serie
    sono decrescenti (non strettamente);
    \item $\lim_{n \to \infty} a_n = 0$, ossia i termini della serie 
    tendono a 0.
\end{enumerate}
Se queste tre ipotesi sono verificate, allora la serie $\sum a_n$ converge
semplicemente.

\section{Serie armonica generalizzata}

La serie armonica generalizzata \`e una serie del tipo:
\[
\sum_{n = n_0}^{\infty} \frac{1}{n^p}
\]
Si distinguono due casi per la convergenza:
\[
\sum_{n = n_0}^{\infty} \frac{1}{n^p} =
\begin{cases}
\text{converge } &  \text{se } p > 1 \\
\text{diverge } & \text{se } p \le 1
\end{cases}
\]

\section{Serie di potenze}

Una serie di potenze \`e una serie del tipo:
\[
\sum_{n = n_0}^{\infty} \frac{\left[ f(x) \right]^{a \, n + b}}{a_n}
\]
Il centro di una serie di potenze \`e un valore $x_0$ tale per cui
$f(x_0) = 0$. Tipicamente le serie che si incontrano hanno la forma:
\[
\sum_{n = n_0}^{\infty} \frac{\left( x - x_0 \right)^{a \, n + b}}{a_n}
\]
In questo caso il centro \`e proprio $x_0$.

Di una serie di potenze, oltre al centro, vogliamo conoscere il raggio
di convergenza (per cui la serie converge assolutamente), e vogliamo poi
studiare se la serie converge semplicemente, converge assolutamente, o
diverge agli estremi dell'intervallo di convergenza.

Sia $R$ il raggio di convergenza di una serie di potenze, l'intervallo di 
convergenza \`e l'intervallo $\ooint{x_0 - R}{x_0 + R}$. Per trovare il 
raggio di convergenza di una serie di potenze, si risolve il seguente
limite:
\[
\lim_{n \to \infty} \abs{\frac{\left[f(x)\right^{a \, (n+1) + b}}{\left[f(x)\right]^{{a \, n + b}}} \cdot \frac{a_n}{a_{n+1}}} = R(x)
\]
Il risultato sar\`a una funzione $R(x)$. Si impone $R(x) < 1$, per trovare
per quali valori di $x$ la serie converge assolutamente.

Si studiano infine i casi in cui $R(x) = 1$, e per ciascun caso si determina
se la serie converge semplicemente, converge assolutamente, o diverge.


\end{document}
