\documentclass[11pt,a4paper,twoside]{report}
% draft mode prevents link generation

\usepackage[italian]{babel}
% per date e ToC in italiano
\usepackage{hyperref}
\usepackage{nameref}
\usepackage[table]{xcolor}
\usepackage{amsmath}        % matematica
\usepackage{amsthm}         % teoremi
\usepackage{amssymb}        % simboli
\usepackage{amsfonts}       % font matematicosi
\usepackage{mathrsfs}
\usepackage{mathtools}
\usepackage{thmbox}
\usepackage{centernot}      % per semplificare
\usepackage{pgfplots}       % per i grafici
\pgfplotsset{compat=1.5}    % consigliata compatibilita'
                            % con la versione 1.5

\usepackage{fullpage}       % troppo margine normalmente
\usepackage{parskip}        % preferisco spazio fra i paragrafi
                            % all'indentazione sulla prima riga
\usepackage{fancyhdr}
\usepackage[makeroom]{cancel}

% algorithms
\usepackage{algorithm}
\usepackage{algpseudocode}

\floatname{algorithm}{Algoritmo}
\renewcommand{\algorithmicrequire}{\textbf{Input:}}
\renewcommand{\algorithmicensure}{\textbf{Output:}}

\theoremstyle{plain}
\newtheorem{axiom}{Assioma}
\newtheorem{theorem}{Teorema}[section]
\newtheorem{lem}[theorem]{Lemma}
\newtheorem{prop}[theorem]{Proposizione}
\newtheorem{cor}[theorem]{Corollario}

\theoremstyle{definition}
\newtheorem{defn}{Definizione}[section]
\newtheorem{esercizio}{Esercizio}

\theoremstyle{remark}
\newtheorem{remark}{Remark}
\newtheorem{oss}{Osservazione}
\newtheorem{caso}{Caso}
\newtheorem{fact}{Fatto}
\newtheorem{claim}{Claim}

\newenvironment{exmp}[1][]
{ \par\medskip\textbf{Esempio\ #1\unskip\,:}\\*[1pt]\rule{\textwidth}{0.4pt}\\*[1pt] }
{ \\*[1pt]\rule{\textwidth}{0.4pt}\medskip }

\newenvironment{smallpmatrix}
{\left( \begin{smallmatrix}}
{\end{smallmatrix} \right)}

\newenvironment{completepmatrix}[1]
{\left(\begin{array}{*{#1}{c}|c}}
{\end{array}\right)}

\pagestyle{fancyplain}
\fancyhead{} % clear all header fields
\fancyfoot{} % clear all footer fields
\fancyfoot[C]{\today}
\fancyfoot[LE,RO]{\thepage}
\renewcommand{\headrulewidth}{0pt}
\renewcommand{\footrulewidth}{0pt}

\usetikzlibrary{shapes,arrows,calc,fit,backgrounds,positioning,automata}

\newcommand{\parity}[1]{\mathcal{Par}\left( { #1 } \right)}
% Languages & Automata
\newcommand{\langs}[1]{\mathcal{L}\left( {#1} \right)}
\newcommand{\lang}[1]{L\left( {#1} \right)}
\newcommand{\langor}{\, | \,}
\newcommand{\kleenestar}{\star}
\newcommand{\emptystring}{\varepsilon}

\newcommand{\bigo}[1]{O\left( {#1} \right)}
\newcommand{\probp}{\code{P}}
\newcommand{\probnp}{\code{NP}}
\newcommand{\probnpc}{\code{NP-C}}
\newcommand{\probexp}{\code{EXP}}
\newcommand{\covers}{<\!\cdot}      % simbolo di 'copre'
\newcommand{\scalar}{\centerdot}
\newcommand{\compl}{\mathsf{c}}     % C complementare
\newcommand{\dotcup}{\mathaccent\cdot\cup}  % unione disgiunta
\newcommand{\lateralsx}[1]{{}_{#1}\!\sim}   % equivalenza sinistra
\newcommand{\lateraldx}[1]{\sim_{#1}}       % equivalenza destra
\newcommand{\mcm}{\operatorname{mcm}}       % mcm
\newcommand{\mcd}{\operatorname{MCD}}       % MCD
\newcommand{\pow}[1]{<\!{#1}\!>}            % gruppo delle potenze (<a>)
\newcommand{\abs}[1]{\left\lvert{#1}\right\rvert}   % valore assoluto
\newcommand{\intsup}[1]{\left\lceil{#1}\right\rceil}   % intero superiore
\newcommand{\intinf}[1]{\left\lfloor{#1}\right\rfloor}   % intero inferiore
\newcommand{\norm}[1]{\left\lVert{#1}\right\rVert}  % norma di un vettore
\newcommand{\divides}{\bigm|}       % simbolo di 'divide'
\newcommand{\definition}{\overset{\underset{\mathrm{def}}{}}{=}}    % uguale con 'definizione' sopra
\newcommand{\supop}{\vee}           % simbolo dell'operatore sup
\newcommand{\infop}{\wedge}         % simbolo dell'operatore inf
\newcommand{\subgroupset}{\mathscr{S}}      % insieme dei sottogruppi (S corsiva)
\newcommand{\solutions}{\mathscr{S}}
\newcommand{\parts}[1]{\mathbb{P} \left( {#1} \right)}             % insieme delle parti
\newcommand{\naturals}{\mathbb{N}}          % insieme dei naturali
\newcommand{\integers}{\mathbb{Z}}          % insieme degli interi
\newcommand{\rationals}{\mathbb{Q}}         % insieme dei razionali
\newcommand{\reals}{\mathbb{R}}             % insieme dei reali
\newcommand{\complexes}{\mathbb{C}}         % insieme dei complessi
\newcommand{\field}{\mathbb{K}}             % simbolo di campo generico
\newcommand{\subfield}{\mathbb{F}}          % simbolo di sottocampo generico
\newcommand{\matrices}{\mathfrak{M}}        % insieme delle matrici
\newcommand{\nullelement}{\underline{0}}    % elemento neutro
\newcommand{\var}{\operatorname{Var}}       % varianza 
\newcommand{\seq}[3]{{#1}_{#2},\ldots,\,{#1}_{#3}}
\newcommand{\prob}[2][P]{#1 \left( {#2} \right)}
\newcommand{\probcond}[3][P]{#1 \left( #2 | #3 \right)}
\newcommand{\expect}[1]{E \left( #1 \right)}
\newcommand{\mass}[2][X]{p_{#1} \left( #2 \right)}
\newcommand{\distr}[2][X]{F_{#1} \left( #2 \right)}
\newcommand{\image}[1]{Im_{#1}}
\newcommand{\cov}{\operatorname{Cov}}
\newcommand{\detname}{\operatorname{det}}
\newcommand{\sgn}{\operatorname{sgn}}
\renewcommand{\det}[1]{\detname \left( {#1} \right)}
\newcommand{\algcompl}[2][A]{{\mathcal{#1}}_{#2}}
\newcommand{\agg}[1]{\operatorname{Agg} \left( {#1} \right)}
\newcommand{\id}{id}
\newcommand{\sfrac}[2]{{#1}/{#2}}
\newcommand{\requiv}[1][]{\ \varepsilon_{#1} \ }
\newcommand{\abslog}[1]{\log \left( \abs{#1} \right)}
\newcommand{\substitute}[2]{\left. {#1} \; \right|_{#2}}
\newcommand{\increment}[1]{{\left. {#1} \; \right|}}
\newcommand{\adj}{\sim}
\newcommand{\code}[1]{\texttt{#1}}
\newcommand{\isa}{\code{is-a}}
% intervalli aperti e chiusi
\newcommand{\ooint}[2]{\left( {#1}, {#2} \right)}
\newcommand{\coint}[2]{\left[ {#1}, {#2} \right)}
\newcommand{\ocint}[2]{\left( {#1}, {#2} \right]}
\newcommand{\ccint}[2]{\left[ {#1}, {#2} \right]}
\newcommand{\stereotipe}[1]{\ll \code{#1} \gg}

\renewcommand{\labelitemi}{$-$}
\let\oldforall\forall
\renewcommand{\forall}{\ \oldforall \ }
\let\oldexists\exists
\renewcommand{\exists}{\ \oldexists \ }
\renewcommand{\implies}{\Rightarrow}
\renewcommand{\iff}{\Leftrightarrow}

\makeatletter
\let\orgdescriptionlabel\descriptionlabel\renewcommand*{\descriptionlabel}[1]{%
\let\orglabel\label
\let\label\@gobble
\phantomsection
\edef\@currentlabel{#1}%
%\edef\@currentlabelname{#1}%
\let\label\orglabel
\orgdescriptionlabel{#1}%
}
\makeatother

\makeatletter
\providecommand*{\diff}%
    {\@ifnextchar^{\DIfF}{\DIfF^{}}}
\def\DIfF^#1{%
    \mathop{\mathrm{\mathstrut d}}%
        \nolimits^{#1}\gobblespace}
\def\gobblespace{%
        \futurelet\diffarg\opspace}
\def\opspace{%
    \let\DiffSpace\!%
    \ifx\diffarg(%
        \let\DiffSpace\relax
    \else
        \ifx\diffarg[%
            \let\DiffSpace\relax
        \else
            \ifx\diffarg\{%
                \let\DiffSpace\relax
            \fi\fi\fi\DiffSpace}

\providecommand*{\deriv}[1][]{%
    \frac{\diff}{\diff #1}}
%\providecommand*{\deriv}[2][]{%
%    \frac{\diff^{#1}}{\diff #2^{#1}}}
\providecommand*{\fderiv}[3][]{%
    \frac{\diff^{#1}#2}{\diff #3^{#1}}}
\providecommand*{\pderiv}[3][]{%
    \frac{\partial^{#1}#2}%
        {\partial #3^{#1}}}

\newcommand{\dx}[1][x]{\, \diff {#1}}

\author{Michele Laurenti \\ \href{mailto:asmeikal@me.com}{asmeikal@me.com}}
\date{\today}


\begin{document}

\section{Teorema fondamentale del calcolo}

Il teorema fondamentale del calcolo vale per ogni $f \in C^{0} ([a,b])$ (ossia per ogni funzione continua nell'intervallo chiuso e limitato $[a,b]$).
In generale, $f \in C^{n} ([a,b])$ per $n \ge 0$ significa che $f$ \`e una funzione derivabile $n$ volte, e che sia $f$ che le sue prime $n$ derivate sono continue nell'intervallo chiuso e limitato $[a,b]$.

\begin{theorem}[Teorema fondamentale del Calcolo]
Sia $F(x) = \int_{a}^{x} f(t) \dx[t]$, per $a \le x \le b$, ossia $F$ \`e la funzione che calcola l'area sotto il grafico della funzione $f$ dal punto $a$ al punto (variabile) $x$, allora:
\begin{enumerate}
    \item $\exists F'(x) = f(x)$, ossia la derivata prima di $F$ \`e $f$;
    \item $\forall G$ primitiva di $f$ in $[a,b]$, risulta che:
    \[
    \int_{a}^{b} f(x) \dx = G(b) - G(a) = \increment{\int f(x) \dx}_{a}^{b}
    \]
	Ossia l'integrale definito da $a$ a $b$ di $f$ \`e la differenza fra $G(b)$ e $G(a)$.
    L'ultima parte si legge ``incremento dell'integrale indefinito di $f$ da $a$ a $b$''.
\end{enumerate}
\end{theorem}
Nella premessa, $x$ indica l'estremo superiore di integrazione, mentre la $t$ \`e una variabile muta di integrazione (e quindi non \`e minimamente rilevante la lettera che gli si associa).

La nozione di primitiva di una funzione e quella di area individuata dal grafico della funzione sono legate a doppio filo dal teorema fondamentale del Calcolo.

\newpage
\section{Integrali impropri}

Sappiamo calcolare questo integrale indefinito:
\[
\int \log (x) \dx = x \, \log (x) - x + c
\]
Poi, $\forall a \ge 0$, sappiamo calcolare questo integrale definito:
\[
\int_{a}^{1} \log (x) \dx = - 1 - a \, \log (a) + a
\]
Facendo il limite per $a$ che tende a $0^+$, vediamo che quest'area vale:
\[
\lim_{a \to 0^+} \int_{a}^{1} \log (x) \dx = -1
\]
Sappiamo calcolare qualunque area di questo tipo, $\forall a, b \ge 0$:
\[
\int_{a}^{b} \log (x) \dx
\]
Abbiamo parlato solo di intervalli $[a,b]$ chiusi e limitati. La cosa pi\`u estrema che abbiamo fatto \`e stato togliere un punto dall'estremo dell'intervallo. Ma in questo caso, se studiamo $\log (x)$ nell'intervallo $(0,1]$, la funzione \`e continua \emph{ma non} limitata.

Prendendo per\`o un qualunque $a \in (0,1]$, la funzione nel punto $a$ \`e sia continua che limitata. Facendo tendere $a$ a $0^+$ vediamo che l'area sotto la funzione \`e limitata, e tende a $-1$. La cosa rilevante \`e che abbiamo attribuito un'area limitata al grafico di una funzione su un insieme illimitato. Adesso possiamo ammettere questo:
\[
\int_{0}^{1} \log (x) \dx = -1
\]
Questo \`e un integrale improprio. Un integrale improprio \`e il limite di un integrale proprio.

\begin{exmp}
Quello che segue \`e un normale integrale definito, per $b \in \reals$.
\[
\int_{1}^{b} \frac{\diff x}{1 + x^2} = \arctan (b) - \frac{\pi}{4}
\]
Ma se facciamo il limite per $b \to \infty$, l'integrale diventa un integrale improprio.
\[
\lim_{b \to \infty} \int_{1}^{b} \frac{\diff x}{1 + x^2} = \frac{\pi}{2} - \frac{\pi}{4} = \frac{\pi}{4}
\]
\end{exmp}

\begin{exmp}
Questo \`e immediatamente riconoscibile come un integrale improprio:
\[
\int_{0}^{1} \frac{\diff x}{x}
\]
La funzione integranda, infatti, non \`e definita in uno degli estremi di integrazione (nello specifico, in 0). Quindi trovare un integrale simile vuol dire trovare questo integrale:
\[
\lim_{a \to 0^+} \int_{a}^{1} \frac{\diff x}{x} =
\lim_{a \to 0^+} \log(1) - \log(a) = 
- \lim_{a \to 0^+} \log(a) = \infty
\]
Qui abbiamo quindi una funzione illimitata su un intervallo limitato, e il suo integrale non esiste.
\end{exmp}

\begin{exmp}
Variazioni sullo stesso integrale improprio.
\[
\int_{1}^{\infty} \frac{\diff x}{x}
\]
Anche questo \`e improprio. Abbiamo una funzione integranda che (in $[1, \infty)$) \`e limitata, nonostante l'intervallo preso non lo sia.
\[
\lim_{b \to \infty} \int_{1}^{b} \frac{\diff x}{x} = \lim_{b \to \infty} \log(b) = \infty
\]
\end{exmp}

Quindi l'integrale definito di $\frac{1}{x}$ non esiste n\'e fra 0 e 1, n\'e fra 1 e $\infty$. $\frac{1}{x}$ \`e una funzione illimitata in 0 e all'infinito. Sar\`a sempre cos\`i, con tutte le funzioni non limitate?

\begin{exmp}
L'integrale che segue sembra simile ai casi precedenti, ma... 
\[
\int \frac{\diff x}{x^2} = - \frac{1}{x} + c
\]
Rifacciamo entrambi i ``test'' fatti prima.
\begin{align*}
\int_{0}^{1} \frac{\diff x}{x^2} \\
\int_{1}^{\infty} \frac{\diff x}{x^2}
\end{align*}
Sempre due integrali impropri. Ma vogliamo capire quale integrale converge e quale diverge.
\[
\lim_{a \to 0^+} \int_{a}^{1} \frac{\diff x}{x^2} =
\lim_{a \to 0^+} - 1 + \frac{1}{a} = \infty
\]
Diverge fra 0 e 1...
\[
\lim_{b \to \infty} \int_{1}^{b} \frac{\diff x}{x^2} =
\lim_{b \to 0^+} - \frac{1}{b} + 1 = 1
\]
E converge a 1 fra 1 e $\infty$.
\end{exmp}

\begin{exmp}
Vediamo un altro caso apparentemente simile ai precedenti. 
\[
\int \frac{\diff x}{\sqrt{x}} = 2 \, \sqrt{x} + c
\]
Quale integrale improprio converge, e quale diverge?
\[
\lim_{a \to 0^+} \int_{a}^{1} \frac{\diff x}{\sqrt{x}} =
\lim_{a \to 0^+} 2 - 2 \, \sqrt{a} = 2
\]
Converge (a 2) fra 0 e 1...
\[
\lim_{b \to \infty} \int_{1}^{b} \frac{\diff x}{\sqrt{x}} =
\lim_{b \to 0^+} 2 \, \sqrt{b} - 2 = \infty
\]
E diverge fra 1 e $\infty$.
\end{exmp}

\subsection{Funzioni razionali e integrali impropri}
\label{integrali_impropri_razionali}

Conosciamo l'integrale di questa funzione generica:
\[
\int \frac{\diff x}{x^p} = 
\begin{cases}
\dfrac{x^{-p + 1}}{- p + 1} + c & \text{ se } p \neq 1 \\
\abslog{x} + c & \text{ se } p = 1
\end{cases}
\]
Vogliamo studiare due integrali impropri a partire da questa funzione generica.
\begin{align*}
\int_{0}^{1} \frac{\diff x}{x^p} \qquad
\int_{1}^{\infty} \frac{\diff x}{x^p}
\end{align*}
\`E chiaro che, per $p$ negativo o nullo ($p \le 0$), il primo \`e un integrale proprio (la funzione, infatti, \`e continua e limitata anche in 0). I casi interessanti sono quelli per $p > 0$. Vogliamo capire per quali valori di $p$ questo integrale converge e per quali diverge.

Abbiamo visto che se $p = \frac{1}{2}$ il primo converge, il secondo diverge, mentre se $p = 2$ il primo diverge e il secondo converge.

Vediamo il primo integrale improprio nel caso generico, per $p \neq 1$:
\[
\int_{0}^{1} \frac{\diff x}{x^p} = 
\lim_{a \to 0^+} \int_{a}^{1} \frac{\diff x}{x^p} =
\lim_{a \to 0^+} \frac{1}{- p + 1} - \frac{a^{-p + 1}}{-p + 1}
\]
Ora, tutto dipende da $-p + 1$. Per $a$ che tende a $0^+$, $a^n$ tende a $0$ se l'esponente \`e positivo, o a $\infty$ se l'esponente \`e negativo. Quindi:
\[
\lim_{a \to 0^+} \frac{1}{- p + 1} - \frac{a^{-p + 1}}{-p + 1} =
\begin{cases}
\dfrac{1}{- p + 1} & \text{ se } p < 1 \\
\infty & \text{ se } p > 1
\end{cases}
\]
Se $p < 1$, abbiamo che $-p > - 1 \implies - p + 1 > 0$.

Vediamo l'altro integrale.
\[
\int_{1}^{\infty} \frac{\diff x}{x^p} =
\lim_{b \to \infty} \int_{1}^{b} \frac{\diff x}{x^p} = 
\begin{cases}
\text{qualcosa} \in \reals & \text{ se } p > 1 \\
\infty & \text{ se } p \le 1
\end{cases}
\]

\subsection{Conclusione}

Nell'intervallo $\ocint{0}{1}$ vale questo:
\[
\int_{0}^{1} \frac{\diff x}{x^p} = 
\begin{cases}
\dfrac{1}{- p + 1} & \text{ se } p < 1 \\
\infty & \text{ se } p \ge 1
\end{cases}
\]
Nell'intervallo $\coint{1}{\infty}$ vale questo:
\[
\int_{1}^{\infty} \frac{\diff x}{x^p} =
\begin{cases}
\text{qualcosa} \in \reals & \text{ se } p > 1 \\
\infty & \text{ se } p \le 1
\end{cases}
\]
E basta.

\begin{exmp}
Basta integrali impropri di funzioni razionali.
\[
\int_{0}^{\infty} x \, e^{-x} \dx =
\lim_{b \to \infty} \int_{0}^{b} x \, e^{-x} \dx = 
\increment{\lim_{b \to \infty} -x \, e^{-x} - e^{-x}}_{0}^{b} =
\lim_{b \to \infty} -b \, e^{-b} - e^{-b} + 1 = + 1
\]
\end{exmp}

\begin{exmp}
Studiamo questo integrale improprio:
\[
\int_{- \infty}^{\infty} e^{-\abs{x}} \dx
\]
La funzione \`e pari, ed \`e limitata. La scriviamo come somma di due integrali impropri:
\begin{align*}
\int_{0}^{\infty} e^{-x} \dx + \int_{-\infty}^{0} e^{x} \dx =
2 \, \int_{0}^{\infty} e^{-x} \dx &= \\
2 \, \lim_{b \to \infty} \int_{0}^{b} e^{-x} \dx = 
\increment{2 \, \lim_{b \to \infty} - e^{-x}}_{0}^{b} = 
2 \, \lim_{b \to \infty} - e^{-b} + e^{0} &= 2
\end{align*}
\end{exmp}

\newpage
\section{Criterio di convergenza integrale}
\label{sec:criterio_convergenza_integrale}

Consideriamo una successione $a_n$ e una funzione $f(x)$ per cui vale:
\begin{align*}
f(N_0) &= a_{N_0}
f(N_0 + 1) &= a_{N_0 + 1} \\
f(N_0 + 2) &= a_{N_0 + 2} \\
\ldots 
\end{align*}
Nella figura \ref{fig:integrale_maggiora_serie}, la somma dei rettangoli (larghi 1 e alti quanto $a_n$) \`e minore dell'integrale della funzione.

\begin{figure}[h]
\centering
\begin{tikzpicture}[scale=1.0]
\begin{axis}[
    % xtick={0,...,1.5},ytick={0,...,1.5},
    % xmax=2,ymax=1.2,ymin=0,xmin=0,
    % enlargelimits=true,
    xticklabels={$N_0$,$N_0$,$N_0$,$N_0 + 1$,$N_0 + 2$,$N_0 + 3$,$N_0 + 4$},
    axis lines=middle,
    clip=false,
    ymax=0.6,
    ymin=0,
    xmin=1.5,
    xmax=7.5,
    domain=2:7,
    axis on top
    ]

\addplot [draw=green, fill=green!10, const plot mark right, samples=6, domain=2:7]
    {1 / x}\closedcycle;

\addplot[smooth, thick,domain=2:7,samples=40]{1 / x};

\end{axis}
\end{tikzpicture}
\caption{\label{fig:integrale_maggiora_serie}L'integrale della funzione maggiora la serie}
\end{figure}

La serie \`e la somma delle aree dei rettangoli, mentre l'integrale \`e l'area sotto la curva. Vale evidentemente che:
\[
\sum_{n = N_0 + 1}^{\infty} a_n \le \int_{N_0}^{\infty} f(x) \dx
\]
Quindi, se l'integrale improprio converge, converge anche la serie.

Usiamo questa propriet\`a per studiare la convergenza delle serie armoniche generalizzate, come questa:
\[
\sum_{n = 1}^{\infty} \frac{1}{n^2} \le 1 + \int_{1}^{\infty} \frac{\diff x}{x^2} < \infty
\]
Il ragionamento alla base di questi confronti \`e propriamente geometrico. 

Quindi, nel caso generale, vale questo:
\[
\sum_{n = 1}^{\infty} \frac{1}{n^p} \le 1 + \int_{1}^{\infty} \frac{\diff x}{x^p} < \infty \text{ per } p > 1
\]

Come facciamo vedere, ora, che la serie diverge per $p \le 1$? Vogliamo scrivere che la sommatoria sia maggiore o uguale all'integrale, quindi dobbiamo costruire il grafico (e di conseguenza la funzione) in modo che i rettangoli della serie contengano il grafico della funzione.

\begin{figure}[h]
\centering
\begin{tikzpicture}[scale=1.0]
\begin{axis}[
    % xtick={0,...,1.5},ytick={0,...,1.5},
    % xmax=2,ymax=1.2,ymin=0,xmin=0,
    % enlargelimits=true,
    xticklabels={$N_0$,$N_0$,$N_0$,$N_0 + 1$,$N_0 + 2$,$N_0 + 3$,$N_0 + 4$},
    axis lines=middle,
    clip=false,
    ymax=0.6,
    ymin=0,
    xmin=1.5,
    xmax=7.5,
    domain=2:7,
    axis on top
    ]

\addplot [draw=red, fill=red!10, ybar interval, samples=6, domain=2:7]
    {1 / x}\closedcycle;

\addplot[smooth, thick,domain=2:7,samples=40]{1 / x};
\end{axis}
\end{tikzpicture}
\caption{\label{fig:serie_maggiora_integrale}La serie maggiora l'integrale della funzione}
\end{figure}

La disuguaglianza che ci interessa ora:
\[
	\int_{N_0}^{\infty} f(x) \dx \le a_{N_0} + \sum_{n = N_0 + 1}^{\infty} a_n 
\]

Quello che ci interessa \`e che la serie converge se e solo se converge anche l'integrale (improprio).

\begin{exmp}[di applicazione del criterio integrale]
Questa serie \`e impossibile da maggiorare o da minorare:
\[
\sum_{n = 2}^{\infty} \frac{1}{n \, \log (n) \, {\left[ \log \left( \log (n) \right) \right]}^{2}}
\]
Funziona per\`o il criterio integrale, che ci mostra che l'integrale converge:
\[
\int_{2}^{\infty} \frac{\diff x}{x \, \log(x) \, {\left[ \log \left( \log (x) \right) \right]}^{2}} = \increment{- \frac{1}{\log \left( \log(x) \right)}}^{\infty}_{2} = \frac{1}{\log \left( \log(2) \right)}
\]
Osservando la serie iniziale, e pensando all'integrale che le si pu\`o associare, si nota la presenza di una funzione e della sua derivata:
\[
\frac{1}{f'(x) \, {\left[ f(x) \right]}^2}
\]
Con $f(x) = \log \left( \log(x) \right)$, e quindi $f'(x) = \frac{1}{x \, log \left( x \right)}$.
\end{exmp}

\newpage
\section{Serie di Taylor}

Consideriamo una generica serie di potenze:
\[
\sum_{n = 0}^{\infty} a_n \, (x - x_0)^n
\]
con raggio di convergenza $R \in \ocint{0}{\infty}$ (ossia, strettamente maggiore di 0).  Vuol dire che la serie converge per $- R < x - x_0 < R$. $R$ pu\`o anche essere infinito: in questo caso la serie converge in tutto $\reals$.

Possiamo vedere questa serie come una funzione $f(x)$, definita per $\abs{x - x_0} < R$, ossia per $x \in \ooint{x_0 - R}{x_0 + R}$ (che \`e chiamato \emph{intervallo di convergenza}). 
\[
f(x) = \sum_{n = 0}^{\infty} a_n \, (x - x_0)^n
\]
Questa funzione \`e derivabile, e la sua derivata \`e la somma delle derivate:
\[
f'(x) = \sum_{n = 1}^{\infty} n \, a_{n} \, (x - x_0)^{n - 1}
\]
E non solo: essendo derivabile, questa funzione \`e anche integrabile:
\[
\int_{x_0}^{x} f(x) \dx = 
\sum_{n = 0}^{\infty} \int_{x_0}^{x} a_n \, (x - x_0)^n =
\sum_{n = 0}^{\infty} a_n \, \frac{(x - x_0)^{n + 1}}{n + 1}
\]
Le serie di potenze sono integrabili e derivabili termine a termine \emph{all'interno dell'intervallo di convergenza}. E il discorso, poi, vale anche per la derivata della derivata, e per la derivata della derivata della derivata... La derivata $k$-esima \`e:
\[
	f^{(k)} (x) = \sum_{n = k}^{\infty} n \cdot (n - 1) \cdot \ldots \cdot (n - k + 1) \, a_n \, (x - x_0)^{n - k}
\]
Dicendo che $0^0 = 1$ (\`e un abuso di notazione, ma a noi va bene), vediamo quanto vale la funzione (e le sue derivate) in $x = x_0$:
\begin{align*}
f(x_0) &= a_0 \\
f'(x_0) &= a_1 \\
f''(x_0) &= 2 \, a_2 \\
&\vdots \\
f^{(k)}(x_0) &= k! \, a_k
\end{align*}
Abbiamo scritto poco sopra, infatti:
\[
f^{(k)} (x_0) = \sum_{n = k} n \cdot (n - 1) \cdot \ldots \cdot (n - k + 1) \, a_n \, (x_0 - x_0)^{n - k} =
k \cdot (k - 1) \cdot \ldots \cdot 1 \, a_{k} = k! \, a_{k}
\]
$0^0$ \`e un abuso di notazione perch\'e quello che si dovrebbe scrivere \`e:
\[
a_0 + \sum_{n = 1}^{\infty} a_n \, (x - x_0)^n
\]
Ma, siccome $n^0 = 1 \forall n \in \reals - \{ 0 \}$, si preferisce abusare della notazione matematica e scrivere:
\[
\sum_{n = 0}^{\infty} a_n \, (x - x_0)^n
\]
Quindi, sapendo che $a_k = \frac{f^{(k)}(x_0)}{k!}$, possiamo riscrivere:
\[
f(x) = \sum_{n = 0}^{\infty} a_n \cdot (x - x_0)^n =
\sum_{n = 0}^{\infty} \frac{f^{(n)} (x_0)}{n!} \cdot (x - x_0)^{n}
\]
Che \`e la famosa serie di Taylor.

Ogni serie di potenze \`e, all'interno dell'intervallo di convergenza, la serie di Taylor della sua somma.

\subsection{Esempi}

A cosa ci serve questo? Sappiamo una cosa:
\[
\sum_{n = 0}^{\infty} x^n = \frac{1}{1 - x}
\]
Quindi, $f(x)$ vale:
\[
f(x) = \frac{1}{1 - x}
\]
Quale \`e la derivata $k$-esima di questa funzione nell'origine? L'abbiamo visto prima: $a_n$ \`e sempre 1, e quindi:
\[
f^{(k)} (0) = k!
\]
Guardiamo questa funzione:
\[
f(x) = \frac{1}{(1 - x)^2}
\]
\`E il quadrato \emph{e} la derivata di $\frac{1}{1 - x}$. Quale \`e il suo sviluppo in serie?
\[
\frac{1}{(1 - x)^2} = \left[ \frac{1}{1 - x} \right]' = 
\sum_{n = 0}^{\infty} \left[ x^{n} \right]' = \sum_{n = 1}^{\infty} n \, x^{n - 1}
\]
La derivata della funzione (che \`e la serie), \`e la serie delle derivate! Il termine con $n = 0$ scompare, una volta derivato: \`e una costante.

Prendiamo quest'altra funzione:
\[
\frac{1}{1 + x} = \sum_{n = 0}^{\infty} (-1)^n \, x^n
\]
Si vede chiaramente, infatti \`e sufficiente sostituire $x$ con $-x$ nell'equazione di poco sopra...

Quale sar\`a lo sviluppo in serie di potenze di questa funzione?
\[
f(x) = \log (1 + x)
\]
\`E una primitiva della funzione vista sopra... Quindi il suo sviluppo in serie di potenze sar\`a:
\[
\log (1 + x) = \sum_{n = 0}^{\infty} (-1)^n \, \frac{x^{n+1}}{n+1}
\]
Sviluppandolo viene questo:
\[
\log (1 + x) = x - \frac{x^2}{2} + \frac{x^3}{3} - \frac{x^4}{4} + \ldots
\]
Tutto questo vale all'interno dell'intervallo di convergenza. Quest'ultimo caso sarebbe vero anche per $x = 1$, anche se l'intervallo di convergenza \`e $-1 < x < 1$. Sostituendo $x = 1$ troviamo questo:
\[
\log (2) = \sum_{n = 1}^{\infty} \frac{(-1)^{n+1}}{n}
\]
La serie armonica a termini di segno alterno vale $\log (2)$.

Sappiamo quanto vale questa serie:
\[
\sum_{n = 0}^{\infty} (-1)^n x^{2 \, n} = \frac{1}{1 + x^2}
\]
La primitiva di una somma \`e la somma delle primitive... E la primitiva di una serie \`e la serie delle primitive.
\[
\arctan (x) = \sum_{n = 0}^{\infty} (-1)^n \frac{x^{2 \, n + 1}}{2 \, n + 1}
\]
Anche qui si pu\`o fare qualche operazione impropria: per $x = 1$, abbiamo uno sviluppo in serie di $\arctan (1) = \frac{\pi}{4}$.

Quando scriviamo un polinomio di Taylor come questo:
\[
f(x) = \sum_{n = 0}^{N} \frac{f^{(n)} (x_0)}{n!} \, (x - x_0)^n
\]
Stiamo trascurando un errore, dipendente da $N$:
\[
E_N (x) = \frac{f^{(N + 1)} (\xi)}{(N + 1)!} \, (x - x_0)^{N + 1}
\]
La serie di Taylor \`e il limite del polinomio di Taylor, per $N \to \infty$.

Studiamo questa serie:
\[
\sum_{n = 0}^{\infty} \frac{x^n}{n!}
\]
Il raggio di convergenza \`e $\infty$, quindi converge per ogni $x$. Questo polinomio qui:
\[
\sum_{n = 0}^{N} \frac{x^n}{n!} + E_N (x)
\]
\`e il polinomio di Taylor di $e^x$ di centro uguale a 0. Le derivate di $e^x$ sono tutte $e^x$, e nell'origine valgono sempre 1. L'errore varr\`a:
\[
E_N (x) = e^\xi \cdot \frac{x^{N+1}}{(N+1)!}
\]
Maggioriamo l'errore cos\`i:
\[
\abs{E_N (x)} \le \frac{e^{\abs{x}} \, \abs{x}^{N+1}}{(N + 1)!}
\]
Per $N$ che tende all'infinito, l'errore tende a 0. Quindi la serie di prima \`e la serie di Taylor di $e^x$:
\[
e^x = \sum_{n = 0}^{\infty} \frac{x^n}{n!}
\]
Ripeschiamo il polinomio di Taylor della funzione seno:
\[
\sin (x) = x - \frac{x^3}{3!} + \frac{x^5}{5!} - \frac{x^7}{7!} + \ldots
\]
La serie di Taylor della funzione sar\`a il limite per $n \to \infty$ del polinomio di Taylor.

\end{document}

