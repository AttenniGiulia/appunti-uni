\section{Ricerca delle primitive}

Nota: con ``primitiva'' e ``antiderivata'' si indica la stessa cosa, ma antiderivata \`e un termine usato pi\`u dagli inglesi.

\begin{defn}[Primitiva di una funzione]
Data una funzione $f(x)$ continua in un intervallo $I$, una funzione $F(x)$ \`e primitiva di $f(x)$ in $I$ se $F'(x) = f(x) \forall x \in I$.
\end{defn}
Cercare una primitiva di una funzione vuol dire cercare una funzione la cui derivata \`e la funzione data.

\`E importante la presenza dell'intervallo nella definizione. Se $F'(x) = f(x) \forall x \in I$, anche $F(x) + c$ \`e un'antiderivata di $f(x)$, e questo \`e banale, ma in un intervallo vale anche il viceversa: date due funzioni $F(x)$ e $G(x)$ primitive di $f(x)$ in un intervallo $I$, allora $F(x) = G(x) + c$, ossia le due primitive differiscono di una costante, e quindi l'espressione $F(x) + c$ \`e la totalit\`a delle primitive, ci d\`a tutte e sole le primitive di una funzione.

Se non si sta in un intervallo, non \`e pi\`u vero che due primitive di una funzione differiscono di una costante. Si sfrutta un semplice fatto:
\[
F(x) - G(x) = c \iff \left[ F(x) - G(x) \right]' = 0 \forall x \in I
\]
Se la derivata della differenza di due funzioni \`e nulla \emph{in un intervallo}, allora le due funzioni differiscono di una costante. Viene tutto dal teorema del valor medio:
\[
g(x_1) - g(x_2) = g'(\xi) \cdot (x_1 - x_2) \text{ per un } \xi \in [x_1, x_2]
\]
Dobbiamo sempre essere in un intervallo! Se la funzione $g(x) = F(x) - G(x)$ ha derivata sempre nulla, allora presi comunque due punti $x_1$ e $x_2$ vale sempre che $g(x_1) - g(x_2) = 0$, ossia $g(x_1) = g(x_2)$, quindi $g(x)$ \`e costante. Fuori da un intervallo questo non vale pi\`u: prendiamo la funzione $f(x) = \sgn(x)$, che \`e definita per ogni $x \neq 0$ e vale:
\[
\sgn(x) = 
\begin{cases}
1 \text{ se } x > 0 \\
-1 \text{ se } x < 0
\end{cases}
\]
La funzione \`e definita su tutta la retta meno un punto. \`E costante a destra dello 0, \`e costante a sinistra dello 0, ma non \`e \emph{costante}.

\section{Integrali indefiniti immediati}
\label{integrali_indefiniti_immediati}

\begin{defn}[Integrale indefinito]
L'integrale indefinito di una funzione \`e l'insieme di tutte le sue primitive, e si indica cos\`i:
\[
\int f(x) \dx = F(x) + c
\]
\end{defn}
Il problema dell'integrazione indefinita affronta il problema inverso della derivazione: solitamente i problemi inversi sono i pi\`u difficili. Non c'\`e approccio algoritmico, servono intuizioni su quale potrebbe essere la funzione $F(x)$ che verifica l'identit\`a di cui sopra, ossia quali funzioni $F(x)$ possiamo trovare che sappiamo essere primitive della funzione $f(x)$ data.

Alcuni integrali indefiniti sono immediati, conoscendo qualche derivata:
\begin{align*}
\int \cos (x) \dx &= \sin (x) + c \\
\int \sin (x) \dx &= - \cos (x) + c
\end{align*}
In questi due casi stiamo a cavallo, le funzioni integrande sono definite su tutta la retta $\reals$.

Queste due funzioni, invece, sono definite una per $x > 0$ e una per $x \neq 0$:
\[
\int \frac{\diff x}{\sqrt{x}} \qquad
\int \frac{\diff x}{x}
\]
Entrambe le funzioni sono nella forma:
\[
\int x^{\alpha} \dx
\]
Se $\alpha$ \`e un numero intero positivo, siamo felici. Se non lo \`e, bisogna escludere lo 0 dall'intervallo di integrazione. Se $\alpha$ \`e un numero razionale, per esempio $\frac{7}{4}$, pu\`o esserci di mezzo una radice pari, quindi necessariamente deve essere $x > 0$.

L'intervallo di integrazione \`e l'intervallo in cui \emph{la funzione integranda} ha senso.

Stabilito l'intervallo in cui la funzione $x^{\alpha}$ \`e integrabile, ed escludendo per ora il caso in cui $\alpha = -1$, cosa possiamo dire? 

Vedendo la derivata di $x^{\beta}$ capiamo una cosa:
\[
\left[ x^{\beta} \right]' = \beta \cdot x^{\beta - 1} \text{ per } \beta \neq 0 \implies
\int \beta \cdot x^{\beta - 1} \dx = x^{\beta} + c
\]
Quindi la risposta alla nostra domanda, per $\alpha \neq -1$, \`e:
\[
\int x^{\alpha} \dx = \frac{x^{\alpha + 1}}{\alpha + 1} + c
\]
Nel caso, per\`o, in cui $\alpha = -1$, abbiamo una funzione integranda definita per ogni $x \neq 0$. Una funzione che ha per derivata $\frac{1}{x}$ \`e il logaritmo. Sapendo come \`e fatto il logaritmo, vediamo che:
\[
\int \frac{\diff x}{x} =
\begin{cases}
\log (x) + c \text{ per } x > 0 \\
\log (-x) + c \text{ per } x < 0
\end{cases}
= \abslog{x} + c
\]
Con le nostre conoscenze, adesso, sappiamo trovare questo integrale:
\[
\int x^{\frac{3}{2}} \dx = \frac{2}{5} \, x^{\frac{5}{2}}
\]
Qualcuno ora potrebbe pensare che anche questo \`e un integrale corretto:
\[
\int (- x)^{\frac{3}{2}} \dx = \frac{2}{5} (- x)^{\frac{5}{2}} + c
\]
Ma non \`e vero! Infatti, derivando il risultato, otteniamo qualcosa di leggermente diverso:
\[
\deriv{x} \frac{2}{5} (- x)^{\frac{5}{2}} = - (-x)^{\frac{3}{2}}
\]
Ora \`e facile capire quale correzione bisogna fare per trovare l'integrale che cercavamo: \`e sufficiente cambiare il segno a quanto avevamo trovato.

\subsubsection{Linearit\`a dell'integrale}

E con questo integrale, come ci si deve comportare?
\[
\int \left( 12 \, x^3 - \frac{1}{\sqrt{3x}} \right) \dx
\]
Siccome la derivata della somma \`e la somma delle derivate, e l'integrale \`e l'inverso della derivata, anche l'integrale di una somma \`e la somma degli integrali. Quindi:
\[
\int 12 x^3 \dx - \int \frac{\diff x}{\sqrt{3x}} = 
\frac{12 \, x^4}{4} - \frac{2}{3} \sqrt{3x} + c =
3 \, x^4 - \frac{2 \sqrt{x}}{\sqrt{3}} + c
\]
Questa propriet\`a si chiama linearit\`a dell'integrale: \`e lo stesso motivo per cui ai fini dell'integrazione le costanti si ``portano fuori''.

Per fortuna, con gli integrali indefiniti \`e sempre possibile controllare l'esattezza del risultato, derivandolo. E spesso \`e bene farlo.

In generale, se l'argomento della funzione integranda \`e moltiplicato per un coefficiente $\alpha$, quello che abbiamo \`e questo:
\[
\int f(\alpha \, x) \dx = \frac{F(\alpha \, x)}{\alpha} + c
\]
Un ragionamento del genere diventa automatico col tempo.

\subsection{Integrali trigonometrici immediati}

Abbiamo visto due integrali trigonometrici all'inizio della sezione \ref{integrali_indefiniti_immediati}. Passiamo a cose pi\`u brutte ma ancora immediate, come questi due:
\begin{align*}
\int \tan (x) \dx &= \int \frac{\sin(x)}{\cos(x)} \dx \\
\int \frac{\diff x}{\cos^2 (x)} &= \int \sec^2 (x) \dx
\end{align*}
Piccolo promemoria: la secante \`e l'inverso del coseno, la cosecante \`e l'inverso del seno.

Il coseno non si annulla in $(-\frac{\pi}{2}, \frac{\pi}{2}) \cup (\frac{\pi}{2}, \frac{3}{2} \pi)$. La seconda funzione quindi \`e definita in questo intervallo, che si pu\`o scrivere anche come $(- \frac{\pi}{2} + k \cdot \pi, \frac{\pi}{2} + k \cdot \pi)$. Una funzione che ha come derivata $\sec^2 (x)$ \`e $\tan(x)$. Infatti, facendo la derivata di un rapporto:
\[
[\tan(x)]' = \deriv{x} \frac{\sin(x)}{\cos(x)} = \frac{\cos^2 (x) + \sin^2 (x)}{\cos^2(x)} = \frac{1}{\cos^2(x)} = \sec^2 (x)
\]
Derivando la cotangente invece abbiamo:
\begin{align*}
[\cot(x)]' = \deriv{x} \frac{\cos (x)}{\sin (x)} = \frac{- \sin^2(x) - \cos^2 (x)}{\sin^2(x)} = - \frac{1}{\sin^2 (x)} = - \csc^2 (x)
\end{align*}
Quindi l'integrale della secante al quadrato \`e la tangente e l'integrale della cosecante al quadrato \`e ``meno cotangente''.

Come comportarsi con funzioni trigonometriche che hanno per argomento una $x$ con coefficienti? Un caso pu\`o essere:
\[
\left[ \cos (14 \, x) \right]' = - 14 \, \sin (14 \, x) \implies
\int \sin (14 \, x) \dx = - \frac{\cos (14 \, x)}{14}  + c
\]
Nel caso seguente, l'integrale non \`e semplicemente $\tan(-3 \, x) + c$. Il coefficiente $-3$ va messo al posto giusto, e un buon modo per rendersi conto di cosa manca \`e derivare il risultato.
\[
\int \frac{\diff x}{\cos^2 (-3 \, x)} = -\frac{\tan (-3 \, x)}{3} + c
\]

Ritorniamo all'integrale della tangente. Abbiamo visto come pu\`o riscriversi:
\[
\int \tan (x) \dx = \int \frac{\sin (x)}{\cos (x)} \dx
\]
Considerando $y = \cos(x)$, possiamo avere un incontro ravvicinato con l'integrazione per sostituzione. Infatti l'integrale della tangente \`e simile a questo integrale:
\[
\int \frac{\diff y}{y} = \abslog{y} + c 
\]
Sostituendo $\cos (x)$ alla $y$ e derivando vediamo di essere vicini alla soluzione:
\[
\left[ \abslog{\cos(x)} \right]' = - \frac{1}{\cos(x)} \sin (x)
\]
E quindi, la soluzione \`e:
\[
\int \tan (x) \dx = - \abslog{ \cos(x) } + c
\]
Passiamo a questa:
\[
\int \sec (x) \, \tan (x) \dx = \int \frac{\sin(x)}{\cos^2(x)} \dx
\]
Questo integrale indefinito \`e simile all'integrale indefinito:
\[
\int \frac{\diff y}{y^2} = \int y^{-2} \dx[y] = - \frac{1}{y} + c
\]
Quindi:
\[
\int \sec (x) \, \tan (x) \dx = - \frac{1}{\cos(x)} + c
\]
Si verifica rapidamente:
\[
\left[ - \frac{1}{\cos (x)} \right]' = - \left[ \frac{1}{\cos (x)} \right]' = - \frac{1}{\cos^2 (x)} \cdot \left( - \sin (x) \right) = \frac{\sin(x)}{\cos^2(x)}
\]
Non abbiamo ancora metodi per risolvere gli integrali. Quindi facciamo congetture, e poi verifichiamo che siano vere.

Ci sono due integrali che non sembrerebbero avere a che fare con funzioni trigonometriche, e che invece, guarda un po':
\begin{align*}
\int \frac{\diff x}{\sqrt{1 - x^2}} = \arcsin(x) + c \\
\int \frac{\diff x}{1 + x^2} = \arctan(x) + c
\end{align*}
Per trovarli basta svolgere le derivate delle funzioni trigonometriche inverse, tenendo a mente questo:
\[
\left[ f^{-1} (x) \right]' = \frac{1}{f' \left( f^{-1} (x) \right)}
\]
Con qualche semplificazione si trovano (miracolosamente) funzioni razionali.

\subsection{Infinite antiderivate}

Come mai ci sono infinite antiderivate? Conoscendo lo spostamento di un oggetto, si pu\`o conoscere la sua velocit\`a, come derivata prima dello spostamento. Ma non si pu\`o ricostruire lo spostamento semplicemente dalla velocit\`a: serve sapere da dove si \`e partiti. Quindi, abbiamo infinite antiderivate che differiscono di una costante. Stesso discorso vale per l'accelerazione, derivata seconda dello spostamento. Un problema del genere pu\`o essere:
\begin{align*}
y'' &= \sin (x) \\
y(x_0) &= 3 \\
y'(x_0) &= -2
\end{align*}
$y$ \`e lo spostamento. Introduciamo una variabile intermedia $z = y'$, per chiarire il lavoro. Abbiamo infinite funzioni $z$ che hanno per derivata $\sin (x)$, tutte nella forma $z(x) = - \cos(x) + c$. Siccome deve valere che $- \cos (x_0) + c = -2$, troviamo che $c = \cos (x_0) - 2$. Ora dobbiamo trovare $y(x)$.
\[
y' = - \cos(x) + \cos (x_0) - 2
\]
Di nuovo infinite soluzioni:
\[
y = - \sin (x) + (\cos (x_0) - 2) \cdot x + k
\]
Imponendo che $y(x_0) = 3$, abbiamo che:
\[
k = 3 + \sin (x_0) - \cos(x_0) \, x_0 + 2 \, x_0
\]

\section{Integrazione per sostituzione}

Per calcolare le derivate bisognerebbe stare continuamente a calcolare limiti di rapporti incrementali, se non ci fossero le regole di derivazione per le funzioni composte. L'integrazione per sostituzione utilizza proprio la regola per la derivazione delle funzioni composte.

Un esempio di integrazione per sostituzione \`e l'esercizio, gi\`a fatto:
\[
\int \frac{2x}{\sqrt{1 + x^2}} \dx = \substitute{\int \frac{\diff y}{\sqrt{y}}}{y = 1 + x^2} = \substitute{2 \, \sqrt{y} + c}{y = 1 + x^2} = 2 \, \sqrt{1 + x^2} + c
\]
O anche questo:
\[
\int \sin(x) \, \cos(x) \dx = \substitute{\int y \dx[y]}{y = \sin (x)} = \substitute{\frac{y^2}{2} + c}{y = \sin (x)} = \frac{\sin^2(x)}{2} + c
\]
Quello che abbiamo fatto in questi due casi \`e una sostituzione di questo tipo:
\[
\int f(x) \, f'(x) \dx = \substitute{\int y \dx[y]}{y = f(x)}
\]
per poi svolgere l'integrale pi\`u semplice che abbiamo trovato. La funzione integranda \`e una funzione per la derivata della stessa.

\begin{defn}[Integrazione per sostituzione]
Questa \`e la regola generale dell'integrazione per sostituzione:
\[
\int g(f(x)) \, f'(x) \dx = \substitute{\int g(y) \dx[y]}{y = f(x)}
\]
Il risultato sar\`a la famiglia di funzioni $G(f(x)) + c$, con $G'(x) = g(x)$, ossia una primitiva qualsiasi di $g(x)$. Infatti, per la regola di derivazione delle funzioni composte:
\[
\left[ G(f(x)) \right]' = G'(f(x)) \, f'(x) = g(f(x)) \, f'(x)
\]
\end{defn}
La formula dell'integrazione per sostituzione si usa anche ``nel senso inverso''. A volte, infatti, partiamo da un integrale in cui non \`e evidente quale sostituzione fare. In questi casi si sceglie la sostituzione, come $y = \sqrt{x}$, e con qualche manipolazione si riscrive l'espressione, sostituendo ad esempio $\diff x = 2 \, y \dx[y]$.

\begin{oss}
L'integrazione per sostituzione vale perch\'e vale la regola di derivazione delle funzioni composte.
\end{oss}

Procediamo con un esempio.
\[
\int \frac{x^3}{1 - 5 \, x^4} \dx
\]
Per risolvere integrali per sostituzione, conviene anzitutto isolare gli elementi dell'integrale che si ritengono utili. Si procede a tentoni: si pu\`o vedere che $x^3 \dx \sim \left[ 1 - 5 \, x^4 \right]'$, ossia le due funzioni sono simili. Quindi, continuando a fare tentativi, derivando a caso e pregando Ges\`u troviamo questo:
\[
\left[ 1 - 5 \, x^4 \right]' = - 20 x^3 \implies
x^3 \dx = - \frac{1}{20} \, \left[ 1 - 5 \, x^4 \right]'
\]
Da qui si pu\`o procedere per sostituzione.
\[
\int \frac{x^3}{1 - 5 \, x^4} = \substitute{- \frac{1}{20} \, \int \frac{\diff y}{y}}{y = 1 - 5 \, x^4} = \substitute{- \frac{\abslog{y}}{20} + c}{y = 1 - 5 \, x^4} = - \frac{\abslog{1 - 5 \, x^4}}{20} + c
\]

\subsection{Scegliere la sostituzione a priori}
\label{sostituzione_a_priori}

Vediamo un integrale difficile:
\[
\int \frac{\cos(x)}{\sqrt{\sin(x)} + 1} \dx = 
\substitute{\int \frac{\diff y}{\sqrt{y} + 1}}{y = \sin (x)}
\]
E ora? Facciamo integrazione per sostituzione, ma non cercando l'espressione di una funzione da sostituire e la sua derivata. Decidiamo a priori di sostituire $y = s^2$, e vediamo che con questa sostituzione vale che $\diff y = 2 \, s \dx[s]$.
\[
\substitute{\int \frac{\diff y}{\sqrt{y} + 1}}{y = \sin (x)} =
\substitute{\int \frac{2 \, s \dx[s]}{s + 1}}{s = \sqrt{y}}
\]
Da qui, si procede con i soliti mezzi, ossia a casaccio. Vedremo meglio pi\`u avanti come spezzare funzioni razionali in funzioni pi\`u semplici:
\[
\int \frac{2 \, s \dx[s]}{s + 1} = 
2 \, \int \frac{s + 1 - 1}{s + 1} \dx[s] = 
2 \, \left( \int \diff s - \int \frac {\diff s}{s + 1} \right) = 2 \, s - 2 \, \abslog{s + 1} + c
\]
Sostituiamo di nuovo, procedendo a ritroso:
\[
2 \, \sqrt{y} - 2 \, \abslog{\sqrt{y} + 1} + c =
2 \, \sqrt{\sin(x)} - 2 \, \abslog{\sqrt{\sin (x)} + 1 } + c
\]

Quest'altro modo di sostituire cerca di individuare la funzione inversa di una delle funzioni presenti nell'integrando, poich\'e una sostituzione simile pu\`o potenzialmente portare a semplificazioni utili. Nel caso di prima, avendo un $\sqrt{y}$, si \`e scelta la sostituzione $y = s^2$, per ottenere poi che $\sqrt{y} = s$.

\subsection{Potenze di seno e coseno}

Consideriamo questo integrale:
\[
\int \frac{\sin^3(x)}{\cos^4(x)} \dx
\]
In generale, quando c'\`e una potenza dispari del seno e una potenza del coseno (o viceversa), la potenza dispari del seno \`e il prodotto di una potenza del coseno per la derivata del coseno... Ossia, se c'\`e un $\sin^{2 n + 1} (x)$, questo si pu\`o sostituire con $(1 - \cos^2(x))^{n} \, \sin(x)$.
\[
\int \frac{\sin^3(x)}{\cos^4(x)} \dx =
\int \frac{\sin(x) \, (1 - \cos^2(x))}{\cos^4(x)} \dx =
\substitute{\int \frac{1 - y^2}{y^4} \dx[y]}{y = \cos(x)}
\]
Una volta fatte le giuste sostituzioni, risolvere l'integrale diventa un lavoro meccanico, riconducibile a integrali gi\`a visti.

Cerchiamo di calcolare questo integrale. C'\`e una formula che si potrebbe imparare a memoria, ma non vogliamo impararla. Per risolverlo bisogna moltiplicare e dividere, e ottenere un integrale ``migliore''.
\[
\int \sec (x) \dx = \int \frac{\diff x}{\cos (x)} = 
\int \sec (x) \, \frac{\sec(x) + \tan (x)}{\sec (x) + \tan (x)} \dx
\]
Adesso sostituiamo con $u = \sec (x) + \tan (x)$. Fidiamoci. Calcoliamo $\diff u$.
\[
\diff u = \left[ \sec(x) + \tan (x) \right]' =
\frac{\sin (x)}{\cos^2 (x)} + \frac{1}{\cos^2 (x)} =
\sec (x) \, (\sec (x) + \tan (x))
\]
Questo ci torna utile, ora! Quindi possiamo sostituire in questo modo:
\[
\int \sec (x) \, \frac{\sec(x) + \tan (x)}{\sec (x) + \tan (x)} \dx =
\substitute{\int \frac{\diff u}{u}}{u = \sec(x) + \tan(x)} =
\abslog{\sec(x) + \tan(x)} + c
\]
Non c'\`e nulla di immediato in questo integrale, come in quello di $\csc(x)$.

\subsection{Sostituzione con seno e coseno}

A volte, come visto nella sezione \ref{sostituzione_a_priori}, non c'\`e l'espressione di una funzione $f(x)$ con la sua derivata che ci renda facile la sostituzione. Si pu\`o provare quindi una sostituzione $x = f(y)$. Una sostituzione tipica di questo tipo \`e quella $x = \cos(\theta)$ o $x = \sin(\theta)$. Consideriamo ad esempio questo integrale:
\[
\int \sqrt{1 - x^2} \dx
\]
Sappiamo che qualcosa come $1 - f(\theta)^2$, se $f(\theta) = \cos (\theta)$, si pu\`o scrivere come $\sin^2 (\theta)$. Allora si pu\`o provare a sostituire, vedendo che $\diff \theta = - \sin (x) \dx$:
\[
\substitute{ - \int \sqrt{1 - \cos^2 (\theta)} \, \sin(\theta) \dx[\theta]}{x = \cos(\theta)} =
- \int \abs{\sin(\theta)} \, \sin(\theta) \dx[\theta]
\]
Questo integrale va ora spezzato in due, a seconda degli intervalli in cui $\sin(\theta)$ \`e positivo o negativo. Da qui \`e un integrale ``semplice'' come:
\[
\int \sin^2 (\theta) \dx[\theta]
\]
Una volta risolto l'integrale, bisogner\`a sostituire di nuovo $\theta$ con $\arccos(x)$.

\subsection{Funzione inverse delle funzioni trigonometriche}
\label{integrale_arcoseno}

Sappiamo che la derivata dell'arcoseno \`e $\frac{1}{\sqrt{1 - x^2}}$. Vediamo come trovare l'integrale di un caso pi\`u generale, come $\frac{1}{\sqrt{a - b \, x^2}}$.
\begin{align*}
\int \frac{\diff x}{\sqrt{3 - 2 \, x^2}} &= 
\frac{1}{\sqrt{3}} \int \frac{\diff x}{\sqrt{1 - \frac{2}{3} \, x^2}} = \tag{mettendo in evidenza} \\
&= \substitute{\frac{1}{\cancel{\sqrt{3}}} \, \frac{\cancel{\sqrt{3}}}{\sqrt{2}} \int \frac{\diff y}{1 - y^2}}{y = \sqrt{\frac{2}{3}} \, x} = \tag{sostituendo} \\
&= \frac{1}{\sqrt{2}} \, \arcsin (y) + c =
\frac{1}{\sqrt{2}} \, \arcsin \left(\sqrt{\frac{2}{3}} \, x \right) + c
\end{align*}
Volendo tirar fuori una regola generale:
\[
\int \frac{\diff x}{\sqrt{a^2 - b^2 \, x^2}} =
\frac{1}{a} \, \arcsin \left( \frac{a}{b} \, x \right) + c
\]

\section{Integrazione per parti}

Sappiamo che c'\`e una regola per la derivata di un prodotto di funzioni.
\[
\left[ u(x) \, v(x) \right]' = u'(x) \, v(x) + u(x) \, v'(x)
\]
Se integriamo entrambi i membri abbiamo questo:
\[
u(x) \, v(x) + c = 
\int u'(x) \, v(x) \dx + \int u(x) \, v'(x) \dx
\]
Passiamo a un esempio particolare:
\[
\int x \, e^x \dx
\]
Si pu\`o vedere che, considerando $u(x) = x$, abbiamo che $x = \left( \frac{x^2}{x} \right)' = u'(x)$, e che $v(x) = e^x$. Sar\`a una buona sostituzione? La scelta, purtroppo, \`e infelice. Infatti:
\[
\int x \, e^x \dx = \frac{x^2}{x} e^x - \int \frac{x^2}{x} e^x \dx
\]
Peggio di prima! La scelta pi\`u felice \`e al contrario:
\[
\int x \, e^x \dx = x \, e^x - \int e^x \dx = x \, e^x - e^x + c
\]
Le derivate delle potenze della $x$ tendono a ``scomparire'' nella funzione costante. Quindi con l'integrazione per parti, l'integrale $\int x \, e^x \dx$ si riduce a calcolare il banale integrale $\int e^x \dx$.

\begin{defn}[Integrazione per parti]
La formula per l'integrazione per parti \`e questa:
\[
\int u' (x) \, v(x) \dx = u(x) \, v(x) - \int u (x) \, v' (x) \dx
\]
Si devono riuscire a riconoscere due funzioni nell'integrando, una da integrare e una da derivare. Applicando l'integrazione per parti, si risolve parte dell'integrale e ci si ritrova con un nuovo integrale da risolvere, speriamo pi\`u semplice.
\end{defn}

Individuare la funzione da integrare e quella da derivare \`e il primo passo.

Altro caso veloce veloce:
\[
\int x \, \sin(x) \dx = - x \, \cos (x) + \int \cos (x) \dx =
- x \, \cos(x) + \sin (x) + c
\]

Si pu\`o vedere come un integrale da risolvere per parti anche questo integrale:
\[
\int \log(x) \dx
\]
Possiamo prendere infatti $u(x) = x$, e avere quindi $u'(x) = 1$ e $v(x) = \log (x)$. Quindi:
\[
\int 1 \, \log(x) \dx = x \, \log (x) - \int \frac{\cancel{x}}{\cancel{x}} \dx = 
x \, \log (x) - x + c
\]
Abbiamo scelto la $x$ (o meglio, la sua derivata, la funzione costante) come funzione da integrare. Tipicamente la $x$ scompare se scelta come funzione da \emph{derivare}, ma in questo caso si semplifica con la derivata del logaritmo.

Il comportamento da tenere con le funzioni composte \`e simile a quello che si tiene con l'integrazione per sostituzione. Questo integrale potrebbe non essere facilmente risolvibile, nel caso generale:
\[
\int f(g(x)) \, h(x) \dx
\]
Ma a quest'altro integrale si pu\`o sempre applicare l'integrazione per parti:
\[
\int f(g(x)) \, g'(x) \, h(x) \dx =
F(g(x)) \, h(x) - \int F(g(x)) \, h'(x) \dx
\]
Per risolvere integrali simili bisogna riconoscere pattern. In questo caso, come con l'integrazione per sostituzione, bisogna trovare una funzione $g$ composta con una funzione $f$ vicino alla sua derivata $g'$.

\begin{exmp}
Un integrale con un polinomio e un'esponenziale, come quello che segue, si pu\`o risolvere per parti, scegliendo il polinomio come funzione da derivare (in modo da abbassarne il grado) e l'esponenziale come funzione da integrare (ch\'e tanto resta sempre uguale).
\[
\int (x - 3) \, e^{2x} \dx =
(x - 3) \, \frac{e^{2x}}{2} - \frac{1}{2} \int e^{2x} \dx = \dots
\]
E via cos\`i...
\end{exmp}

\begin{exmp}
Un integrale con un logaritmo e una $x$, come quello che segue, si pu\`o risolvere per parti, scegliendo la $x$ come funzione da integrare (a differenza dell'esempio di prima) e il logaritmo come funzione da derivare.
\[
\int x \, \log^3 (x) \dx =
\frac{x^2}{2} \, \log^3 (x) - \frac{3}{2} \int \frac{x^{\cancel{2}}}{\cancel{x}} \, \log^2 (x) \dx = \dots
\]
Poi si procede ripetitivamente, abbassando l'ordine della potenza del logaritmo... Si potrebbe prendere il logaritmo come funzione da integrare, e si potrebbe pensare di farlo per parti, e si potrebbe pensare di integrare il logaritmo (invece di derivarlo).
\[
\int \log^3 (x) \dx =
\int x \, \frac{\log^3 (x)}{x} \dx =
\frac{1}{4} \, x \, \log^4 (x) - \frac{1}{4} \int \log^4 (x) \dx
\]
Ma ci ritroviamo a navigare nella merda. Quindi: \`e bene derivare il logaritmo.
\end{exmp}

\section{Integrali di funzioni razionali}

Siano $p(x)$ e $q(x)$ due polinomi, la funzione $\frac{p(x)}{q(x)}$ \`e una funzione razionale.
\[
\int \frac{p(x)}{q(x)} \dx
\]
Gli integrali di una funzione razionale (come quello sopra) sono \emph{sempre} combinazioni lineari di polinomi, logaritmi e arcotangenti.

Se il grado di $p(x)$ \`e maggiore del grado di $q(x)$, possiamo fare una divisione fra polinomi (cfr. corso di algebra) e ottenere un integrale equivalente della forma $\int h(x) \dx + \int \frac{r(x)}{q(x)} \dx$, tale per cui $h(x)$, $r(x)$ e $q(x)$ sono tutti polinomi e il grado del polinomio $r(x)$ \`e minore del grado di $q(x)$.

\begin{exmp}
Partiamo con una cosa facile facile, giusto per scaldarci:
\[
\int \frac{x^2}{x - 4} \dx =
\int \left[ x + 4 + \frac{16}{x-4} \right] \dx 
\]
\begin{figure}[h]
\centering
\begin{tabular}{rrr|r}
$x^2$ & & & $x - 4$ \\ \cline{4-4}
$-x^2$ & $+4 \, x$ & & $x + 4$ \\ \cline{1-2}
$0$ & $4 \, x$ & & \\
& $- 4 \, x$ & $+ 16$ & \\ \cline{2-3}
& $0$ & $16$ & 
\end{tabular}
\caption{\label{fig:divisione}Divisione fra polinomi...}
\end{figure}
Per la divisione si segue il procedimento in figura \ref{fig:divisione}.

Ora risolvere l'integrale \`e facile...
\end{exmp}

\begin{exmp}
Facciamo un altro esempio di applicazione del teorema di divisione per i polinomi:
\[
\int \frac{x^2}{5 \, x^2 + 3} \dx = 
\frac{1}{5} \, \int \frac{x^2}{x^2 + \frac{3}{5}} \dx = 
\frac{1}{5} \left[ \int \dx - \frac{3}{5} \int \frac{\diff x}{x^2 + \frac{3}{5}} \right] 
\]
Il primo integrale \`e banale, il secondo, come al solito, va ricondotto all'integrale di $\frac{\diff y}{y^2 + 1}$ (che sappiamo essere l'arcotangente).
\[
\int \frac{\diff x}{x^2 + \frac{3}{5}} =
\frac{1}{\frac{3}{5}} \int \frac{\diff x}{\frac{5}{3} \, x^2 + 1} =
\frac{5}{3} \substitute{\int \sqrt{\frac{5}{3}} \frac{\diff y}{y^2 + 1} }{y^2 = \frac{5}{3} \, x^2 \implies y = \sqrt{\frac{5}{3}} \, x}
\]
Le sostituzioni da fare sono sempre le stesse: per trasformare $x^2 + \frac{3}{5}$ in qualcosa come $y^2 + 1$, si mette anzitutto in evidenza $\frac{3}{5}$, e si ottiene quindi $\frac{3}{5} \, (\frac{5}{3} \, x^2 + 1)$, e poi si sostituisce $\frac{5}{3} \, x^2$ con $y^2$, e quindi $y = \sqrt{\frac{5}{3}} \, x$ (e $\diff y = \sqrt{\frac{5}{3}} \dx$).
\end{exmp}

\begin{exmp}[(spezzare integrali razionali)]
Se abbiamo qualcosa come questo?
\[
\int \frac{\diff x}{x^2 - \frac{3}{5}}
\]
Bisogna spezzare in due l'integrale.
\[
x^2 - \frac{3}{5} = 
\left( x - \sqrt{\frac{3}{5}} \right) \cdot \left( x + \sqrt{\frac{3}{5}} \right)
\]
Quindi:
\begin{align*}
\int \frac{\diff x}{x^2 - \frac{3}{5}} &= 
\int \frac{\diff x}{\left( x - \sqrt{\frac{3}{5}} \right) \cdot \left( x + \sqrt{\frac{3}{5}} \right)} = \\
&= \int \frac{A \dx}{x - \sqrt{\frac{3}{5}}} + \int \frac{B \dx}{x + \sqrt{\frac{3}{5}}} = \\
&= \int \frac{A \, \left( x - \sqrt{\frac{3}{5}} \right) + B \, \left( x + \sqrt{\frac{3}{5}} \right)}{x^2 - \frac{3}{5}} \dx 
\end{align*}
Spezziamo l'integrale in una somma con coefficienti $A$ e $B$ sconosciuti, e poi sommiamo i due addendi per avere un modo per scoprire quanto devono valere i coefficienti $A$ e $B$.

Adesso bisogna trovare per quali valori di $A$ e di $B$ vale questo:
\[
A \, \left( x - \sqrt{\frac{3}{5}} \right) + B \, \left( x + \sqrt{\frac{3}{5}} \right) = 1
\]
I due valori sono $A = - \frac{1}{2} \, \sqrt{\frac{5}{3}}$, e $B = -A$.
\end{exmp}

\subsection{Polinomi di secondo grado (e oltre) al denominatore}

Ribadiamo una verit\`a fondamentale: gli integrali di funzioni razionali sono \emph{sempre} una somma di polinomi, logaritmi e arcotangenti. Chiariamo un po' il ruolo dell'arcotangente in questi integrali, e in generale cosa fare con gli integrali di funzioni razionali con polinomi di secondo grado al denominatore.

\subsubsection{Polinomi di secondo grado senza coefficiente del primo grado}

Come visto nella sezione \ref{integrale_arcoseno} per l'integrale di $\frac{\diff x}{\sqrt{1 - x^2}}$, troviamo una formula generale per l'integrale:
\[
\int \frac{\diff x}{x^2 + a^2}
\]
Come al solito, trasformiamo trasformiamo l'integrale in $\frac{\diff y}{y^2 + 1}$, mettendo prima in evidenza $a^2$ e poi facendo una sostituzione opportuna:
\[
\frac{1}{a^2} \int \frac{\diff x}{\frac{1}{a^2} \, x^2 + 1} =
\frac{1}{a^{\cancel{2}}} \substitute{\int \frac{\cancel{a} \dx[y]}{y^2 + 1}}{y = \frac{1}{a} \, x} =
\frac{1}{a} \arctan \left( \frac{x}{a} \right) + c
\]
Nella sostituzione, essendo $y = \frac{1}{a} \, x$, vale che $\diff x = a \dx[y]$.

Se il segno di $a^2$ \`e negativo ci si comporta in modo diverso.
\[
\int \frac{\diff x}{x^2 - a^2}
\]
Bisogna dividere l'integrale in questo modo:
\[
\int \left( \frac{A}{x + a} + \frac{B}{x - a} \right) \dx
\]
e poi si deve imporre questo:
\begin{align*}
& A \, (x - a) + B \, (x + a) = 1 \implies \\
\implies & (A + B) \, x + (B - A) \, a = 1 \implies \\
\implies & \begin{cases}
A = - B \\
2 \, B = \frac{1}{a}
\end{cases}
\end{align*}
L'integrale quindi si spezza in:
\[
- \frac{1}{2 \,a} \int \frac{\diff x}{x + a}
+ \frac{1}{2 \,a} \int \frac{\diff x}{x - a}
\]

\subsubsection{Polinomi di secondo grado generici}

Abbiamo fatto fuori i casi in cui il coefficiente della $x$ di primo grado \`e 0 (ossia, i casi in cui abbiamo al denominatore un polinomio $x^2 + c$). Vediamo un caso diverso:
\[
\int \frac{\diff x}{x^2 + 2 \, x - 3} =
\int \frac{\diff x}{(x - 1) \cdot (x + 3)}
\]
E qui sappiamo di nuovo come procedere (perch\'e il polinomio ha radici reali distinte): come visto poco sopra, si divide l'integrale:
\[
\int \frac{\diff x}{(x - 1) \cdot (x + 3)} =
\int \left[ \frac{A}{x - 1} + \frac{B}{x + 3} \right] \dx
\]
Poi si trovano i valori di $A$ e $B$ che verificano l'uguaglianza. Ma qui ci ha detto bene...

\subsubsection{Polinomi di secondo grado senza radici reali}

In questo caso non possiamo spezzare l'integrale, il determinante \`e negativo (e il polinomio ha due radici complesse coniugate):
\[
\int \frac{\diff x}{x^2 + 2 \, x + 3}
\]
Ma forse possiamo ricondurre $x^2 + 2 \, x + 3$ a un polinomio del tipo $y^2 + a^2$. Lo possiamo scrivere come segue:
\[
x^2 + 2 \, x + 3 = x^2 + 2 \, x + 1 + 2 =
{(x + 1)}^2 + 2
\]
E quindi l'integrale \`e un arcotangente.

\subsubsection{Polinomi di secondo grado con radici reali coincidenti}

Abbiamo visto il caso in cui abbiamo radici reali \emph{distinte}, e ora vediamo il caso in cui abbiamo radici reali \emph{coincidenti}.
\[
\int \frac{\diff x}{x^2 + 2 + 1} =
\int \frac{\diff x}{{(x + 1)}^2}
\]
La strada corretta \`e per sostituzione, ma a prima (s)vista si potrebbe pensare di ricondurlo a un integrale tipo:
\[
\int \frac{\diff x}{(x + \alpha) \cdot (x + \beta)} =
\int \left[ \frac{A}{x + \alpha} + \frac{B}{x + \beta} \right] \dx
\]
Questo integrale si pu\`o riscrivere cos\`i solo quando $\alpha$ \`e diverso da $\beta$! Infatti si avrebbe questo:
\[
\frac{A}{x + \alpha} + \frac{B}{x + \alpha} = \frac{A + B}{x + \alpha} \neq \frac{\diff x}{{(x + \alpha)}^2}
\]
Totalmente sbagliato!

La decomposizione che funziona, se si vuole procedere per decomposizione, \`e questa:
\[
\frac{1}{{(x + \alpha)}^2} =
\frac{A}{x + \alpha} + \frac{B \, x}{{(x + \alpha)}^2} =
\frac{A \, (x + \alpha) + B \, x}{{(x + \alpha)}^2}
\]
Adesso dobbiamo trovare i valori di $A$ e di $B$ tali per cui:
\[
A \, (x + \alpha) + B \, x = 1
\]
E il gioco \`e fatto.

\subsubsection{Polinomi di primo grado su polinomi del secondo grado}

Integrali sempre pi\`u brutti: ora vediamo funzioni razionali con polinomi del primo grado al numeratore, e polinomi di secondo grado al denominatore.
\[
\int \frac{C \, x + D}{x^2 + bx + c} \dx
\]

\begin{exmp}
Guardiamo questo caso:
\[
\int \frac{x \dx}{x^2 + 2 \,x + 1} =
\frac{1}{2} \int \frac{2 \,x \dx}{x^2 + 2 \, x + 1} =
\frac{1}{2} \int \frac{2 \, x + 2 \dx}{x^2 + 2 \, x + 1} - \int \frac{\dx}{x^2 + 2 \, x + 1}
\]
Il primo \`e un logaritmo, il secondo l'abbiamo gi\`a visto.
\end{exmp}

\subsubsection{Polinomi (speciali) di terzo grado}

Ora abbiamo un polinomio di primo grado per un polinomio del secondo grado (ossia, un polinomio di terzo grado molto particolare).

\begin{exmp}[di polinomio a radici reali distinte]
Qui il secondo polinomio ha due radici reali distinte (per nostra somma fortuna):
\[
\int \frac{\diff x}{x \, (x^2 + 2 \, x - 3)} =
\int \frac{\diff x}{x \cdot (x - 1) \cdot (x + 3)} =
\int \left( \frac{A}{x} + \frac{B}{x - 1} + \frac{C}{x + 3} \right) \dx
\]
\end{exmp}

\begin{exmp}[di polinomio a radici complesse coniugate]
Qui il secondo polinomio ha due radici complesse coniugate:
\[
\int \frac{\diff x}{x \, (x^2 + 2 \, x + 3)}
\]
La tentazione di scrivere questo \`e sbagliatissima (si finisce in uno dei peggio gironi dell'inferno):
\[
\int \left( \frac{A}{x} + \frac{B}{x^2 + 2 \, x + 3} \right) \dx
\]
Non abbiamo speranze. Non troveremo mai due costanti $A$ e $B$ che soddisfano questa uguaglianza: avremmo infatti che $A \, x^2 = 0$, e non ci va bene. Non dobbiamo andare a cercare \emph{due} costanti, ma tre:
\[
\int \left( \frac{A}{x} + \frac{B \, x + C}{x^2 + 2 \, x + 3} \right) \dx
\]
Adesso le condizioni da imporre sono soddisfacibili:
\[
\begin{cases}
A + B = 0 \\
2 \, A + C = 0 \\
3 \, A = 1
\end{cases}
\]
\end{exmp}

\begin{exmp}[di polinomio a radici reali coincidenti]
Qui il secondo polinomio ha due radici reali coincidenti:
\[
\int \frac{\diff x}{x \, (x^2 + 2 \, x + 1)} =
\int \frac{\diff x}{x \, {(x + 1)}^2}
\]
Quel che dobbiamo scrivere adesso \`e questo:
\[
\int \left( \frac{A}{x} + \frac{B}{x + 1} + \frac{C}{{(x+1)}^2} \right) \dx
\]
Andando a svolgere troviamo questo:
\[
\int \frac{A \, {(x + 1)}^2 + B \, x \, (x + 1) + C \, x}{x \, {(x + 1)}^2} \dx
\]
Che vuol dire che le condizioni da imporre sono queste:
\[
\begin{cases}
A + B = 0 \\
2 \, A + B + C = 0 \\
A = 1
\end{cases}
\]
Il che equivale a dire che $A = 1$, $B = -1$ e $C = - 1$.
\end{exmp}

\subsubsection{Polinomi a radici complesse coniugate}

Questi sono forse gli integrali razionali pi\`u brutti. Al denominatore c'\`e un polinomio a radici complesse coniugate \emph{elevato al quadrato}. Sono integrali nella forma:
\[
\int \frac{\diff x}{(x^2 - x + 1)^2}
\]
Se il polinomio non fosse elevato al quadrato, sapremmo risolverlo con una sostituzione opportuna:
\[
\int \frac{\diff x}{x^2 - x + 1} = 
\int \frac{\diff x}{x^2 - x + \frac{1}{4} + \frac{3}{4}} = 
\int \frac{\diff x}{{\left(x - \frac{1}{2}\right)}^2 + \frac{3}{4}} = 
\substitute{\int \frac{\diff y}{y^2 + \frac{3}{4}}}{y = x - \frac{1}{2}}
\]
Anche questi si risolveranno con una sostituzione...

\begin{exmp}[di polinomio a radici complesse, elevato al quadrato]
Vediamo questo integrale.
\[
\int \frac{C \, x + D}{\left( x^2 - x + 1 \right)^2} \dx
\]
Si spezza in questo modo:
\[
\int \frac{C}{2} \frac{2 \, x - 1}{\left( x^2 - x + 1 \right)^2} \dx +
\int \frac{C}{2} \frac{\diff x}{\left( x^2 - x + 1 \right)^2} +
\int \frac{D \dx}{\left( x^2 - x + 1 \right)^2}
\]
Il primo integrale \`e facile, e infatti \`e:
\[
\int \frac{C}{2} \frac{2 \, x - 1}{\left( x^2 - x + 1 \right)^2} \dx =
- \frac{C}{2} \frac{1}{x^2 - x + 1} + c
\]
Gli ultimi due sono quelli cattivi.
\end{exmp}

Si possono distinguere tre casi (prendendo come esempio il polinomio a radici complesse $x^2 - x + 1$), tutti riconducibili all'ultimo:
\begin{enumerate}
    \item Il primo caso ha al numeratore $x^2$, e si pu\`o spezzare cos\`i:
    \begin{align*}
    \int \frac{x^2 \dx}{(x^2 - x + 1)^2} &= 
    \int \frac{\diff x}{x^2 - x + 1} + \int \frac{x - 1}{(x^2 - x + 1)^2} \dx = \\
    &= \int \frac{\diff x}{x^2 - x + 1} + \frac{1}{2} \, \int \frac{2 \, x - 1}{(x^2 - x + 1)^2} \dx - \frac{1}{2} \, \int \frac{\diff x}{(x^2 - x + 1)^2}
    \end{align*}
    Dei tre integrali all'ultimo passaggio, sappiamo risolvere il primo (con la sostituzione vista prima), e il secondo (al numeratore compare la derivata del polinomio). Il terzo integrale ricade nel caso \ref{itm:terzo_caso_polinomi_complessi}.
    \item Il secondo caso ha al numeratore $x$, e si spezza cos\`i:
    \begin{align*}
    \int \frac{x \dx}{(x^2 - x + 1)^2} &= \frac{1}{2} \, \int \frac{2 \, x + 1 - 1 \dx}{(x^2 - x + 1)^2} = \\
    &= \frac{1}{2} \, \int \frac{2 \, x - 1}{(x^2 - x + 1)^2} \dx + \frac{1}{2} \, \int \frac{\diff x}{(x^2 - x + 1)^2} \\
    \end{align*}
    Il primo integrale dell'ultimo passaggio sappiamo risolverlo, il secondo integrale ricade nel caso \ref{itm:terzo_caso_polinomi_complessi}.
    \item\label{itm:terzo_caso_polinomi_complessi} Resta quindi solo il caso:
    \[
    \int \frac{\diff x}{(x^2 - x + 1)^2} 
    \]
    E su questo possiamo fare una sostituzione.
\end{enumerate}

La sostituzione da fare \`e quella gi\`a vista prima:
\[
\int \frac{\diff x}{(x^2 - x + 1)^2} =
\substitute{\int \frac{\diff y}{{\left(y^2 + \frac{3}{4}\right)}^2}}{y = x - \frac{1}{2}}
\]
Sembra allontanarci dalla soluzione, ma adesso prendiamo in esame l'integrale che segue. Possiamo spezzare anche questo, e arrivare ad un nuovo integrale...
\[
\int \frac{\diff y}{(1 + y^2)^2} =
\int \frac{1 + y^2 - y^2}{(1 + y^2)^2} \dx[y] =
\int \frac{\cancel{1 + y^2}}{(1 + y^2)^{\cancel{2}}} \dx[y] -
\int \frac{y^2 \dx[y]}{(1 + y^2)^2}
\]
E questo \`e l'ultimo! Si risolve per parti, e a cascata, ora, sappiamo risolvere uno dopo l'altro tutti gli integrali visti in questa sezione.
\[
\int \frac{y^2 \dx[y]}{(1 + y^2)^2} = 
\frac{1}{2} \, \int y \, \frac{2 \, y}{(1 + y^2)^2} \dx[y] =
- \frac{1}{2} \frac{y}{1 + y^2} + \frac{1}{2} \int \frac{\diff y}{1 + y^2} = 
\frac{1}{2} \, \left[ \arctan(y) - \frac{y}{1 + y^2} \right] + c
\]
Da cui sappiamo la soluzione del penultimo:
\[
\int \frac{\diff y}{(1 + y^2)^2} =
\int \frac{\diff y}{1 + y^2} - \frac{1}{2} \, \left[ \arctan(y) - \frac{y}{1 + y^2} \right] =
\frac{1}{2} \, \left[ \arctan(y) + \frac{y}{1 + y^2} \right] + c
\]
E da cui possiamo trovare la soluzione anche a questo integrale:
\begin{align*}
\int \frac{\diff x}{(x^2 - x + 1)^2} &=
\substitute{\int \frac{\diff y}{{\left(y^2 + \frac{3}{4}\right)}^2}}{y = x - \frac{1}{2}} = \int \frac{\diff y}{y^2 + \frac{3}{4}} -
\int \frac{y^2 \dx[y]}{{\left(y^2 + \frac{3}{4}\right)}^2} = \\
&= \frac{\sqrt{3}}{3} \, \arctan \left[ \frac{2 \, \sqrt{3}}{3} \, \left( x - \frac{1}{2} \right) \right] +
\frac{1}{2} \, \frac{x - \frac{1}{2}}{x^2 - x + 1} + c
\end{align*}
E poi, agli altri...

\section{Complessi}

\begin{defn}[Complessi coniugati]
Due numeri si dicono complessi coniugati se sono nella forma:
\[
\alpha + i \, \beta \qquad \alpha - i \, \beta
\]
con $\alpha, \beta \in \reals$.
\end{defn}

Data una qualsiasi equazione di secondo grado:
\[
a \, x^2 + b \, x + c = 0
\]
le sue soluzioni in generale sono:
\[
x_{1,2} = \frac{-b \pm \sqrt{b^2 - 4 a c}}{2 a}
\]
Se $b^2 < 4 ac$, allora le soluzioni sono:
\[
x_{1,2} = \frac{-b \pm i \, \sqrt{4 a c - b^2}}{2 a}
\]







































