\section{17 ottobre 2014}

\begin{enumerate}
    \item (2.27) Un'urna contiene 3 palline rosse e 7 nere. Due giocatori, A e B, estraggono uno alla volta una pallina dall'urna fino a che non viene estratta la prima pallina rossa. Si determini la probabilit\`a che A estragga la pallina rossa. (A estrae la prima pallina, B la seconda, A la terza e cos\`i via. Le estrazioni vengono fatte senza reinserimento.)
    \item (2.28) Un'urna contiene 5 palline rosse, 6 blu e 8 verdi. Se estraiamo un blocco di 3 palline, qual \`e la probabilit\`a che le palline abbiano
    \begin{enumerate}
        \item il medesimo colore;
        \item tre colori differenti?
    \end{enumerate}
    Si ripeta l'esercizio sotto l'ipotesi che estraiamo le tre palline una alla volta e che dopo ogni estrazione, segnamo che colore \`e uscito e reinseriamo la pallina nell'urna prima del'estrazione successiva. Questo procedimento \`e noto come campionamento con reinserimento.
    \item (2.31) Una squadra di basket \`e formata da 3 giocatori (una guardia, un'ala e un centro).
    \begin{enumerate}
        \item Se scegliamo un giocatore a caso da tre quadre formate nel modo precedente, qual \`e la probabilit\`a che venga selezioanta una nuova squadra con guardia, ala e centro?
        \item Qual \`e la probabilit\`a che i 3 giocatori selezionati giochino tutti e tre nel medesimo ruolo?
    \end{enumerate}
    \item (2.33) Una foresta contiene 20 alci, dei quali 5 vengono catturati, marchiati e quindi liberati. Dopo un certo tempo 4 delle 20 alci vengono catturate. Qual \`e la probabilit\`a che 2 di esse siano marchiate? Che ipotesi state facendo?
    \item (2.35) A un congresso partecipano 30 fisici e 24 matematici. Tre dei 54 partecipanti al congresso vengono scelti a caso per comporre un gruppo di lavoro. Qual \`e la probabilit\`a che almeno un matematico ne faccia parte?
    \item (2.37) Una professoressa assegna agli studenti 10 problemi, informandoli che l'esame finale consister\`a in 5 di questi scelti a caso. Se uno studente \`e riuscito a risolverne 7, qual \`e la probabilit\`a che risponda esattamente a
    \begin{enumerate}
        \item 5 dei problemi dell'esame finale;
        \item almeno 4 dei problemi dell'esame finale?
    \end{enumerate}
    \item (2.39) In una citt\`a ci sono 5 alberghi. Se un giorno 3 persone scelgono a caso un albergo dove pernottare, qual \`e la probabilit\`a che ognuno scenda in un albergo diverso? Che ipotesi state facendo?
\end{enumerate}