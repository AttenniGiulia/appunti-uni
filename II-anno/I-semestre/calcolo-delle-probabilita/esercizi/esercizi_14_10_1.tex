\section{1 ottobre 2014}

\begin{enumerate}
    \item (1.5) Anni fa il prefisso telefonico in Canada e negli Stati Uniti consisteva in una sequenza di 3 cifre: la prima era un intero compreso tra 2 e 9, la seconda era 0 o 1, la terza era un intero tra 1 e 9. 
    \begin{enumerate}
        \item Quanti prefissi erano possibili?
        \item Quanti di essi cominciavano con il 4?
    \end{enumerate}
    \item (1.6) In quanti modi si possono distribuire 4 libri diversi a 7 bambini?
    \item (1.7) Nella prima fila di un'aula devono sedersi 6 studenti (3 ragazzi e 3 ragazze).
    \begin{enumerate}
        \item In quanti modi possono sedersi gli studenti?
        \item In quanti modi possono sedersi gli studenti se sia i maschi che le femmine devono sedere vicini fra loro?
        \item In quanti modi possono sedersi gli studenti se solo i maschi devono stare vicini?
        \item In quanti modi possono sedersi gli studenti se due studenti dello stesso sesso non devono stare vicini?
    \end{enumerate}
    \item (1.11) In quanti modi si possono sistemare in uno scaffale 3 romanzi, 2 testi di matematica e 1 di chimica se:
    \begin{enumerate}
        \item i libri si possono sistemare in qualunque modo?
        \item i testi di matematica vanno messi vicini fra loro e i romanzi vanno messi vicini fra loro?
        \item i romanzi vanno messi vicini fra loro?
    \end{enumerate}
\end{enumerate}