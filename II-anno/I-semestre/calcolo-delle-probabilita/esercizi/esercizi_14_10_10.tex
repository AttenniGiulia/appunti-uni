\section{10 ottobre 2014}

\begin{enumerate}
    \item Estraggo 3 carte da un mazzo di 40 (4 semi, 10 valori). Lo spazio campionario ha cardinalit\`a $\binom{40}{3}$ (tutti i sottoinsiemi di tre elementi di un insieme di 40 elementi), ossia 9880. Per simmetria gli esiti sono equiprobabili.
    \begin{enumerate}
        \item Calcolare la probabilit\`a di estrarre tutti assi. L'evento ha cardinalit\`a $\binom{4}{3} = 4 \Rightarrow $ la probabilit\`a dell'evento $P(E) = \frac{|E|}{|S|} \simeq 0{,}0004$.
        \item Calcolare la probabilit\`a che le tre carte abbiano lo stesso valore. Posso scegliere la prima carta in 40 modi, la seconda comunque ho scelto la prima in 3 modi (dovendo avere lo stesso valore), la terza in 2 modi. I possibili ordinamenti di questi tre elementi sono $3! = 6$. Ho quindi $\frac{240}{6} = 40$ possibili insiemi di tre carte uguali. La probabilit\`a dell'evento \`e $\frac{40}{9880} \simeq 0{,}004$.
        \item Calcolare la probabilit\`a che siano di tre valori differenti. Posso scegliere la prima carta in 40 modi. Per la seconda posso scegliere fra 9 valori con 4 semi, per la terza posso scegliere fra 8 valori con 4 semi. I possibili ordinamenti di questi tre elementi sono $3! = 6$. Ho quindi $\frac{40 \times 36 \times 32}{6} = 7680$ possibili insiemi di tre carte uguali. La probabilit\`a dell'evento \`e $\frac{7680}{9880} \simeq 0{,}777$.
    \end{enumerate}
    \item Lancio 2 dadi.
    \begin{enumerate}
        \item Calcolare la probabilit\`a che il risultato del primo lancio sia maggiore del secondo. La cardinalit\`a dell'evento \`e 15. La probabilit\`a \`e $\frac{15}{36} \simeq 0{,}41$.
    \end{enumerate}
\end{enumerate}