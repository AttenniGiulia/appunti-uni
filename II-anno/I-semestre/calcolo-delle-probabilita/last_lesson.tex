\documentclass[11pt,a4paper,twoside]{report}
% draft mode prevents link generation

\usepackage[italian]{babel}
% per date e ToC in italiano
\usepackage{hyperref}
\usepackage{nameref}
\usepackage[table]{xcolor}
\usepackage{amsmath}        % matematica
\usepackage{amsthm}         % teoremi
\usepackage{amssymb}        % simboli
\usepackage{amsfonts}       % font matematicosi
\usepackage{mathrsfs}
\usepackage{mathtools}
\usepackage{thmbox}
\usepackage{centernot}      % per semplificare
\usepackage{pgfplots}       % per i grafici
\pgfplotsset{compat=1.5}    % consigliata compatibilita'
                            % con la versione 1.5

\usepackage{fullpage}       % troppo margine normalmente
\usepackage{parskip}        % preferisco spazio fra i paragrafi
                            % all'indentazione sulla prima riga
\usepackage{fancyhdr}
\usepackage[makeroom]{cancel}

% algorithms
\usepackage{algorithm}
\usepackage{algpseudocode}

\floatname{algorithm}{Algoritmo}
\renewcommand{\algorithmicrequire}{\textbf{Input:}}
\renewcommand{\algorithmicensure}{\textbf{Output:}}

\theoremstyle{plain}
\newtheorem{axiom}{Assioma}
\newtheorem{theorem}{Teorema}[section]
\newtheorem{lem}[theorem]{Lemma}
\newtheorem{prop}[theorem]{Proposizione}
\newtheorem{cor}[theorem]{Corollario}

\theoremstyle{definition}
\newtheorem{defn}{Definizione}[section]
\newtheorem{esercizio}{Esercizio}

\theoremstyle{remark}
\newtheorem{remark}{Remark}
\newtheorem{oss}{Osservazione}
\newtheorem{caso}{Caso}
\newtheorem{fact}{Fatto}
\newtheorem{claim}{Claim}

\newenvironment{exmp}[1][]
{ \par\medskip\textbf{Esempio\ #1\unskip\,:}\\*[1pt]\rule{\textwidth}{0.4pt}\\*[1pt] }
{ \\*[1pt]\rule{\textwidth}{0.4pt}\medskip }

\newenvironment{smallpmatrix}
{\left( \begin{smallmatrix}}
{\end{smallmatrix} \right)}

\newenvironment{completepmatrix}[1]
{\left(\begin{array}{*{#1}{c}|c}}
{\end{array}\right)}

\pagestyle{fancyplain}
\fancyhead{} % clear all header fields
\fancyfoot{} % clear all footer fields
\fancyfoot[C]{\today}
\fancyfoot[LE,RO]{\thepage}
\renewcommand{\headrulewidth}{0pt}
\renewcommand{\footrulewidth}{0pt}

\usetikzlibrary{shapes,arrows,calc,fit,backgrounds,positioning,automata}

\newcommand{\parity}[1]{\mathcal{Par}\left( { #1 } \right)}
% Languages & Automata
\newcommand{\langs}[1]{\mathcal{L}\left( {#1} \right)}
\newcommand{\lang}[1]{L\left( {#1} \right)}
\newcommand{\langor}{\, | \,}
\newcommand{\kleenestar}{\star}
\newcommand{\emptystring}{\varepsilon}

\newcommand{\bigo}[1]{O\left( {#1} \right)}
\newcommand{\probp}{\code{P}}
\newcommand{\probnp}{\code{NP}}
\newcommand{\probnpc}{\code{NP-C}}
\newcommand{\probexp}{\code{EXP}}
\newcommand{\covers}{<\!\cdot}      % simbolo di 'copre'
\newcommand{\scalar}{\centerdot}
\newcommand{\compl}{\mathsf{c}}     % C complementare
\newcommand{\dotcup}{\mathaccent\cdot\cup}  % unione disgiunta
\newcommand{\lateralsx}[1]{{}_{#1}\!\sim}   % equivalenza sinistra
\newcommand{\lateraldx}[1]{\sim_{#1}}       % equivalenza destra
\newcommand{\mcm}{\operatorname{mcm}}       % mcm
\newcommand{\mcd}{\operatorname{MCD}}       % MCD
\newcommand{\pow}[1]{<\!{#1}\!>}            % gruppo delle potenze (<a>)
\newcommand{\abs}[1]{\left\lvert{#1}\right\rvert}   % valore assoluto
\newcommand{\intsup}[1]{\left\lceil{#1}\right\rceil}   % intero superiore
\newcommand{\intinf}[1]{\left\lfloor{#1}\right\rfloor}   % intero inferiore
\newcommand{\norm}[1]{\left\lVert{#1}\right\rVert}  % norma di un vettore
\newcommand{\divides}{\bigm|}       % simbolo di 'divide'
\newcommand{\definition}{\overset{\underset{\mathrm{def}}{}}{=}}    % uguale con 'definizione' sopra
\newcommand{\supop}{\vee}           % simbolo dell'operatore sup
\newcommand{\infop}{\wedge}         % simbolo dell'operatore inf
\newcommand{\subgroupset}{\mathscr{S}}      % insieme dei sottogruppi (S corsiva)
\newcommand{\solutions}{\mathscr{S}}
\newcommand{\parts}[1]{\mathbb{P} \left( {#1} \right)}             % insieme delle parti
\newcommand{\naturals}{\mathbb{N}}          % insieme dei naturali
\newcommand{\integers}{\mathbb{Z}}          % insieme degli interi
\newcommand{\rationals}{\mathbb{Q}}         % insieme dei razionali
\newcommand{\reals}{\mathbb{R}}             % insieme dei reali
\newcommand{\complexes}{\mathbb{C}}         % insieme dei complessi
\newcommand{\field}{\mathbb{K}}             % simbolo di campo generico
\newcommand{\subfield}{\mathbb{F}}          % simbolo di sottocampo generico
\newcommand{\matrices}{\mathfrak{M}}        % insieme delle matrici
\newcommand{\nullelement}{\underline{0}}    % elemento neutro
\newcommand{\var}{\operatorname{Var}}       % varianza 
\newcommand{\seq}[3]{{#1}_{#2},\ldots,\,{#1}_{#3}}
\newcommand{\prob}[2][P]{#1 \left( {#2} \right)}
\newcommand{\probcond}[3][P]{#1 \left( #2 | #3 \right)}
\newcommand{\expect}[1]{E \left( #1 \right)}
\newcommand{\mass}[2][X]{p_{#1} \left( #2 \right)}
\newcommand{\distr}[2][X]{F_{#1} \left( #2 \right)}
\newcommand{\image}[1]{Im_{#1}}
\newcommand{\cov}{\operatorname{Cov}}
\newcommand{\detname}{\operatorname{det}}
\newcommand{\sgn}{\operatorname{sgn}}
\renewcommand{\det}[1]{\detname \left( {#1} \right)}
\newcommand{\algcompl}[2][A]{{\mathcal{#1}}_{#2}}
\newcommand{\agg}[1]{\operatorname{Agg} \left( {#1} \right)}
\newcommand{\id}{id}
\newcommand{\sfrac}[2]{{#1}/{#2}}
\newcommand{\requiv}[1][]{\ \varepsilon_{#1} \ }
\newcommand{\abslog}[1]{\log \left( \abs{#1} \right)}
\newcommand{\substitute}[2]{\left. {#1} \; \right|_{#2}}
\newcommand{\increment}[1]{{\left. {#1} \; \right|}}
\newcommand{\adj}{\sim}
\newcommand{\code}[1]{\texttt{#1}}
\newcommand{\isa}{\code{is-a}}
% intervalli aperti e chiusi
\newcommand{\ooint}[2]{\left( {#1}, {#2} \right)}
\newcommand{\coint}[2]{\left[ {#1}, {#2} \right)}
\newcommand{\ocint}[2]{\left( {#1}, {#2} \right]}
\newcommand{\ccint}[2]{\left[ {#1}, {#2} \right]}
\newcommand{\stereotipe}[1]{\ll \code{#1} \gg}

\renewcommand{\labelitemi}{$-$}
\let\oldforall\forall
\renewcommand{\forall}{\ \oldforall \ }
\let\oldexists\exists
\renewcommand{\exists}{\ \oldexists \ }
\renewcommand{\implies}{\Rightarrow}
\renewcommand{\iff}{\Leftrightarrow}

\makeatletter
\let\orgdescriptionlabel\descriptionlabel\renewcommand*{\descriptionlabel}[1]{%
\let\orglabel\label
\let\label\@gobble
\phantomsection
\edef\@currentlabel{#1}%
%\edef\@currentlabelname{#1}%
\let\label\orglabel
\orgdescriptionlabel{#1}%
}
\makeatother

\makeatletter
\providecommand*{\diff}%
    {\@ifnextchar^{\DIfF}{\DIfF^{}}}
\def\DIfF^#1{%
    \mathop{\mathrm{\mathstrut d}}%
        \nolimits^{#1}\gobblespace}
\def\gobblespace{%
        \futurelet\diffarg\opspace}
\def\opspace{%
    \let\DiffSpace\!%
    \ifx\diffarg(%
        \let\DiffSpace\relax
    \else
        \ifx\diffarg[%
            \let\DiffSpace\relax
        \else
            \ifx\diffarg\{%
                \let\DiffSpace\relax
            \fi\fi\fi\DiffSpace}

\providecommand*{\deriv}[1][]{%
    \frac{\diff}{\diff #1}}
%\providecommand*{\deriv}[2][]{%
%    \frac{\diff^{#1}}{\diff #2^{#1}}}
\providecommand*{\fderiv}[3][]{%
    \frac{\diff^{#1}#2}{\diff #3^{#1}}}
\providecommand*{\pderiv}[3][]{%
    \frac{\partial^{#1}#2}%
        {\partial #3^{#1}}}

\newcommand{\dx}[1][x]{\, \diff {#1}}

\author{Michele Laurenti \\ \href{mailto:asmeikal@me.com}{asmeikal@me.com}}
\date{\today}


\begin{document}

\section{Disuguaglianza di Markov}

\begin{prop}[Disuguaglianza di Markov]
Sia $X$ una variabile aleatoria tale per cui $X \ge 0$, allora $\forall a > 0$ vale che:
\[
\prob{X \ge a} \le \frac{\expect{X}}{a}
\]
\end{prop}
\begin{proof}
Sia $Y$ una variabile aleatoria definita come:
\[
Y = 
\begin{cases}
1 \text{ se } X \ge a \\
0 \text{ altrimenti}
\end{cases}
\]
Che \`e anche:
\[
Y(s) = 
\begin{cases}
1 \text{ se } X(s) \ge a \\
0 \text{ altrimenti}
\end{cases}
\]
\`E la funzione caratteristica dell'evento $X \ge a$. Si osservi che $Y \le \frac{X}{a}$. Infatti, se $X(s) \ge a$, allora $Y(s) = 1$, mentre $\frac{X(s)}{a} \ge \frac{a}{a} = 1$. Quindi \`e verificato che $Y(s) \le \frac{X(s)}{a}$.

Se invece $X(s) < a$, allora $Y(s) = 0$. Per ipotesi $X(s) \ge 0$ e $a > 0$, quindi $\frac{X(s)}{a} \ge 0 = Y(s)$.

Concludiamo che \`e vero che $Y \le \frac{X}{a}$. Per il lemma \ref{monotonia_valore_atteso}, $\expect{Y} \le \expect{\frac{X}{a}}$.
\begin{gather*}
\expect{Y} = \prob{X \ge a} \\
\expect{\frac{X}{a}} = \frac{\expect{X}}{a}
\end{gather*}
Quindi viene che:
\[
\prob{X \ge a} \le \frac{\expect{X}}{a}
\]
\end{proof}

\begin{exmp}
Sia $X$ una variabile aleatoria $\operatorname{Poisson} (100)$. Qual \`e la probabilit\`a che $X \ge 1000$? $X$ \`e positiva, a \`e positivo, quindi:
\[
\prob{X \ge 1000} \le \frac{100}{1000} = \frac{1}{10}
\]
\end{exmp}

\begin{lem}[Monotonia del valore atteso]\label{monotonia_valore_atteso}
Siano $X, Y : S \to \reals$ con $X \le Y$. $X \le Y$ significa che $X(s) \le Y(s) \forall s \in S$. Allora $\expect{X} \le \expect{Y}$.
\end{lem}
\begin{proof}
Sia $Z = Y - X$, abbiamo che $Z(s) = Y(s) - X(s) \ge 0$ per le ipotesi. Quindi $Z : S \to \reals$ ed \`e non negativa, ossia i suoi possibili valori $\{ z_i \}_{i}$ sono tutti maggiori di o uguali a zero.
\begin{align*}
\expect{Z} &= \expect{Y - X} = \expect{Y} - \expect{X} = \tag{per linearit\`a} \\
= \sum_{z_i} \underbrace{z_i}_{\ge 0} \cdot \underbrace{\prob{Z = z_i}}_{\ge 0} \ge 0
\end{align*}
Finito, avendo dimostrato che $\expect{Y} - \expect{X} \ge 0$.
\end{proof}

Isoliamo un passaggio della dimostrazione precedente: la disuguaglianza di Chebyshev. Sia $X$ una variabile aleatoria di media $\mu$ e di varianza $\sigma^2$. Allora $\forall s > 0$ vale che:
\[
\prob{\abs{X - \mu} > s} \le \frac{\sigma^2}{s^2}
\]
\begin{proof}
\[
\abs{X - \mu} \ge s \iff (X - \mu)^2 \ge s^2
\]
Applichiamo Markov considerando $(X - \mu)^2$ come variabile aleatoria non negativa (\`e un quadrato) e $s^2$ come il valore $a$.
\[
\prob{(X - \mu)^2 \ge s^2} \le \frac{\expect{(X - \mu)^2}}{s^2} = \frac{\var (X)}{s^2} = \frac{\sigma^2}{s^2}
\]
\end{proof}

Quando scrivemmo questo:
\[
\prob{\abs{\frac{\sum_{i = 1}^{n} X_i}{n} - \mu} \ge \varepsilon}
\]
lo stimammo come minore di:
\[
\frac{1}{\varepsilon^2} \var \left( \frac{\sum_{i = 1}^{n} X_i}{n} \right)
\]
proprio grazie alla disuguaglianza di Chebyshev.

\section{Variabili aleatorie continue}

\begin{defn}[Variabile aleatoria continua]
Una variabile aleatoria $X : S \to \reals$ si dice \emph{continua} se esiste una funzione non negativa $f : \reals \to \left[ 0, + \infty \right)$ definita su tutto $\reals$, tale che:
\[
\forall B \subseteq \reals \prob{X \in B} = \int_{B} f(u) \, du
\]
Ossia la probabilit\`a che $X$ sia in $B$ \`e uguale all'integrale sull'intervallo $B$. $B$ deve essere misurabile, e tipicamente \`e un intervallo o una semiretta. $f$ \`e chiamata \emph{funzione di densit\`a}.
\end{defn}

Se una variabile aleatoria \`e continua, lo spazio campionario non \`e numerabile.

La funzione di densit\`a sta alla variabile aleatoria continua come la densit\`a discreta sta alla variabile aleatoria discreta.

Se $X$ \`e una variabile aleatoria continua, il suo valore atteso \`e:
\[
\expect{X} = \int x \cdot f(x) \, dx
\]
Il parallelismo arriva fino a un certo punto. Se \`e vero che $\mass{u} = \prob{X = u}$, non \`e invece assolutamente vero che $f(u) = \prob{X = u}$. Infatti, se $X$ \`e una variabile aleatoria continua, allora:
\[
\prob{X = u} = \prob{X \in \{u\}} = \int_{\{u\}} f(z) \, dz
\]
L'integrale su un intervallo degenere (come $[u,u]$) vale 0. Le variabili aleatorie continue sono tali che $\prob{X = u} = 0 \forall u \in \reals$. I possibili valori di una variabile aleatoria continua non sono un insieme numerabile.

\begin{oss}
\[
\int_{-\infty}^{+\infty} f(z) \, dz = 1
\]
\end{oss}

\begin{proof}
\[
1 = \prob{X \in \reals}
\]
L'evento certo \`e che $X$ assuma valore nei numeri reali.
\[
1 = \int_{\reals} f(z) \, dz = \int_{-\infty}^{+\infty} f(z) \, dz
\]
\end{proof}

\subsection{Variabile aleatoria uniformemente distribuita}

\begin{defn}[Variabile aleatoria (continua) uniformemente distribuita]
Dati $a < b$ reali, la variabile aleatoria $X$ uniforme su $[a,b]$ (o uniformemente distribuita su $[a,b]$) \`e la variabile aleatoria $X$ continua con funzione di densit\`a cos\`i definita:
\[
f(u) =
\begin{cases}
\frac{1}{b - a} \text{ se } a \le u \le b \\
0 \text{ altrimenti}
\end{cases}
\]
\end{defn}

\begin{figure}[ht]
\centering
\begin{tikzpicture}
    \coordinate (Origin)   at (0,0);
    \coordinate (XAxisMin) at (-3,0);
    \coordinate (XAxisMax) at (5,0);
    \coordinate (YAxisMin) at (0,-1);
    \coordinate (YAxisMax) at (0,2);
    \draw [thin, gray,-latex] (XAxisMin) -- (XAxisMax);% Draw x axis
    \draw [thin, gray,-latex] (YAxisMin) -- (YAxisMax);% Draw y axis
    \draw (-1,0) node [below] {$a$};
    \draw (2,0) node [below] {$b$};
    \draw [ultra thick, red] (XAxisMin) -- (-1, 0);
    \draw [ultra thick, red] (-1, 1) -- (2, 1);
    \draw [ultra thick, red] (2, 0) -- (XAxisMax);
\end{tikzpicture}
\caption{\label{fig:distribuzione}Esempio di variabile aleatoria uniformemente distribuita su $[a,b]$}
\end{figure}
\[
\prob{X < a \lor X > b} =
\int_{(-\infty,a) \cup (b,+\infty)} f(z) \, dz = 0
\]
E inoltre:
\[
\prob{X \in [a,b]} =
\int_{[a,b]} f(z) \, dz = \int_{a}^{b} \frac{1}{b - a} \, dz = 
\frac{b - a}{b - a} = 1
\]
Nel caso particolare in cui $a = 0$ e $b = 1$, qual \`e la probabiit\`a che $X$ sia in $[0, \sfrac{1}{2}]$? Proprio $\sfrac{1}{2}$. La probabilit\`a che la variabile aleatoria assuma valori in un intervallo \`e la lunghezza dell'intervallo. La funzione di probabilit\`a \`e omogenea.

\subsection{Variabile aleatoria normale (o gaussiana)}

\begin{defn}[Variabile aleatoria gaussiana]
Una variabile aleatoria $X$ si dice gaussiana di media $\mu \in \reals$ e varianza $\sigma^2$ se $X$ \`e continua con:
\[
f(z) = \frac{1}{\sqrt{2 \pi \sigma^2}} \cdot e^{-\frac{(z - mu)^2}{2 \sigma^2}}
\]
\end{defn}

La funzione di densit\`a \`e una funzione a campana, sempre positiva, centrata rispetto a $\mu$, e fa cos\`i:

\pgfmathdeclarefunction{gauss}{2}{%
  \pgfmathparse{1/(#2*sqrt(2*pi))*exp(-((x-#1)^2)/(2*#2^2))}%
}

\begin{figure}[ht]
\centering
\begin{tikzpicture}
\begin{axis}[every axis plot post/.append style={
  mark=none,domain=-2:3,samples=50,smooth}, % All plots: from -2:2, 50 samples, smooth, no marks
  axis x line*=bottom, % no box around the plot, only x and y axis
  axis y line*=left, % the * suppresses the arrow tips
  enlargelimits=upper] % extend the axes a bit to the right and top
  \addplot {gauss(0,0.5)};
\end{axis}
\end{tikzpicture}
\caption{Esempio di variabile aleatoria normale}
\end{figure}

Quando $\mu = 0$ e $\sigma^2 = 1$, la variabile aleatoria \`e detta \emph{normale standard}. La sua funzione di densit\`a \`e:
\[
\frac{1}{\sqrt{2 \pi}} \cdot e^{-\frac{x^2}{2}}
\]
\begin{theorem}[Teorema del limite centrale]
Sia $X_1, X_2, X_3, \ldots$ una successione di variabili aleatorie i.i.d. Sia $\mu = \expect{X_i}$ il loro valore atteso comune, e sia la loro varianza $\var(X_i) = \sigma^2$. $S_n$ \`e la somma delle prime $n$ variabili aleatorie.
\[
S_n = X_1 + \ldots + X_n
\]
Per la legge debole dei grandi numeri:
\[
\lim_{n \to \infty} \prob{ \abs{ \frac{S_n}{n} - \mu} > \varepsilon} = 0 \forall \varepsilon > 0
\]
Si pu\`o anche scrivere come:
\[
\prob{ \abs{\frac{S_n - n \cdot \mu}{n}} > \varepsilon} \to 0
\]
\[
n \cdot \mu = \expect{S_n}
\]
Se divido per qualcosa di pi\`u ``debole'' di $n$, ossia di ordine minore, quando $n$ tende all'infinito, cosa succede? Ho un limite diverso? Se si divide per $\sqrt{n}$ succede questo, $\forall a < b$:
\[
\lim_{n \to \infty} 
\prob{a \le \frac{S_n - n \cdot \mu}{\sigma \sqrt{n}} \le b} = 
\frac{1}{\sqrt{2 \pi}} \int_{a}^{b} e^{-\frac{z^2}{2}} \, dz = 
\prob{a \le Z \le b}
\]
Con $Z$ una variabile aleatoria gaussiana standard.
\end{theorem}

Abbiamo visto che $S_n = X_1 + \ldots + X_n$, e la sua media quindi \`e $\expect{S_n} = \expect{X_1} + \ldots + \expect{X_n} = n \cdot \mu$. La sua varianza, essendo tutte variabili i.i.d., \`e:
\[
\var{S_n} = \var{X_1} + \ldots + \var{X_n} = \sigma^2 n
\]
Quindi $\sigma \sqrt{n}$ \`e la sua deviazione standard.

Quindi:
\[
\frac{S_n - n \cdot \mu}{\sigma \sqrt{n}}
\]
Questo mostro ha media 0 e varianza 1. Si pu\`o controllare. La sottrazione e la divisione \`e servita per \emph{centrare} questo mostro, e confrontarlo con la gaussiana.

\begin{exmp}
Lancio una moneta 10000 volte. Testa esce 5800 volte. La moneta \`e truccata? Consideriamo le variabili $X_i$:
\[
X_i =
\begin{cases}
1 \text{ se al lancio $i$ esce Testa} \\
0 \text{ altrimenti}
\end{cases}
\]
$S_n = X_1 + \ldots + X_n$ \`e il numero di teste nei primi $n$ lanci. Immaginiamo la moneta sia onesta, quindi $\mu = \frac{1}{2}$, e $\sigma^2 = p \cdot (1 - p) = \frac{1}{4}$.

Per il teorema del limite centrale:
\[
\frac{S_n - n \cdot \mu}{\sigma \sqrt{n}} \approx Z
\]
Con $Z$ \`e una gaussiana standard, e per $n$ grande approssima il mostro a sinistra. $n = 10000$ lo consideriamo grande a sufficienza.
\[
\frac{S_{10000} - 5000}{50}
\]
Quindi il numero di teste meno 5000 diviso 50 \`e ben approssimato da una gaussiana standard. Qual \`e la probabilit\`a che questo valore sia maggiore di $a$?
\[
\prob{ \frac{S_{10000} - 5000}{50} > a} \approx \prob{Z > a}
\]
In particolar modo, si \`e verificato che:
\[
\frac{S_{10000} - 5000}{50} \ge \frac{800}{50} = 16
\]
Quindi:
\[
\prob{\frac{S_{10000} - 5000}{50} \ge 16} \approx \prob{Z \ge 16}
\]
\end{exmp}

\end{document}
