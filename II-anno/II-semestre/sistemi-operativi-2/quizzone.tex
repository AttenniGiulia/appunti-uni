\documentclass[a4paper,twoside]{article}
\usepackage[italian]{babel}
\usepackage{hyperref}
\usepackage{nameref}
\usepackage{amsmath}        % matematica
\usepackage{amsthm}         % teoremi
\usepackage{amssymb}        % simboli
\usepackage{amsfonts}       % font matematicosi
\usepackage{mathrsfs}
\usepackage{mathtools}
\usepackage{listings}
\usepackage{fullpage}
\usepackage{parskip}

\newcommand{\code}[1]{\texttt{#1}}

\begin{document}

\begin{enumerate}
    \item Il comando \code{size /bin/ls} ha il seguente effetto:

    Stampa la dimensione dei segmenti text, data e bss del binario \code{/bin/ls}
    \item Nel sistema operativo UNIX, l'istruzione \code{echo prova > pippo} ha la seguente semantica: 

    Scrive nel file \code{pippo} la stringa prova. Il precedente contenuto di \code{pippo} \`e sovrascritto. 
    \item L'effetto dell'istruzione \code{man cmd} \`e:

    Stampare su stdout una descrizione del comando \code{cmd}.
    \item Si consideri il seguente script:
    \begin{lstlisting}[language=sh]
    verb=sing
    echo I like ${verb}
    \end{lstlisting}
    Una possibile esecuzione dello script di cui sopra \`e: 

    \code{I like sing} 
    \item Nel sistema operativo UNIX, il comando crontab serve per:

    Schedulare l'esecuzione di comandi che vengono eseguiti periodicamente

    \item L'effetto dell'istruzione 
    \begin{lstlisting}[language=sh]
    eval 'echo pwd'
    \end{lstlisting}
    nella shell di login \`e: 

    Stampare \code{pwd} su stdout. 
    \item Nel sistema operativo UNIX, il comando 
    \begin{lstlisting}[language=sh,literate={-}{{-}}1]
    find . -name “*.exe” -ls -exec rm {} \;
    \end{lstlisting}
    ha la seguente semantica:

    Cancella tutti i file che hanno estensione \code{.exe} dalla directory corrente e sue sottodirectory
    \item Nel sistema operativo UNIX, l'istruzione \code{diff pippo pluto} ha la seguente semantica: 

    Stampa su stdout le differenze tra i files \code{pippo} e \code{pluto}.
    \item L'effetto dell'istruzione \code{cat pippo} \`e: 

    Stampare il contenuto del file \code{pippo} su stdout
    \item Nel sistema operativo UNIX, l'istruzione \code{sort pippo} ha la seguente semantica: 

    Ordina le righe del file \code{pippo} in ordine lessicografico crescente e stampa su stdout il risultato. 
    \item Nel sistema operativo UNIX, l'istruzione:
    \begin{lstlisting}[language=sh]
    gawk '($1 == "Stampa") {print $0}' pippo 
    \end{lstlisting}
    ha la seguente semantica: 

    Stampa su stdout tutte le righe del file \code{pippo} che contengono la parola \code{Stampa} come prima parola
    \item Nel sistema operativo UNIX, l'istruzione \code{uniq pippo} ha la seguente semantica: 

    Stampa su stdout il file \code{pippo} collassando tutte le righe adiacenti ripetute in un unica riga. 
    \item Un file detto Makefile rappresenta:

    Rappresenta un file di configurazione per l'utility \code{make}
    \item Nel sistema operativo UNIX, l'istruzione \code{rm -fr pippo} ha la seguente semantica: 

    Rimuove ricorsivamente il contenuto della la directory \code{pippo} e la directory \code{pippo} stessa. 
    \item L'effetto dell'istruzione \code{pwd} \`e: 

    Stampare la directory corrente. 

    \item Si considerino i seguenti file attributes nel tipico formato UNIX:
    \begin{lstlisting}[language=sh,literate={-}{{-}}1]
    -rw-r--r-- 1 enrico tronci 278 Jul 5 13:12 pippo.tex
    \end{lstlisting}
    Lo user del file \`e: 

    \code{enrico}
    \item Nel sistema operativo UNIX, l'istruzione per spostarsi dalla directory corrente alla directory TARGET ́\`e:

    \code{cd TARGET}
    \item Nel sistema operativo UNIX, l'istruzione per stampare su stdout i files di una directory \`e: 

    \code{ls}
    \item Si considerino i seguenti file attributes nel tipico formato UNIX: 
    \begin{lstlisting}[language=sh,literate={-}{{-}}1]
    -rw-r--r-- 1 enrico tronci 278 Jul 5 13:12 pippo.tex
    \end{lstlisting}
    Lo user del file ha i seguenti permessi:

    Lettura e scrittura, ma non esecuzione. 
    \item Nel sistema operativo UNIX, l'istruzione \code{rmdir pippo} ha la seguente semantica: 

    Rimuove la directory \code{pippo} se \`e vuota, altrimenti ritorna errore. 
    \item Il file \emph{file descriptor} ritornato dall \emph{system call} \code{open()} \`e di tipo:

    \code{int}
    \item Una \emph{pipe} \`e uno pseudo file usato per: 

    Comunicazione tra processi sullo stesso computer. 
    \item Un programma setuid a root \`e:

    Un programma eseguibile di propriet\`a di root e con il setuid bit impostato. L'esecuzione di questo programma genera un processo che ha effective user id uguale a 0. 
    \item L'istruzione: \code{alarm(Numero);} ha il seguente effetto

    Crea un allarme che viene generato allo scadere di \code{Numero} secondi. Allo scadere del tempo il segnale \code{SIGALRM} viene generato.
    \item Un processo \code{P} diventa \emph{zombie} quando: 

    \code{P} termina, il \emph{parent process} di \code{P} non termina e non esegue mai una \code{wait}. 

    \item È possibile allocare memoria nello stack? se si, come?

    È possibile allocare memoria nello stack con l'istruzione \code{alloca}
    \item L'effetto dell'istruzione \code{p = malloc(N);} \`e:

    Alloca  uno spazio di memoria di \code{N} bytes. La variabile \code{p} rappresenta il puntatore che punta all'area di memoria allocata.
    \item L'istruzione \code{realloc(A,B);} ha il seguente effetto:

    Modifica l'area di memoria puntata da \code{A} dandogli come nuova size \code{B}. L'operazione potrebbe non andare a buon fine. In questo caso l'area di memoria originaria (puntata da \code{A}) non viene modificata.
    \item Nel sistema operativo UNIX, l'istruzione \code{echo prova >> pippo} ha la seguente semantica: 

    Appende al file \code{pippo} la stringa \code{prova}. Il precedente contenuto di \code{pippo} \`e conservato. 
    \item Nel sistema operativo UNIX, l'istruzione \code{cp -r pippo pluto} ha la seguente semantica: 

    Copia ricorsivamente nella directory \code{pluto} il contenuto della directory \code{pippo} 
    \item Nel sistema operativo UNIX, l'istruzione \code{kill}: 

    Pu\`o mandare un segnale ad un processo.
    \item Nel sistema operativo UNIX, l'istruzione \code{find . -name "*.tex"} ha la seguente semantica:

    Stampa su stdout, ricorsivamente a partire dalla directory corrente, tutti i files il cui nome termina con la stringa \code{.tex}.
    \item Nel sistema operativo UNIX, l'istruzione \code{chmod +x pippo} ha la seguente semantica:

    Rende il file \code{pippo} eseguibile.
    \item La semantica dell'istruzione
    \begin{lstlisting}[language=sh]
    ls -1 *.qry | wc | gawk '{print $1}'
    \end{lstlisting}
    \`e la seguente:

    Stampa su stdout il numero di files della directory corrente il cui nome termina con \code{.qry}.
    \item L'effetto dell'istruzione \code{exec echo 3} nella shell di login \`e:

    Stampare 3 su stdout e terminare la sessione.
    \item Una socket di tipo \code{AF\_INET} pu`o essere usata per la comunicazione tra processi:

    Su macchine ovunque su Internet, anche sulla stessa macchina.
    \item L'istruzione: \code{signal(SIGINT, myhandler);} ha il seguente effetto:

    Installa myhandler come handler per il segnale \code{SIGINT}.
    \item Nel sistema operativo UNIX, l'istruzione \code{egrep sck pippo} ha la seguente semantica:

    Mostra su stdout tutte le righe del file \code{pippo} che contengono il pattern \code{sck}.
    \item Nel sistema operativo UNIX, l'istruzione per spostare un file \code{pippo} dalla directory corrente alla directory \code{TARGET} \`e:

    \code{mv pippo TARGET}
    \item In un sistema Unix l'istruzione \code{bind()} ha il seguente effetto:

    Associa il socket name con il socket address.
    \item Un \emph{unnamed socket} \`e:

    Un file descriptor.
    \item Per una socket internet, tra le informazioni contenute nel socket address, c'\`e:

    L'internet address.
    \item L'istruzione \code{listen(p, k)} ha il seguente effetto:

    Definisce come \code{k} il massimo numero di connessioni per la socket \code{p}.
    \item L'istruzione:
    \begin{lstlisting}[language=C]
    p = socket(PF_LOCAL, SOCK_STREAM, DEFAULT_PROTOCOL);
    \end{lstlisting}
    Ha il seguente effetto:

    Alla variabile \code{p} viene assegnato il valore di un file descriptor (int) associato ad una \code{unnamed socket}.
    \item Si consideri il seguente script:
    \begin{lstlisting}[language=sh]
    (sleep 30; echo done 1) &
    (sleep 30; echo done 2) &
    echo done 3; wait; echo done 4
    \end{lstlisting}
    Una possibile esecuzione dello script di cui sopra \`e:

    \begin{lstlisting}[language=sh]
    done 3
    done 1
    done 2
    done 4
    \end{lstlisting}
    \item Nel sistema operativo UNIX l'istruzione 
    \begin{lstlisting}[language=sh]
    gawk '{print NF $0}' pippo
    \end{lstlisting}
    ha la seguente semantica:

    Stampa su stdout tutte le righe del file \code{pippo} anteponendo ad ogni riga il numero di campi in essa contenuta.
    \item Nel sistema operativo Unix il path name di un file viene considerato relativo alla directory di lavoro quando:

    Il path name NON inizia con \code{/}.
    \item Il comando \code{make} \`e utile per:

    È di ausilio alla compilazione di programmi.
    \item Il comando \code{nice}:

    Consente di eseguire un comando specificandone la priorit\`a
\end{enumerate}

\end{document}




