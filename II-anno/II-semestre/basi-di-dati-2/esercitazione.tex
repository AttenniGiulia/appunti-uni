\section{Esercitazione 3 marzo}

\subsection{Tornei di Go}

Cosa ci serve:
\begin{enumerate}
    \item Giocatore
    \begin{enumerate}[label*=\arabic*.]
        \item Nickname (identificativo)
        \item Nome
        \item Cognome
        \item Rank dichiarato
        \item Indirizzo
        \begin{enumerate}[label*=\arabic*.]
            \item Via
            \item Numero civico
            \item CAP
            \item Citt\`a (req. \ref{itm:req_citta})
        \end{enumerate}
    \end{enumerate}
    \item \label{itm:req_citta} Citt\`a
    \begin{enumerate}[label*=\arabic*.]
        \item Nome
        \item Provincia
        \item Stato
    \end{enumerate}
    \item \label{itm:partita} Partita
    \begin{enumerate}[label*=\arabic*.]
        \item Data
        \item Luogo (req. \ref{itm:req_citta}) - consideriamo il luogo unicamente come una citt\`a
        \item Regole usate (giapponesi o cinesi)
        \item Giocatore con pietre bianche
        \item Giocatore con pietre nere
        \item Fattore di deficit o \emph{komi} - numero reale compreso tra 0 e 10
        \item Esito della partita
        \begin{enumerate}[label*=\arabic*.]
            \item Rinuncia del bianco
            \item Rinuncia del nero
            \item Nessuno rinuncia
            \begin{enumerate}[label*=\arabic*.]
                \item Punteggio del bianco
                \item Punteggio del nero
            \end{enumerate}
        \end{enumerate}
    \end{enumerate}
    \item Torneo
    \begin{enumerate}[label*=\arabic*.]
        \item Nome
        \item Descrizione testuale
        \item Anno (intero)
        \item Insieme delle partite giocate (req \ref{itm:partita})
    \end{enumerate}
\end{enumerate}
Serve sempre capire, quando si analizzano i requisiti, qual \`e il livello di dettaglio che si intende raggiungere.

Il fatto che le partite si \emph{possa} riferire a un torneo, vuol dire che le partite possono essere giocate fuori da un torneo. Modifica la specifica. Se non ci fosse, le partite sarebbero solo all'interno di un torneo.

\begin{figure}[h]
\centerin
\begin{tikzpicture}
    \node[entity] (giocatore) {Giocatore};
    \node[attribute] (nickname) [above left of=entity] {Nickname} edge (giocatore);
    \node[attribute] (nome) [above of=entity] {Nome} edge (giocatore);
    \node[attribute] (cognome) [above right of=entity] {Cognome} edge (giocatore);
    \node[attribute] (rank) [right of=entity] {Rank} edge (giocatore);

    \node[entity] 

\end{tikzpicture}
\end{figure}

Si usa un ``fiammifero nero'' per identificativi.

\subsection{eBuy}

Analisi dei requisiti.

\begin{enumerate}
    \item \label{itm:ebuy_utente} Utente registrato
    \begin{enumerate}
        \item Nome
        \item Data di registrazione
        \item Tipo utente (privato o professionale)
        \item Dei venditori privati interessano:
        \begin{enumerate}
            \item URL del sito
        \end{enumerate}
    \end{enumerate}
    \item Post
    \begin{enumerate}
        \item Descrizione oggetto in vendita
        \item Categoria (req. \ref{itm:ebuy_categoria})
        \item Pagamenti accettati
        \item Utente venditore (req. \ref{itm:ebuy_utente})
        \item Modalit\`a di vendita
        \begin{enumerate}
            \item Asta
            \begin{enumerate}
                \item Prezzo iniziale
                \item Data e ora di scadenza
                \item Ammontare del rialzo
                \item Elenco dei \emph{bid} (req. \ref{itm:ebuy_bid})
            \end{enumerate}
            \item Compralo subito
            \begin{enumerate}
                \item Prezzo di vendita
            \end{enumerate}
        \end{enumerate}
    \end{enumerate}
    \item \label{itm:ebuy_bid} Bid
    \item \label{itm:ebuy_categoria} Categoria
\end{enumerate}

Il prezzo finale \`e determinato univocamente dal prezzo iniziale, dall'ammontare del rialzo e dall'elenco dei bid. Quindi non serve inserirlo. Non dobbiamo mettere qui le funzionalit\`a del sistema.

\section{Esercitazione 5 marzo}

\subsection{Database universit\`a}

Analizziamo i requisiti.
\begin{enumerate}
    \item Studente
    \begin{enumerate}
        \item Nome
        \item Matricola
        \item Data di nascita
        \item Luogo di nascita (req...)
        \item Facolt\`a
        \begin{enumerate}
            \item Anno di iscrizione
        \end{enumerate}
        \item Corsi superati
    \end{enumerate}
    \item Citt\`a
    \begin{enumerate}
        \item Nome
        \item Regione (req...)
    \end{enumerate}
    \item Regione
    \begin{enumerate}
        \item Nome
    \end{enumerate}
    \item Facolt\`a
    \item Professore
    \item Corso
\end{enumerate}

\begin{tikzpicture}

\node[entity] (studente) {Studente};
\node[entity] (professore) [above = of nascitaProfessore] {Professore};
\node[entity] (corso) [right = of insegna] {Corso};
\node[entity] (corso2) [below = of esame] {Corso};
\node[entity] (citta) [below right = of nascitaStudente] {Citt\`a};
\node[entity] (regione) [right = of regioneCitta] {Regione};
\node[entity] (facolta) [left = of facoltaStudente] {Facolt\`a};
\node[entity] (facolta2) [right = of facoltaCorso] {Facolt\`a}; % tratteggiare

% studente
\node[attribute] (studenteNome) [above = of studente, label=Nome] edge (studente);
\node[attribute] (studenteCognome) [above = of studente, label=Cognome] edge (studente);
\node[attribute] (studenteMatricola) [above = of studente, label=Matricola] edge (studente);
\node[attribute] (studenteData) [above = of studente, label=Data di nascita] edge (studente);

% studente
\node[attribute] (professoreNome) [above = of professore, label=Nome] edge (professore);
\node[attribute] (professoreCognome) [above = of professore, label=Cognome] edge (professore);
\node[attribute] (professoreData) [above = of professore, label=Data di nascita] edge (professore);

% citta
\node[attribute] (cittaNome) [above = of citta, label=Nome] edge (citta);

% regione
\node[attribute] (regioneNome) [above = of regione, label=Nome] edge (regione);

% facolta
\node[attribute] (facoltaNome) [above = of facolta, label=Nome] edge (facolta);
\node[attribute] (facoltaTipo) [above = of facolta, label=Tipo] edge (facolta);

% corso
\node[attribute] (corsoCodice) [above = of corso, label=Codice] edge (corso);
\node[attribute] (corsoNome) [above = of corso, label=Nome] edge (corso);
\node[attribute] (corsoOre) [above = of corso, label=Ore] edge (corso);


% relationships
\node[relationship] (regioneCitta) [right = of citta] {RegioneCitt\`a};
\node[relationship] (nascitaStudente) [below right = of studente] {NascitaStudente};
\node[relationship] (nascitaProfessore) [above = of citta] {NascitaProfessore};
\node[relationship] (facoltaStudente) [left = of studente] {Facolt\`aaStudente};
\node[relationship] (insegna) [right = of professore] {Insegna};
\node[relationship] (facoltaCorso) [below right = of corso] {InsegFacolt\`aCorso};
\node[relationship] (esame) [below = of studente] {Esame};

% facoltaStudente
\node[attribute] [label=Anno Iscrizione];

% links
\draw[link] (facolta) -- node [pos=0.1, auto] {(0,N)} (facoltaStudente);
\draw[link] (studente) -- node [pos=0.1, auto] {(1,1)} (facoltaStudente);
\draw[link] (regione) -- node [pos=0.1, auto] {(0,N)} (regioneCitta);
\draw[link] (citta) -- node [pos=0.1, auto] {(1,1)} (regioneCitta);
\draw[link] (studente) -- node [pos=0.1, auto] {(1,1)} (nascitaStudente);
\draw[link] (citta) -- node [pos=0.1, auto] {(0,N)} (nascitaStudente);
\draw[link] (professore) -- node [pos=0.1, auto] {(1,1)} (insegna);
\draw[link] (corso) -- node [pos=0.1, auto] {(0,N)} (insegna);
\draw[link] (corso) -- node [pos=0.1, auto] {(1,1)} (facoltaCorso);
\draw[link] (facolta2) -- node [pos=0.1, auto] {(0,N)} (facoltaCorso);

\end{tikzpicture}

Studente
Nome & stringa
Cognome & stringa
Matricola & stringa
Data di nascita & data

Citt\`a
Nome & stringa

Regione
Nome & stringa


\subsection{Esercitazioni universitarie}

Raffinamento dei requisiti:
\begin{enumerate}
    \item Corso
    \begin{enumerate}
        \item Nome
        \item CFU (intero $> 0$)
        \item Anno accademico (intero $> 0$)
        \item Docenti di riferimento (req \ref{itm:esercitazioni_docente})
        \item Lista delle esercitazioni (req \ref{itm:esercitazioni_esercitazioni})
    \end{enumerate}
    \item \label{itm:esercitazioni_docente} Docente
    \begin{enumerate}
        \item Nome
        \item Cognome
    \end{enumerate}
    \item \label{itm:esercitazioni_esercitazioni} Esercitazione
    \begin{enumerate}
        \item Codice (una stringa)
        \item Lista degli esercizi presentati (req \ref{itm:esercitazioni_esercizi})
        \item Lista degli esercizi risolti (req \ref{itm:esercitazioni_esercizi})
    \end{enumerate}
    \item \label{itm:esercitazioni_esercizi} Esercizio
    \begin{enumerate}
        \item Testo
        \item Soluzione
    \end{enumerate}
\end{enumerate}


\node[entity] (corso) {Corso};

\node[attribute] (corsoNome) [label=Nome] {};
\node[attribute] (corsoCFU) [label=CFU] {};
\node[attribute] (corsoAA) [label=AA] {};
\node[attribute] (corsoNome) [label=Nome] {};


Corso
Nome & stringa
CFU & intero > 0
Anno Accademico & intero > 0

\subsection{Dormo da te}

Raffinamento dei requisiti:
\begin{enumerate}
    \item Utente
    \begin{enumerate}
        \item Nome
        \item Cognome
        \item Sesso
        \item Data di nascita
        \item Citt\`a di residenza
        \item Abitazioni a disposizione (req. \ref{itm:dormodate_abitazione})
    \end{enumerate}
    \item \label{itm:dormodate_abitazione} Abitazione
    \begin{enumerate}
        \item Citt\`a
        \item Distanza dal centro
        \item Distanza stazione (metro, autobus, treno) pi\`u vicina
        \item Numero membri famiglia (adulti e bambini)
        \item Sistemazioni offerte (req. \ref{itm:dormodate_sistemazioni})
    \end{enumerate}
    \item \label{itm:dormodate_sistemazioni} Sistemazioni
    \begin{enumerate}
        \item Tipologia (letto o divano)
        \item Numero di posti
        \item Stanza (req. \ref{itm:dormodate_stanza})
    \end{enumerate}
    \item \label{itm:dormodate_stanza} Stanza
    \begin{enumerate}
        \item Abitazione in cui si trova (req. \ref{itm:dormodate_abitazione})
        \item Tipologia (privata o pubblica)
    \end{enumerate}
    \item Il sistema offre le seguenti funzionalit\`a
    \begin{enumerate}
        \item Un utente deve potersi registrare
        \item Un utente deve poter inserire i propri dati
    \end{enumerate}
\end{enumerate}

\section{Esercitazione 17 marzo}

\subsection{Banca del Tempo}

Analisi dei requisiti:
\begin{enumerate}
    \item Correntista
    \begin{enumerate}
        \item Nome
        \item Cognome
        \item Codice Fiscale
        \item Telefono
        \item Indirizzo
        \item Scambi richiesti (req \ref{itm:bdt_scambio})
        \item Scambi offerti (req \ref{itm:bdt_scambio})
        \item Abilit\`a possedute (req \ref{itm:bdt_abilita})
        \begin{enumerate}
            \item Competenza in una abilit\`a (intero da 6 a 10)
        \end{enumerate}
        \item Categoria (req \ref{itm:bdt_categoria})
        \item Monte ore mensile (intero > 0)
    \end{enumerate}
    \item \label{itm:bdt_categoria} Categoria
    \begin{enumerate}
        \item Nome
        \item Valore ora
    \end{enumerate}
    \item \label{itm:bdt_abilita} Abilit\`a
    \begin{enumerate}
        \item Nome
    \end{enumerate}
    \item \label{itm:bdt_scambio} Scambio di tempo
    \begin{enumerate}
        \item Richiedente (req \ref{itm:bdt_abilita})
        \item Offerente (req \ref{itm:bdt_abilita})
        \item Feedback del richiedente sull'offerente
        \item Data e ora di inizio
        \item Durata in ore (intero o reale? sicuramente > 0)
        \item Abilit\`a richiesta (req \ref{itm:bdt_abilita})
    \end{enumerate}
    \item Giudizio sul servizio offerto
    \item Il saldo non pu\`o mai essere negativo
    \item Il saldo iniziale \`e 10 ore
    \item Il saldo viene calcolato attraverso una funzionalit\`a
    \item Il sistema deve impedire prenotazioni se il monte ore mensile dell'offerente non \`e sufficiente
    \item Fra le categorie devono essere presenti le seguenti categorie
    \begin{enumerate}
        \item Lavoratore - Valore ora: 1.5
        \item Non lavoratore - Valore ora: 1.2
        \item Pensionato - Valore ora: 1
    \end{enumerate}
\end{enumerate}

\subsection{TuTubi}

Analisi dei requisiti.

\begin{enumerate}
    \item Utente
    \item Video
    \item Categoria
    \item Tag
\end{enumerate}



































