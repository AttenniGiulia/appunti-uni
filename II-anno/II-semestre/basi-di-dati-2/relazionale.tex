Il modello relazionale favorisce l'indipendenza dei dati dalla struttura di memorizzazione fisica. Si basa sulle relazioni matematiche, sottoinsiemi del prodotto cartesiano. Il modello relazionale \`e basato sui valori, non ci sono puntatori. Anche i riferimenti avvengono attraverso valori.

Nel modello relazionale dei dati il concetto di relazione viene esteso, associando a ciascun dominio un nome (unico nella relazione) che descrive il ruolo del dominio nella relazione. L'ordinamento degli attributi \`e irrilevante. La struttura non \`e posizionale.

R (A_1 : D_1 , \ldots , A_n : D_n)

A_i sono i nomi dei domini. Se t \in R, t[A] o t.A identifica il valore della n-upla t per il dominio di nome A. Si pu\`o estendere a t[A,B], per insiemi di attributi. Si possono anche vedere le n-uple come n-uple etichettate:

t = \langle A_1 : t[A_1] , \ldots , A_n : t[A_n] \rangle

Le relazioni sono rappresentabili come tabelle (ma non tutte le tabelle rappresentano relazioni). Perch\'e una tabella sia una relazione, i nomi delle colonne devono essere distinti, i valori delle colonne devono essere tutti dello stesso dominio, e le righe devono essere tutte diverse.

I riferimenti avvengono per valore!

Uno schema di relazione \`e la lista di nomi di attributi e domini di una singola relazione. Uno schema di basi di dati \`e un insieme di schemi di relazione. Un'istanza di relazione su uno schema di relazione \`e un insieme di n-uple sul prodotto cartesiano dei domini della relazione. Un'istanza di base di dati \`e un insieme di istanze di relazione sulle sue relazioni.

L'assenza di valori in una riga va codificata. Una tecnica rudimentale ma efficace per gestire l'assenza di informazione \`e il valore NULL. Quindi una n-upla pu\`o avere, al posto i-esimo, un valore nel dominio D_i oppure NULL. \`E un po' pericoloso, per\`o: il valore potrebbe essere inesistente, o sconosciuto, o non ancora noto (senza informazione se esiste o meno). Tutti e tre questi casi vengono trattati allo stesso modo, e ricadono nel terzo.

Ad un sistema di basi di dati sono associati dei vincoli di integrit\`a, che devono essere verificate da tutte le istanze (legali). I vincoli possono essere inter-relazionali o intra-relazionali (se coinvolgono o meno pi\`u relazioni).

I vincoli sono usati dal DBMS per scegliere la \emph{strategia} di esecuzione delle interrogazioni.

Vincoli di n-upla esprimono condizioni sui valori di una singola n-upla. Ciascuna n-upla deve soddisfare i vincoli. Vincoli su un solo attributo sono detti vincoli di dominio.

Poi ci sono i vincoli di chiave, famosi. Una super-chiave \`e un insieme K di attributi di una relazione per cui non possono esistere n-uple che coincidono su K e non fuori da K. Una chiave di relazione \`e una super-chiave minimale rispetto all'inclusione insiemistica fra attributi.

Tutte e sole le istanze legali della base di dati devono soddisfare i vincoli.

Ogni singolo valore \`e univocamente accessibile tramite il nome della relazione, il valore della chiave e il nome dell'attributo.

Il modello relazionale ha bisogno di chiavi primarie. Le chiavi primarie non possono avere valore NULL.

Vincoli di integrit\`a referenziale (o \emph{foreign key}) fra l'insieme X di attributi sulla relazione R, e la relazione S: ogni n-upla t di R deve avere valori per gli attributi X che compaiono come valori degli attributi della chiave primaria di S. In presenza di valori NULL, i vincoli di integrit\`a devono essere meno restrittivi.

In presenza di vincoli di integrit\`a referenziale, come deve comportarsi il DBMS in caso di cancellazione di una n-upla di S? Assumendo che il valore della chiave primaria di S per quell'n-upla corrisponda al valore su X di una n-upla in R.

Il DBMS deve rifiutare l'operazione, di default, perch\'e violerebbe il vincolo di integrit\`a. Per\`o possono esserci azioni compensative, ad esempio l'introduzione di valori NULL. In caso di violazioni dei vincoli di integrit\`a, si inserisce NULL come valore. Un'altra soluzione \`e l'eliminazione a cascata di tutte le n-uple violate dopo un'operazione.

\section{SQL}

SQL ha tre livelli:
\begin{enumerate}
    \item livello interno, o fisico
    \item livello logico, riguarda il livello di organizzazione logica dei dati, ed \`e proprio il modello relazionale
    \item livello esterno, come la base viene vista esternamente
\end{enumerate}

A livello logico si lavora con due costrutti: DDL (data definition language), per creare schemi di relazione, e DML (data manipulation language), per i dati.

create database nome_database [opzioni]

create schema nome_schema [opzioni]
Uno schema \`e un namespace

create table nome_tabella ( 
    nome_attributo dominio [vincoli di dominio],
    nome_attributo dominio [vincoli di dominio],
    \ldots
    nome_attributo dominio [vincoli di dominio],
    [altri vincoli intra-relazionali],
    [vincoli inter-relazionali]
)

primary key (nome_attributo)
foreign key (nome_attributo) references nome_tabella(nome_attributo)

create domain nome_dominio as tipo_base
    [valore di default]
    [vincolo]

insert into tabella(attributo1, \ldots, attributoN)
values (valore1, \ldots, valoreN)

delete form tabella
where condizione

select  tabella.attributo1,     //target list
        tabella.attributo2,
        \ldots,
        tabella.attributoN
from tabella                    // clausola from
where condizione                // clausola where

select Officina.indirizzo
from Officina
where Officina.nome = `fixit'

\pi_{indirizzo} ( \sigma_{nome = `fixit'} (Officina) )

Esistono operatori aggregati che semplificano il lavoro. Prendono tutte le ennuple e le aggregano in un valore solo. Un esempio di operatore aggregato \`e \code{select count(*)}. Per contare i valori distinti, si pu\`o fare count(distinct attributo). count(attributo) esclude automaticamente i valori NULL. Altri operatori aggregati sono sum(), avg(), min(), max(). In tutti questi aggregati i valori NULL sono ignorati, e il dominio degli attributi deve essere, a seconda dei casi, sommabile o ordinabile.

La target list di una select deve essere \emph{omogenea}. Un principio generale \`e: se la target-list contiene operatori aggregati, non pu\`o contenere attributi, e viceversa. Eccezione alla regola: gli attributi nella target list devono appartenere a un group-by.

Gli operatori aggregati si possono applicare a partizioni delle ennuple. I valori degli operatori aggregati sono calcolati gruppo per gruppo.

select attrA, attrB, ... aggr1(...), ... aggrN(...)
from Tabella1, ... TabellaM
where condizione
group by attrA, attrB, attrC, ... attrG
having condizione_sui_gruppi

Se voglio mettere degli attributi nella target list assieme agli operatori aggregati, gli attributi \emph{devono} essere raggruppati.


Trigger: tempo, evento, tabelle, propriet\`a.























